\documentclass[
	% tamaño papel
	letterpaper,
	% tamaño fuentes 
	12pt,
	% documento a una cara (twoside incia capitulos en cara derecha)
	oneside]
	% estilo del documento
	{article}
%% ~~~~~~~~~~~~~~~~~~~~~~~~~~~~~~~~~~~~~~~
%% Idioma y fuentes
%% ~~~~~~~~~~~~~~~~~~~~~~~~~~~~~~~~~~~~~~~
% manejo del idioma español
\usepackage[spanish]{babel}
% manejo de fuentes
\usepackage{fontspec}
% define fuente garamond
% http://www.georgduffner.at/ebgaramond/
% los glyfos 'bold' son (c) Adobe Systems
\setmainfont[
	Path = ../master/fonts/,
	BoldFont = AGaramondPro-Bold,
	BoldItalicFont = AGaramondPro-BoldItalic,
	ItalicFont = EBGaramond12-Italic,
	SmallCapsFont = EBGaramond12-SC]{EBGaramond12-Regular}
\usepackage[
	kerning=false,
	tracking=false,
	protrusion=true,
	expansion=false]{microtype}
% estilo de pagina
\pagestyle{plain}
% identacion  	
\parindent 2em
% espacio entre parrafos
\parskip 0.4em
% permite saltos de linea en las notas al pie
\usepackage{bigfoot}
% guionado  de algunas palabras que LaTeX no ha dividido como deberia
%\hyphenation{
% a
%  ...
% b
%  ...
%}

%% ~~~~~~~~~~~~~~~~~~~~~~~~~~~~~~~~~~~~~~~~~~~~~~~~~~~~~
%%             \\\ COMIENZA EL DOCUMENTO ///                        
%% ~~~~~~~~~~~~~~~~~~~~~~~~~~~~~~~~~~~~~~~~~~~~~~~~~~~~~
\begin{document}
\title{\Huge\sc El sujeto y el poder\footnote{Traducción de  Jorge Álvarez Yágüez, con la revisión de Isabel Sobrino Mosteyrín. Recogido en {\it La ética del pensamiento}, 1994. Las notas a pie de página pretenden ayudar al lector en la interpretación del texto, por eso no son solo relativas a datos, cronología, nombres citados, referencias bibliográficas, sino también a perspectivas de lectura; lógicamente cuando la nota es del mismo texto original así se hace constar.\newline Este texto fue publicado por primera vez en Dreyfus, Hubert y Rabinow, Paul, {\it Michel Foucault. Beyond Structuralism and Hermeneutics}, Chicago, The University of Chicago Press, 1983. En francés ha sido recogido en {\it Dits et Ecrits. 1954–1988}. t. {\sc iv}, París, Gallimard, 1994. p. 222–243. El texto se compone de dos partes, la primera («Way Study Power: The Question of the Subject») fue escrita en inglés, y la segunda («How is Power exercised?») en francés. Traducimos, pues, del inglés la primera parte, y del francés la segunda tal y como apareció en la edición francesa del libro citado de Dreyfus y Rabinow ({\it Michel Foucault: un parcours philosophique: audelà de l’objectivité et de la subjectivité}, París, Gallimard, 1984).\newline Es este, en su conjunto, un trabajo fundamental, la segunda y última gran reflexión teórica de Foucault sobre el tema del poder. Siendo la primera la contenida en {\it La voluntad de saber} (1976), ahí aun muy condicionada por la investigación sobre las {\it disciplinas} hecho en {\it Vigilar y castigar} (1975), si bien ya abriéndose a incluir nuevas dimensiones como las ligadas al tratamiento de la {\it población} como ser vivo, que introducía el concepto de {\it biopoder}. Este segundo análisis, sin contradecir en absoluto al anterior, está más influido por sus trabajos iniciados en torno a 1978 sobre el concepto de {\it gobierno}, concepto central para entender este desarrollo y buena parte de las investigaciones del último período foucaultiano. En la primera parte («¿Por qué estudiar el poder?») Foucault, hace una serie de consideraciones de carácter general que son de la mayor importancia para el buen entendimiento del conjunto de su obra. Al respecto sitúa su enfoque del poder, y en particular su relación con lo que estima el núcleo de todo su trabajo que no es otro que el {\it sujeto}. Me parece que esta interpretación es la más acertada de las varias que de sí ofreció su autor, y deshace alguna de las lecturas más extendidas sobre su obra. Al mismo tiempo aborda el punto capital de su crítica de la racionalidad, y su posición respecto del fenómeno de la Ilustración, en diálogo y confrontación con la perspectiva francfortiana. A ello hay que añadir las significativas precisiones que sitúan su posición respecto del marxismo. Por lo que atañe al concepto mismo de poder, es destacable la precisa exposición del concepto de {\it poder pastoral}, un concepto sorprendentemente pasado por alto por tantos estudios sobre el autor. Ya en la segunda parte del texto («¿Cómo se ejerce el poder?») se penetra decisivamente en lo que es la lógica del poder, más allá de las preguntas clásicas por su naturaleza u origen. Es capital al respecto la idea de la necesidad de incorporar al funcionamiento del poder la {\it libertad} del sujeto, sin la que aquel no puede existir, que distancia a Foucault de toda idea de un poder total o cerrado, o de todo {\it conductismo} respecto a los que el poder somete.}}
\author{Michel Foucault}
\date{}
\maketitle

\section*{Por qué estudiar el poder: La cuestión del sujeto}

Las ideas que me gustaría discutir aquí no representan ni una teoría ni una metodología\footnote{Esta precaución en lo tocante al estudio del poder era la que tomaba en {\it La voluntad de saber} al oponer una «analítica del poder» a una teoría del poder: {\it La volonté de savoir, histoire de la sexualité}, París, Gallimard, 1976. p. 109.}.

Me gustaría decir, antes de nada, cuál ha sido la meta de mis trabajos durante los últimos veinte años\footnote{Foucault aborda su obra desde la publicación de la {\it Historia de la locura} en 1961. Como ha sido habitual en él deja fuera los textos anteriores, particularmente su libro {\it Enfermedad mental y personalidad}, de 1955.}. No ha consistido en analizar los fenómenos del poder, ni en elaborar los fundamentos para un tal análisis\footnote{La totalización de su obra como «filosofía del poder» fue frecuente una vez publicado {\it Vigilar y castigar}. Téngase en cuenta que aún no se habían editado los dos últimos tomos de su {\it Historia de la sexualidad}, ambos aparecidos en 1984. {\it El uso de los placeres}, y {\it La inquietud de sí}, ni ninguno de sus cursos en el {\it Collège de France}.}

Mi objetivo, en cambio, ha sido producir una historia de los diferentes modos por los cuales, en nuestra cultura, los seres humanos son constituidos como sujetos. Mi trabajo ha tratado de tres modos de objetivación que transforman a los seres humanos en sujetos. El primero está formado por los modos de investigación que intentan darse a sí mismos el estatuto de ciencia\footnote{Este fue el trabajo expuesto en {\it Las palabras y las cosas}.}; por ejemplo, la objetivación del sujeto hablante en la {\it grammaire générale} [gramática general], la filología y la lingüística. O, dentro aún de este primer modo, la objetivación del sujeto productivo, el sujeto que trabaja, en el análisis de la riqueza y en la economía. O, un tercer ejemplo, la objetivación del puro hecho de ser vivo, en la historia natural o la biología.

En la segunda parte de mi trabajo, he estudiado la objetivación del sujeto a través de lo que llamaré «prácticas divisorias»\footnote{ Segunda parte, que no implica sucesión cronológica respecto de la primera, pues está comprendida por sus obras: {\it Historia de la locura} (1961), {\it Nacimiento de la clínica} (1963), y {\it Vigilar y castigar} (1975).}. El sujeto es dividido en el interior de sí mismo o respecto de los otros. Este proceso lo objetiva. Constituyen ejemplos de ello: el loco y el cuerdo, el enfermo y el sano, los criminales y los «buenos muchachos».

Finalmente, he intentado estudiar —es mi trabajo actual— el modo en que un ser humano se convierte a sí mismo o a sí misma en sujeto\footnote{Parte compuesta por la {\it Historia de la sexualidad}, muy en particular sus dos últimos tomos, cuya temática ya había iniciado a la sazón Foucault.}. Por ejemplo, he elegido el dominio de la sexualidad —cómo los hombres han aprendido a reconocerse a sí mismos como sujetos de «sexualidad».

De modo que no es el poder, sino el sujeto, el tema general de mi investigación\footnote{Esta autointerpretación de su obra, como centrada en el sujeto, deshace la extendida interpretación de un Foucault que solo al final de su obra descubriría este y que además adoptaría una concepción tradicional, dándose un Foucault de «retorno al sujeto» frente al Foucault «filósofo del poder».}.

Es verdad que he llegado a estar completamente involucrado en la cuestión del poder. Pronto me di cuenta de que el ser humano al mismo tiempo que está situado en relaciones de producción y de significación, también lo está en relaciones de poder que son muy complejas. Ahora bien, me parecía que la teoría e historia económica aportaban un buen instrumento para el estudio de las relaciones de producción; que la lingüística y la semiótica ofrecían instrumentos para el estudio de las relaciones de significación; pero para las relaciones de poder no teníamos herramientas de estudio. Solo podíamos recurrir a formas de pensar el poder basadas en modelos jurídicos, esto es: ¿Qué legitima el poder? O podíamos recurrir a formas de pensar el poder basadas en modelos institucionales, esto es: ¿Qué es el Estado?

Era, por tanto, necesario ampliar las dimensiones de una definición del poder si queríamos usar esa definición para estudiar la objetivación del sujeto.

¿Necesitamos una teoría del poder? Desde el momento en que una teoría asume una objetivación anterior, no puede ser afirmada como una base para un trabajo analítico. Pero este trabajo analítico no puede proceder sin una continua conceptualización. Y esta conceptualización implica pensamiento crítico —un examen constante.

Lo primero a examinar es lo que llamaría «necesidades conceptuales». Me refiero a que la conceptualización no debería estar fundada en una teoría del objeto —el objeto conceptualizado no es el único criterio de una buena conceptualización. Tenemos que conocer las condiciones históricas que motivan nuestra conceptualización. Necesitamos una conciencia histórica de nuestra circunstancia presente.

Lo segundo a examinar es el tipo de realidad con la que estamos tratando.

Un escritor de un conocido periódico francés expresó una vez su sorpresa: «¿Por qué tanta gente plantea hoy la noción de poder? ¿Es un tema tan importante? ¿Es tan independiente como para poder ser abordada sin tener en cuenta otros problemas?».

La sorpresa de este escritor me asombró. Soy escéptico a la hora de asumir que esta cuestión se haya planteado por primera vez en el siglo {\sc xx}. De todos modos, para nosotros no es solo una cuestión teórica, sino una parte de nuestra experiencia. Me gustaría mencionar solo dos «formas patológicas», esas dos «enfermedades del poder» —fascismo y stalinismo\footnote{Foucault se ha referido varias veces a estas dos formas de poder en términos semejantes: «La philosophie analytique de la politique», 1978, {\it Dits et Ecrits. 1954–1988}. t. {\sc iii}, París, Gallimard, 1994 p. 535.}. Una de las numerosas razones por las que son, para nosotros, tan desconcertantes es que, a pesar de su singularidad histórica, no son completamente originales. Usaban y extendían mecanismos ya presentes en la mayor parte de otras sociedades. Más que eso: pese a su propia locura interna, utilizaron en gran medida las ideas y dispositivos de nuestra racionalidad política.

Lo que necesitamos es una nueva {\it economía} de las relaciones de poder —usando la palabra economía en su sentido teórico y práctico. Para decirlo en otras palabras: desde Kant, el papel de la filosofía es impedir a la razón ir más allá de los límites de lo que es dado en la experiencia; pero desde el mismo momento— esto es, desde el desarrollo del Estado moderno y de la organización política de la sociedad— el papel de la filosofía es también vigilar los poderes excesivos de la racionalidad política. Lo que supone un desafío bastante grande.

Todo el mundo es consciente de hechos tan banales. Pero el hecho de que sean banales no significa que no existan. Lo que tenemos que hacer con los hechos banales es descubrir —o intentar descubrir— qué específico y quizá original problema está conectado con ellos.

La relación entre racionalización y exceso de poder político es evidente. Y no debiéramos necesitar esperar a la burocracia o los campos de concentración para reconocer la existencia de tal relación. Pero el problema es: ¿Qué hacer con un hecho tan evidente?

¿Haremos un proceso a la razón? Según mi punto de vista, nada sería más estéril. En primer lugar, porque este campo nada tiene que ver con la culpa o la inocencia. En segundo lugar, porque es absurdo considerar la razón como la entidad opuesta a la sinrazón. Finalmente, porque un proceso de ese tipo nos metería en la trampa de interpretar el arbirario y aburrido papel del racionalista o del irracionalista.

¿Investigaremos esta clase de racionalismo que parece ser específico de nuestra cultura moderna y que se origina en la {\it Aufklärung} [Ilustración]? Creo que ese era el enfoque de algunos miembros de la Escuela de Fráncfort. Mi propósito, sin embargo, no es comenzar una discusión de sus obras, aunque son de lo más importante y valioso. Sugeriría, más bien, otro modo de investigar los lazos entre racionalización y poder.

Quizá sea prudente no tomar como un todo la racionalización de la sociedad o de la cultura, sino analizar ese proceso en diversos campos, cada uno en referencia a una experiencia fundamental: locura, enfermedad, muerte, crimen, sexualidad, etc.

Creo que la palabra {\it racionalización} es peligrosa. Lo que tenemos que hacer es analizar racionalidades específicas más que invocar constantemente el progreso de la racionalización en general\footnote{Foucault había presentado estas tesis sobre la crítica de la racionalidad política en «Omnes et singulatim, Towards a criticism of Political Reason», 1979, ob. cit.}.

Aunque la {\it Aufklärung} haya sido una importantísima fase en nuestra historia y en el desarrollo de la tecnología política, creo que tenemos que referirnos a un proceso mucho más remoto si queremos entender cómo hemos quedado atrapados en nuestra propia historia.

Me gustaría sugerir otro camino para avanzar hacia una nueva economía de las relaciones de poder, un camino que es más empírico, más directamente relacionado con nuestra situación presente y que implica más relaciones entre teoría y práctica. Consiste en tomar las formas de resistencia frente a las distintas formas de poder como punto de partida. Para emplear otra metáfora, consiste en usar esta resistencia como un catalizador químico para sacar a la luz relaciones de poder, localizar su posición, descubrir su punto de aplicación y los métodos usados. Más que analizar el poder desde el punto de vista de su racionalidad interna, consiste en analizar las relaciones de poder a través del antagonismo de las estrategias.

Por ejemplo, para descubrir qué entiende nuestra sociedad por cordura, quizá debiéramos investigar qué está sucediendo en el campo de la locura.

Y descubrir lo que entendemos por legalidad a través del campo de la ilegalidad.

Y, para comprender lo relativo a las relaciones de poder, quizá debamos investigar las formas de resistencia y los intentos que haya habido de disociar esas relaciones.

Como punto de partida, tomemos una serie de oposiciones que se han desarrollado alrededor de los últimos años: oposición al poder de los hombres sobre las mujeres, de los padres sobre los hijos, de la psiquiatría sobre la enfermedad mental, de la medicina sobre la población, de la administración sobre los modos de vivir de la gente.

No basta con decir que son luchas antiautoritarias; tenemos que intentar definir con más precisión lo que tienen en común:

\begin{enumerate}
\item En primer lugar, son luchas «transversales»; esto es, no están limitadas a un país. Por supuesto, se desarrollan más fácilmente y con mayor amplitud en algunos países, pero no están limitadas a una forma de gobierno político o económico en particular.

\item El objetivo de esas luchas está en los efectos de poder como tales. Por ejemplo, la profesión médica no es criticada principalmente porque sea un asunto de obtención de beneficios, sino porque ejerce un poder incontrolado sobre los cuerpos de la gente, su salud, su vida y su muerte.

\item Son luchas inmediatas, por dos razones: en tales luchas, la gente critica las instancias de poder que le son más próximas, aquellas que ejercen su acción sobre los individuos. No enfilan al «enemigo principal» sino al enemigo más cercano. Tampoco esperan encontrar una solución a su problema en una fecha futura (esto es, la liberación, la revolución, el fin de la lucha de clases). En comparación con una escala teórica de explicaciones o un orden revolucionario que polariza el historiador, son luchas anárquicas.

  Pero estos no son sus puntos más originales. Los siguientes me parecen más específicos.

\item Son luchas que cuestionan el estatus del individuo: por un lado, afirman el derecho a ser diferentes y subrayan todo lo que hace al individuo verdaderamente individual. Por otro lado, atacan a todo lo que separa al individuo, lo que rompe sus lazos con los otros, escinde la vida comunitaria, fuerza al individuo a volverse sobre sí y le ata a su propia identidad de una manera constrictiva.

  No son luchas exactamente a favor o en contra del «individuo», sino que son luchas en contra del «gobierno de la individualización».

\item Son una oposición a los efectos de poder que están vinculados al conocimiento, a la competencia, a la cualificación: luchas en contra de los privilegios del conocimiento. Pero son también una oposición al secreto, la deformación y las representaciones mistificadoras impuestas a la gente.

  No hay nada de «cientificista» en esto (esto es, una creencia dogmática en el valor del conocimiento científico), pero tampoco es un rechazo escéptico o relativista de toda verdad verificada. Lo que se cuestiona es el modo en que el conocimiento circula y funciona, sus relaciones con el poder. En suma, el {\it régime du savoir} [régimen de saber]

\item Finalmente, todas estas luchas presentes giran en torno a la cuestión: ¿Quiénes somos? Son un rechazo de estas abstracciones, de la violencia estatal económica e ideológica que ignora quienes somos individualmente, y también un rechazo de una inquisición científica o administrativa que determine quién es uno.
\end{enumerate}

Para resumir, el principal objetivo de esas luchas es atacar no tanto a «tal o cual» institución de poder, o grupo, élite o clase, sino más bien a una técnica, a una forma de poder.

\bigskip

Esta forma de poder se aplica a la vida cotidiana inmediata que categoriza al individuo, lo marca en su propia individualidad, lo ata a su propia identidad, le impone una ley de verdad que ha de reconocer y que los otros han de reconocer en él. Es una forma de poder que transforma a los individuos en sujetos. Hay dos significados de la palabra {\it sujeto}: sujetado a algún otro por el control y la dependencia, y atado a su propia identidad por la conciencia o el autoconocimiento. Ambos significados sugieren una forma de poder que somete y hace sujeto.

En general, puede decirse que hay tres tipos de luchas: contra las formas de dominación (étnica, social y religiosa); contra las formas de explotación que separan a los individuos de lo que producen; o contra lo que ata al individuo a sí mismo y de ese modo le somete a otros (luchas contra la sujeción, contra las formas de subjetividad y sumisión).

Pienso que en la historia se pueden encontrar numerosos ejemplos de esas tres clases de luchas sociales, ya aisladas una de otra, o entremezcladas. Pero incluso cuando están mezcladas, una de ellas la mayoría de las veces prevalece. Por ejemplo, en las sociedades feudales, prevalecían las luchas contra las formas de dominación étnica o social, aun cuando la explotación económica podría haber sido muy importante entre las causas de revuelta.

En el siglo {\sc xix}, la lucha contra la explotación ocupó el primer plano.

Y hoy en día, la lucha contra las formas de sujeción —contra la sumisión de la subjetividad— ha llegado a ser cada vez más importante, aunque las luchas contra la dominación y la explotación no han desaparecido. Muy al contrario.

Sospecho que no es la primera vez que nuestra sociedad se ve confrontada con esta clase de lucha. Todos aquellos movimientos que tuvieron lugar en los siglos {\sc xv} y {\sc xvi} y que han tenido a la Reforma como su principal expresión y resultado debieran ser analizados como una gran crisis de la experiencia occidental de la subjetividad y una revuelta contra la clase de poder religioso y moral que dió forma, durante la Edad Media, a esa subjetividad. La necesidad de tomar parte directa en la vida espiritual, en la obra de la salvación, en la verdad que se encuentra en el Libro —todo eso fue una lucha por una nueva subjetividad\footnote{Foucault nos dejó algunas preciosas observaciones de este movimiento en los trabajos que rodearon al que habría de ser el cuarto volumen de la {\it Historia de la sexualidad, Las confesiones de la carne} (Les Aveux de la chair) que no ha llegado a publicarse.}.

Me doy cuenta de qué objeciones pueden hacerse. Podemos decir que todos los tipos de sujeción son fenómenos derivados, que son simplemente las consecuencias de otros procesos, económicos y sociales: fuerzas de producción, luchas de clases y estructuras ideológicas que determinan la forma de subjetividad.

Es cierto que los mecanismos de sujeción no pueden ser estudiados al margen de su relación con los mecanismos de explotación y de dominación. Pero no constituyen simplemente la «terminal» de mecanismos más fundamentales. Mantienen relaciones complejas y circulares con otras formas\footnote{Una sencilla aclaración que ayuda a situar con más precisión el pensamiento foucaultiano respecto de los enfoques marxistas.}.

La razón de que esta clase de lucha tienda a prevalecer en nuestra sociedad es debida al hecho de que desde el siglo {\sc xvi}, una nueva forma política de poder se ha estado desarrollando sin cesar. Esta nueva estructura política, como todo el mundo sabe, es el Estado. Pero la mayoría de las veces, el Estado es visto como una clase de poder político que ignora a los individuos, atendiendo solo a los intereses de la totalidad o, debería decir, de una clase o un grupo de entre los ciudadanos.

Esto es completamente cierto. Pero me gustaría señalar el hecho de que el poder del Estado (y esta es una de las razones de su fuerza) es una forma de poder a la vez individualizante y totalizante. Nunca, creo, en la historia de las sociedades humanas —ni siquiera en la vieja sociedad china— ha habido una combinación tan compleja de técnicas de individualización y procedimientos de totalización en las mismas estructuras políticas.

Esto es debido al hecho de que el Estado moderno occidental ha integrado en una nueva figura política, un viejo poder técnico que se originó en instituciones cristianas. Podemos llamar a este poder técnico, poder pastoral\footnote{El último concepto relativo al poder imperante en nuestras sociedades que Foucault desarrolló, no fue el tan difundido, como mal entendido, de {\it biopolítica}, sino este que aquí nos presenta, de {\it poder pastoral}.}.

En primer lugar, unas pocas palabras sobre este poder pastoral:

A menudo se ha dicho que el cristianismo dio lugar a un código ético fundamentalmente diferente al del mundo antiguo. Menos énfasis se pone habitualmente en el hecho de que presentó y expandió nuevas relaciones de poder a lo largo del mundo antiguo.

El cristianismo es la única religión que se ha organizado a sí misma como una iglesia. Y como tal, postula en principio que ciertos individuos puedan, por su cualidad religiosa, servir a otros, no en tanto que príncipes, magistrados, profetas, adivinos, benefactores, educadores, etcétera, sino en tanto que pastores. Sin embargo, esta palabra designa una forma muy especial de poder.

\begin{enumerate}
\item Es una forma de poder cuyo objetivo último es asegurar la salvación individual en el otro mundo.

\item El poder pastoral no es simplemente una forma de poder que impone; también tiene que estar preparado para sacrificarse por la vida y la salvación del rebaño. Por consiguiente, es distinto del poder real, que exige el sacrificio de sus sujetos para salvar el trono.

\item Es una forma de poder que no cuida solo de la comunidad como un todo, sino de cada individuo en particular, durante toda su vida.

\item Finalmente, esta forma de poder no puede ser ejercida sin conocer el interior de la mente de las personas, sin explorar sus almas, sin hacer que revelen sus más íntimos secretos. Eso implica un conocimiento de la conciencia y una capacidad para dirigirla.
\end{enumerate}

Esta forma de poder está orientada a la salvación (en tanto que opuesta al poder político). Es oblativa (en tanto que opuesta al principio de la soberanía). Es individualizante (en tanto que opuesta al poder jurídico); es coextensiva y continua respecto de la vida; está vinculada a una producción de verdad —la verdad del individuo mismo.

Pero todo esto es parte de la historia, dirán ustedes; el pastorado, si bien no ha desaparecido, al menos ha perdido la parte principal de su eficiencia.

Eso es verdad, pero creo que debemos distinguir entre dos aspectos del poder pastoral —entre la institucionalización eclesiástica, que ha cesado, o al menos perdido su vitalidad desde el siglo {\sc xviii}, y su función, que se ha extendido y multiplicado fuera de la institución eclesiástica.

Un fenómeno importante tuvo lugar en torno al siglo {\sc xviii}, fue una nueva distribución, una nueva organización de esta clase de poder individualizante.

No creo que debamos considerar el «Estado moderno» como una entidad que fue desarrollada por encima de los individuos, ignorando lo que son e incluso su misma existencia, sino, por el contrario, como una muy sofisticada estructura en la cual los individuos pueden ser integrados bajo una condición: que esta individualidad esté modelada en una nueva forma, y sometida a un conjunto de patrones muy específicos.

En cierto modo, podemos ver el Estado como una matriz moderna de individualización, o una nueva forma de poder pastoral.

Unas pocas palabras más sobre este nuevo poder pastoral.

\begin{enumerate}
\item Podemos observar un cambio en su objetivo. Ya no se trataba por más tiempo de guiar a la gente a su salvación en el otro mundo, sino más bien de asegurarla en este. Y en este contexto, la palabra {\it salvación} adquiere diferentes sentidos: salud, bienestar (esto es, riqueza suficiente, nivel de vida) seguridad, protección contra accidentes. Una serie de objetivos «mundanos» tomó el lugar de los objetivos religiosos del pastorado tradicional, todo de lo más fácilmente, porque este, por varias razones, siguió de manera accesoria un cierto número de estos objetivos; solamente tenemos que pensar en el papel de la medicina y su función de bienestar asegurada durante largo tiempo por la Iglesia católica y protestante.

\item Al mismo tiempo, los funcionarios del poder pastoral se incrementan. A veces esta forma de poder fue ejercida por aparatos estatales, o, en cualquier caso, por una institución pública como la policía (no olvidemos que en el siglo {\sc xviii} la fuerza policial no fue creada solamente para mantener la ley y el orden, ni para asistir a los gobiernos en su lucha contra sus enemigos, sino para asegurar el aprovisionamiento urbano, la higiene, la salud y los estándares considerados necesarios para la artesanía y el comercio)\footnote{Respecto al estudio de la policía, véase el curso de 1977–1978, {\it Sécurité, territoire, population}, Gallimard, París, 2004 (hay trad. en Akal y {\sc fce}) especialmente las lecciones del 29 de marzo y del 5 de abril de 1978.}. A veces el poder era ejercido por empresas privadas, sociedades de asistencia, benefactores y en general por filántropos. Pero las instituciones antiguas, por ejemplo, la familia, eran también movilizadas en esa época para asumir funciones pastorales. Era también ejercido por estructuras complejas como la medicina, que incluía iniciativas privadas con la venta de servicios según los principios de la economía de mercado, pero que también incluía instituciones públicas tales como los hospitales.

\item Finalmente, la multiplicación de los objetivos y de los agentes del poder pastoral centró el desarrollo del conocimiento sobre el hombre en dos papeles: uno, globalizante y cuantitativo, referente a la población; el otro analítico, referente al individuo.
\end{enumerate}

Y esto implica que el poder de tipo pastoral que durante siglos —más de un milenio— ha estado vinculado a instituciones religiosas definidas, se extendió rápidamente a la totalidad del cuerpo social; encontró apoyo en una multitud de instituciones. Y, en vez de un poder pastoral y un poder político más o menos vinculados uno al otro, más o menos rivales, hubo una «táctica» individualizante que caracterizó a una serie de poderes: los de la familia, medicina, psiquiatría, educación, empresa.

A finales del siglo {\sc xviii}, Kant escribió en una revista alemana —{\it Berliner Monatschrift} [Revista mensual de Berlín]— un breve texto. El título era {\it Was heisst Aufklärung?} [Qué significa Ilustración] Fue considerado durante mucho tiempo, y todavía lo es, un trabajo de relativamente poca importancia. Pero yo no puedo evitar encontrarlo muy interesante y sorprendente, porque fue la primera vez que un filósofo proponía como tarea filosófica investigar no solo el sistema metafísico o los fundamentos del conocimiento científico, sino un acontecimiento histórico —un acontecimiento reciente, incluso contemporáneo.

Cuando en 1784, Kant preguntaba {\it Was heisst Aufklärung}?, quería decir, ¿Qué está ocurriendo hoy? ¿Qué nos está pasando? ¿Cómo es este mundo, este período, este preciso momento en el que estamos viviendo?

O, en otras palabras: ¿Qué somos nosotros como {\it Aufklärer}, como parte de la Ilustración? Comparen esto con la pregunta cartesiana: ¿Quién soy yo? ¿Yo, como sujeto único, pero universal y ahistórico? Yo, para Descartes, es cualquiera, en cualquier lugar y en cualquier momento.

Pero Kant pregunta algo distinto: ¿Qué somos? En un muy preciso momento de la historia. La pregunta de Kant aparece como un análisis a la vez de nosotros y de nuestro tiempo.

Pienso que esta cuestión de la filosofía cobró cada vez más importancia, Hegel, Nietzsche…

La otra cuestión de la «filosofía universal» no desapareció. Pero la tarea de la filosofía como un análisis crítico de nuestro mundo es algo cada vez más importante. Quizá el más importante de todos los problemas filosóficos es el problema del tiempo presente, y de lo que somos en este preciso momento.

Quizá el objetivo hoy no sea descubrir lo que somos, sino rechazar lo que somos. Tenemos que imaginar y construir lo que podríamos ser para librarnos de esta clase de política de {\it double bind} [doble vínculo], que es la simultánea individualización y totalización de las estructuras del poder moderno.

La conclusión sería que el problema ético, político, social, filosófico de nuestros días no es intentar liberar al individuo del Estado, y de las instituciones estatales, sino liberarnos a la vez del Estado y del tipo de individualización que está vinculada al Estado. Tenemos que promover nuevas formas de subjetividad a través del rechazo de esta clase de individualidad que nos ha sido impuesta durante varios siglos.

\section*{¿Cómo se ejerce el poder?}

Para algunos, interrogarse sobre el «cómo» del poder significaría limitarse a describir sus efectos sin relacionarlos nunca ni con causas ni con una naturaleza. Eso sería hacer de ese poder una sustancia misteriosa que no queremos investigar en sí misma, sin duda porque preferimos no «poner en cuestión». En este proceder, del que no se da razón, ellos sospechan un fatalismo. Pero su misma desconfianza ¿no muestra que ellos suponen que el Poder es algo que existe con su origen de una parte, su naturaleza de otra y, en fin, sus manifestaciones?

Si concedo un cierto privilegio provisional a la cuestión del «cómo», no es que quiera eliminar la cuestión del «qué» y del «por qué». Es para plantearlas de otro modo; mejor aún: para saber si es legítimo imaginar un «Poder» que une en sí un qué, un por qué y un cómo. Hablando sin rodeos, diría que comenzar el análisis por el «cómo» es introducir la sospecha de que el «Poder» como tal no existe; es preguntarse en todo caso a qué contenidos asignables se puede apuntar cuando se hace uso de ese término majestuoso, globalizante y sustantificador; es sospechar que se deja escapar un conjunto de realidades muy complejas cuando se da vueltas indefinidamente sobre la doble cuestión: ¿Qué es el Poder? y ¿de dónde viene? La modesta cuestión, completamente llana y empírica: ¿Cómo ocurre?, enviada como exploradora, no tiene por función hacer pasar de contrabando una «metafísica», o una «ontología» del poder, sino intentar una investigación crítica en la temática del poder.

\subsection*{«Cómo», no en el sentido de «¿cómo se manifiesta?», sino de ¿cómo se ejerce? Y, «¿qué ocurre cuando los individuos ejercen, como se dice, su poder sobre otros?}

De este «poder», hay que distinguir, en primer lugar, el que se ejerce sobre las cosas, y que proporciona la capacidad de modificarlas, de utilizarlas, de consumirlas o de destruirlas —un poder que remite a aptitudes directamente inscritas en el cuerpo o mediatizadas por recursos instrumentales. Digamos que aquí se trata de «capacidad». Lo que caracteriza, en cambio, al «poder» que se trata de analizar aquí, es que pone en juego relaciones entre individuos (o entre grupos). Pues no hay que engañarse: si se habla del poder de las leyes, de las instituciones o de las ideologías, si se habla de estructuras o de mecanismos de poder, es solamente en la medida en que se supone que «algunos» ejercen un poder sobre otros. El término «poder» designa relaciones entre «participantes» (y, por esto, no pienso en un sistema de juego, sino simplemente, y permaneciendo por el momento en la mayor generalidad, en un conjunto de acciones que se inducen y se responden las unas a las otras).

Hay que distinguir también las relaciones de poder de las relaciones de comunicación que transmiten una información a través de una lengua, un sistema de signos o cualquier otro medio simbólico. Sin duda comunicar es siempre una cierta manera de actuar sobre el otro o los otros. Pero la producción y la puesta en circulación de elementos significantes puede bien tener por objetivo o por consecuencia efectos de poder; estos no son simplemente un aspecto de aquellas. Pasen o no a través de sistemas de comunicación, las relaciones de poder tienen su especificidad.

«Relaciones de poder», «relaciones de comunicación», «capacidades objetivas», no deben, pues, confundirse. Lo que no quiere decir que se trate de tres dominios separados, y que, por una parte, esté el dominio de las cosas, de la técnica finalista, del trabajo y de la transformación de lo real, por otra, el de los signos, de la comunicación, de la reciprocidad y de la fabricación del sentido, y, finalmente, el de los medios de coerción, de la desigualdad y de la acción de los hombres sobre los hombres\footnote{Cuando Habermas distingue entre dominación, comunicación y actividad de fines, no ve en ello, creo, tres dominios diferentes, sino tres «trascendentales». (Nota del autor).}. Se trata de tres tipos de relaciones que, de hecho, están siempre imbricadas unas en otras, proporcionándose apoyo recíproco y sirviéndose mutuamente de instrumento. La puesta en práctica de capacidades objetivas en sus formas más elementales implica relaciones de comunicación (se trate de información previa o de trabajo compartido); está también vinculada a relaciones de poder (se trate de tareas obligatorias, de gestos impuestos por una tradición o un aprendizaje, de subdivisiones o de reparto más o menos obligatorio de trabajo). Las relaciones de comunicación implican actividades finalistas (aunque no fuera más que la «correcta» puesta en circulación de los elementos significantes) y, por el mero hecho de que modifican el campo informativo de los participantes, inducen efectos de poder. En cuanto a las relaciones de poder propiamente dichas, se ejercen en una parte extremadamente importante a través de la producción y el intercambio de signos; y no son apenas disociables tampoco de las actividades finalistas, se trate de las que permiten el ejercicio de ese poder (como las técnicas de adiestramiento, los procedimientos de dominación, las maneras de obtener obediencia), o de aquellas que apelan, para desarrollarse, a relaciones de poder (como en la división del trabajo y la jerarquía de las tareas).

Por supuesto, la coordinación entre estos tres tipos de relaciones no es ni uniforme ni constante. No hay en una sociedad dada un tipo general de equilibrio entre las actividades finalistas, los sistemas de comunicación y las relaciones de poder. Hay más bien diversas formas, diversos lugares, diversas circunstancias u ocasiones en que esas interrelaciones se establecen sobre un modelo específico. Pero hay también «bloques» en los cuales el ajuste de las capacidades, las redes de comunicación y las relaciones de poder constituyen sistemas reglados y concertados. Sea, por ejemplo, una institución escolar: su disposición espacial, el reglamento meticuloso que rige su vida interna, las diferentes actividades que se organizan en ella, los diversos personajes que allí viven o se encuentran, cada uno con una función, un lugar, un rostro bien definido; todo eso constituye un «bloque» de capacidad–comunicación–poder. La actividad que asegura el aprendizaje y la adquisición de las aptitudes o de los tipos de comportamiento se desarrolla ahí a través de todo un conjunto de comunicaciones regladas (lecciones, preguntas y respuestas, órdenes, exhortaciones, signos codificados de obediencia, marcas diferenciales del «valor» de cada uno y de los niveles de saber) y a través de toda una serie de procedimientos de poder (encierro, vigilancia, recompensa y castigo, jerarquía piramidal).

Estos bloques donde la puesta en práctica de capacidades técnicas, el juego de las comunicaciones y las relaciones de poder son ajustados los unos a los otros según fórmulas pensadas, constituyen lo que se puede llamar, ampliando un poco el sentido de la palabra, «disciplinas». El análisis empírico de algunas disciplinas tal como se han constituido históricamente presenta, por eso mismo, un cierto interés. En primer lugar, porque las disciplinas muestran, según esquemas artificialmente claros y decantados, la manera en la que pueden articularse, unos sobre otros, los sistemas de finalidad objetiva, de comunicaciones y de poder. Porque muestran también diferentes modelos de articulaciones (ya con preminencia de las relaciones de poder y de obediencia, como en las disciplinas de tipo monástico o de tipo penitenciario, ya con preminencia de las actividades finalistas como en las disciplinas de talleres o de hospitales, ya con preminencia de las relaciones de comunicación como en las disciplinas de aprendizaje; ya también con una saturación de los tres tipos de relaciones como quizá en la disciplina militar, donde una plétora de signos marca hasta la redundancia relaciones de poder densas y cuidadosamente calculadas para procurar un cierto número de efectos técnicos).

Y lo que hay que entender por el disciplinamiento de las sociedades desde el siglo {\sc xviii} en Europa no es, por supuesto, que los individuos que forman parte de ellas se vuelvan cada vez más obedientes, ni que aquellas empiecen a asemejarse a cuarteles, escuelas o prisiones, sino que en ellas se ha buscado un ajuste cada vez mejor controlado —cada vez más racional y económico— entre las actividades productivas, las redes de comunicación y el juego de las relaciones de poder.

Abordar el tema del poder por un análisis del «cómo» es, pues, operar, en relación a la suposición de un «Poder» fundamental, varios desplazamientos críticos. Significa adoptar por objeto de análisis las {\it relaciones de poder} y no un poder; relaciones de {\it poder} que son distintas tanto de las capacidades objetivas como de las relaciones de comunicación; relaciones de poder, en fin, que pueden tomarse en la diversidad de sus encadenamientos con esas capacidades y esas relaciones.

\subsection*{¿En qué consiste la especificidad de las relaciones de poder?}

El ejercicio del poder no es simplemente una relación entre participantes individuales o colectivos; es un modo de acción de algunos sobre algunos otros. Lo que quiere decir, claro está, que no hay algo como el «Poder» o un «poder» que pueda existir globalmente, masivamente o en estado difuso, concentrado o distribuido: no hay más poder que el ejercido por los «unos» sobre los «otros»; el poder no existe más que en acto, incluso si se inscribe en un campo de posibilidades dispersas que se apoya en estructuras permanentes. Eso quiere decir también que el poder no es del orden del consentimiento; no es en sí mismo renuncia a una libertad, transferencia de derecho, poder de todos y de cada uno delegado a algunos (lo que no impide que el consentimiento pueda ser una condición para que la relación de poder exista y se mantenga); la relación de poder puede ser el efecto de un consentimiento anterior o permanente; no es, en su propia naturaleza, la manifestación de un consenso.

¿Quiere esto decir que haya que buscar el carácter propio de las relaciones de poder en una violencia que sería su forma primitiva, su secreto permanente y su último recurso —lo que aparece, en última instancia, como su verdad, cuando se ve obligado a quitarse la máscara y mostrarse tal cual es? De hecho, lo que define una relación de poder es un modo de acción que no actúa directa e inmediatamente sobre los otros, sino que actúa sobre la propia acción de éstos. Una acción sobre la acción, sobre acciones eventuales, o actuales, futuras o presentes. Una relación de violencia actúa sobre un cuerpo, sobre cosas: fuerza, pliega, quiebra, destruye; cierra todas las posibilidades; no tiene, pues, en torno a ella otro polo que el de la pasividad; y si encuentra una resistencia no tiene otra opción que intentar reducirla. Una relación de poder, en cambio, se articula sobre dos elementos que le son indispensables para ser precisamente una relación de poder: que «el otro» (aquel sobre el que esta se ejerce) sea claramente reconocido y mantenido hasta el final como sujeto de acción; y que se abra, ante la relación de poder, todo un campo de respuestas, reacciones, efectos e invenciones posibles.

La puesta en juego de relaciones de poder no es más exclusiva del uso de la violencia que de la adquisición de los consentimientos; sin duda ningún ejercicio del poder puede prescindir del uno y de la otra, ni, frecuentemente, de los dos a la vez. Pero, aunque son sus instrumentos o sus efectos, no constituyen su principio ni tampoco su naturaleza. El ejercicio del poder puede suscitar tanta aceptación como se quiera: puede acumular los muertos y protegerse tras todas las amenazas que pueda imaginar. No es en sí mismo una violencia que sepa a veces ocultarse, o un consentimiento que implícitamente se reconduzca. Es un conjunto de acciones sobre acciones posibles: opera sobre el campo de posibilidad en donde viene a inscribirse el comportamiento de sujetos actuantes; incita, induce, disuade, facilita o hace más difícil, amplía o limita, vuelve más o menos probable; en el extremo, obliga o impide absolutamente; pero es siempre una manera de actuar sobre uno o varios sujetos actuantes y sobre lo que hacen o son capaces de hacer. Una acción sobre acciones.

El término «conducta» a pesar de su equivocidad misma es quizá uno de los que mejor permiten aprehender lo que hay de específico en las relaciones de poder. La «conducta» es a la vez el acto de «llevar» a los otros (según mecanismos de coerción más o menos estrictos) y la manera de comportarse en un campo más o menos abierto de posibilidades. El ejercicio del poder consiste en «conducir conductas» y en disponer la probabilidad. El poder, en el fondo, es menos del orden del enfrentamiento entre dos adversarios, o del compromiso del uno respecto del otro, que del orden del «gobierno». Hay que dejar a esta palabra la significación amplísima que tenía en el siglo {\sc xvi}\footnote{Sobre el concepto de «gobierno», clave para entender toda esta última reflexión de Foucault sobre el poder, véase: «Omnes et singulatim: Toward a Criticism of Political Reason», en {\it The Tanner Lectures on Human Values}, t. {\sc ii}, University of Utha Press, 1981, págs., 223–254; hay trad. al español en Paidós; véase también el Curso de 1977–1978, {\it Securité, territoire, population}, ob. cit.}. No se refería solo a estructuras políticas y a la gestión de los Estados, sino que designaba la manera de dirigir la conducta de individuos o de grupos: gobierno de los niños, de las almas, de las comunidades, de las familias, de los enfermos. No abarcaba simplemente formas instituidas y legítimas de sujeción [assujettissement] política o económica; sino modos de acción más o menos pensados y calculados, pero todos destinados a actuar sobre las posibilidades de acción de otros individuos. Gobernar, en este sentido, es estructurar el campo de acción eventual de los otros. El modo de relación propia del poder no habría, pues, que buscarlo del lado de la violencia y de la lucha, ni del lado del contrato y del lazo voluntario (que todo lo más pueden ser sus instrumentos), sino del lado de este modo de acción singular —ni guerrera ni jurídica— que es el gobierno.

Cuando se define el ejercicio del poder como un modo de acción sobre las acciones de los otros, cuando se lo caracteriza por el «gobierno» de los hombres los unos por los otros —en el sentido más amplio de esta palabra— se incluye un elemento muy importante: el de la libertad. El poder no se ejerce más que sobre «sujetos libres» y en tanto que son «libres» —entendemos por ello sujetos individuales o colectivos que tienen ante sí un campo de posibilidades, donde pueden tener lugar varias conductas, varias reacciones y diversos modos de comportamiento. Allí donde las determinaciones están saturadas no hay relaciones de poder: la esclavitud no es una relación de poder cuando el hombre está encadenado (se trata entonces de una relación física de coerción) sino justamente cuando puede desplazarse y, en el extremo, escapar\footnote{«Un hombre encadenado y batido está sometido a la fuerza que se ejerce sobre él. No al poder. Pero si se puede inducirle a hablar, cuando su último recurso sería cerrar la boca prefiriendo la muerte, entonces se le ha empujado a comportarse de una determinada manera. Su libertad ha sido sujeta al poder», «Omnes et singulatim», ob. cit., {\it Dits et Ecrits. 1954–1988}. t. {\sc iv}, París, Gallimard, 1994. p. 360.}. No hay pues un cara a cara del poder y la libertad, con una relación de exclusión entre sí (donde quiera que el poder se ejerza la libertad desaparece), sino un juego mucho más complejo: en ese juego, la libertad va a aparecer claramente como condición de existencia del poder (a la vez su condición previa, puesto que es necesario que haya libertad para que el poder se ejerza, y también su soporte permanente puesto que si se hurtase enteramente al poder que se ejerce sobre ella este desaparecería por ese mismo hecho y debería encontrar un sustituto en la coerción pura y simple de la violencia); pero aparece también como lo que no podrá más que oponerse a un ejercicio del poder que tiende a fin de cuentas a determinarla enteramente.

La relación de poder y la insumisión de la libertad no pueden, pues, ser separadas. El problema central del poder no es el de la «servidumbre voluntaria»\footnote{Referencia al clásico de 1549 de La Boetie, {\it Discours de la servitude volontaire}, hay varias traducciones al español, por ejemplo, la de Pedro Lomba en Trotta.} (¿cómo podemos desear ser esclavos?): en el corazón de la relación de poder, «provocándola» sin cesar, están la renuencia del querer y la intransitividad de la libertad. Más que de un «antagonismo» esencial, sería mejor hablar de un «agonismo» —de una relación que es a la vez de incitación recíproca y de lucha— menos de una oposición término a término que los bloquea uno frente a otro que de una provocación permanente.

\subsection*{¿Cómo analizar la relación de poder?}

Se puede, quiero decir, es perfectamente legítimo analizarla en instituciones muy determinadas; estas constituyen un observatorio privilegiado para captarlas [las relaciones de poder], diversificadas, concentradas, puestas en orden y llevadas, al parecer, a su más alto punto de eficacia; es ahí, en una primera aproximación, donde se puede esperar ver aparecer la forma y la lógica de sus mecanismos elementales. Sin embargo, el análisis de las relaciones de poder en espacios institucionales cerrados presenta un cierto número de inconvenientes. En primer lugar, el hecho de que una parte importante de los mecanismos puestos en práctica por una institución estén destinados a asegurar su propia conservación comporta el riesgo de descifrar, sobre todo en las relaciones de poder «intra–institucionales», funciones esencialmente reproductivas. En segundo lugar, uno se expone, al analizar las relaciones de poder a partir de las instituciones, a buscar en estas la explicación y el origen de aquellas, es decir, en suma, a explicar el poder por el poder. Y en fin, en la medida en que las instituciones actúan esencialmente por la puesta en juego de dos elementos: reglas (explícitas o silenciosas) y un aparato, a riesgo de dar a la una y al otro un privilegio exagerado en la relación de poder y, por tanto, no ver en esta más que modulaciones de la ley y de la coerción.

No se trata de negar la importancia de las instituciones en la disposición de las relaciones de poder, sino de sugerir que es preciso, más bien, analizar las instituciones a partir de las relaciones de poder y no a la inversa, y que el punto de anclaje fundamental de estas, aun si toman cuerpo y se cristalizan en una institución, hay que buscarlo más acá de ella.

Volvamos a hablar de la definición según la cual el ejercicio del poder sería una manera para unos de estructurar el campo de acción posible de los otros. Lo que sería, entonces, lo propio de una relación de poder es que sería un modo de acción sobre acciones. Es decir, que las relaciones de poder arraigan hondo en el nexo social, y que no conforman por encima de la «sociedad» una estructura suplementaria cuya radical desaparición quizá podría soñarse. Vivir en sociedad es, de todas formas, vivir de manera que sea posible actuar los unos sobre la acción de los otros. Una sociedad sin «relaciones de poder» no puede ser más que una abstracción. Lo que, dicho sea de paso, hace todavía más necesario políticamente el análisis de lo que son en una sociedad dada, de su formación histórica, de lo que las torna sólidas o frágiles, de las condiciones que son necesarias para transformar unas y abolir las otras. Pues decir que no puede haber sociedad sin relación de poder no quiere decir ni que las que existen sean necesarias, ni que, de todos modos, el «Poder» constituya, en el corazón de las sociedades, una fatalidad ineludible; sino que el análisis, la elaboración, la puesta en cuestión recurrente de las relaciones de poder, y del «agonismo» entre relaciones de poder e intransitividad de la libertad, son una tarea política incesante, e incluso que esa es la tarea política inherente a toda existencia social.

\bigskip 

Concretamente, el análisis de las relaciones de poder exige que se establezca un cierto número de puntos:

\begin{list}{}{}
 \item[\bf\it 1) El sistema de las diferenciaciones] que permiten actuar sobre las acciones de los otros: diferencias jurídicas o tradicionales de posición y de privilegio; diferencias económicas en la apropiación de las riquezas y de los bienes, diferencias de lugar en los procesos de producción, diferencias lingüísticas o culturales, diferencias en el saber hacer y en las competencias, etc. Toda relación de poder pone en práctica diferencias que son para ella a la vez condiciones y efectos.

\item[\bf\it 2) El tipo de objetivos] perseguidos por los que actúan sobre la acción de los otros: mantenimiento de privilegios, acumulación de beneficios, puesta en práctica de autoridad estatutaria, ejercicio de una función o de un oficio.

\item[\bf\it 3) Las modalidades instrumentales:] según que el poder sea ejercido mediante la amenaza de las armas, por los efectos de la palabra, a través de las diferencias económicas, por mecanismos más o menos complejos de control, por sistemas de vigilancia, con o sin archivos, según reglas explícitas o no, permanentes o modificables, con o sin dispositivos materiales, etc.

\item[\bf\it 4) Las formas de institucionalización:] estas pueden mezclar disposiciones tradicionales, estructuras jurídicas, fenómenos de hábito o de moda (como se ve en las relaciones de poder que atraviesan la institución familiar); pueden también tomar el aspecto de un dispositivo cerrado sobre sí mismo con sus lugares específicos, sus reglamentos propios, sus estructuras jerárquicas cuidadosamente diseñadas, y una relativa autonomía funcional (como en las instituciones escolares o militares); pueden formar sistemas muy complejos dotados de aparatos múltiples, como en el caso del Estado, que tiene por función constituir la envoltura general, la instancia de control global, el principio de regulación y, en una cierta medida, también de distribución de todas las relaciones de poder en un conjunto social dado.

\item[\bf\it 5) Los grados de racionalización:] pues la puesta en juego de relaciones de poder como acción sobre un campo de posibilidad puede ser más o menos elaborada en función de la eficacia de los instrumentos y de la certeza del resultado (refinamientos tecnológicos mayores o menores en el ejercicio del poder) o incluso en función del coste eventual (se trate del coste económico de los medios puestos en práctica o el coste «de respuesta» constituido por las resistencias encontradas). El ejercicio del poder no es un hecho bruto, un dato institucional, o una estructura que se mantenga o se destruya; se elabora, se transforma, se organiza, se dota de procedimientos más o menos adecuados.
\end{list}

Se ve por qué el análisis de las relaciones de poder en una sociedad no puede reducirse al estudio de una serie de instituciones, ni siquiera al estudio de todas las que merecerían el nombre de «política». Las relaciones de poder enraízan en el conjunto de la red social. Esto no quiere decir, sin embargo, que haya un principio de Poder primero y fundamental que domine hasta el menor elemento de la sociedad, sino que, a partir de esta posibilidad de acción sobre la acción de los otros que es coextensiva a toda relación social, de las múltiples formas de disparidad individual, objetivos, instrumentaciones dadas sobre nosotros y sobre los otros, institucionalización más o menos sectorial o global, organización más o menos pensada, se definen las diferentes formas de poder. Las formas y los lugares de «gobierno» de los hombres unos por otros son múltiples en una sociedad; se superponen, se entrecruzan, se limitan y se anulan a veces, se refuerzan en otros casos. Que el Estado en las sociedades contemporáneas no es simplemente una de las formas o uno de los lugares —aunque fuese el más importante— sino que de una cierta manera todos los demás tipos de relación de poder se refieren a él, es un hecho cierto. Pero esto no es porque cada una se derive de él. Es más bien porque se ha producido una estatalización continua de las relaciones de poder (aunque no haya tomado la misma forma en el orden pedagógico, judicial, económico, familiar). En el sentido esta vez restringido de la palabra «gobierno», se podría decir que las relaciones de poder han sido progresivamente gubernamentalizadas [gouvernementalisées], es decir, elaboradas, racionalizadas y centralizadas en la forma o bajo la caución de las instituciones estatales.

\subsection*{Relaciones de poder y relaciones estratégicas}

La palabra estrategia se emplea normalmente en tres sentidos. En primer lugar, para designar la elección de los medios empleados para llegar a un fin; se trata de la racionalidad puesta en práctica para lograr un {\it objetivo}. [En segundo lugar,] para designar la manera en la que un participante, en un juego dado, actúa en función de lo que piensa que será la acción de los otros, y de lo que considera que los otros pensarán que es la suya; en suma, la manera en que se intenta tener {\it ventaja sobre el otro}. Y finalmente, para designar el conjunto de los procedimientos utilizados en un enfrentamiento con el fin de privar al adversario de sus medios de combate y obligarle a renunciar a la lucha; se trata entonces de los medios destinados a obtener la {\it victoria}. Estos tres sentidos se reúnen en las situaciones de enfrentamiento —guerra o juego— donde el objetivo es actuar sobre un adversario de tal manera que la lucha sea para él imposible. La estrategia se define entonces por la elección de las soluciones «ganadoras». Pero hay que tener presente que se trata aquí de un tipo muy particular de situación, y que hay otras en las que es preciso mantener la distinción entre los diferentes sentidos de la palabra estrategia.

En el primer sentido indicado, se puede llamar «estrategia de poder» al conjunto de los medios puestos en práctica para hacer funcionar o para mantener un dispositivo de poder. Se puede también hablar de estrategia propia de las relaciones de poder en la medida en que estas constituyen modos de acción sobre la acción posible, eventual, supuesta, de otros. Se puede, pues, descifrar en términos de «estrategias» los mecanismos puestos en práctica en las relaciones de poder. Pero el punto más importante es, evidentemente, la relación entre relaciones de poder y estrategias de enfrentamiento. Pues si es verdad que en el corazón de las relaciones de poder y como condición permanente de su existencia hay una «insumisión» y libertades esencialmente renuentes, no hay relación de poder sin resistencia, sin escapatoria o huida, sin inversión eventual; toda relación de poder implica, pues, al menos de manera virtual, una estrategia de lucha, sin que, sin embargo, vengan a superponerse, a perder su especificidad y finalmente a confundirse. Constituyen una para otra una especie de límite permanente, de punto de inversión posible. Una relación de enfrentamiento encuentra su término, su momento final (y la victoria de uno de los dos adversarios) cuando el juego de las relaciones antagonistas viene a ser sustituido por los mecanismos estables por los que uno puede conducir de manera bastante constante y con suficiente certidumbre la conducta de los otros; para una relación de enfrentamiento, desde el momento en que ya no es de lucha a muerte, la fijación de una relación de poder constituye un punto de mira —a la vez su consumación y su propia puesta en suspenso. Y a la inversa, para una relación de poder, la estrategia de lucha constituye también ella una frontera: aquella en que la inducción calculada de las conductas en los otros ya no puede ir más allá de la réplica a su propia acción. Como no podría haber ahí relaciones de poder sin puntos de insumisión que por definición le escapan, toda intensificación, toda extensión de las relaciones de poder para someterlos no pueden sino conducir a los límites del ejercicio del poder; este encuentra, entonces, su tope ya en un tipo de acción que reduce al otro a la impotencia total (una «victoria» sobre el adversario sustituye al ejercicio del poder), ya en un cambio de aquellos a los que se gobierna y su transformación en adversarios. En suma, toda estrategia de enfrentamiento sueña con convertirse en relación de poder; y toda relación de poder tiende, tanto si sigue su propia línea de desarrollo como si evita resistencias frontales, a convertirse en estrategia ganadora.

De hecho, entre relación de poder y estrategia de lucha hay una apelación recíproca, encadenamiento indefinido e inversión perpetua. A cada instante la relación de poder puede convertirse, y en ciertos puntos se convierte, en un enfrentamiento entre adversarios. A cada instante también las relaciones de antagonismo, en una sociedad dada, dan lugar a la puesta en práctica de mecanismos de poder. Inestabilidad, pues, que hace que los mismos procesos, los mismos acontecimientos, las mismas transformaciones puedan descifrarse tanto en el interior de una historia de luchas como en la de las relaciones y de los dispositivos de poder. No serán ni los mismos elementos significativos, ni los mismos encadenamientos ni los mismos tipos de inteligibilidad los que aparecerán, aunque se refieran al mismo tejido histórico y aunque cada uno de los dos análisis deba reenviar al otro. Y es justamente la interferencia de las dos lecturas la que hace aparecer esos fenómenos fundamentales de «dominación» que presenta la historia de una gran parte de las sociedades humanas. La dominación es una estructura global de poder de la que se pueden encontrar a veces las ramificaciones y las consecuencias hasta en la trama más tenue de la sociedad; pero es al mismo tiempo una situación estratégica más o menos adquirida y consolidada en un enfrentamiento de largo alcance histórico entre adversarios. Puede bien ocurrir que un hecho de dominación no sea más que la transcripción de uno de los mecanismos de poder de una relación de enfrentamiento y sus consecuencias (una estructura política derivada de una invasión); puede también que una relación de lucha entre dos adversarios sea el efecto del desarrollo de relaciones de poder con los límites y divisiones que conlleva. Pero lo que hace de la dominación de un grupo, de una casta o de una clase, y de las resistencias o revueltas que encara, un fenómeno central en la historia de las sociedades es que manifiestan bajo una forma global y masiva, a escala del cuerpo social entero, el engranaje de relaciones de poder con las relaciones estratégicas, y sus efectos de arrastre recíproco.
\end{document}
