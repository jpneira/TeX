\documentclass[
	% Tamaño papel
	letterpaper,
	% Tamaño fuentes 
	12pt,
	% Documento a una cara (twoside incia capitulos en cara derecha)
	oneside]
	% Estilo del documento
	{article}
%%---------------------------------------
%% Idioma y fuentes
%%---------------------------------------
% Manejo del idioma español
\usepackage[spanish]{babel}
% Manejo de fuentes
\usepackage{fontspec}
% define fuente garamond
% http://www.georgduffner.at/ebgaramond/
% los glyfos 'bold' son (c) Adobe Systems
\setmainfont[
	Path = ../master/fonts/,
	BoldFont = AGaramondPro-Bold,
	BoldItalicFont = AGaramondPro-BoldItalic,
	ItalicFont = EBGaramond12-Italic,
	SmallCapsFont = EBGaramond12-SC]{EBGaramond12-Regular}
\usepackage[
	kerning=false,
	tracking=false,
	protrusion=true,
	expansion=false]{microtype}

% Tamaño de pagina
%\usepackage[letterpaper]{geometry}
%	\geometry{
%		top=2cm,
%		left=3cm,
%		right=3cm,
%		bottom=2cm}

% Estilo de pagina
\pagestyle{plain}
% Identacion  	
\parindent 2em
% Espacio entre parrafos
\parskip 0.4em
% Guionado  de algunas palabras que LaTeX no ha dividido como deberia
%\hyphenation{
% e
%  edu-ca-cio-nal
%}


%% -----------------------------------------------------
%%             \\\ COMIENZA EL DOCUMENTO ///                        
%% -----------------------------------------------------
\begin{document}
\title{\Huge\sc De los espacios otros\footnote{{\it «Des espaces autres»}, Conferencia dicada en el Cercle des études architecturals, 14 de marzo de 1967, publicada en Architecture, Mouvement, Continuité, nº 5, octubre de 1984.}}
\author{Michel Foucault\footnote{Tradución de Pablo Blitstein y Tadeo Lima.}}
\date{}
\maketitle

La gran obsesión que tuvo el siglo {\sc xix} fue, como se sabe, la historia: temas del desarrollo y de la interrupción, temas de la crisis y del ciclo, temas de la acumulación del pasado, gran sobrecarga de los muertos, enfriamiento amenazante del mundo. En el segundo principio de la termodinámica el siglo {\sc xix} encontró lo esencial de sus recursos mitológicos. La época actual quizá sea sobre todo la época del espacio. Estamos en la época de lo simultáneo, estamos en la época de la yuxtaposición, en la época de lo próximo y lo lejano, de lo uno al lado de lo otro, de lo disperso. Estamos en un momento en que el mundo se experimenta, creo, menos como una gran vida que se desarrolla a través del tiempo que como una red que une puntos y se entreteje.

Tal vez se pueda decir que algunos de los conflictos ideológicos que animan las polémicas actuales se desarrollan entre los piadosos descendientes del tiempo y los habitantes encarnizados del espacio. El estructuralismo, o al menos lo que se agrupa bajo este nombre algo general, es el esfuerzo por establecer, entre elementos repartidos a través del tiempo, un conjunto de relaciones que los hace aparecer como yuxtapuestos, opuestos, implicados entre sí, en suma, que los hace aparecer como una especie de configuración; y a decir verdad, no se trata de negar el tiempo, sino de una manera de tratar lo que llamamos tiempo y lo que llamamos historia.

Se debe señalar sin embargo que el espacio que aparece hoy en el horizonte de nuestras preocupaciones, de nuestra teoría, de nuestros sistemas no es una innovación; el espacio mismo, en la experiencia occidental, tiene una historia, y no es posible desconocer este entrecruzamiento fatal del tiempo con el espacio. Se podría decir, para trazar muy groseramente esta historia del espacio, que en la Edad Media había un conjunto jerarquizado de lugares: lugares sagrados y lugares profanos, lugares protegidos y lugares por el contrario abiertos y sin prohibiciones, lugares urbanos y lugares rurales (esto en lo que concierne a la vida real de los hombres). Para la teoría cosmológica, había lugares supracelestes opuestos al lugar celeste; y el lugar celeste se oponía a su vez al lugar terrestre. Estaban los lugares donde las cosas se encontraban ubicadas porque habían sido desplazadas violentamente, y también los lugares donde, por el contrario, las cosas encontraban su ubicación o su reposo naturales. Era esta jerarquía, esta oposición, este entrecruzamiento de lugares lo que constituía aquello que se podría llamar muy groseramente el espacio medieval: un espacio de localización. Este espacio de localización se abrió con Galileo, ya que el verdadero escándalo de la obra de Galileo no es tanto el haber descubierto, o más bien haber redescubierto que la Tierra giraba alrededor del Sol, sino el haber constituido un espacio infinito, e infinitamente abierto; de tal forma que el espacio medieval, de slgún modo, se disolvía, el lugar de una cosa no era más que un punto en su movimiento, así como el reposo de una cosa no era más que su movimiento indefinidamente desacelerado. Dicho de otra manera, a partir de Galileo, a partir del siglo {\sc xvii}, la extensión sustituye a la localización.

En nuestros días, el emplazamiento sustituye a la extensión que por su cuenta ya había reemplazado a la localización. El emplazamiento se define por las relaciones de proximidad entre puntos o elementos; formalmente, se las puede describir como series, árboles, enrejados.

Por otra parte, es conocida la importancia de los problemas de emplazamiento en la técnica contemporánea: almacenamiento de la información o de los resultados parciales de un cálculo en la memoria de una máquina, circulación de elementos discretos, con salida aleatoria (como los automóviles, simplemente, o los sonidos a lo largo de una línea telefónica), identificación de elementos, marcados o codificados, en el interior de un conjunto que está distribuido al azar, o clasificado en una clasificación unívoca, o clasificado según una clasificación plurívoca, etc. De una manera todavía más concreta, el problema del sitio o del emplazamiento se plantea para los hombres en términos de demografía; y este último problema del emplazamiento humano no plantea simplemente si habrá lugar suficiente para el hombre en el mundo —problema que es después de todo bastante importante—, sino también el problema de qué relaciones de proximidad, qué tipo de almacenamiento, de circulación, de identificación, de clasificación de elementos humanos deben ser tenidos en cuenta en tal o cual situación para llegar a tal o cual fin. Estamos en una época en que el espacio se nos da bajo la forma de relaciones de emplazamientos. 

En todo caso, creo que la inquietud actual concierne fundamentalmente al espacio, sin duda mucho más que al tiempo; el tiempo no aparece probablemente sino como uno de los juegos de distribución posibles entre los elementos que se reparten en el espacio.

Ahora bien, a pesar de todas las técnicas que lo invisten, a pesar de toda la red de saber que permite determinarlo o formalizarlo, el espacio contemporáneo tal vez no está todavía enteramente desacralizado —a diferencia sin duda del tiempo, que ha sido desacralizado en el siglo {\sc xix}. Es verdad que ha habido una cierta desacralización teórica del espacio (aquella cuya señal es la obra de Galileo), pero tal vez no accedimos aún a una desacralización práctica del espacio. Y tal vez nuestra vida está controlada aún por un cierto número de oposiciones que no se pueden modificar, contra las cuales la institución y la práctica aún no se han atrevido a rozar: oposiciones que admitimos como dadas: por ejemplo, entre el espacio privado y el espacio público, entre el espacio de la familia y el espacio social, entre el espacio cultural y el espacio útil, entre el espacio del ocio y el espacio del trabajo, todas dominadas por una sorda sacralización.

La obra —inmensa— de Bachelard, las descripciones de los fenomenólogos nos han enseñado que no vivimos en un espacio homogéneo y vacío, sino, por el contrario, en un espacio que está cargado de cualidades, un espacio que tal vez esté también visitado por fantasmas; el espacio de nuestra primera percepción, el de nuestras ensoñaciones, el de nuestras pasiones guardan en sí mismos cualidades que son como intrínsecas; es un espacio liviano, etéreo, transparente, o bien un espacio oscuro, rocalloso, obstruido: es un espacio de arriba, es un espacio de las cimas, o es por el contrario un espacio de abajo, un espacio del barro, es un espacio que puede estar corriendo como el agua viva, es un espacio que puede estar fijo, detenido como la piedra o como el cristal.

Sin embargo, estos análisis, aunque fundamentales para la reflexión contemporánea, conciernen sobre todo al espacio del adentro. Es del espacio del afuera que quisiera hablar ahora.

El espacio en el que vivimos, que nos atrae hacia fuera de nosotros mismos, en el que se desarrolla precisamente la erosión de nuestra vida, de nuestro tiempo y de nuestra historia, este espacio que nos carcome y nos agrieta es en sí mismo también un espacio heterogéneo. Dicho de otra manera, no vivimos en una especie de vacío, en el interior del cual podrían situarse individuos y cosas. No vivimos en un vacío diversamente tornasolado, vivimos en un conjunto de relaciones que definen emplazamientos irreductibles los unos a los otros y que no deben superponerse.

Por supuesto, se podría emprender la descripción de estos diferentes emplazamientos, buscando el conjunto de relaciones por el cual se los puede definir. Por ejemplo, describir el conjunto de relaciones que definen los emplazamientos de pasaje, las calles, los trenes (un tren es un extraordinario haz de relaciones, ya que es algo a través de lo cual se pasa, es algo mediante lo cual se puede pasar de un punto a otro y además es también algo que pasa). Se podría describir, por el haz de relaciones que permiten definirlos, estos emplazamientos de detención provisoria que son los cafés, los cines, las playas. Se podría también definir, por su red de relaciones, el emplazamiento de descanso, cerrado o medio cerrado, constituido por la casa, la habitación, la cama, etc. Pero los que me interesan son, entre todos los emplazamientos, algunos que tienen la curiosa propiedad de estar en relación con todos los otros emplazamientos, pero de un modo tal que suspenden, neutralizan o invierten el conjunto de relaciones que se encuentran, por sí mismos, designados, reflejados o reflexionados. De alguna manera, estos espacios, que están enlazados con todos los otros, que contradicen sin embargo todos los otros emplazamientos, son de dos grandes tipos.

Están en primer lugar las utopías. Las utopías son los emplazamientos sin lugar real. Mantienen con el espacio real de la sociedad una relación general de analogía directa o inversa. Es la sociedad misma perfeccionada o es el reverso de la sociedad, pero, de todas formas, estas utopías son espacios fundamental y esencialmente irreales.

También existen, y esto probablemente en toda cultura, en toda civilización, lugares reales, lugares efectivos, lugares que están diseñados en la institución misma de la sociedad, que son especies de contra–emplazamientos, especies de utopías efectivamente realizadas en las cuales los emplazamientos reales, todos los otros emplazamientos reales que se pueden encontrar en el interior de la cultura están a la vez representados, cuestionados e invertidos, especies de lugares que están fuera de todos los lugares, aunque sean sin embargo efectivamente localizables. Estos lugares, porque son absolutamente otros que todos los emplazamientos que reflejan y de los que hablan, los llamaré, por oposición a las utopías, las heterotopías; y creo que entre las utopías y estos emplazamientos absolutamente otros, estas heterotopías, habría sin duda una suerte de experiencia mixta, medianera, que sería el espejo. El espejo es una utopía, porque es un lugar sin lugar. En el espejo, me veo donde no estoy, en un espacio irreal que se abre virtualmente detrás de la superficie, estoy allá, allá donde no estoy, especie de sombra que me devuelve mi propia visibilidad, que me permite mirarme allá donde estoy ausente: utopía del espejo. Pero es igualmente una heterotopía, en la medida en que el espejo existe realmente y tiene, sobre el lugar que ocupo, una especie de efecto de retorno; a partir del espejo me descubro ausente en el lugar en que estoy, puesto que me veo allá. A partir de esta mirada que de alguna manera recae sobre mí, del fondo de este espacio virtual que está del otro lado del vidrio, vuelvo sobre mí y empiezo a poner mis ojos sobre mí mismo y a reconstituirme allí donde estoy; el espejo funciona como una heterotopía en el sentido de que convierte este lugar que ocupo, en el momento en que me miro en el vidrio, en absolutamente real, enlazado con todo el espacio que lo rodea, y a la vez en absolutamente irreal, ya que está obligado, para ser percibido, a pasar por este punto virtual que está allá.

En cuanto a las heterotopías propiamente dichas, ¿cómo se las podría describir, que sentido tienen? Se podría suponer, no digo una ciencia, porque es una palabra demasiado prostituida ahora, sino una especie de descripción sistemática que tuviera por objeto, en una sociedad dada, el estudio, el análisis, la descripción, la «lectura», como se gusta decir ahora, de estos espacios diferentes, estos otros lugares, algo así como una polémica a la vez mítica y real del espacio en que vivimos; esta descripción podría llamarse la heterotopología. Primer principio: no hay probablemente una sola cultura en el mundo que no constituya heterotopías. Es una constante de todo grupo humano. Pero las heterotopías adquieren evidentemente formas que son muy variadas, y tal vez no se encuentre una sola forma de heterotopía que sea absolutamente universal. Sin embargo es posible clasificarlas en dos grandes tipos.

En las sociedades llamadas «primitivas», hay una forma de heterotopías que yo llamaría heterotopías de crisis, es decir que hay lugares privilegiados, o sagrados, o prohibidos, reservados a los individuos que se encuentran, en relación a la sociedad y al medio humano en el interior del cual viven, en estado de crisis. Los adolescentes, las mujeres en el momento de la menstruación, las parturientas, los viejos, etc.

En nuestra sociedad, estas heterotopías de crisis están desapareciendo, aunque se encuentran todavía algunos restos. Por ejemplo, el colegio, bajo su forma del siglo {\sc xix}, o el servicio militar para los jóvenes jugaron ciertamente tal rol, ya que las primeras manifestaciones de la sexualidad viril debían tener lugar en «otra parte», diferente de la familia. Para las muchachas existía, hasta mediados del siglo {\sc xx}, una tradición que se llamaba el «viaje de bodas»; un tema ancestral. El desfloramiento de la muchacha no podía tener lugar «en ninguna parte» y, en ese momento, el tren, el hotel del viaje de bodas eran ese lugar de ninguna parte, esa heterotopía sin marcas geográficas.

Pero las heterotopías de crisis desaparecen hoy y son reemplazadas, creo, por heterotopías que se podrían llamar de desviación: aquellas en las que se ubican los individuos cuyo comportamiento está desviado con respecto a la media o a la norma exigida. Son las casas de reposo, las clínicas psiquiátricas; son, por supuesto, las prisiones, y debería agregarse los geriátricos, que están de alguna manera en el límite de la heterotopía de crisis y de la heterotopía de desviación, ya que, después de todo, la vejez es una crisis, pero igualmente una desviación, porque en nuestra sociedad, donde el tiempo libre se opone al tiempo de trabajo, el no hacer nada es una especie de desviación. 

El segundo principio de esta descripción de las heterotopías es que, en el curso de su historia, una sociedad puede hacer funcionar de una forma muy diferente una heterotopía que existe y que no ha dejado de existir; en efecto, cada heterotopía tiene un funcionamiento preciso y determinado en la sociedad, y la misma heterotopía puede, según la sincronía de la cultura en la que se encuentra, tener un funcionamiento u otro.

Tomaré por ejemplo la curiosa heterotopía del cementerio. El cementerio es ciertamente un lugar otro en relación a los espacios culturales ordinarios; sin embargo, es un espacio ligado al conjunto de todos los emplazamientos de la ciudad o de la sociedad o de la aldea, ya que cada individuo, cada familia tiene parientes en el cementerio. En la cultura occidental, el cementerio existió prácticamente siempre. Pero sufrió mutaciones importantes. Hasta el fin del siglo {\sc xviii}, el cementerio se encontraba en el corazón mismo de la ciudad, a un lado de la iglesia. Existía allí toda una jerarquía de sepulturas posibles. Estaba la fosa común, en la que los cadáveres perdían hasta el último vestigio de individualidad, había algunas tumbas individuales, y también había tumbas en el interior de la iglesia. Estas tumbas eran de dos especies: podían ser simplemente baldosas con una marca, o mausoleos con estatuas. Este cementerio, que se ubicaba en el espacio sagrado de la iglesia, ha adquirido en las sociedades modernas otro aspecto diferente y, curiosamente, en la época en que la civilización se ha vuelto —como se dice muy groseramente— «atea», la cultura occidental inauguró lo que se llama el culto de los muertos.

En el fondo, era muy natural que en la época en que se creía efectivamente en la resurrección de los cuerpos y en la inmortalidad del alma no se haya prestado al despojo mortal una importancia capital. Por el contrario, a partir del momento en que no se está muy seguro de tener un alma, ni de que el cuerpo resucitará, tal vez sea necesario prestar mucha más atención a este despojo mortal, que es finalmente el último vestigio de nuestra existencia en el mundo y en las palabras. En todo caso, a partir del siglo {\sc xix} cada uno tiene derecho a su pequeña caja para su pequeña descomposición personal; pero, por otra parte, recién a partir del siglo {\sc xix} se empezó a poner los cementerios en el límite exterior de las ciudades; correlativamente a esta individualización de la muerte y a la apropiación burguesa del cementerio nació la obsesión de la muerte como «enfermedad». Se supone que los muertos llevan las enfermedades a los vivos, y que la presencia y la proximidad de los muertos al lado de la casa, al lado de la iglesia, casi en el medio de la calle, propaga por sí misma la muerte. Este gran tema de la enfermedad esparcida por el contagio de los cementerios persistió en el fin del siglo {\sc xviii}; y en el transcurso del siglo {\sc xix} comenzó su desplazamiento hacia los suburbios. Los cementerios constituyen entonces no sólo el viento sagrado e inmortal de la ciudad, sino «la otra ciudad», donde cada familia posee su negra morada. 

Tercer principio: la heterotopía tiene el poder de yuxtaponer en un solo lugar real múltiples espacios, múltiples emplazamientos que son en sí mismos incompatibles. Es así que el teatro hace suceder sobre el rectángulo del escenario toda una serie de lugares que son extraños los unos a los otros; es así que el cine es una sala rectangular muy curiosa, al fondo de la cual, sobre una pantalla bidimensional, se ve proyectar un espacio en tres dimensiones; pero tal vez el ejemplo más antiguo de estas heterotopías (en forma de emplazamientos contradictorios) sea el jardín. No hay que olvidar que el jardín, creación asombrosa ya milenaria, tenía en oriente significaciones muy profundas y como superpuestas. El jardín tradicional de los persas era un espacio sagrado que debía reunir, en el interior de su rectángulo, cuatro partes que representaban las cuatro partes del mundo, con un espacio todavía más sagrado que los otros que era como su ombligo, el ombligo del mundo en su medio (allí estaban la fuente y la vertiente); y toda la vegetación del jardín debía repartirse dentro de este espacio, en esta especie de microcosmos. 

En cuanto a las alfombras, ellas eran, en el origen, reproducciones de jardines. El jardín es una alfombra donde el mundo entero realiza su perfección simbólica, y la alfombra, una especie de jardín móvil a través del espacio. El jardín es la parcela más pequeña del mundo y es por otro lado la totalidad del mundo. El jardín es, desde el fondo de la Antigüedad, una especie de heterotopía feliz y universalizante (de ahí nuestros jardines zoológicos).

Cuarto principio: las heterotopías están, las más de las veces, asociadas a cortes del tiempo; es decir que operan sobre lo que podríamos llamar, por pura simetría, heterocronías. La heterotopía empieza a funcionar plenamente cuando los hombres se encuentran en una especie de ruptura absoluta con su tiempo tradicional; se ve acá que el cementerio constituye un lugar altamente heterotópico, puesto que comienza con esa extraña heterocronía que es, para un individuo, la pérdida de la vida, y esa cuasi eternidad donde no deja de disolverse y de borrarse.

En forma general, en una sociedad como la nuestra, heterotopía y heterocronía se organizan y se ordenan de una manera relativamente compleja. Están en primer lugar las heterotopías del tiempo que se acumulan al infinito, por ejemplo los museos, las bibliotecas —museos y bibliotecas son heterotopías en las que el tiempo no cesa de amontonarse y de encaramarse sobre sí mismo, mientras que en el siglo {\sc xvii}, hasta fines del {\sc xvii} incluso, los museos y las bibliotecas eran la expresión de una elección. En cambio, la idea de acumular todo, la idea de constituir una especie de archivo general, la voluntad de encerrar en un lugar todos los tiempos, todas las épocas, todas las formas, todos los gustos, la idea de constituir un lugar de todos los tiempos que esté fuera del tiempo, e inaccesible a su mordida, el proyecto de organizar así una suerte de acumulación perpetua e indefinida del tiempo en un lugar inamovible\ldots todo esto pertenece a nuestra modernidad. El museo y la biblioteca son heterotopías propias de la cultura occidental del siglo {\sc xix}.

Frente a estas heterotopías, ligadas a la acumulación del tiempo, se hallan las heterotopías que están ligadas, por el contrario, al tiempo en lo que tiene de más fútil, de más precario, de más pasajero, según el modo de la fiesta. Son heterotopías no ya eternizantes, sino absolutamente crónicas. Tales son las ferias, esos maravillosos emplazamientos vacíos en el límite de las ciudades, que una o dos veces al año se pueblan de puestos, de barracones, de objetos heteróclitos, de luchadores, de mujeres–serpiente, de adivinas. Muy recientemente también, se ha inventado una nueva heterotopía crónica: las ciudades de veraneo; esas aldeas polinesias que ofrecen tres cortas semanas de desnudez primitiva y eterna a los habitantes de las ciudades; y ustedes ven por otra parte que acá se juntan las dos formas de heterotopías, la de la fiesta y la de la eternidad del tiempo que se acumula: las chozas de Djerba son en un sentido parientes de las bibliotecas y los museos, pues en el reencuentro de la vida polinesia, el tiempo queda abolido, pero es también el tiempo recobrado, toda la historia de la humanidad remontándose desde su origen como en una especie de gran saber inmediato.

Quinto principio: las heterotopías suponen siempre un sistema de apertura y uno de cierre que, a la vez, las aíslan y las vuelven penetrables. En general, no se accede a un emplazamiento heterotópico como accedemos a un molino. O bien uno se halla allí confinado —es el caso de las barracas, el caso de la prisión— o bien hay que someterse a ritos y a purificaciones. Sólo se puede entrar con un permiso y una vez que se ha completado una serie de gestos. Existe, por otro lado, heterotopías enteramente consagradas a estas actividades de purificación, medio religiosa, medio higiénica, como los hammam musulmanes, o bien purificación en apariencia puramente higiénica, como los saunas escandinavos.

Existen otras, al contrario, que tienen el aire de puras y simples aberturas, pero que, en general, ocultan curiosas exclusiones. Todo el mundo puede entrar en los emplazamientos heterotópicos, pero a decir verdad, esto es sólo una ilusión: uno cree penetrar pero, por el mismo hecho de entrar, es excluido. Pienso, por ejemplo, en esas famosas habitaciones que existían en las grandes fincas del Brasil, y en general en Sudamérica. La puerta para acceder a ellas no daba a la pieza central donde vivía la familia, y todo individuo que pasara, todo viajero tenía el derecho de franquear esta puerta, entrar en la habitación y dormir allí una noche. Ahora bien, estas habitaciones eran tales que el individuo que pasaba allí no accedía jamás al corazón mismo de la familia, era absolutamente huésped de pasada, no verdaderamente un invitado. Este tipo de heterotopía, que hoy prácticamente ha desaparecido en nuestras civilizaciones, podríamos tal vez reencontrarlo en las famosas habitaciones de los moteles americanos, donde uno entra con su coche y con su amante y donde la sexualidad ilegal se encuentra a la vez absolutamente resguardada y absolutamente oculta, separada, y sin embargo dejada al aire libre. 

Finalmente, la última nota de las heterotopías es que son, respecto del espacio restante, una función. Ésta se despliega entre dos polos extremos. O bien tienen por rol crear un espacio de ilusión que denuncia como más ilusorio todavía todo el espacio real, todos los emplazamientos en el interior de los cuales la vida humana está compartimentada (tal vez sea éste el rol que durante mucho tiempo jugaran las casas de tolerancia, rol del que se hallan ahora privadas); o bien, por el contrario, crean otro espacio, otro espacio real, tan perfecto, tan meticuloso, tan bien ordenado, como el nuestro es desordenado, mal administrado y embrollado. Ésta sería una heterotopía no ya de ilusión, sino de compensación, y me pregunto si no es de esta manera que han funcionado ciertas colonias. En ciertos casos, las colonias han jugado, en el nivel de la organización general del espacio terrestre, el rol de heterotopía. Pienso por ejemplo, en el momento de la primer ola de colonización, en el siglo {\sc xvii}, en esas sociedades puritanas que los ingleses fundaron en América y que eran lugares otros absolutamente perfectos.

Pienso también en esas extraordinarias colonias jesuíticas que fueron fundadas en Sudamérica: colonias maravillosas, absolutamente reglamentadas, en las que se alcanzaba efectivamente la perfección humana. Los jesuitas del Paraguay habían establecido colonias donde la existencia estaba reglamentada en cada uno de sus puntos. La aldea se repartía según una disposición rigurosa alrededor de una plaza rectangular al fondo de la cual estaba la iglesia; a un costado, el colegio, del otro, el cementerio, y, después, frente a la iglesia se abría una avenida que otra cruzaría en ángulo recto. Las familias tenían cada una su pequeña choza a lo largo de estos ejes y así se reproducía exactamente el signo de Cristo. La cristiandad marcaba así con su signo fundamental el espacio y la geografía del mundo americano.

La vida cotidiana de los individuos era regulada no con un silbato, pero sí por las campanas. Todo el mundo debía despertarse a la misma hora, el trabajo comenzaba para todos a la misma hora; la comida a las doce y a las cinco; después uno se acostaba y a la medianoche sonaba lo que podemos llamar la diana conyugal. Es decir que al sonar la campana cada uno cumplía con su deber. Casas de tolerancia y colonias son dos tipos extremos de heterotopía, y si uno piensa que, después de todo, el barco es un pedazo flotante de espacio, un lugar sin lugar, que vive por él mismo, que está cerrado sobre sí y que al mismo tiempo está librado al infinito del mar y que, de puerto en puerto, de orilla en orilla, de casa de tolerancia en casa de tolerancia, va hasta las colonias a buscar lo más precioso que ellas encierran en sus jardines, ustedes comprenden por qué el barco ha sido para nuestra civilización, desde el siglo {\sc xvi} hasta nuestros días, a la vez no solamente el instrumento más grande de desarrollo económico (no es de eso de lo que hablo hoy), sino la más grande reserva de imaginación. El navío es la heterotopía por excelencia. En las civilizaciones sin barcos, los sueños se agotan, el espionaje reemplaza allí la aventura y la policía a los corsarios.
\end{document}