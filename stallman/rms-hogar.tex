\documentclass[oneside,twocolumn]{article}
\usepackage{lmodern}
\usepackage[T1]{fontenc}
\usepackage{textcomp}
\usepackage[spanish]{babel}
\usepackage[utf8]{inputenc}
\usepackage{fancyhdr}
\fancyhf{}
\pagestyle{fancy}
\fancyhead[LE]{\scriptsize{\bf EMEEQUIS} | Richard Stallman, un hombre sin hogar}
\fancyhead[RO]{\scriptsize{\bf EMEEQUIS} | Richard Stallman, un hombre sin hogar}	 
\fancyfoot[C]{\thepage}
\usepackage{url}
\usepackage{fancybox}
\hyphenation{
es-pe-ci-a-li-za-do
des-pe-gue
des-pu-és
ho-gar
la-bo-ra-to-ri-o
li-ber-tad
ma-yor
o-pe-ra-ti-vo
re-cha-za
res-pon-de
re-sis-tir
so-li-da-ri-dad
u-sa-do
u-sa-ran
us-ted
xo-chi-mil-co
} 
\begin{document}
\title{\bf Richard Stallman, un hombre sin hogar}
\author{Por Diego Mendiburu \\
\texttt{dmendiburu@m-x.com.mx} \\
Fotografías: Eduardo Loza \\
\url{http://m-x.com.mx/}}
\date{13 de Junio de 2011}
\twocolumn[
\begin{@twocolumnfalse}
\maketitle
\begin{abstract}
Para unos es un loco; para otros, un extremista.

Richard Stallman es, sin duda, un personaje trascendental de la era de la información, cuya influencia
podría ser aún mayor después de su propio tiempo de vida. 

Es, también, un hombre que ha perdido a su familia y que cuando pensó que por fin había encontrado 
un lugar especial al cual llamar hogar, lo perdió, a pesar de lo mucho que hizo para evitar su destrucción.

Hace unas semanas Stallman estuvo en el DF y emeequis pudo charlar de manera peculiar con él: dada su 
costumbre de nunca hospedarse en hoteles, reportero y fotógrafo	fungieron como sus choferes y acompañantes.
\end{abstract}
\end{@twocolumnfalse}]

Richard necesita ponerse unos calcetines, y rápido.

Con su característica barba larga y retorcida, una prominente barriga, una cabellera que inicia negra y
termina gris y se extiende hasta ocultarle los hombros, me espera en el umbral de un departamento justo al
inicio de la calle de Bolívar, esquina con Brasil, en el Centro Histórico de la Ciudad de México.

Quienes no lo conozcan dirían que estoy frente a un \emph{hippie} gringo que vino de mochilazo a conocer la 
capital. Si se le juzga por su apariencia y sus hábitos, tendrían razón.

Pero no es un visitante cualquiera, y no se está poniendo unos calcetines para ir a conocer Xochimilco 
o las pirámides de Teotihuacán. Este sujeto de ojos claros y rostro extrañamente aniñado se llama Richard 
Stallman y es una de las personas que mayor influencia ha tenido en el desarrollo de la industria del 
\emph{software}, tanto como Bill Gates o Steve Jobs. No ha sido nombrado la persona del año por la revista
\emph{TIME} ni \emph{Forbes} le ha dedicado media docena de páginas gracias a su éxito económico porque 
a él lo que menos le interesa es el poder o el dinero.

Stallman quiere ser libre y que todos lo seamos, al menos frente a una computadora.

En 1983 comenzó a desarrollar, con apoyo de quienes se le han sumado a su causa, un sistema operativo informático
totalmente libre. A la postre, la iniciativa de Stallman se ha convertido en un movimiento de miles, quizá millones 
de personas, cuyos principios éticos y filosóficos han influenciado no sólo al mundo de la informática, sino a 
la cultura, las artes y la sociedad.

Ahora no hay tiempo para explicar la influencia de este hombre que aún sigue buscando unos calcetines. 
Necesitamos irnos ya. Lo está esperando el senador Francisco Javier Castellón Fonseca desde hace 15 minutos.

Extraña oportunidad para realizar una entrevista: debemos recoger a Stallman en el estudio donde se estaba
quedando, llevarlo a dar una conferencia al Senado y luego tratar de conversar con él camino al aeropuerto.

Al fin se pone unos zapatos negros, cierra su prominente maleta, guarda en una mochila su pequeña computadora
Yeeloong, de fabricación china y única en el mundo por usar sólo \emph{software} libre.

Salimos a la calle para abordar el auto que nos llevará al auditorio Sebastián Lerdo de Tejada del Senado. 
Pero hay un problema.

---¡Pongan la maleta atrás! ---ordena Stallman al tiempo que señala la cajuela.

---No se puede, la traigo llena ---responde Eduardo Loza, fotógrafo y ruletero por una tarde.

---¿Y entonces, dónde?

---Aquí, en los asientos traseros.

---¡No! ¡No se puede! ¡No podemos dejar mi maleta en un auto sin que nadie la vigile! ---Stallman comienza a elevar
el volumen, desesperarse---. ¡En qué están pensando ustedes!

A ratos gruñón, el padre del \emph{software} libre se concentra en el teclado de su computadora una vez en el 
auto. No gusta de la conversación ligera. Después nos enteraremos por qué.

\[\diamond~\diamondsuit~\diamond\]

Imagínese que compra un auto nuevo. Pero usted tiene prohibido abrir el cofre y conocer su motor, 
identificar el tanque de gasolina, la batería, el carburador. Imagine también que no puede cambiarle la radio, 
ponerle rines deportivos o cambiarle el escape. Por último, imagine que ese auto en realidad no es suyo, sino 
que usted sólo tiene una ``licencia'' para usarlo, además no puede prestárselo a nadie.

¿Absurdo? Pues eso es muy parecido a lo que sucede actualmente con la industria del \emph{software}. 
``¿Por qué permitimos que funcione así?'', se pregunta Stallman.

El \emph{software} libre es la alternativa a ese modelo que las más grandes empresas de la 
programación han impuesto.

``El \emph{software} libre es aquel que respeta tu libertad y la solidaridad social de tu comunidad'', 
define Stallman frente a un auditorio lleno. La otra posibilidad, añade, es el \emph{software} privativo, 
llamado así porque ``priva a sus usuarios de la libertad''.

El \emph{software} libre debe reunir cuatro características para poder ser identificado como tal:

\begin{enumerate}
\item La libertad de ejecutar el programa para cualquier propósito.
\item La libertad de estudiar cómo funciona el programa y la posibilidad de adaptarlo de acuerdo a las 
necesidades del usuario (el acceso al código fuente del programa es una precondición para esto).
\item La libertad de distribuir copias como acto de solidaridad con el vecino.
\item La libertad de mejorar el programa y hacer esas mejoras públicas, para que toda la comunidad se
beneficie.
\end{enumerate}

``Es un \emph{software} ético, distribuido de manera ética. Un programa privativo es un yugo, 
una trampa. Si tiene funcionalidades atractivas son el cebo de la trampa para que la gente abandone 
su libertad. El \emph{software} privativo no debería existir. Su existencia, su uso es un problema 
social. Y deberíamos eliminarlo ---conmina Stallman frente a las cámaras del Canal del Congreso---, pues suele tener 
funcionalidades malignas''.

Le dan la razón algunos sucesos recientes:

En 2009, la empresa Amazon borró por accidente y de forma remota la novela \emph{1984}, de George Orwell, 
de las tabletas lectoras de libros Kindle de miles de sus clientes.

Recientemente se supo que los teléfonos iPhone registraban, sin autorización, todas las coordenadas 
del GPS del aparato, lo que significa que potencialmente Apple, o quien vulnerara la seguridad del 
dispositivo, podría haber monitoreado la ubicación precisa de cada usuario.

Eso no pasaría con el \emph{software} libre, ya que al ser público su código cualquiera puede
detectar funciones inconvenientes y eliminarlas de inmediato.

(``No tengo teléfono portátil porque reconocí, cuando aparecieron, que son el sueño de Stalin, 
son dispositivos de vigilancia y seguimiento de la gente ---me dirá en algún momento Stallman---. Si 
hace 20 años alguien te hubiera preguntado `¿quieres llevar un dispositivo que diga cada minuto donde 
estás al gran hermano?', hubieras dicho que esa era una pregunta absurda, que evidentemente no'').

El senador Castellón agradece al físico y programador en cuanto éste finaliza su intervención. 
Entonces vemos otra de las escenas que han conformado la polémica reputación de Stallman: su 
obsesión por jamás utilizar software privativo, incluso cuando se le hace 
una entrevista para radio o televisión.

---Quiero agradecerle a Richard, que tiene las puertas abiertas del Senado de la República; su
conferencia y el foro están siendo transmitidos en vivo por el Canal del Congreso y\ldots

---¿Pero es un canal de emisión de ondas electromagnéticas o por internet? ---interrumpe ansioso 
Stallman.

---Ambos ---le responde el senador.

---Pero, en internet ¿qué formato usan?

La gente echa a reír y Richard aumenta el volumen de su voz. Aquí no hay ninguna broma. Stallman 
rechaza entrevistas que serán publicadas en formatos de video o audio que no son de \emph{software}
libre.

---Los formatos que suelen usarse para videos son un obstáculo para el \emph{software} libre. 
Este es un asunto muy importante, el Senado tiene que cambiar por un formato que no imponga el uso 
de software privativo, como WebM u Ogg Vorbis\ldots

---Gracias por la recomendación, Richard ---le responde el legislador---. Vamos a hacer un receso.

\[\diamond~\diamondsuit~\diamond\]

La vida de Stallman está marcada por la pérdida. El nacimiento del \emph{software} libre, si 
bien tiene profundas raíces filosóficas y éticas, obedece también a la necesidad de este hombre de 
encontrar un hogar.

Stallman nació el 16 de marzo de 1953 en Manhattan, Nueva York, hijo de una pareja de judíos. 
Su madre, profesora substituta de arte; su padre, un veterano de la Segunda Guerra Mundial que 
siempre estaba enojado, según recuerda Richard.

``Nunca gritaba ---ha comentado en entrevistas previas--- pero tenía una manera fría y ofensiva de 
criticarte''.

Sus padres se divorciaron cuando Stallman apenas rebasaba los cinco años. Pasó el resto de su 
infancia viviendo de lunes a viernes con su madre y los fines de semana con su padre. Ahí perdió 
su primer hogar.

En mi tristeza, solía pensar «`quiero ir a casa’\ldots Me refería a un lugar inexistente que nunca 
encontraré», dijo Stallman en una entrevista publicada en 2002.

También pasó mucho tiempo con sus abuelos paternos. Cuando cumplió 10 años, ambos murieron 
consecutivamente. Esta pérdida fue devastadora para Stallman, quien ha dicho que solía encontrar 
con ellos un ambiente amoroso y gentil. Había vuelto a perder un hogar.

No encontraría el siguiente sino hasta que entró a la universidad, porque sus años en la educación 
media fueron terribles. Nunca logró socializar con sus compañeros de escuela, era inhábil para los 
deportes y se rehusaba a escribir tareas que no fueran de matemáticas y ciencias exactas.

---De adolescente no comprendía las relaciones con los otros. No sabía participar. No soy buen 
conversador, no me interesa hablar del clima o cosas sin importancia, y no comprendo por qué otros 
lo hacen. Para mí es pura molestia si alguien me invita a hacerlo ---platica mientras viajamos rumbo 
al aeropuerto, luego de que el mal humor se esfumó junto con el estrés de su conferencia.

---Pero esas pláticas ayudan a establecer un vínculo con las personas ---le digo.

---Para mí eso no es un vínculo, es sólo una molestia ---responde en perfecto español, idioma 
que aprendió empíricamente, como el francés.

Su madre, ya fallecida, admitió en 2002 que su hijo tenía ``algunas de las características de un 
niño autista\ldots lamento no haber sabido más sobre el autismo en ese entonces''.

Stallman se describió como ``casi autista'' en una entrevista, y cree que si hubiera nacido 40 
años después hubiera sido diagnosticado con síndrome de Asperger, que se diferencia del autismo en 
tanto que el sujeto no observa retraso en el desarrollo del lenguaje.

Considerado un niño prodigio en las ciencias exactas y siendo un insaciable devorador de libros, 
Stallman ingresó a un curso de ciencias y matemáticas de fin de semana en la Universidad de Columbia, 
especialmente diseñado para estudiantes superdotados. Inclusive ahí fue considerado por sus compañeros 
como ``demasiado intenso'' y que ``asustaba''.

A los 12 años comenzó su fascinación por las computadoras al recibir de un maestro el manual 
de una IBM 7094, pudo tocar una hasta los 16. Entonces cayó sobre él la amenaza de la guerra de Vietnam.

``Temía ser enviado a Vietnam, el miedo me aplastó durante años ---relata rumbo al aeropuerto---. 
Pero tuve suerte. No me enrolaron porque al último momento conseguí una exención de estudiante''.
	
A pesar de oponerse a la guerra en Vietnam, Stallman no participó en las protestas de la época. 
Sus fuertes convicciones políticas se manifestarían mucho después. Ingresó a la prestigiosa Universidad 
de Harvard, donde estudió una licenciatura en Física.

En Harvard su curiosidad por la informática aumentó, al punto que sus visitas a los laboratorios 
de cómputo se hacían cada vez más frecuentes. Pronto comenzó a programar, y en cuanto terminó su 
licenciatura decidió acudir al laboratorio de inteligencia artificial del Instituto Tecnológico de 
Massachussets, el famoso MIT. De inmediato fue contratado.

Haí encontró un laboratorio completamente distinto a los que había conocido antes: había una 
verdadera ``cultura \emph{hacker}'', como él mismo la define, un ambiente en donde mentes brillantes
encontraban soluciones ingeniosas a problemas complejos, donde cualquiera podía sentarse frente a una 
computadora y empezar a programar, y donde todos compartían un objetivo: escribir el mejor \emph{software}.
Nada de hacer dinero, obtener fama o demostrar ser mejor que los demás. Un ambiente de camaradería, 
fraternidad\ldots una nueva familia.

\[\diamond~\diamondsuit~\diamond\]

Richard Matthew Stallman creía haber hallado, al fin, su verdadero hogar. Estaba equivocado. 
Impotente, lo vio desintegrarse ante sus ojos. 

El \emph{software} libre, contrario a lo novedoso que puede ser el término para algunos, precede 
al \emph{software} privativo. Antes, cuando Stallman comenzó a programar en los laboratorios universitarios, 
el único \emph{software} que existía era aquel que se podía estudiar, copiar, modificar y mejorar 
con total libertad, ya sea dentro de las aulas o afuera mediante Arpanet, la red de computadoras que
posteriormente se convertiría en la red de redes, la internet.

``Participaba en una comunidad de \emph{software} libre ya madura, que ya usaba un sistema operativo 
libre, y me emplearon para participar en su desarrollo. No era tan grande, quizá unos cientos de personas. 
Teníamos costumbres muy arraigadas, como la consideración ética de los asuntos, tradiciones de cómo 
colaborar, cooperar con otros''.

La primera vez que Stallman se topó con su enemigo, el \emph{software} privativo, fue cuando 
un programador de la empresa Xerox se negó a darle el código fuente de una impresora. Richard quería 
modificarlo para que le avisara a los usuarios cuando el papel se había atascado, una modificación 
sencilla que mejoraría la vida de todos dentro del laboratorio, ya que la impresora estaba muy lejos.

La segunda vez fue mucho peor. A principios de los ochenta los programadores del laboratorio se 
dividieron en dos grupos, presionados por quienes querían comercializar lo que producían. Unos fundaron 
la compañía Symbolics, cuya intención era reemplazar el \emph{software} libre del laboratorio con 
su propio \emph{software} privativo.

``Cuando fueron las protestas de Vietnam no me imaginaba actuando políticamente. Gané la fuerza 
para actuar durante los años en el MIT, porque tenía que actuar de pequeñas maneras dentro del 
laboratorio para mantener la libertad tradicional. Eso me preparó para una lucha más fuerte, pues 
comenzó a morir la comunidad por la división entre sus miembros. ¿Qué habría podido hacer yo para 
que los ex empleados volvieran? ¿Cómo convencerlos?''.

Stallman, quien solía dormir en el laboratorio, se quedó prácticamente solo. ``Eso me puso muy 
triste, pero encontré una manera de contraatacar, de resistir. Nos dieron un ultimátum: Symbolics, 
empresa a la que no prefería, exigió a todos en el laboratorio elegir un lado u otro, así que la 
única reacción era elegir el lado opuesto y batirse''.

Symbolics quería que todas las computadoras usaran su sistema operativo y que se abandonara el 
anterior, que era \emph{software} libre. Para evitarlo, Stallman integraba todas las características 
del sistema de Symbolics al sistema operativo anterior, con tal de evitar su obsolescencia.

Trabajó arduamente dos años. Al final el MIT decidió comprar unas máquinas que sólo funcionaban 
con el \emph{software} de Symbolics. Stallman comprendió que su tiempo en el laboratorio había terminado. 
Había vuelto a perder su hogar.

Pronto anunció en grupos de noticias de Arpanet el desarrollo de GNU, un sistema operativo totalmente 
libre que podría ser utilizado en cualquier máquina del planeta.

El nombre GNU es ``un acrónimo recursivo\ldots pero no tengo tiempo de explicárselos'', abrevió en su 
conferencia del Senado.

En realidad es una broma, porque significa ``GNU No es Unix'', en referencia al muy añejo Unix que, 
sin embargo, ha sido cimiento de muchos sistemas operativos modernos, como Mac OS X.

Stallman fundó la organización no lucrativa Free Software Foundation para coordinar la creación de 
su sistema operativo. ``Quería crear otra comunidad de \emph{software} libre, una que reemplazara la 
comunidad perdida. No deseaba pasar toda mi vida sufriendo un acto injusto, quería construir algo nuevo. 
Por eso comencé el desarrollo de GNU''.

La Free Software Foundation redactó en 1989 la Licencia Pública General de GNU ---más conocida por su 
nombre en inglés GNU General Public License o GPL---, cuyo propósito es declarar que el \emph{software}
cubierto por esta licencia es libre y, por tanto, está protegido contra intentos de apropiación que 
restrinjan la libertad de los usuarios.

Cabe aclarar que el software libre no necesariamente es gratuito. Los programadores son libres de 
cobrar o no por su programación.

Stallman desarrolló muchos programas del sistema operativo, pero aún no terminaba de escribir el 
núcleo, es decir, el \emph{software} que se encarga de que el resto de los programas tengan acceso 
al \emph{hardware} de la computadora. En 1989 un ingeniero finlandés liberó el núcleo Linux bajo la 
licencia GPL, colocando la última pieza del rompecabezas. Así nació el sistema operativo GNU/Linux, que 
hoy es utilizado aproximadamente por 1.5 por ciento de las computadoras del mundo.

Es un número pequeño, pero es, por mucho, el sistema más utilizado en servidores de internet por su 
confiabilidad ---63 por ciento, según el sitio especializado W3Techs--- y tiene 33 por ciento del mercado de 
teléfonos móviles inteligentes a través de Android, de Google, basado en GNU/Linux.

Stallman todavía extraña aquellos años en el laboratorio. A mediados de los noventa dejó de programar 
y hoy se la pasa viajando por todo el planeta predicando las bondades del \emph{software} libre, que 
no se desarrolla dentro de su fundación sino que más bien es impulsado por miles, quizá millones de 
personas en todos los países.

``No tengo un hogar. Después de la muerte de la comunidad del laboratorio del MIT nunca he tenido un 
hogar. Estoy algo triste, pero ¿qué puedo hacer?''.

\[\diamond~\diamondsuit~\diamond\]

Conversamos en el auto. Richard se toma una de las hebras retorcidas de su cabellos canos y la recorre 
con sus dedos índice y pulgar hasta la punta. Se mete la punta a la boca y la cercena delicadamente con 
sus dientes frontales. Luego la saca de su boca con los dedos. Lo hace una y otra vez. Manía no tan extraña 
para un hombre no tan ordinario.

Llegamos al aeropuerto. Luego de hacer el papeleo, Richard dice que tiene hambre. Quiere una sopa. 
Al no dar con alguna que le apetezca, opta por comprarse un par de bizcochos de una famosa cadena 
estadunidense. 

Stallman nunca se hospeda en hoteles, siempre duerme en las casas de simpatizantes de la causa del 
\emph{software} libre que le abren las puertas de sus hogares e inclusive le pagan o le cocinan sus
alimentos.

``Después de 10 años en el MIT, que no pagaba muy bien, seguí la vida de estudiante. Cuando lancé el 
movimiento del \emph{software} libre decidí vivir barato para no ser esclavo del dinero; para poder
hacer lo que me pareciera justo necesitaba, primero, no depender del dinero. Sé que hubiera podido 
dedicarme a programar en el mundo del \emph{software} privativo y ganar mucho dinero, pero esa 
hubiera sido una vida fea, de vergüenza. Habría pasado el resto de mi vida lamentando lo que hacía''.

Sentados en el área de comida de la terminal aérea, Richard deshace el \emph{brownie} de chocolate y
el \emph{cruller} glaseado que compró, para llevarse los pedacitos a la boca.

``Me pagan por dar conferencias. No tengo ingresos enormes y no los necesito, mi vida no es muy cara. 
Lo que deseaba realmente no era lo material, lo que se vende. Las cosas que más faltaban en mi vida no 
se vendían. ¿Por qué buscar más dinero? Estoy muy contento, los lujos no podrían haberme hecho tan feliz
como haber logrado algo''.

Y vaya que lo ha logrado. Stallman ha recibido múltiples reconocimientos por su férrea defensa del
\emph{software} libre, como una beca de la MacArthur Foundation en 1990, y media docena de doctorados 
honoris causa de parte de universidades de todo el mundo.

La influencia de Stallman es tal que su filosofía ha inundado también el mundo de la cultura; el académico 
Lawrence Lessig creó las licenciaturas Creative Commons, que son el equivalente de la licencia GPL para los 
productos culturales. En 2004 publicó su libro \emph{Cultura libre}, que trata sobre los excesos de las 
leyes de derechos de autor, la piratería y el \emph{copyleft} ---término acuñado por Stallman---, que consiste 
en permitir la libre distribución de copias y versiones modificadas de una obra u otro trabajo, exigiendo que 
los mismos derechos sean preservados en las versiones modificadas.

Falta una hora para que el avión de Richard despegue. Se dirige a Boston, Massachussetts, donde tiene un 
pequeño departamento. Estará poco tiempo: en menos de una semana volará de nuevo, esta vez hacia Innsbruck, 
Austria.

Es hora de la última pregunta.

---¿La comunidad del \emph{software} libre puede innovar a la misma velocidad que lo hacen las grandes
empresas?

---No sé, pero es un asunto secundario, porque la libertad es más importante que la innovación. De hecho la 
innovación no vale nada si viene sin libertad. Sacrificaría sin dudarlo toda la innovación por la libertad.

Un muchacho delgado y moreno deambula por nuestra mesa. Al fin se anima. Se acerca a Richard, le pide un 
autógrafo.

---¿Esta aquí por coincidencia? ---le pregunta Stallman.

---No, sabía que su avión sale a la una de la tarde ---le responde el joven.

---Este es un suceso único en mi vida ---dice Stallman, con una profunda sonrisa---. Nunca me había buscado nadie en el aeropuerto. Raras veces alguien me reconoce, y es agradable. Pero no todas las semanas. Soy un poquito célebre, no mucho ---festeja antes pasar a la sala de abordaje y subirse al avión que lo llevará a ese lugar donde vive unos cuantos meses del año, pero al que nunca podrá llamar hogar.	
\end{document}

