\errorcontextlines=10
\documentclass[12pt]{article}
\usepackage[english]{babel}
\begin{document}

A.~Einstein, Ann. Phys. {\bf 17,} 132-148 \hfill {\large \bf 1905}\\[2cm]

\begin{center}
{\Large Concerning an Heuristic Point of View Toward\\
the Emission and Transformation of Light}\\
\end{center}

\vspace{0.5cm}

\begin{center}
A.~Einstein\\
Bern, 17 March 1905
\end{center}

\centerline{--- ---~~~$\diamond~\diamondsuit~\diamond$~~~--- ---}

\vspace{1cm}

A profound formal distinction  exists between the theoretical concepts
which  physicists have  formed  regarding  gases and  other ponderable
bodies  and the  Maxwellian  theory of   electromagnetic  processes in
so--called empty  space. While we  consider the  state of a body to be
completely determined by the positions and velocities of a very large,
yet finite, number of  atoms and electrons,  we make use of continuous
spatial  functions to  describe the  electromagnetic  state of a given
volume,  and a  finite  number of   parameters  cannot be  regarded as
sufficient for the  complete determination of  such a state. According
to the  Maxwellian  theory, energy  is to be  considered  a continuous
spatial function in the case of  all purely electromagnetic phenomena
including  light,  while the  energy of  a  ponderable  object should,
according to the present  conceptions of physicists, be represented as
a sum carried over the atoms and electrons. The energy of a ponderable
body cannot be  subdivided into arbitrarily  many or arbitrarily small
parts,  while  the  energy of  a beam  of  light from  a  point source
(according  to the   Maxwellian theory  of light  or, more  generally,
according   to any  wave theory) is continuously  spread  an ever
increasing volume.

The wave  theory  of light,  which  operates  with  continuous spatial
functions,  has worked well  in the  representation of  purely optical
phenomena and  will probably never  be replaced by  another theory. It
should be kept in  mind, however, that the  optical observations refer
to time  averages rather  than  instantaneous values.  In spite of the
complete  experimental   confirmation of the  theory as  applied to
diffraction,  reflection,  refraction,  dispersion,  etc., it is still
conceivable that  the theory of  light which  operates with continuous
spatial functions may  lead to contradictions  with experience when it
is applied to the phenomena of emission and transformation of light. 

It seems to me that the observations associated with blackbody radiation,
fluorescence, the production of cathode rays by ultraviolet light, and other
related phenomena connected with the emission or transformation of light are
more readily understood if one assumes that the energy of light is
discontinuously distributed in space. In accordance with the assumption to be
considered here, the energy of a light ray spreading out from a point source is
not continuously distributed over an increasing space but consists of a finite
number of energy quanta which are localized at points in space, which move
without dividing, and which can only be produced and absorbed as complete units.

In the following I wish to present the line of thought and the facts which have
led me to this point of view, hoping that this approach may be useful to some
investigators in their research.

\vspace{0.5cm}
\section*{{\bf 1. Concerning a Difficulty with Regard to the Theory of Blackbody Radiation}}
\vspace{0.5cm}

We start first with the point of view taken in the Maxwellian and the electron
theories and consider the following case. In a space enclosed by completely
reflecting walls, let there be a number of gas molecules and electrons which are
free to move and which exert conservative forces on each other on close
approach: i.e. they can collide with each other like molecules in the kinetic
theory of gases.\footnote{This assumption is equivalent to the supposition that
the average kinetic energies of gas molecules and electrons are equal to each
other at thermal equilibrium. It is well known that, with the help of this
assumption, Herr Drude derived a theoretical expression for the ratio of thermal
and electrical conductivities of metals.} Furthermore, let there be a number of
electrons which are bound to widely separated points by forces proportional to
their distances from these points. The bound electrons are also to participate
in conservative interactions with the free molecules and electrons when the
latter come very close. We call the bound electrons ``oscillators'': they emit
and absorb electromagnetic waves of definite periods.

According  to the  present  view  regarding the  origin of  light, the
radiation in the  space we are  considering  (radiation which is found
for the case of dynamic  equilibrium in accordance with the Maxwellian
theory) must be identical with the  blackbody radiation --- 
at least if
oscillators  of all  the  relevant  frequencies  are  considered to be
present.

For the time being, we disregard the radiation emitted and absorbed by
the    oscillators and   inquire  into  the   condition  of  dynamical
equilibrium    associated  with  the   interaction (or   collision) of
molecules and electrons. The  kinetic theory of gases asserts that the
average kinetic energy of an  oscillator electron must be equal to the
average kinetic energy  of a translating gas  molecule. If we separate
the motion of an  oscillator electron into  three components at angles
to each other, we find  for the average  energy $\overline{E}$ of one
of these linear components the expression 
$$
\overline{E} = (R/N) ~ T,
$$
where $R$ denotes the universal gas constant. $N$ denotes the number of ``real
molecules'' in a gram equivalent, and $T$ the absolute temperature. The energy
$\overline{E}$ is equal to two-thirds the kinetic energy of a free monatomic gas
particle because of the equality the time average values of the kinetic and
potential energies of the oscillator. If through any 
cause---in our case through
radiation processes---it should occur that 
the energy of an oscillator takes on a
time-average value greater or less than $\overline{E}$, then the collisions with
the free electrons and molecules would lead to a gain or loss of energy by the
gas, different on the average from zero. Therefore, in the case we are
considering, dynamic equilibrium is possible only when each oscillator has the
average energy $\overline{E}$.

We shall now proceed to present a similar argument regarding the interaction
between the oscillators and the radiation present in the cavity. Herr Planck has
derived\footnote{M. Planck, Ann. Phys. 1, 99 (1900).} the condition for the
dynamics equilibrium in this case under the supposition that the radiation can
be considered a completely random process.\footnote{This problem can be
formulated in the following manner. We expand the $Z$ component of the
electrical force ($Z$) at an arbitrary point during the time interval between $t
= 0$ and $t = T$ in a Fourier series in which $A_{\nu} \geq 0$ and $0 
\leq \alpha_{\nu} \leq 2 \pi$: the time $T$ is taken to be very large relative
to all the periods of oscillation that are present:
$$
Z = \sum^{\nu = \infty}_{\nu = 1} A_{\nu} \sin \left( 2 \pi \nu \frac{t}{T}
+ \alpha_{\nu} \right),
$$
If one imagines making this expansion arbitrary often at a given point in space
at randomly chosen instants of time, one will obtain various sets of values of $
A_{\nu}$ and $\alpha_{\nu}$. There then exist for the frequency of occurrence of
different sets of values of $A_{\nu}$ and $\alpha_{\nu}$ (statistical)
probabilities $dW$ of the form:
$$
dW = f (a_1, A_2, \ldots, \alpha_1, \alpha_2, \ldots) dA_1 dA_2 \ldots d\alpha_1
d \alpha_2 \ldots,
$$
The radiation is then as disordered as conceivable if
$$
f (A_1, A_2, \ldots \alpha_1, \alpha_2, \ldots) = F_1 (A_1) F_2 (A_2) \ldots f_1
(\alpha_1) f_2 (\alpha_2) \ldots,
$$
i.e., if the probability of a particular value of $A$ or $\alpha$
is independent of other values of $A$ or $\alpha$.
The more
closely this condition is fulfilled (namely, that the individual pairs of values
of $A_{\nu}$ and $\alpha_{\nu}$ are dependent upon the emission and absorption
processes of specific groups of oscillators) the more closely will radiation in
the case being considered approximate a perfectly random state.} He found 
$$
(\overline{E}_{\nu}) = (L^3/ 8 \pi \nu^2) \rho_{\nu},
$$
where $(\overline{E}_{\nu})$ is the average energy (per degree of freedom) of an
oscillator with eigenfrequency $\nu$, $L$ the velocity of light, $\nu$ the
frequency, and $\rho_{\nu} d \nu$ the energy per unit volume of that portion of
the radiation with frequency between $\nu$ and $\nu + d\nu$.

If the radiation energy of frequency $\nu$ is 
not continually increasing or
decreasing, the following relations must obtain:
$$
(R/N)~ T = \overline{E} = \overline{E}{\nu} = (L^3/8 \pi \nu^2) 
\rho_{\nu},
$$
$$
\rho_{\nu} = (R/N) (8\pi \nu^2/L^3) ~ T.
$$
These relations, found to be the conditions of dynamic equilibrium, not only
fail to coincide with experiment, but also state that in our model there can be
not talk of a definite energy distribution between ether and matter. The wider
the range of wave numbers  of the oscillators, the greater will be the radiation
energy of the space, and in the limit we obtain
$$
\int \limits^{\infty}_0~ \rho_{\nu}~ d \nu = \frac{R}{N} \cdot \frac{8 \pi}{L^3}
\cdot T~ \int \limits^{\infty}_0 ~ \nu^2 ~d \nu = \infty.
$$

\vspace{0.5cm}
\section*{
{\bf 2. Concerning Planck's Determination of the Fundamental Constants}}
\vspace{0.5cm}

We wish to show in the following that Herr Planck's determination of the
fundamental constants is, to a certain extent, independent of his theory of
blackbody radiation.

Planck's formula,\footnote{M. Planck, Ann. Phys. 4, 561 (1901).} which has
proved adequate up to this point, gives for $\rho_{\nu}$
$$
\rho_{\nu} = \frac{\alpha \nu^3}{e^{\beta \nu /T} - 1},
$$
$$
\alpha = 6.10 \times 10^{-56},
$$
$$
\beta = 4.866 \times 10^{-11}.
$$
For large values of $T/\nu$; i.e. for large wavelengths and radiation
densities, this equation takes the form
$$
\rho_{\nu} = (\alpha/\beta) ~\nu^2  T.
$$
It is evident that this equation is identical with the one obtained in Sec.~1
from the Maxwellian and electron theories. By equating the coefficients of both
formulas one obtains 
$$
(R/N)  (8 \pi/L^3) = (\alpha/\beta)
$$
or
$$
N = (\beta/\alpha)  (8 \pi R/L^3) = 6.17 \times 10^{23}.
$$
i.e., an atom of hydrogen weighs $1/N$ grams $= 1.62 \times 10^{-24}$ g. This is
exactly the value found by Herr Planck, which in turn agrees with values found
by other methods. 

We therefore arrive at the conclusion: the greater the energy density and the
wavelength of a radiation, the more useful do the theoretical principles we have
employed turn out to be: for small wavelengths and small radiation densities,
however, these principles fail us completely.

In the following we shall consider the experimental facts concerning blackbody
radiation without invoking a model for the emission and propagation of the
radiation itself.  

\vspace{0.5cm}
\section*{
{\bf 3. Concerning the Entropy of Radiation}}
\vspace{0.5cm}

The following treatment is to be found in a famous work by Herr W. Wien and is
introduced here only for the sake of completeness.

Suppose we have radiation occupying a volume $v$. We assume that the observable
properties of the radiation are completely determined when the radiation density
$\rho(\nu)$ is given for all frequencies.\footnote{This assumption is an
arbitrary one. One will naturally cling to this simplest assumption as long as
it is not controverted experiment.} Since radiation of different frequencies are
to be considered independent of each other when there is no transfer of heat or
work, the entropy of the radiation can be represented by 
$$
S = v \int \limits^{\infty}_0~ \varphi (\rho, \nu)~ d \nu,
$$
where $\varphi$ is a function of the variables $\rho$ and $\nu$. 

$\varphi$ can be  reduced to a function of a  single  variable through
formulation  of the  condition  that the  entropy of the  radiation is
unaltered during  adiabatic  compression between  reflecting walls. We
shall  not  enter into  this   problem,  however, but  shall  directly
investigate the  derivation of the function  
$\varphi$ from the blackbody
radiation law.

In the case of blackbody radiation, $\rho$ is such 
a function of $\nu$ that the
entropy is maximum for a fixed value of energy; i.e.,
$$
\delta~ \int \limits^{\infty}_0~ \varphi~(\rho, \nu)~ d \nu = 0,
$$
providing
$$
\delta~ \int \limits^{\infty}_0~ \rho d \nu = 0.
$$

>From this it follows that for every choice of $\delta \rho$  as a function of 
$\nu$
$$
\int \limits^{\infty}_0~ \left( \frac{\partial \varphi}{\partial \rho} - 
\lambda \right)  \delta \rho d \nu = 0,
$$
where $\lambda$ is independent of $\nu$. In the case of blackbody radiation,
therefore, 
$\partial \varphi/ \partial \rho$ is independent of $\nu$.

The following equation applies when the temperature of a unit volume of
blackbody radiation increases by $dT$
$$
dS = \int \limits^{\nu = \infty}_{\nu = 0}~ \left( \frac{\partial \varphi}
{\partial \rho} \right)~d \rho d \nu,
$$
or, since $\partial \varphi / \partial \rho$ is independent of $\nu$.
$$
dS = (\partial \varphi/ \partial \rho) ~ dE.
$$
Since $dE$ is equal to the heat added and since the 
process is reversible, the
following statement also applies
$$
dS = (1/T) ~ dE.
$$
By comparison one obtains
$$
\partial \varphi/ \partial \rho = 1/T.
$$

This is the law of blackbody radiation. Therefore one can derive the law of
blackbody radiation from the function 
$\varphi$, and, inversely, one can derive the
function $\varphi$ by integration, keeping in mind the fact 
that $\varphi$ vanishes
when $\rho = 0$.

\vspace{0.5cm}
\section*{
{\bf 4. Asymptotic from for the Entropy of Monochromatic 
Radiation at Low Radiation Density}} 
\vspace{0.5cm}

>From existing observations of the blackbody radiation, it is clear that the law
originally postulated by Herr W. Wien,
$$
\rho = \alpha \nu^3 e^{- \beta \nu/T},
$$
is not exactly valid. It is, however, well confirmed experimentally for large
values of $\nu/T$. We shall base our analysis on this formula, keeping in mind
that our results are only valid within certain limits.

This formula gives immediately
$$
(1/T) = - (1/\beta \nu) ~ \mbox{ln}~ (\rho/\alpha \nu^3)
$$
and then, by using the relation obtained in the preceeding section,
$$
\varphi(\rho, \nu) = - \frac{\rho}{\beta \nu}  \left[ \mbox{ln} ~ 
\left( \frac{\rho}
{\alpha \nu^3} \right) - 1 \right].
$$
Suppose that we have radiation of energy $E$, with frequency between $\nu$ and
$\nu  + d \nu$, enclosed in volume $v$. 
The entropy of this radiation is:
$$
S = v \varphi (\rho, \nu) d\nu = - \frac{E}{\beta \nu}  \left[\mbox{ln}~
 \left( \frac{E}{v \alpha \nu^3 d \nu} \right) - 1 \right].
$$

If we confine ourselves to investigating the dependence of the entropy on the
volume occupied by the radiation, and if we denote by $S_0$ the entropy of the
radiation at volume $v_0$, we obtain 
$$
S - S_0 = (E/\beta \nu) ~ \mbox{ln}~ (v/v_0).
$$

This equation shows that the entropy of a monochromatic radiation of
sufficiently low density varies with the volume in the same manner as the
entropy of an ideal gas or a dilute solution. In the following, this equation
will be interpreted in accordance with the principle introduced into physics by
Herr Boltzmann, namely that the entropy of a system is a function of the
probability its state.

\vspace{0.5cm}
\section*{
{\bf 5. Molecular--Theoretic Investigation of the Dependence
of the Entropy of Gases and Dilute solutions on the volume}}
\vspace{0.5cm}

In the calculation of entropy by molecular--theoretic methods we frequently use
the word ``probability'' in a sense differing from that employed in the calculus
of probabilities. In particular ``gases of equal probability'' have frequently
been hypothetically established when one theoretical models being utilized
are definite enough to permit a deduction rather than a conjecture. I will show
in a separate paper that the so-called ``statistical probability'' is fully
adequate for the treatment of thermal phenomena, and I hope that by doing so I
will eliminate a logical difficulty that obstructs the application of Boltzmann'
s Principle. here, however, only a general formulation and application to very
special cases will be given.

If it is reasonable to speak of the probability of the state of a system, and
futhermore if every entropy increase can be understood as a transition to a
state of higher probability, then the entropy $S_1$ of a system is a function of
$W_1$, the probability of its instantaneous state. If we have two 
noninteracting systems $S_1$ and $S_2$, we can write 
$$
S_1 = \varphi_1 (W_1),
$$
$$
S_2 = \varphi_2  (W_2).
$$
If one considers these two systems as a single system of entropy $S$ and
probability $W$, it follows that 
$$
S = S_1 + S_2 = \varphi (W)
$$
and
$$
W = W_1 \cdot W_2.
$$

The last equation says that the states of the two  systems are independent of
each other.

>From these equation it follows that
$$
\varphi (W_1 \cdot W_2) = \varphi_1 (W_1) + \varphi_2 (W_2)
$$
and finally
$$
\varphi_1(W_1) = C ~\mbox{ln} (W_1) + \mbox{const},
$$
$$
\varphi_2(W_2) = C ~ \mbox{ln} (W_2) + \mbox{const},
$$
$$
\varphi (W) = C ~ \mbox{ln} (W) + \mbox{const}.
$$
The quantity $C$ is therefore a universal constant; the kinetic theory of gases
shows its value to be $R/N$, where the constants $R$ and $N$ have been defined
above. If $S_0$ denotes the entropy of a system in some initial state and $W$
denotes the relative probability of a state of entropy $S$, we obtain in general
$$
S - S_0 = (R/N) ~ \mbox{ln}~W.
$$ 
First we treat the following special case. We consider a number $(n)$ of movable
points (e.g., molecules) confined in a volume $v_0$. Besides these points, there
can be in the space any number of other movable points of any kind. We shall not
assume anything concerning the law in accordance with which the points move in
this space except that with regard to this motion, no part of the space (and no
direction within it) can be distinguished from any other. Further, we take the
number of these movable points to be so small that we can disregard interactions
between them.

This system, which, for example, can be an ideal gas or a dilute solution,
possesses an entropy $S_0$. Let us imagine transferring all $n$ movable points
into a volume $v$ (part of the volume $v_0$) without anything else being changed
in the system. This state obviously possesses a different entropy $(S)$, and now
wish to evaluate the entropy difference with the help of the Boltzmann
Principle.

We inquire: How large is the probability of the latter state relative to the
original one? Or: How large is the probability that at a randomly chosen instant
of time all $n$ movable points in the given volume $v_0$ will be found by chance
in the volume $v$?

For this probability, which is a ``statistical probability'', one obviously
obtains:
$$
W = (v/v_0)^n;
$$
By applying the Boltzmann Principle, one then obtains
$$
S - S_0 = R ~ (n/N) ~ \mbox{ln}~ (v/v_0).
$$

It is noteworthy that in the derivation of this equation, from which one can
easily obtain the law of Boyle and Gay--Lussac as well as the analogous law of
osmotic pressure thermodynamically,\footnote{If $E$ is the energy of the system,
one obtains:
$$
- d \cdot (E - TS) = pdv = TdS = RT \cdot (n/N) \cdot (dv/v);
$$
therefore
$$
pv = R \cdot (n/N) \cdot T.
$$} no assumption had to be made as to a law of motion of the molecules.

\vspace{0.5cm}
\section*{
{\bf 6. Interpretation of the Expression for the volume Dependence of the
entropy of Monochromatic Radiation in Accordance with Boltzmann's Principle}}
\vspace{0.5cm}

In Sec. 4, we found the following expression for the dependence of the entropy
of monochromatic radiation on the volume
$$
S - S_0 = (E/\beta \nu) ~ \mbox{ln}~ (v/v_0).
$$
If one writes this in the from
$$
S - S_0 = (R/N) ~ \mbox{ln} \left[ (v/v_0)^{(N/R) (E/\beta \nu)}
\right].
$$
and if one compares this with the general formula for the Boltzmann principle
$$
S - S_0 = (R/N) ~ \mbox{ln} W,
$$
one arrives at the following conclusion:

If monochromatic radiation of frequency $\nu$ and energy $E$ is enclosed by
reflecting walls in a volume $v_0$, the probability that the total radiation
energy will be found in a volume $v$ (part of the volume $v_0$) at any randomly
chosen instant is 
$$
W = (v/v_0)^{(N/R)  (E/\beta \nu)}.
$$

>From this we further conclude that: Monochromatic radiation of low density (
within the range of validity of Wien's radiation formula) behaves
thermodynamically as though it consisted of a number of independent energy
quanta of magnitude $R \beta \nu/N$.

We still wish to compare the average magnitude of the energy quanta of the
blackbody radiation with the average translational kinetic energy of a molecule
at the same temperature. The latter is 
$^3/_2 (R/N)  T$, while,
according to the Wien formula, one obtains for the average magnitude of an
energy quantum
$$
\int \limits^{\infty}_0~ \alpha \nu^3 e^{- \beta \nu/T} d\nu\bigg/
\int \limits^{\infty}_0~ \frac{N}{R \beta \nu} ~ \alpha \nu^3 
e^{- \beta \nu/ T} d \nu = 3 (RT/N).
$$

If the entropy of monochromatic radiation depends on volume as though the
radiation were a discontinuous medium consisting of energy quanta of magnitude $
R \beta \nu/N$, the next obvious step is to investigate whether the laws of
emission and transformation of light are also of such a nature that they can be
interpreted or explained  by considering light to consist of such energy quanta.
We shall examine this question in the following.

\vspace{0.5cm}
\section*{
{\bf 7. Concerning Stokes's Rule}}
\vspace{0.5cm}

According to the result just obtained, let us assume that, when monochromatic
light is transformed through photoluminescence into light of a different
frequency, both the incident and emitted light consist of energy quanta of
magnitude $R \beta \nu/N$, where $\nu$ denotes the relevant frequency. The
transformation process is to be interpreted in the following manner. Each
incident energy quantum of frequency $\nu_1$ is absorbed and generates by 
itself--at least at sufficiently low densities of incident energy quanta -- a
light quantum of frequency $\nu_2$; it is possible that the absorption of the
incident light quanta can give rise to the simultaneous emission of light quanta
of frequencies $\nu_3, \nu_4$ etc., as well as to energy of other kinds, e.g.,
heat. It does not matter what intermediate processes give rise to this final
result. If the fluorescent substance is not a perpetual source of energy, the
principle of conservation of energy requires that the energy of an emitted
energy quantum cannot be greater than that of the incident light quantum; it
follows that 
$$
R~ \beta \nu_2/N \leq R~ \beta \nu_1/N
$$
or
$$ 
\nu_2 \leq \nu_1.
$$
This is the well--known Stokes's Rule.

It should be strongly emphasized that according to our conception the quantity
of light emitted under conditions of low illumination (other conditions
remaining constant) must be proportional to the strength of the incident light,
since each incident energy quantum will cause an elementary process of the
postulated kind, independently of the action of other incident energy quanta. In
particular, there will be no lower limit for the intensity of incident light
necessary to excite the fluorescent effect.

According to the conception set forth above, deviations from Stokes's Rule are
conceivable in the following cases:

1. when the number of simultaneously interacting energy quanta per unit volume
   is so large that an energy quantum of emitted light can receive its energy
   from several incident energy quanta;

2. when the incident (or emitted) light is not of such a composition that it
   corresponds to blackbody radiation within the range of validity of Wien's 
Law, that is to say, for example, when the incident light is produced by a body
of such high temperature that for the wavelengths under consideration 
Wien's Law is no longer valid. 

The last-mentioned possibility commands especial interest. According to the
conception we have outlined, the possibility is not excluded that a ``non-Wien
radiation'' of very low density can exhibit an energy behavior different from
that of a blackbody radiation within the range of validity of Wien's Law.

\vspace{0.5cm}
\section*{
{\bf 8. Concerning the Emission of Cathode Rays\\ Through  Illumination of
Solid Bodies}}
\vspace{0.5cm}
 
The usual conception that the energy of light is continuously distributed over
the space through which it propagates, encounters very serious difficulties when
one attempts to explain the photoelectric phenomena, as has been pointed out in
Herr Lenard's pioneering paper.\footnote{P. Lenard, Ann. Phys., 8, 169, 170 (
1902).}

According to the concept that the incident light consists of energy quanta of
magnitude $R \beta \nu/N$, however, one can conceive of the ejection of
electrons by light in the following way. Energy quanta penetrate into the
surface layer of the body, and their energy is transformed, at least in part,
into kinetic energy of electrons. The simplest way to imagine this is that a
light quantum delivers its entire energy to a single electron: we shall assume
that this is what happens. The possibility should not be excluded, however, that
electrons might receive their energy only in part from the light quantum. 

An electron to which kinetic energy has been imparted in the interior of the
body will have lost some of this energy by the time it reaches the surface.
Furthermore, we shall assume that in leaving the body each electron must perform
an amount of work $P$ characteristic of the substance. The ejected electrons
leaving the body with the largest normal velocity will be those that were
directly at the surface. The kinetic energy of such electrons is given by 
$$
R~ \beta \nu/N - P.
$$

In the body is charged to a positive potential  $\Pi$ and is surrounded by
conductors at zero potential, and if $\Pi$ is just large 
enough to prevent
loss of electricity by the body, if follows that: 
$$
\Pi \epsilon = R \beta \nu/N - P
$$
where $\epsilon$ denotes the electronic charge, or 
$$
\Pi E = R \beta \nu - P'
$$
where $E$ is the charge of a gram equivalent of a monovalent ion and $P'$ is the
potential of this quantity of negative electricity relative to the 
body.\footnote{If one assumes that the individual electron is detached from a
neutral molecule by light with the performance of a certain amount of work,
nothing in the relation derived above need be changed; one can simply consider $
P'$ as the sum of two terms.}

If one takes $E = 9.6 \times 10^3$, then $\Pi \cdot 10^{-8}$ is 
the potential
in volts which the body assumes when irradiated in a vacuum.

In order to see whether the derived relation yields an 
order of magnitude
consistent with experience, we take 
$P' = 0$, $\nu = 1.03 \times 10^{15}$
(corresponding to the limit of the 
solar spectrum toward the ultraviolet) and 
$\beta = 4.866 \times 10^{-11}$. 
We obtain $\Pi \cdot 10^7 = 4.3$ volts, a
result agreeing in order magnitude with those 
of Herr Lenard.\footnote{P.Lenard, Ann. Phys. 8, 
pp. 163, 185, and Table I, Fig. 2 (1902).}

If the derived formula is correct, then $\Pi$, when represented in Cartesian
coordinates as a function of the frequency of the incident light, must be a
straight line whose slope is independent of the nature of the emitting
substance.

As far as I can see, there is no contradiction between these conceptions and the
properties of the photoelectric observed by Herr Lenard. If each energy quantum
of the incident light, independently of everything else, delivers its energy of
electrons, then the velocity distribution of the ejected electrons will be
independent of the intensity of the incident light; on the other hand the number
of electrons leaving the body will, if other conditions are kept constant, be
proportional to the intensity of the incident light.\footnote{P. Lenard, Ref. 9,
p. 150 and p. 166--168.}

Remarks similar to those made concerning hypothetical deviations from Stokes's
Rule can be made with regard to hypothetical boundaries 
of validity of the law set forth above.

In the foregoing it has been assumed that the energy of at least some of the
quanta of the incident light is delivered completely to individual electrons. If
one does not make this obvious assumption, one obtains, in place of the last
equation:
$$
\Pi E + P' \leq R \beta \nu.
$$

For fluorescence induced by cathode rays, which is the inverse process to the
one discussed above, one obtains by analogous considerations:
$$
\Pi E + P' \geq R \beta \nu.
$$
In the case, of the substances investigated by 
Herr Lenard, $PE$ \footnote{Should
be $\Pi E$ (translator's note).}is always significantly greater than $R \beta
\nu$, since the potential difference, which the cathode rays must traverse in
order to produce visible light, amounts in some cases to hundreds and in others
to thousands of volts.\footnote{P. Lenard, Ann. Phys., 12, 469 (1903).} It
is therefore to be assumed that the kinetic energy of an electron goes into the
production of many light energy quanta.

\vspace{0.5cm}
\section*{
{\bf 9. Concerning the Ionization of Gases by Ultraviolet Light
Solid Bodies}}
\vspace{0.5cm}

We shall have to assume that, the ionization of a gas by ultraviolet light, an
individual light energy quantum is used for the ionization of an individual gas
molecule. From this is follows immediately that the work of ionization (i.e.,
the work theoretically needed for ionization) of a molecule cannot be greater
than the energy of an absorbed light quantum capable of producing this effect.
If one denotes by $J$ the (theoretical) work of ionization per gram equivalent,
then it follows that:
$$
R ~\beta \nu \geq J.
$$
According to Lenard's measurements, however, the largest  effective wavelength
for air is approximately $1.9 \times 10^{-5}$ cm: therefore:
$$
R ~\beta \nu = 6.4 \cdot 10^{12}~ \mbox{erg} \geq J.
$$
An upper limit for the work of ionization can also be obtained from the
ionization potentials of rarefied gases. according to J. Stark\footnote{J.
Stark, Die Electrizit\"et in Gasen (Leipzig, 1902, p. 57)} the smallest observed
ionization potentials for air (at platinum anodes) is about 10 V.\footnote{In
the interior of gases the ionization potential for negative ions is, however,
five times greater.} One therefore obtains $9.6 \times 10^{12}$ as an upper limit
for $J$, which is nearly equal to the value found above.

There is another consequence the experimental testing of which seems to me to be
of great importance. If every absorbed light energy quantum ionizes a molecule,
the following relation must obtain between the quantity of absorbed light $L$
and the number of gram molecules of ionized gas $j$:
$$
j = L/R \beta \nu.
$$

If our conception is correct, this relationship must be valid for all gases
which (at the relevant frequency) show no appreciable absorption without
ionization.

\end{document}
