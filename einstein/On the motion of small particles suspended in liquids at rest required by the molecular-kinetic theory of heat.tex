\errorcontextlines=10
\documentclass{article}
\usepackage[english]{babel}

\begin{document}

A.~Einstein, Ann. Phys. {\bf 17,} 549-560 \hfill {\large \bf 1905}\\[2cm]

\begin{center}
{\Large On the motion of small particles suspended in liquids at rest required by the molecular-kinetic theory of heat}\\
\end{center}

\vspace{0.5cm}

\begin{center}
A.~Einstein\\
Received May 11, 1905\\
\end{center}

\centerline{--- ---~~~$\diamond~\diamondsuit~\diamond$~~~--- ---}

\vspace{1cm}

In this paper it will be shown that, according to the molecular-kinetic theory of heat,
bodies of a microscopically visible size suspended in liquids must, as a result of thermal molecular
motions, perform motions of such magnitude that they can be easily observed with a microscope.
It is possible that the motions to be discussed here are identical with so-called Brownian molecular
motion; however, the data available to me on the latter are so imprecise that I could not form a
judgment on the question. If the motion to be discussed here can actually be observed, together
with the laws it is expected to obey, then classical thermodynamics can no longer be viewed as
applying to regions that can be distinguished even with a microscope, and an exact determination
of actual atomic sizes becomes possible. On the other hand, if the prediction of this motion were
to be proved wrong, this fact would provide a far-reaching argument against the molecular-kinetic
conception of heat.

\section{On the Osmotic Pressure to be Ascribed to Suspended Particles}

Let $z$ gram-molecules of a non-electrolyte be dissolved in a part $V^{*}$ of the total volume $V$ of a
liquid. If the volume $V^{*}$ is separated from the pure solvent by a wall that is permeable to the
solvent but not to the solute, then this wall is subjected to a so-called osmotic pressure, which for
sufficiently large values of $V^{*}/z$ satisfies the equation

\[pV^{*}=RTz\]

But if instead of the solute, the partial volume $V^{*}$ of the liquid contains small suspended bodies
that also cannot pass through the solvent-permeable wall, then, according to the classical theory
of thermodynamics, we should not expect–at least if we neglect the force of gravity, which does not
interest us here–any pressure to be exerted on the wall; for according to the usual interpretation,
the “free energy” of the system does not seem to depend on the position of the wall and of the
suspended bodies, but only on the total mass and properties of the suspended substance, the liquid,
and the wall, as well as on the pressure and temperature. To be sure, the energy and entropy of
the interfaces (capillary forces) should also be considered when calculating the free energy; but we
can disregard them here because changes in the position of the wall and suspended bodies will not
cause changes in the size and state of the contact surfaces.

But a different interpretation arises from the standpoint of the molecular-kinetic theory of heat.
According to this theory, a dissolved molecule differs from a suspended body only in size, and it
is difficult to see why suspended bodies should not produce the same osmotic pressure as an equal
number of dissolved molecules. We have to assume that the suspended bodies perform an irregular,
albeit very slow, motion in the liquid due to the liquid’s molecular motion; if prevented by the wall
from leaving the volume V ∗ , they will exert pressure upon the wall just like molecules in solution.
Thus, if n suspended bodies are present in the volume $V^{*}$ , i.e., $n/V^{*}=\nu$ in a unit volume, and
if neighboring bodies are sufficiently far separated from each other, there will be a corresponding
osmotic pressure p of magnitude

\[p=\frac{RT}{V^{*}}\frac{n}{N}=\frac{RT}{N}\cdot\nu,\]

where $N$ denotes the number of actual molecules per gram-molecule. It will be shown in the next
section that the molecular-kinetic theory of heat does indeed lead to this broader interpretation of
osmotic pressure.

%\section{Osmotic Pressure from the Standpoint of the Molecular-Kinetic Theory of Heat\footnote{this section it is assumed that the reader is familiar with the author's papers on the foundations of thermo-dynamics (cf. \textit{Ann}. d. \textit{Phys}. 9 [1902]: 417 and 11 [1903]: 170). Knowledge of these Papers and of this section of the present paper is not essential for an understanding of the results in the present paper.}}

If $p_{1}p_{2}\ldots p_{l}$ are state variables of a physical system that completely determine the system’s instan-
taneous state (e.g., the coordinates and velocity components of all atoms of the system), and if the
complete system of equations for changes of these variables is given in the form

\[\frac{\partial p_{\nu}}{\partial t}=\varphi_{\nu}(p_{1}\ldots p_{l}) (\nu=1,2,\ldots l)\]

with $\sum \frac{\partial _{\varphi\mu}}{\partial _{p \mu}}=0$ then the entropy of the system is given by the expression

\[ S = \frac{\bar{E}}{T} + 2\kappa \ln \int e^{- \frac{E}{2^{\kappa}T}} dp_{1} \ldots dp_{l} \]

Here $T$ denotes the absolute temperature, $\bar{E}$ the energy of the system, and $E$ the energy as a
function of the $p_{\nu}$. The integral extends over all possible values of $p_{\nu}$ consistent with the conditions of the problem. $\kappa$ is connected with the constant $N$ mentioned above by the relation $2\kappa N = R$. Hence we get for the free energy $F$

\[F= -\frac{R}{N}T \ln \int e^{- \frac{EN}{RT}}dp_{1} \ldots dp_{l} = - \frac{RT}{N} \ln B. \]

Let us now imagine a liquid enclosed in volume $V$; let a part $V^{*}$ of the volume $V$ contain n
solute molecules or suspended bodies, which are retained in the volume $V^{*}$ by a semipermeable
wall; the integration limits of the integral $R$ occurring in the expressions for $S$ and $F$ will be affected
accordingly. Let the total volume of the solute molecules or suspended bodies be small compared
with $V^{*}$. In accordance with the theory mentioned, let this system be completely described by the
variables $p_{1} \ldots p_{l}$.

Even if the molecular picture were extended to include all details, the calculation of the integral
$B$ would he so difficult that an exact calculation of $F$ is hardly conceivable. However, here we only
need to know how $F$ depends on the size of the volume $V^{*}$ in which all the solute molecules or
suspended bodies (hereafter called “particles” for brevity) are contained.

Let us call the rectangular coordinates of the center of gravity of the first particle $x1 , y1 , z1$, those
of the second $x2 , y2 , z2$, etc., and those of the last particle $x_{n}, y_{n}, z_{n}$, and assign to the centers of gravity of the particles the infinitesimally small parallelepiped regions $dx_{1}dy_{1}dz_{1}$,   $dx_{2}dy_{2}dz_{2} \ldots dx_{n}dy_{n}dz_{n}$,
all of which lie in $V^{*}$. We want to evaluate the integral occurring in the expression for $F$ , with the
restriction that the centers of gravity of the particles shall lie in the regions just assigned to them.
In any case, this integral can be put into the form

\[dB= dx_{1}dy_{1} \ldots dz_{n} \cdot J, \]

where $J$ is independent of $dx_{1}dy_{1}$, etc., as well as of $V^{*}$, i.e., of the position of the semiperme-
able wall. But J is also independent of the particular choice of the \textit{positions} of the center of
gravity regions and of the value of $V^{*}$, as will be shown immediately. For if a second system of
infinitesimally small regions were assigned to the centers of gravity of the particles and denoted by
$dx'_{1}dy'_{1}dz'_{1}$, $dx'_{2}dy'_{2}dz'_{2} \ldots dx'_{n}dy'_{n}dz'_{n}$, and if these regions differed from the originally assigned ones by their position alone, but not by their size, and if, likewise, all of them were contained in $V^{*}$, we would similarly have

\[dB'=dx'_{1}dy'_{1} \ldots dz'_{n} \cdot J', \]

where

\[dx_{1}dy_{1} \ldots dz_{n}=dx'_{1}dy'_{1} \ldots dz'_{n}.\]

Hence,

\[\frac{dB}{dB'}=\frac{J}{J'} \]

But from the molecular theory of heat, as presented in the papers cited,\footnote{A. Einstein, \textit{Ann. d. Phys}. 11 (1903): 170.} it is easily deduced that
$dB/B$ and $dB′/B$ are respectively equal to the probabilities that at an arbitrarily chosen moment
the centers of gravity of the particles will be found in the regions ($dx_{1} \ldots dz_{n}$) and ($dx'_{l} \ldots dz'_{n}$)
respectively. If the motions of the individual particles are independent of one another (to a sufficient
approximation) and if the liquid is homogeneous and no forces act on the particles, then for regions
of the same size the probabilities of the two systems of regions will be equal, so that

\[\frac{dB}{B}=\frac{dB'}{B} \]

But from this equation and the previous one it follows that

\[J=J' \]

This proves that $J$ does not depend on either $V^{*}$ or $x_{1}, y_{1}, \ldots z_{n}$. By integrating, we get

\[B= \int Jdx_{l} \ldots dz_{n}=JV^{*n}, \]

and hence

\[F=-\frac{RT}{N}\lbrace\ln J+n \ln V^{*}\rbrace \]

and

\[p=\frac{\partial F}{\partial V^{*}}=\frac{RT}{V^{*}}\frac{n}{N}=\frac{RT}{N}\nu \]

This analysis shows that the existence of osmotic pressure can be deduced from the molecular-
kinetic theory of heat, and that, at high dilution, according to this theory, equal numbers of solute
molecules and suspended particles behave identically as regards osmotic pressure.

\section{Theory of Diffusion of Small Suspended Spheres}

Suppose suspended particles are randomly distributed in a liquid. We will investigate their state
of dynamic equilibrium under the assumption that a force $K$, which depends on the position but
not on the time, acts on the individual particles. For the sake of simplicity, we shall assume that
the force acts everywhere in the direction of the $X$-axis.

If the number of suspended particles per unit volume is $\nu$, then in the case of thermodynamic
equilibrium $\nu$ is a function of x such that the variation of the free energy vanishes for an arbitrary
virtual displacement $\delta x$ of the suspended substance. Thus

\[\delta F = \delta E - T\delta S =0\]

Let us assume that the liquid has a unit cross section perpendicular to the $X$-axis, and that it is
bounded by the planes $x=0$ and $x=l$. We then have

\[\delta E= -\int^{1}_{0} K\nu\,\delta x\, dx\]

\clearpage

and

\[\delta S= \int^{l}_{0} R \frac{\nu}{N} \frac{\partial\delta x}{\partial x}dx = - \frac{R}{N}\int^{l}_{0}
\frac{\partial\nu}{\partial x}\delta xdx \]

The required equilibrium condition is therefore

\begin{equation}
\label{eqn:1}
-K\nu + \frac{RT}{N}\frac{\partial\nu}{\partial x} = 0
\end{equation}

or

\[K\nu - \frac{\partial p}{\partial x} = 0 \]

\noindent The last equation asserts that the force K is equilibrated by the force of osmotic pressure.

We can use equation (\ref{eqn:1}) to determine the diffusion coefficient of the suspended substance. We
can look upon the dynamic equilibrium state considered here as a superposition of two processes
proceeding in opposite directions, namely:

\begin{enumerate}
\item A motion of the suspended substance under the influence of the force K that acts on each
suspended particle.
\item A process of diffusion, which is to be regarded as the result of the disordered motions of the
particles produced by thermal molecular motion.
\end{enumerate}

If the suspended particles have spherical form (where $P$ is the radius of the sphere) and the
coefficient of viscosity of the liquid is $k$, then the force $K$ imparts to an individual particle the
velocity\footnote{Cf., e.g., G. Kirchhoff, \textit{Vorlesungen uber Mechanik}, 26. Vorl., 4 (\textit{Lectures on Mechanics}, Lecture 26, sec. 4).}

\[\frac{K}{6\pi kP}, \]

and 

\[\frac{\nu K}{6\pi kP}\]

\noindent particles will pass through a unit area per unit time.

Further, if $D$ denotes the diffusion coefficient of the suspended substance and $\mu$ the mass of a
particle, then

\[-D\frac{\partial (\mu\nu)}{\partial x}\mathrm{grams}\]

\clearpage

or

\[-D\frac{\partial\nu}{\partial x}\]

\noindent particles will pass across a unit area per unit time as the result of diffusion. Since dynamic
equilibrium prevails, we must have

\begin{equation}
\label{eqn:2}
\frac{\nu K}{6\pi kP}-D \frac{\partial\nu}{\partial x}=0 
\end{equation}

From the two conditions (\ref{eqn:1}) and (\ref{eqn:2}) found for dvnamic equilibrium, we can calculate the diffusion coefficient. We get

\[D=\frac{RT}{N} \cdot \frac{1}{6\pi kP} \]

Thus, except for universal constants and the absolute temperature, the diffusion coefficient of the
suspended substance depends only on the viscosity of the liquid and on the size of the suspended
particles.

\section{On the Disordered Motion of Particles Suspended in a Liquid and its Relation to Diffusion}

We shall now turn to a closer examination of the disordered motions that arise from thermal
molecular motion and give rise to the diffusion investigated in the last section.

Obviously, we must assume that each individual particle executes a motion that is independent
of the motions of all the other particles; the motions of the same particle in different time intervals
must also be considered as mutually independent processes, so long as we think of these time
intervals as chosen not to be too small.

   We now introduce a time interval $\tau$ , which is very small compared with observable time intervals
but still large enough that the notions performed by a particle during two consecutive time intervals
$\tau$ can be considered as mutually independent events.

Suppose, now, that a total of n suspended particles is present in a liquid. In a time interval $\tau$ ,
the $X$-coordinates of the individual particles will increase by $\Delta$, where $\Delta$ has a different (positive or negative) value for each particle. A certain probability distribution law will hold for $\Delta$: the number dn of particles experiencing a displacement that lies between $\Delta$ and $\Delta+d\Delta$ in the time interval $\tau$ will be expressed by an equation of the form

\[dn=n\varphi(\Delta)d\Delta\]

where

\[\int^{+\infty}_{-\infty} \varphi(\Delta)d\Delta=1, \]

and $\varphi$ differs from zero only for very small values of $\Delta$ and satisfies the condition

\[\varphi(\Delta)=\varphi(-\Delta).\]

We will now investigate how the diffusion coefficient depends on $\varphi$, restricting ourselves again
to the ease where the number $\nu$ of particles per unit volume only depends on $x$ and $t$.

Let $\nu = f(x, t)$ be the number of particles per unit volume; we calculate the distribution of
particles at time $t + \tau$ from their distribution at time $t$. From the definition of the function $\varphi(\Delta)$ we can easily obtain the number of particles found at time $t + \tau$ between two planes perpendicular
to the $X$-axis with abscissas $x$ and $x + dx$. One obtains

\[f(x,t+\tau)dx=dx \cdot \int^{\Delta = +\infty}_{\Delta = -\infty} f(x,\Delta)\varphi(\Delta)d\Delta \]

But since $\tau$ is very small, we can put

\[f(x,t+\tau)=f(x,t)+\tau\frac{\partial f}{\partial t}.\]

Further, let us expand $f (x + \Delta, t)$ in powers of $\Delta$:

\[f(x+\Delta, t) =f(x,t)+\Delta \frac{\partial f(x,t)}{\partial x} + \frac{\Delta ^{2}}{2!}
\frac{\partial^{2}f(x,t)}{\partial x^{2}} \ldots \mathrm{ad\: inf} \]

We can bring this expansion under the integral sign since only very small values of $\Delta$ contribute
anything to the latter. We obtain

\[f \, +\frac{\partial f}{\partial t} \, \cdot \, \tau = f \, \cdot \, \int^{+\infty}_{-\infty} \varphi(\Delta)d\Delta  \, +  \, \frac{\partial f}{\partial x} \int^{+\infty}_{-\infty} \Delta\varphi(\Delta)d\Delta \, + \, \frac{\partial ^{2}f}{\partial x^{2}}
\int^{+\infty}_{-\infty} \frac{\Delta ^{2}}{2} \varphi(\Delta)d\Delta \ldots  \]

On the right-hand side, the second, fourth, etc., terms vanish since $\varphi(x) = \varphi (-x)$, while for the
first, third, fifth, etc., terms, each successive term is very small compared with the one preceding
it. From this equation, by taking into account that

\[\int^{+\infty}_{-\infty} \varphi(\Delta)d\Delta = 1, \]

\noindent and putting

\[\frac{1}{\tau} \int^{+\infty}_{-\infty} \frac{\Delta^{2}}{2} \varphi(\Delta)d\Delta=D, \]

and taking into account only the first and third terms of the right-hand side, we get

\[\frac{\partial f}{\partial t} = D \frac{\partial^{2} f}{\partial x^{2}} \]

This is the well-known differential equation for diffusion, and we recognize that $D$ is the diffusion
coefficient.

Another important point can be linked to this argument. We have assumed that the individual
particles are all referred to the same coordinate system. However, this is not necessary since the
motions of the individual particles are mutually independent. We will now refer the motion of each
particle to a coordinate system whose origin coincides with the position of the center of gravity of
the particle in question at time $t = 0$, with the difference that $f (x, t)dx$ now denotes the number of
particles whose $X$-coordinate has increased between the times $t = 0$ and $t = t$ by a quantity that
lies somewhere between $x$ and $x + dx$. Thus, the function $f$ varies according to equation (\ref{eqn:2}) in
this case as well. Further, it is obvious that for $x\neq 0$ and $t = 0$ we must have

\[f(x,t)= 0 \; \mathrm{and} \; \int^{+\infty}_{-\infty} f(x,t)dx=n\]

The problem, which coincides with the problem of diffusion outwards from a point (neglecting
the interaction between the diffusing particles), is now completely determined mathematically; its
solution is

\[f(x,t) = \frac{n}{\sqrt{4\pi D}} \frac{e^{-\frac{x^{2}}{4Dt}}}{\sqrt{t}} \]

   The probability distribution of the resulting displacements during an arbitrary time $t$ is thus
the same as the distribution of random errors, which was to be expected. What is important,
however, is how the constant in the exponent is related to the diffusion coefficient. With the help
of this equation we can now calculate the displacement $\lambda_{x}$ in the direction of the $X$-axis that a
particle experiences on the average, or, to be more precise, the root-mean-square displacement in
the $X$-direction; it is

\[\lambda_{x} = \sqrt{\overline{x^{2}}} = \sqrt{2Dt} \]

The mean displacement is thus proportional to the square root of the time. It can easily be shown
that the root-mean-square of the total displacements of the particles has the value $\lambda_{x} \sqrt{3}$.

\section{Formula for the Mean Displacement of Suspended Particles. A New Method of  Determining the Actual Size of Atoms}

In section 3 we found the following value for the diffusion coefficient $D$ of a substance suspended
in a liquid in the form of small spheres of radius $P$:

\[D= \frac{RT}{N} \frac{1}{6\pi kP}.\]

\noindent Further, we found in section 4 that the mean value of the displacements of the particles in the
$X$-direction at time $t$ equals

\[\lambda_{x} = \sqrt{2Dt} \]

\noindent By eliminating $D$, we get:

\[\lambda_{x} = \sqrt{t} \cdot \sqrt{\frac{RT}{N} \frac{1}{3\pi kP}}. \]


\noindent This equation shows how $\lambda _{x}$ depends on $T$, $k$, and $P$.

We will now calculate how large $\lambda _{x}$ is for one second if $N$ is taken to be 6 · 10$^{23}$ in accordance
with the results of the kinetic theory of gases; water at 17$^{\circ}$C ($k = 1.35 \cdot 10^{−2}$) is chosen as the
liquid, and the diameter of the particles is 0.001 mm. We get

\[\lambda_{x} = 8 \cdot 10^{-5} \: \mathrm{cm} = 0.8 \: \mathrm{micron}. \]

\noindent Therefore, the mean displacement in one minute would be about 6 microns.

Conversely the relation can be used to determine $N$ . We obtain

\[\frac{t}{\lambda^{2}_{x}} \cdot \frac{RT}{3\pi KP}.\]

Let us hope that a researcher will soon succeed in solving the problem presented here, which is so
important for the theory of heat.





\end{document}


