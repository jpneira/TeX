%% http://www.gutenberg.org/ebooks/36114
\documentclass[spanish]{book}
\usepackage[spanish]{babel}
\usepackage{amsmath}    

\begin{document}
\title{\textbf{\Huge Sobre la Teoría}\\
       \textbf{\Huge de la}\\
       \textbf{\Huge Relatividad}\\
       \textbf{\Huge Especial \& General}}
\author{{\Large Albert Einstein}}
\date{1916}


\maketitle

\chapter*{Prefacio}
\addcontentsline{toc}{chapter}{Prefacio}

El presente librito pretende dar una idea lo más exacta posible de la teoría de la relatividad, pensando en aquellos que, sin dominar el aparato matemático de la física teórica, tienen interés en la teoría desde el punto de vista científico o filosófico general. La lectura exige una formación de bachillerato aproximadamente y ---pese a la brevedad del librito--- no poca paciencia y voluntad por parte del lector. El autor ha puesto todo su empeño en resaltar con la máxima claridad y sencillez las ideas principales, respetando por lo general el orden y el contexto en que realmente surgieron. En aras de la claridad me pareció inevitable repetirme a menudo, sin reparar lo más mínimo en la elegancia expositiva; me atuve obstinadamente al precepto del genial teórico L. Boltzmann, de dejar la elegancia para los sastres y zapateros. Las dificultades que radican en la teoría propiamente dicha creo no habérselas ocultado al lector, mientras que las bases físicas empíricas de la teoría las he tratado deliberadamente con cierta negligencia, para que al lector alejado de la física no le ocurra lo que al caminante, a quien los árboles no le dejan ver el bosque. Espero que el librito depare a más de uno algunas horas de alegre entretenimiento.

\bigskip

{\raggedleft Diciembre, 1916\\
{\sc A. Einstein}}

\part{Sobre la Teoría de la Relatividad Especial}

\chapter{El Contenido Físico de los Teoremas Geométricos}

Seguro que también tú, querido lector, entablaste de niño conocimiento con el
soberbio edificio de la Geometría de Euclides y recuerdas, quizá con más respeto que
amor, la imponente construcción por cuyas altas escalinatas te pasearon durante horas
sin cuento los meticulosos profesores de la asignatura. Y seguro que, en virtud de ese
tu pasado, castigarías con el desprecio a cualquiera que declarase falso incluso el más
recóndito teoremita de esta ciencia. Pero es muy posible que este sentimiento de
orgullosa seguridad te abandonara de inmediato si alguien te preguntara: «¿Qué
entiendes tú al afirmar que estos teoremas son verdaderos?». Detengámonos un rato
en esta cuestión.

La Geometría parte de ciertos conceptos básicos, como el de plano, punto, recta, a
los que estamos en condiciones de asociar representaciones más o menos claras, así
como de ciertas proposiciones simples (axiomas) que, sobre la base de aquellas
representaciones, nos inclinamos a dar por «verdaderas». Todos los demás teoremas
son entonces referidos a aquellos axiomas (es decir, son demostrados) sobre la base de
un método lógico cuya justificación nos sentimos obligados a reconocer. Un teorema
es correcto, o «verdadero», cuando se deriva de los axiomas a través de ese método
reconocido. La cuestión de la «verdad» de los distintos teoremas geométricos remite,
pues, a la de la «verdad» de los axiomas. Sin embargo, se sabe desde hace mucho que
esta última cuestión no sólo no es resoluble con los métodos de la Geometría, sino que
ni siquiera tiene sentido en sí. No se puede preguntar si es verdad o no que por dos
puntos sólo pasa \em{una recta}. Únicamente cabe decir que la Geometría euclídea trata de
figuras a las que llama «rectas» y a las cuales asigna la propiedad de quedar
unívocamente determinadas por dos de sus puntos. El concepto de «verdadero» no se
aplica a las proposiciones de la Geometría pura, porque con la palabra «verdadero»
solemos designar siempre, en última instancia, la coincidencia con un objeto «real»; la
Geometría, sin embargo, no se ocupa de la relación de sus conceptos con los objetos
de la experiencia, sino sólo de la relación lógica que guardan estos conceptos entre sí.

El que, a pesar de todo, nos sintamos inclinados a calificar de «verdaderos» los
teoremas de la Geometría tiene fácil explicación. Los conceptos geométricos se
corresponden más o menos exactamente con objetos en la naturaleza, que son, sin
ningún género de dudas, la única causa de su formación. Aunque la Geometría se
distancie de esto para dar a su edificio el máximo rigor lógico, lo cierto es que la
costumbre, por ejemplo, de ver un segmento como dos lugares marcados en un cuerpo
prácticamente rígido está muy afincada en nuestros hábitos de pensamiento. Y
también estamos acostumbrados a percibir tres lugares como situados sobre una recta
cuando, mediante adecuada elección del punto de observación, podemos hacer
coincidir sus imágenes al mirar con un solo ojo.

Si, dejándonos llevar por los hábitos de pensamiento, añadimos ahora a los
teoremas de la Geometría euclídea un único teorema más, el de que a dos puntos de
un cuerpo prácticamente rígido les corresponde siempre la misma distancia
(segmento), independientemente de las variaciones de posición a que sometamos el
cuerpo, entonces los teoremas de la Geometría euclídea se convierten en teoremas
referentes a las posibles posiciones relativas de cuerpos prácticamente rígidos\footnote{
De esta manera se le asigna también a la línea recta un objeto de la naturaleza. 
Tres puntos de un cuerpo rígido A, B, C se hallan situados sobre una línea recta
cuando, dados los puntos A y C, el punto B está elegido de tal manera que la suma 
de las distancia $\overline{AB}$ y $\overline{BC}$ es lo más pequeña posible. 
Esta definición, defectuosa desde luego, puede bastar en este contexto.}. La 
Geometría así ampliada hay que contemplarla como una rama de la física. Ahora sí
cabe preguntarse por la «verdad» de los teoremas geométricos así interpretados,
porque es posible preguntar si son válidos o no para aquellos objetos reales que
hemos asignado a los conceptos geométricos. Aunque con cierta imprecisión,
podemos decir, pues, que por «verdad» de un teorema geométrico entendemos en este
sentido su validez en una construcción con regla y compás.

Naturalmente, la convicción de que los teoremas geométricos son «verdaderos» en
este sentido descansa exclusivamente en experiencias harto incompletas. De entrada
daremos por supuesta esa verdad de los teoremas geométricos, para luego, en la última
parte de la exposición (la teoría de la relatividad general), ver que esa verdad 
tiene sus límites y precisar cuáles son éstos.


\chapter{El Sistema de Coordenadas}

Basándonos en la interpretación física de la distancia que acabamos de señalar
estamos también en condiciones de determinar la distancia entre dos puntos de un
cuerpo rígido por medio de mediciones. Para ello necesitamos un segmento (regla S)
que podamos utilizar de una vez para siempre y que sirva de escala unidad. Si A y B
son dos puntos de un cuerpo rígido, su recta de unión es entonces construible según
las leyes de la Geometría; sobre esta recta de unión, y a partir de A, llevamos el
segmento S tantas veces como sea necesario para llegar a B. El número de repeticiones
de esta operación es la medida del segmento AB. Sobre esto descansa toda medición
de longitudes.\footnote{Se ha supuesto, sin embargo, que la medición es exacta, es 
decir, que da un número entero. De esta dificultad se deshace uno empleando escalas 
subdivididas, cuya introducción no exige ningún método fundamentalmente nuevo.}

Cualquier descripción espacial del lugar de un suceso o de un objeto consiste en
especificar el punto de un cuerpo rígido (cuerpo de referencia) con el cual coincide el
suceso, y esto vale no sólo para la descripción científica, sino también para la vida
cotidiana. Si analizo la especificación de lugar «en Berlín, en la Plaza de Potsdam», veo
que significa lo siguiente. El suelo terrestre es el cuerpo rígido al que se refiere la
especificación de lugar; sobre él, «Plaza de Potsdam en Berlín» es un punto
marcado, provisto de nombre, con el cual coincide espacialmente el suceso.\footnote{
No es preciso entrar aquí con más detenimiento en el significado de «coincidencia 
espacial», pues este concepto es claro en la medida en que, en un caso real, 
apenas habría división de opiniones en torno a su validez}

  Este primitivo modo de localización sólo atiende a lugares situados en la superficie
de cuerpos rígidos y depende de la existencia de puntos distinguibles sobre aquélla.
Veamos cómo el ingenio humano se libera de estas dos limitaciones sin que la esencia
del método de localización sufra modificación alguna. Si sobre la Plaza de Potsdam flota
por ejemplo una nube, su posición, referida a la superficie terrestre, cabrá fijarla sin
más que erigir en la plaza un mástil vertical que llegue hasta la nube. La longitud del
mástil medida con la regla unidad, junto con la especificación del lugar que ocupa el
pie del mástil, constituyen entonces una localización completa. El ejemplo nos
muestra de qué manera se fue refinando el concepto de lugar:
\begin{enumerate}
\item Se prolonga el cuerpo rígido al que se refiere la localización, de modo que el
cuerpo rígido ampliado llegue hasta el objeto a localizar.
\item Para la caracterización del lugar se utilizan \textit{números}, y no la nomenclatura de
puntos notables (en el caso anterior, la longitud del mástil medida con la regla).
\item Se sigue hablando de la altura de la nube aun cuando no se erija un mástil
que llegue hasta ella. En nuestro caso, se determina --mediante fotografías de la nube
desde diversos puntos del suelo y teniendo en cuenta las propiedades de
propagación de la luz-- qué longitud habría que dar al mástil para llegar a la nube.
\end{enumerate}

De estas consideraciones se echa de ver que para la descripción de lugares es
ventajoso independizarse de la existencia de puntos notables, provistos de nombres y
situados sobre el cuerpo rígido al que se refiere la localización, y utilizar en lugar de
ello números. La física experimental cubre este objetivo empleando el sistema de
coordenadas cartesianas.

Este sistema consta de tres paredes rígidas, planas, perpendiculares entre sí y ligadas
a un cuerpo rígido. El lugar de cualquier suceso, referido al sistema de coordenadas,
viene descrito (en esencia) por la especificación de la longitud de las tres verticales o
coordenadas ($x, y, z$) que pueden trazarse desde el suceso hasta esas
tres paredes. Las longitudes de estas tres perpendiculares pueden determinarse
mediante una sucesión de manipulaciones con reglas rígidas, manipulaciones que
vienen prescritas por las leyes y métodos de la Geometría euclidiana.

En las aplicaciones no suelen construirse realmente esas paredes rígidas que forman
el sistema de coordenadas; y las coordenadas tampoco se determinan realmente por
medio de construcciones con reglas rígidas, sino indirectamente. Pero el sentido físico
de las localizaciones debe buscarse siempre en concordancia con las consideraciones
anteriores, so pena de que los resultados de la física y la astronomía se diluyan en la
falta de claridad.\footnote{No es sino en la teoría de la relatividad general, estudiada 
en la segunda parte del libro, donde se hace necesario afinar y modificar esta concepción.}

La conclusión es, por tanto, la siguiente: toda descripción espacial de sucesos se sirve
de un cuerpo rígido al que hay que referirlos espacialmente. Esa referencia presupone
que los «segmentos» se rigen por las leyes de la Geometría euclídea, viniendo
representados físicamente por dos marcas sobre un cuerpo rígido.


\chapter{Espacio y Tiempo en la Mecánica Clásica}

Si formulo el objetivo de la Mecánica diciendo que «la Mecánica debe describir
cómo varía con el tiempo la posición de los cuerpos en el espacio», sin añadir grandes
reservas y prolijas explicaciones, cargaría sobre mi conciencia algunos pecados
capitales contra el sagrado espíritu de la claridad. Indiquemos antes que nada estos
pecados.

No está claro qué debe entenderse aquí por «posición» y «espacio». Supongamos que
estoy asomado a la ventanilla de un vagón de ferrocarril que lleva una marcha
uniforme, y dejo caer una piedra a la vía, sin darle ningún impulso. Entonces veo
(prescindiendo de la influencia de la resistencia del aire) que la piedra cae en línea
recta. Un peatón que asista a la fechoría desde el terraplén observa que la piedra cae a
tierra según un arco de parábola. Yo pregunto ahora: las «posiciones» que recorre la
piedra ¿están «realmente» sobre una recta o sobre una parábola? Por otro lado, ¿qué
significa aquí movimiento en el «espacio»? La respuesta es evidente después de lo
dicho en §2. Dejemos de momento a un Lado la oscura palabra «espacio», que, para ser
sinceros, no nos dice absolutamente nada; en lugar de ella ponemos «movimiento
respecto a un cuerpo de referencia prácticamente rígido». Las posiciones con relación al
cuerpo de referencia (vagón del tren o vías) han sido ya definidas explícitamente en el
epígrafe anterior. Introduciendo en lugar de «cuerpo de referencia» el concepto de
«sistema de coordenadas», que es útil para la descripción matemática, podemos decir:
la piedra describe, con relación a un sistema de coordenadas rígidamente unido al
vagón, una recta; con relación a un sistema de coordenadas rígidamente ligado a las
vías, una parábola. En este ejemplo se ve claramente que en rigor no existe una
trayectoria\footnote{Es decir, una curva a lo largo de la cual se mueve el cuerpo.},
sino sólo una trayectoria con relación a un cuerpo de referencia 
determinado.
   
Ahora bien, la descripción \textit{completa} del movimiento no se obtiene sino al especificar
cómo varía la posición del cuerpo \textit{con el tiempo}, o lo que es lo mismo, para cada
punto de la trayectoria hay que indicar en qué momento se encuentra allí el cuerpo.
Estos datos hay que completarlos con una definición del tiempo en virtud de la cual
podamos considerar estos valores temporales como magnitudes esencialmente
observables (resultados de mediciones). Nosotros, sobre el suelo de la Mecánica
clásica, satisfacemos esta condición —con relación al ejemplo anterior— de la siguiente
manera. Imaginemos dos relojes exactamente iguales; uno de ellos lo tiene el hombre
en la ventanilla del vagón de tren; el otro, el hombre que está de pie en el terraplén.
Cada uno de ellos verifica en qué lugar del correspondiente cuerpo de referencia se
encuentra la piedra en cada instante marcado por el reloj que tiene en la mano.
Nos abstenemos de entrar aquí en la imprecisión introducida por el carácter finito de
la velocidad de propagación de la luz. Sobre este extremo, y sobre una segunda
dificultad que se presenta aquí, hablaremos detenidamente más adelante.


\chapter{El Sistema de Coordenadas de Galileo}

Como es sabido, la ley fundamental de la Mecánica de Galileo y Newton, conocida
por la ley de inercia, dice: un cuerpo suficientemente alejado de otros cuerpos
persiste en su estado de reposo o de movimiento rectilíneo uniforme. Este principio
se pronuncia no sólo sobre el movimiento de los cuerpos, sino también sobre qué
cuerpos de referencia o sistemas de coordenadas son permisibles en la Mecánica y
pueden utilizarse en las descripciones mecánicas. Algunos de los cuerpos a los que sin
duda cabe aplicar con gran aproximación la ley de inercia son las estrellas fijas. Ahora
bien, si utilizamos un sistema de coordenadas solidario con la Tierra, cada estrella fija
describe, con relación a él y a lo largo de un día (astronómico), una circunferencia de
radio enorme, en contradicción con el enunciado de la ley de inercia. Así pues, si uno
se atiene a esta ley, entonces los movimientos sólo cabe referirlos a sistemas de
coordenadas con relación a los cuales las estrellas fijas no ejecutan movimientos
circulares. Un sistema de coordenadas cuyo estado de movimiento es tal que con
relación a él es válida la ley de inercia lo llamamos sistema de coordenadas de
Galileo. Las leyes de la Mecánica de Galileo-Newton sólo tienen validez para
sistemas de coordenadas de Galileo.

\chapter{El Principio de la Relatividad (En Sentido Restringido)}

Para conseguir la mayor claridad posible, volvamos al ejemplo del vagón de tren que
lleva una marcha «uniforme». Su movimiento decimos que es una traslación uniforme
(uniforme, porque es de velocidad y dirección constantes; traslación, porque
aunque la posición del vagón varía con respecto a la vía, no ejecuta ningún giro).
Supongamos que por los aires vuela un cuervo en línea recta y uniformemente
(respecto a la vía). No hay duda de que el movimiento del cuervo es --respecto al
vagón en marcha-- un movimiento de distinta velocidad y diferente dirección, pero
sigue siendo rectilneo y uniforme. Expresado de modo abstracto: si una masa $m$ se
mueve en línea recta y uniformemente respecto a un sistema de coordenadas $K$,
entonces tambin se mueve en línea recta y uniformemente respecto a un segundo
sistema de coordenadas $K'$ siempre que éste ejecute respecto a $K$ un movimiento de
traslación uniforme. Teniendo en cuenta lo dicho en el párrafo anterior, se desprende
de aquí lo siguiente:

Si $K$ es un sistema de coordenadas de Galileo, entonces también lo es cualquier otro
sistema de coordenadas $K'$ que respecto a $K$, se halle en un estado de traslación
uniforme. Las leyes de la Mecánica de Galileo-Newton valen tanto respecto a
$K'$ según idénticas leyes generales que con respecto a $K$. Esta proposición es lo
que llamaremos el principio de relatividad (en sentido restringido).

Demos un paso más en la generalización y enunciemos el siguiente principio: Si $K'$,
es un sistema de coordenadas que se mueve uniformemente y sin rotación respecto a $K$
entonces los fenómenos naturales transcurren con respecto a $K'$ según idénticas leyes
generales que con respecto a $K$. Esta proposición es lo que llamaremos el \emph{principio de
relatividad} (en sentido restringido).

Mientras se mantuvo la creencia de que todos los fenómenos naturales se podían
representar con ayuda de la Mecánica clásica, no se podía dudar de la validez de este
principio de relatividad. Sin embargo, los recientes adelantos de la Electrodinámica y
de la óptica hicieron ver cada vez más claramente que la Mecánica clásica, como base
de toda descripción física de la naturaleza, no era suficiente. La cuestión de la validez
del principio de relatividad se tornó así perfectamente discutible, sin excluir la
posibilidad de que la solución fuese en sentido negativo. Existen, con todo, dos
hechos generales que de entrada hablan muy a favor de la validez del principio de
relatividad. 

En efecto, aunque la mecánica clásica no proporciona una base
suficientemente ancha para representar teóricamente \textit{todos} los fenómenos físicos,
tiene que poseer un contenido de verdad muy importante, pues da con admirable
precisión los movimientos reales de los cuerpos celestes. De ahí que en el campo de la
\textit{Mecánica} tenga que ser válido con gran exactitud el principio de relatividad. Y que un
principio de generalidad tan grande y que es válido, con tanta exactitud, en un
determinado campo de fenmenos fracase en otro campo es, a priori, poco
probable.

El segundo argumento, sobre el que volveremos más adelante, es el siguiente. Si el
principio de relatividad (en sentido restringido) no es válido, entonces los sistemas de
coordenadas de Galileo $K$, $K'$, $K''$, etc., que se mueven uniformemente unos respecto a 
los otros, no serán \textit{equivalentes} para la descripción de los fenómenos naturales. En ese
caso no tendremos más remedio que pensar que las leyes de la naturaleza sólo pueden
formularse con especial sencillez y naturalidad si de entre todos los sistemas de
coordenadas de Galileo eligiésemos como cuerpo de referencia \textit{uno} $K_{0}$ 
que tuviera un estado de movimiento determinado. A éste lo calificaramos, y con razón (por sus
ventajas para la descripción de la naturaleza), de absolutamente en reposo, mientras
que de los demás sistemas galileanos $K$ diríamos que son «móviles». Si la vía fuese el
sistema $K_{0}$ pongamos por caso, entonces nuestro vagón de ferrocarril sera un 
sistema $K$, respecto al cual regiran leyes menos sencillas que respecto a $K_{0}$.
Esta menor simplicidad habra que atribuirla a que el vagón $K$ se mueve respecto a
$K_{0}$ (es decir, «realmente»). En estas leyes generales de la naturaleza formuladas respecto a
$K$,tendran que desempeñar un papel el módulo y la dirección de la velocidad del vagón.
Sera de esperar, por ejemplo, que el tono de un tubo de órgano fuese distinto
cuando su eje fuese paralelo a la dirección de marcha que cuando estuviese
perpendicular.

Ahora bien, la Tierra, debido a su movimiento orbital alrededor del
Sol, es equiparable a un vagón que viajara a unos 30 km por segundo. Por
consiguiente, caso de no ser válido el principio de relatividad, sería de esperar que la
dirección instantnea del movimiento terrestre interviniera en las leyes de la
naturaleza y que, por lo tanto, el comportamiento de los sistemas físicos dependiera de
su orientacin espacial respecto a la Tierra; porque, como la velocidad del movimiento
de rotación terrestre varía de dirección en el transcurso del año, la Tierra no puede
estar todo el año en reposo respecto al hipotético sistema $K_{0}$. Pese al esmero que se ha
puesto en detectar una tal anisotropía del espacio físico terrestre, es decir, una no
equivalencia de las distintas direcciones, jamás ha podido ser observada. Lo cual es un
argumento de peso a favor del principio de la relatividad.

\chapter{El Teorema de Adición de Velocidades Según la Mecánica Clásica}

Supongamos que nuestro tan trado y llevado vagón de ferrocarril viaja con
velocidad constante $v$, por la línea, e imaginemos que por su interior camina un
hombre en la dirección de marcha con velocidad $w$. ¿Con qué velocidad $W$ avanza el
hombre respecto a la vía al caminar? La única respuesta posible parece desprenderse
de la siguiente consideración:
Si el hombre se quedara parado durante un segundo, avanzara, respecto a la vía, un
trecho $v$ igual a la velocidad del vagón. Pero en ese segundo recorre además, respecto
al vagón, y por tanto también respecto a la vía, un trecho $w$ igual a la velocidad con
que camina. Por consiguiente, en ese segundo \textit{avanza} en total el trecho
$W=v+w$ respecto a la vía. Más adelante veremos que este razonamiento, que expresa el teorema
de adición de velocidades según la Mecánica clásica, es insostenible y que la ley que
acabamos de escribir no es válida en realidad. Pero entre tanto edificaremos sobre su
validez.

\chapter{La Aparente Incompatibilidad de la Ley de Propagación de la Luz con
el Principio de la Relatividad}

Apenas hay en la física una ley más sencilla que la de propagacin de la luz en el
espacio vacío. Cualquier escolar sabe (o cree saber) que esta propagación se produce
en línea recta con una velocidad de $c=300,000$ km./s. En cualquier caso, sabemos
con gran exactitud que esta velocidad es la misma para todos los colores, porque si no
fuera así, el mínimo de emisión en el eclipse de una estrella fija por su compañera
oscura no se observaría simultáneamente para los diversos colores. A través de un
razonamiento similar, relativo a observaciones de las estrellas dobles, el astrónomo
holandés De Sitter consiguió también demostrar que la velocidad de propagación de la
luz no puede depender de la velocidad del movimiento del cuerpo emisor. La hipótesis
de que esta velocidad de propagación depende de la dirección en el espacio es de
suyo improbable.

Supongamos, en resumen, que el escolar cree justificadamente en la sencilla ley de la
constancia de la velocidad de la luz $c$ (en el vaco). Quin diría que esta ley tan simple
ha sumido a los físicos más concienzudos en grandsimas dificultades conceptuales? Los
problemas surgen del modo siguiente.

Como es natural, el proceso de la propagación de la luz, como cualquier otro, hay que
referirlo a un cuerpo de referencia rígido (sistema de coordenadas). Volvemos a elegir
como tal las vías del tren e imaginamos que el aire que había por encima de ellas lo
hemos eliminado por bombeo. Supongamos que a lo largo del terraplén se emite un
rayo de luz cuyo vértice, según lo anterior, se propaga con la velocidad $c$ respecto a
aquél. Nuestro vagón de ferrocarril sigue viajando con la velocidad
$v$,en la misma dirección en que se propaga el rayo de luz, pero naturalmente mucho más despacio.
Lo que nos interesa averiguar es la velocidad de propagación del rayo de luz respecto
al vagón. Es fácil ver que el razonamiento del epígrafe anterior tiene aquí aplicación,
pues el hombre que corre con respecto al vagón desempeña el papel del rayo de luz.
En lugar de su velocidad $w$ respecto al terraplén aparece aquí la velocidad de la luz
respecto a éste; la velocidad $w$ que buscamos, la de la luz respecto al vagón, es por
tanto igual a:

\[w=c-v.\]

Así pues, la velocidad de propagación del rayo de luz respecto al vagón resulta ser
menor que $c$.

Ahora bien, este resultado atenta contra el principio de la relatividad expuesto en \S 5,
porque, según este principio, la ley de propagación de la luz en el vacío, como
cualquier otra ley general de la naturaleza, debera ser la misma si tomamos el vagón
como cuerpo de referencia que si elegimos las vías, lo cual parece imposible según
nuestro razonamiento. Si cualquier rayo de luz se propaga respecto al terraplén con la
velocidad $c$, la ley de propagación respecto al vagón parece que tiene que ser, por eso
mismo, otra distinta... en contradicción con el principio de relatividad.

A la vista del dilema parece ineludible abandonar, o bien el principio de relatividad,
o bien la sencilla ley de la propagación de la luz en el vacío. El lector que haya seguido
atentamente las consideraciones anteriores esperará seguramente que sea el principio
de relatividad --que por su naturalidad y sencillez se impone a la mente como algo casi
ineludible-- el que se mantenga en pie, sustituyendo en cambio la ley de la propagación
de la luz en el vacío por una ley más complicada y compatible con el principio de
relatividad. Sin embargo, la evolución de la física teórica demostró que este camino
era impracticable. Las innovadoras investigaciones teóricas de H. A. Lorentz sobre los
procesos electrodinámicos y ópticos en cuerpos móviles demostraron que las
experiencias en estos campos conducen con necesidad imperiosa a una teoría de los
procesos electromagnéticos que tiene como consecuencia irrefutable la ley de la
constancia de la luz en el vacío. Por eso, los teóricos de vanguardia se inclinaron más
bien por prescindir del principio de relatividad, pese a no poder hallar ni un solo hecho
experimental que lo contradijera.

Aquí es donde entrá la teoría de la relatividad. Mediante un análisis de los
conceptos de espacio y tiempo se vio que en realidad \textit{no existía ninguna 
incompatibilidad entre el principio de la relatividad y la ley de propagación de la luz}, sino que,
atenindose uno sistemáticamente a estas dos leyes, se llegaba a una teoría lógicamente
impecable. Esta teoría, que para diferenciarla de su ampliación (comentada más
adelante) llamamos teora de la relatividad especial, es la que expondremos a
continuación en sus ideas fundamentales.

\chapter{Sobre el Concepto de Tiempo en la Física}

Un rayo ha caído en dos lugares muy distantes $A$ y $B$ de la vía. Yo añado la
afirmación de que ambos impactos han ocurrido \textit{simultáneamente}. Si ahora te
pregunto, querido lector, si esta afirmación tiene o no sentido, me contestarás con un
«sí» contundente. Pero si luego te importuno con el ruego de que me expliques con
más precisión ese sentido, advertirás tras cierta reflexión que la respuesta no es
tan sencilla como parece a primera vista.

Al cabo de algún tiempo quizá te acuda a la mente la siguiente respuesta: «El
significado de la afirmación es claro de por sí y no necesita de ninguna aclaración; sin
embargo, tendra que reflexionar un poco si se me exige determinar, mediante
observaciones, si en un caso concreto los dos sucesos son o no simultáneos». Pero
con esta respuesta no puedo darme por satisfecho, por la siguiente razón.
Suponiendo que un experto meteorlogo hubiese hallado, mediante agudísimos
razonamientos, que el rayo tiene que caer siempre simultáneamente en los lugares $A$ y
$B$, se planteara el problema de comprobar si ese resultado teórico se corresponde o
no con la realidad. Algo análogo ocurre en todas las proposiciones físicas en las que
interviene el concepto de simultáneo. Para el físico no existe el concepto mientras
no se brinde la posibilidad de averiguar en un caso concreto si es verdadero o no.
Hace falta, por tanto, una definición de simultaneidad que proporcione el método
para decidir experimentalmente en el caso presente si los dos rayos han
caído simultáneamente o no. Mientras no se cumpla este requisito, me estaré 
entregando como físico (¡y también como no físico!) a la ilusión de creer que
puedo dar sentido a esa afirmación de la simultaneidad. (No sigas leyendo,
querido lector, hasta concederme esto plenamente convencido.)

Tras algún tiempo de reflexión haces la siguiente propuesta para constatar la
simultaneidad. Se mide el segmento de unión $\overline{AB}$ a lo largo de la vía y se coloca en
su punto medio $M$ a un observador provisto de un dispositivo (dos espejos formando
$90^{\circ}$ entre sí, por ejemplo) que le permite la visualización óptica simultánea de ambos
lugares $A$ y $B$. Si el observador percibe los dos rayos simultáneamente, entonces es
que son simultáneos.

Aunque la propuesta me satisface mucho, sigo pensando que la cuestión no queda
aclarada del todo, pues me siento empujado a hacer la siguiente objeción: 

Tu definición sera necesariamente correcta si yo supiese ya que la luz que la
percepción de los rayos transmite al observador en $M$ se propaga con la misma
velocidad en el segmento $A~\longrightarrow~M$ que en el segmento $B~\longrightarrow~M$.
Sin embargo, la comprobación de este supuesto sólo sera posible si se dispusiera ya
de los medios para la medición de tiempos. Parece, pues, que nos movemos en un
círculo lógico.

Después de reflexionar otra vez, me lanzas con toda razón una mirada algo
despectiva y me dices: A pesar de todo, mantengo mi definición anterior, porque en
realidad no presupone nada sobre la luz. A la definición de simultaneidad solamente
hay que imponerle una condición, y es que en cualquier caso real permita tomar \textit{una}
decisión empírica acerca de la pertinencia o no pertinencia del concepto a definir. Que
mi definición cubre este objetivo es innegable. Que la luz tarda el mismo tiempo en
recorrer el camino $A~\longrightarrow~M$ que el $~B\longrightarrow~M$ no es en 
realidad ningn \textit{supuesto previo ni hipótesis} sobre la naturaleza 
física de la luz, sino una \textit{estipulación} que puedo hacer a
discreción para llegar a una definición de simultaneidad.

Está claro que esta definición se puede utilizar para dar sentido exacto al enunciado
de simultaneidad, no sólo de dos sucesos, sino de un número arbitrario de ellos, sea
cual fuere su posición con respecto al cuerpo de referencia\footnote{ Suponemos además
que cuando ocurren tres fenmenos $A$, $B$, $C$ en lugares distintos y $A$ es simultáneo
a $B$ y $B$ simultáneo a $C$ (en el sentido de la definición anterior), entonces se 
cumple también el criterio de simultaneidad para la pareja de sucesos $A-C$. Este supuesto
es una hipótesis física sobre la ley de propagación de la luz; tiene que cumplirse 
necesariamente para poder mantener en pie la ley de la constancia de la velocidad de
la luz en el vacío.}. Con ello se llega también a una definición del tiempo en la Física.
Imaginemos, en efecto, que en los puntos $A, B, C$ de la vía (sistema de coordenadas)
existen relojes de idéntica constitución y dispuestos 
de tal manera que las posiciones de las manillas sean simultáneamente (en el
sentido anterior) las mismas. Se entiende entonces por tiempo de un suceso la hora
(posición de las manillas) marcada por aquel de esos relojes que está inmediatamente
contiguo (espacialmente) al suceso. De este modo se le asigna a cada suceso un valor
temporal que es esencialmente observable.

Esta definición entraña otra hipótesis física de cuya validez, en ausencia de razones
empricas en contra, no se podrá dudar. En efecto, se supone que todos los relojes
marchan igual de rápido si tienen la misma constitución. Formulándolo exactamente:
si dos relojes colocados en reposo en distintos lugares del cuerpo de referencia son
puestos en hora de tal manera que la posición de las manillas del uno sea \textit{simultnea}
(en el sentido anterior) a la \textit{misma} posición de las manillas del otro, entonces
posiciones iguales de las manillas son en general simultáneas (en el sentido de la
definición anterior).


\chapter{La Relatividad de la Simultaneidad}

Hasta ahora hemos referido nuestros razonamientos a un determinado cuerpo de
referencia que hemos llamado «terraplén» o «vías». Supongamos que por los carriles
viaja un tren muy largo, con velocidad constante $v$ y en la dirección señalada en la Fig.
\ref{fig:1}. Las personas que viajan en este tren hallarán ventajoso utilizar el tren como cuerpo
de referencia rígido (sistema de coordenadas) y referirán todos los sucesos al tren.
Todo suceso que se produce a lo largo de la vía, se produce también en un punto
determinado del tren. Incluso la definición de simultaneidad se puede dar
exactamente igual con respecto al tren que respecto a las vías. Sin embargo, se
plantea ahora la siguiente cuestión:

%                       Fig. 01:
%   v --->         M' ----->      v --->   Tren
%   -------|------------|------------|----/
% ---------|------------|------------|-------
%          A            M            B   Vías
%
\begin{figure}[hbtp]
 \centering

\caption{}


\label{fig:1}

\begin{picture}(250,100)(0,0) \thicklines \put(0,40){\line(1,0){250}}
\put(230,30){Vías} \put(15,50){\line(1,0){210}} \put(225,50){\line(1,1){10}}
\put(235,60){Tren} \put(40,35){\line(0,1){20}} \put(37,23){A}
\put(125,48){\line(0,1){7}} \put(125,35){\line(0,1){7}}
\put(121,23){M} \put(210,35){\line(0,1){20}} \put(207,23){B}
\thinlines \put(15,60){$v$} \put(22,62){\vector(1,0){25}}
\put(195,60){$v$} \put(202,62){\vector(1,0){25}} \put(105,65){M$'$}
\put(122,67){\vector(1,0){25}} \end{picture} 
\end{figure}

Dos sucesos (p. ej., los dos rayos A y B) que son simultáneos \textit{respecto al terraplén},
son también simultáneos \textit{respecto al tren}? En seguida demostraremos que la respuesta
tiene que ser negativa.

Cuando decimos que los rayos A y B son simultáneos respecto a las vías, queremos
decir: los rayos de luz que salen de los lugares A y B se reúnen en el punto medio
M del tramo de vía A $\longrightarrow$ B Ahora bien, los sucesos A y B se corresponden también con
lugares A y B en el tren. Sea M$'$ el punto medio del segmento A $\longrightarrow$ B
del tren en marcha. Este punto M$'$ es cierto que en el instante de la caída de los rayos coincide
con el punto M, pero, como se indica en la figura, se mueve hacia la derecha con la
velocidad $v$ del tren. Un observador que estuviera sentado en el tren en M$'$ pero que no
poseyera esta velocidad, permanecería constantemente en M, y los rayos de luz que
parten de las chispas A y B lo alcanzarían simultneamente, es decir, estos dos rayos
de luz se reuniran precisamente en él. La realidad es, sin embargo, que (juzgando la
situación desde el terraplén) este observador va al encuentro del rayo de luz que
viene de B, huyendo en cambio del que avanza desde A. Por consiguiente, verá antes
la luz que sale de B que la que sale de A. En resumidas cuentas, los observadores
que utilizan el tren como cuerpo de referencia tienen que llegar a la conclusión de
que la chispa elctrica B ha caído antes que la A. Llegamos así a un resultado
importante:

Sucesos que son simultáneos respecto al terraplén no lo son respecto al tren, y
viceversa (relatividad de la simultaneidad). Cada cuerpo de referencia (sistema de
coordenadas) tiene su tiempo especial; una localización temporal tiene sólo sentido
cuando se indica el cuerpo de referencia al que remite.

Antes de la teoría de la relatividad, la Física suponía siempre implícitamente que el
significado de los datos temporales era absoluto, es decir, independiente del estado de
movimiento del cuerpo de referencia. Pero acabamos de ver que este supuesto es
incompatible con la definición natural de simultaneidad; si prescindimos de él,
desaparece el conflicto, expuesto en \S 7, entre la ley de la propagación de la luz y el
principio de la relatividad.

En efecto, el conflicto proviene del razonamiento del epígrafe 6, que ahora resulta
insostenible. Inferimos allí que el hombre que camina por el vagón y recorre el trecho $w$
\textit{en un segundo} recorre ese mismo trecho también en \textit{un segundo}
respecto a las vías. Ahora bien, toda vez que, en virtud de las reflexiones anteriores, el tiempo que
necesita un proceso con respecto al vagón no cabe igualarlo a la duración del mismo
proceso juzgada desde el cuerpo de referencia del terraplén, tampoco se puede
afirmar que el hombre, al caminar respecto a las vías, recorra el trecho $w$ 
en un tiempo que --juzgado desde el terraplén-- es igual a un segundo. 

Digamos de paso que el razonamiento de \S 6 descansa además en un segundo supuesto que, a la luz de
una reflexión rigurosa, se revela arbitrario, lo cual no quita para que, antes de
establecerse la teoría de la relatividad, fuese aceptado siempre (de modo implícito).

\chapter{Sobre la Relatividad del Concepto de Distancia Espacial}

Observamos dos lugares concretos del tren\footnote{El centro de los vagones 
primero y centésimo, por ejemplo.} que viaja con velocidad $v$, por la
línea y nos preguntamos qué distancia hay entre ellos. Sabemos ya que para medir
una distancia se necesita un cuerpo de referencia respecto al cual hacerlo. Lo más
sencillo es utilizar el propio tren como cuerpo de referencia (sistema de coordenadas).
Un observador que viaja en el tren mide la distancia, transportando en línea recta una
regla sobre el suelo de los vagones, por ejemplo, hasta llegar desde uno de los puntos
marcados al otro. El número que indica cuántas veces transportó la regla es entonces la
distancia buscada.

Otra cosa es si se quiere medir la distancia desde la vía. Aquí se ofrece el método
siguiente. Sean A$'$ y B$'$ los dos puntos del tren de cuya distancia se trata; estos
dos puntos se mueven con velocidad $v$ a lo largo de la vía. Preguntámonos primero
por los puntos A y B de la vía por donde pasan A$'$ y B$'$ en un momento determinado $t$
(juzgado desde la vía). En virtud de la definición de tiempo dada en \S 8, estos puntos
A y B de la vía son determinables. A continuación se mide la distancia entre A y B
transportando repetidamente el metro a lo largo de la vía.

A priori no está dicho que esta segunda medición tenga que proporcionar el mismo
resultado que la primera. La longitud del tren, medida desde la vía, puede ser distinta
que medida desde el propio tren. Esta circunstancia se traduce en una segunda
objeción que oponer al razonamiento, aparentemente tan meridiano, de \S 6. Pues si el
hombre en el vagón recorre en una unidad de tiempo el trecho $w$ \textit{medido desde el tren}, 
este trecho \textit{medido desde la vía}, no tiene por qué ser igual a $w$.


\chapter{La Transformación de Lorentz}

Las consideraciones hechas en los tres últimos epígrafes nos muestran que la aparente
incompatibilidad de la ley de propagación de la luz con el principio de relatividad en
\S 7 está deducida a través de un razonamiento que tomaba a préstamo de la Mecánica
clásica dos hipótesis injustificadas; estas hipótesis son:

\begin{enumerate}
\item El intervalo temporal entre dos sucesos es independiente del estado de
movimiento del cuerpo de referencia.
\item El intervalo espacial entre dos puntos de un cuerpo rgido es
independiente del estado de movimiento del cuerpo de referencia.
\end{enumerate}

Si eliminamos estas dos hipótesis, desaparece el dilema de \S 7, porque el teorema de
adición de velocidades deducido en \S 6 pierde su validez. Ante nosotros surge la
posibilidad de que la ley de la propagación de la luz en el vacío sea compatible con el
principio de relatividad. Llegamos así a la pregunta: ¿cómo hay que modificar el
razonamiento de \S 6 para eliminar la aparente contradicción entre estos dos resultados
fundamentales de la experiencia? Esta cuestión conduce a otra de índole general. En
el razonamiento de \S 6 aparecen lugares y tiempos con relación al tren y con relación a
las vías. ¿Cómo se hallan el lugar y el tiempo de un suceso con relación al tren cuando
se conocen el lugar y el tiempo del suceso con respecto a las vías? ¿Esta pregunta
tiene alguna respuesta de acuerdo con la cual la ley de la propagación en el vaco no
contradiga al principio de relatividad? ¿O expresado de otro modo: cabe hallar alguna
relación entre las posiciones y tiempos de los distintos sucesos con relación a ambos
cuerpos de referencia, de manera que todo rayo de luz tenga la velocidad de
propagación $c$ respecto a las vías y respecto al tren? Esta pregunta conduce a una respuesta
muy determinada y afirmativa, a una ley de transformación muy precisa para las
magnitudes espacio-temporales de un suceso al pasar de un cuerpo de referencia a
otro.

% Figura 2

%          z'
%          | ---> 
%  z       |    y'  
%  |       |   / --->
%  |    y  |  /  v
%  |   /   | / --->
%  |  /    |/______________x'
%  | /     K'
%  |/______________x
%  K

\begin{figure}[hbtp]
 \centering

\caption{}


\label{fig:2}

\begin{picture}(200,220)(0,0) \thicklines \put(15,10){$K$} \put(20,20){\line(1,0){125}}
\put(149,17){$x$} \put(20,20){\line(0,1){125}} \put(17,150){$z$}
\put(20,20){\line(1,2){40}} \put(55,105){$y$}

\put(85,25){$K'$} \put(90,35){\line(1,0){125}} \put(219,32){$x'$}
\put(90,35){\line(0,1){125}} \put(87,165){$z'$} \put(90,35){\line(1,2){40}}
\put(125,120){$y'$}

\thinlines \put(95,155){\vector(1,0){35}} \put(135,110){\vector(1,0){35}}
\put(110,40){\vector(1,0){35}} \end{picture} 
\end{figure}

Antes de entrar en ello, intercalemos la siguiente consideración. Hasta ahora
solamente hemos hablado de sucesos que se producían a lo largo de la vía, la cual
desempeñaba la función matemática de una recta. Pero, siguiendo lo indicado en el
epígrafe 2, cabe imaginar que este cuerpo de referencia se prolonga hacia los lados y
hacia arriba por medio de un andamiaje de varillas, de manera que cualquier suceso,
ocurra donde ocurra, puede localizarse respecto a ese andamiaje. Análogamente, es
posible imaginar que el tren que viaja con velocidad $v$ 
se prolonga por todo el espacio, de manera que cualquier suceso, por lejano que esté, también pueda
localizarse respecto al segundo andamio. Sin incurrir en defecto teórico, podemos
prescindir del hecho de que en realidad esos andamios se destrozaran uno contra el
otro debido a la impenetrabilidad de los cuerpos sólidos. En cada uno de estos
andamios imaginamos que se erigen tres paredes mutuamente perpendiculares que
denominamos «planos coordenados» («sistema de coordenadas»). Al terraplén le 
corresponde entonces un sistema de coordenadas $K$, y al tren otro $K'$ Cualquier
suceso, dondequiera que ocurra, viene fijado espacialmente respecto a $K$ por las tres
perpendiculares $x$, $y$, $z$ a los planos coordenados, y temporalmente por un valor $t$. \textit{Ese mismo suceso}
viene fijado espacio-temporalmente respecto a $K'$, por valores
correspondientes $x'$, $y'$, $z'$, $t'$, que, como es natural, no coinciden con $x$, $y$, $z$, $t$.
Ya explicamos antes con detalle cómo interpretar estas magnitudes como resultados de
mediciones físicas.

Es evidente que el problema que tenemos planteado se puede formular exactamente
de la manera siguiente: Dadas las cantidades $x$, $y$, $z$, $t$, de un suceso respecto a $K$,
cules son los valores $x'$, $y'$, $z'$, $t'$ del mismo suceso respecto a $K'$ Las relaciones hay
que elegirlas de tal modo que satisfagan la ley de propagación de la luz en el vacío
para uno y el mismo rayo de luz (y además para cualquier rayo de luz) respecto $K$ y $K'$.
Para la orientación espacial relativa indicada en el dibujo de la Figura \ref{fig:2}, el
problema queda resuelto por las ecuaciones:

\begin{eqnarray*}
x' & = & \frac{x-vt}{\sqrt{I-\frac{v^{2}}{c^{2}}}}\\
y' & = & y\\
z' & = & z\\
t' & = & \frac{t-\frac{v}{c^{2}}x}{\sqrt{I-\frac{v^{2}}{c^{2}}}}
\end{eqnarray*}

\noindent   Este sistema de ecuaciones se designa con el nombre de «transformación de
Lorentz».\footnote{En el Apéndice se da una derivación sencilla de la transformación de Lorentz}

Ahora bien, si en lugar de la ley de propagación de la luz hubiésemos tomado como
base los supuestos implícitos en la vieja mecánica, relativos al carcter absoluto de los
tiempos y las longitudes, en vez de las anteriores ecuaciones de transformación
habríamos obtenido estas otras:

\begin{eqnarray*}
x' & = & x-vt\\
y' & = & y\\
z' & = & z\\
t' & = & t
\end{eqnarray*}

\noindent sistema que a menudo se denomina «transformación de Galileo». La transformación de
Galileo se obtiene de la de Lorentz igualando en ésta la velocidad de la luz
$c$ a un valor infinitamente grande.

El siguiente ejemplo muestra claramente que, según la transformación de Lorentz,
la ley de propagación de la luz en el vacío se cumple tanto respecto al cuerpo de
referencia $K$ como respecto al cuerpo de referencia $K'$. Supongamos que se envía una
señal luminosa a lo largo del eje $x$ positivo, propagndose la excitación luminosa según
la ecuación

\[x=ct,\]

\noindent es decir, con velocidad $c$. De acuerdo con las ecuaciones de la transformación de
Lorentz, esta sencilla relación entre $x$ y $t$ determina una relacin entre  $x'$ y $t'$. En efecto,
sustituyendo $x$ por el valor $ct$ en las ecuaciones primera y cuarta de la transformación
de Lorentz obtenemos:

\begin{eqnarray*}
x' & = & \frac{(c-v)t}{\sqrt{I-\frac{v^{2}}{c^{2}}}}\\
t' & = & \frac{(1-\frac{v}{c})t}{\sqrt{I-\frac{v^{2}}{c^{2}}}}
\end{eqnarray*}

\noindent de donde, por división, resulta inmediatamente

\[x'=ct'\]

\noindent La propagación de la luz, referida al sistema $K'$, se produce según esta ecuación.
Se comprueba, por tanto, que la velocidad de propagación \textit{es} tambin igual a
$c$. respecto al cuerpo de referencia $K'$; y análogamente para rayos de luz que se
propaguen en cualquier otra dirección. Lo cual, naturalmente, no es de extrañar,
porque las ecuaciones de la transformación de Lorentz están derivadas con este
criterio.


\chapter{El Comportamiento de Reglas y Relojes Móviles}

Coloco una regla de un metro sobre el eje $x'$ de $K'$ de manera que un extremo
coincida con el punto $x'=0$ y el otro con el punto $x'=1$. ¿Cuál es la longitud de la
regla respecto al sistema $K$? Para averiguarlo podemos determinar las posiciones de
ambos extremos respecto a $K$ en un momento determinado $t$ De la primera ecuación
de la transformación de Lorentz, para $t=0$ se obtiene para estos dos puntos:

\begin{eqnarray*}
x_{\mbox{(origen de la escala)}} & = & 0\overline{\sqrt{I-\frac{v^{2}}{c^{2}}}}\\
x_{\mbox{(extremo de la escala)}} & = & 1\overline{\sqrt{I-\frac{v^{2}}{c^{2}}}}
\end{eqnarray*}
 ~

\noindent estos dos puntos distan entre sí $\sqrt{I-v^{2}/c^{2}}$.

Ahora bien, el metro se mueve respecto a $K$ con velocidad $v$. de donde se
deduce que la longitud de una regla rígida de un metro que se mueve con velocidad
$v$ en el sentido de su longitud es de $\sqrt{I-v^{2}/c^{2}}$ metros.

La regla rígida en movimiento es más corta que la misma regla cuando está en estado de reposo, y es
tanto más corta cuando más rápidamente se mueva. Para la velocidad $v=c$ 
seria $\sqrt{I-v^{2}/c^{2}}=0$, para velocidades aún mayores la raíz se hara imaginaria. De aquí
inferimos que en la teoría de la relatividad la velocidad $c$ desempeña
el papel de una velocidad límite que no puede alcanzar ni sobrepasar
ningún cuerpo real.

Añadamos que este papel de la velocidad $c$ como velocidad límite se sigue de las
propias ecuaciones de la transformación de Lorentz, porque éstas pierden todo sentido
cuando $v$ se elige mayor que $c$.

Si hubisemos procedido a la inversa, considerando un metro que se halla en reposo
respecto a $K$ sobre el eje $x$, habríamos comprobado que en relación a $K'$ tiene la
longitud de $\sqrt{I-v^{2}/c^{2}}$; Lo cual está totalmente de acuerdo con el principio de la relatividad, en el cual hemos
basado nuestras consideraciones.

A priori es evidente que las ecuaciones de transformación tienen algo que decir
sobre el comportamiento físico de reglas y relojes, porque las cantidades $x$, $y$, $z$, $t$ no son
otra cosa que resultados de medidas obtenidas con relojes y reglas. Si hubisemos
tomado como base la transformación de Galileo, no habríamos obtenido un
acortamiento de longitudes como consecuencia del movimiento.

Ímaginemos ahora un reloj con segundero que reposa constantemente en el origen
($x'=0$) de $K'$. Sean $t'=0$ y $t'=1$  dos señales sucesivas de este reloj. Para estos dos tics,
las ecuaciones primera y cuarta de la transformación de Lorentz darán:

\[t=0\]

\noindent y

\[t'=\frac{I}{\sqrt{I-\frac{v^{2}}{c^{2}}}}\]
 ~

Juzgado desde $K$, el reloj se mueve con la velocidad v; respecto a este cuerpo
de referencia, entre dos de sus seales transcurre, no un segundo, sino

\[\frac{I}{\sqrt{I-\frac{v^{2}}{c^{2}}}}\]
 ~

\noindent segundos, o sea un tiempo algo mayor.
Como consecuencia de su movimiento, el reloj marcha algo más despacio que en estado
de reposo. La velocidad de la luz $c$ desempeña, también aquí, el papel de una
velocidad límite inalcanzable.


\chapter{Teorema de Adición de velocidades. \protect \protect \\
Experimento de Fizeau}

Dado que las velocidades con que en la práctica podemos mover relojes y reglas son
pequeñas frente a la velocidad de la luz $c$, es difcil que podamos comparar los
resultados del epígrafe anterior con la realidad. Y puesto que, por otro lado, esos
resultados le parecerán al lector harto singulares, voy a extraer de la teoría otra
consecuencia que es muy fácil de deducir de lo anteriormente expuesto y que los
experimentos confirman brillantemente.
  
En \S 6 hemos deducido el teorema de adición para velocidades de la misma
dirección, tal y como resulta de las hipótesis de la Mecánica clásica. Lo mismo se
puede deducir fácilmente de la transformación de Galileo (\S 11). En lugar del hombre
que camina por el vagón introducimos un punto que se mueve respecto al sistema de
coordenadas $K'$ según la ecuación

\[x'=wt'.\]
 ~

Mediante las ecuaciones primera y cuarta de la transformacin de Galileo se pueden
expresar $x'$ y $t'$ en función de $x$ y $t$, obteniendo

\[x=(v+w)t.\]
 ~

Esta ecuación no expresa otra cosa que la ley de movimiento del punto respecto al
sistema $K$ (del hombre respecto al terraplén), velocidad que designamos por $W$, con
lo cual se obtiene, como en \S 6:

\begin{equation}
W=v+w\label{eqnA}
\end{equation}

Pero este razonamiento lo podemos efectuar igual de bien basándonos en la teoría de
la relatividad. Lo que hay que hacer entonces es expresar $x'$ y $t'$ en la ecuación

\begin{equation}
x'=wt'\label{eqnB}
\end{equation}

\noindent en función de $x$ y $t$, utilizando las ecuaciones primera y cuarta de la transformación de
Lorentz. En lugar de la ecuación \ref{eqnA} se obtiene entonces esta otra:

\[W=\frac{v+w}{I+\frac{vw}{c^{2}}}\]
 ~

\noindent que corresponde al teorema de adición de velocidades de igual dirección según la teoría
de la relatividad. La cuestión es cuál de estos dos teoremas resiste el cotejo con la
experiencia. Sobre el particular nos instruye un experimento extremadamente
importante, realizado hace más de medio siglo por el genial físico Fizeau y desde
entonces repetido por algunos de los mejores físicos experimentales, por lo cual el
resultado es irrebatible. El experimento versa sobre la siguiente cuestión. Supongamos
que la luz se propaga en un cierto líquido en reposo con una determinada velocidad $w$.
¿Con qué velocidad se propaga en el tubo $R$ (de la figura \ref{fig:3}) en la dirección de la flecha,
cuando dentro de ese tubo fluye el líquido con velocidad $v$?

% Figura 3

%                       T
%                     /
%  --------------------------------------
%        v --------->
%  --------------------------------------
% 

\begin{figure}[hbtp]
\centering
\caption{}
\label{fig:3}
\begin{picture}(200,75)(0,0) \thicklines \put(0,15){\line(1,0){200}}
\put(0,35){\line(1,0){200}} \put(100,35){\line(1,3){5}}
\put(107,52){T}
\thinlines \put(40,25){\vector(1,0){50}} \put(60,26){$v$}
\end{picture} 
\end{figure}

En cualquier caso, fieles al principio de relatividad, tendremos que sentar el supuesto
de que, \textit{respecto al líquido}, la propagación de la luz se produce siempre con la
misma velocidad $w$, muvase o no el líquido respecto a otros cuerpos. Son conocidas,
por tanto, la velocidad de la luz respecto al líquido y la velocidad de éste respecto al
tubo, y se busca la velocidad de la luz respecto al tubo.

Está claro que el problema vuelve a ser el mismo que el de \S 6. El tubo
desempeña el papel de las vías o del sistema de coordenadas $K$; el líquido, el
papel del vagón o del sistema de coordenadas $K$'; la luz, el del hombre que
camina por el vagón o el del punto móvil mencionado en este apartado. Así
pues, si llamamos $W$ a la velocidad de la luz respecto al tubo, ásta vendrá dada
por la ecuación \ref{eqnA} o por la \ref{eqnB}, según que sea la transformación 
de Galileo o la de Lorentz la que se corresponde con la realidad. El experimento\footnote{
Fizeau halló $W=w+v\left(I-\frac{I}{n^{2}}\right)$, donde $n=\frac{c}{w}$ es el 
índice de refracción del líquido. Por otro lado, debido a que $\frac{vw}{c^{2}}$
es muy pequeño frente a $I$, se puede sustituir \ref{eqnB}, por $W=(w+v)\left(I-\frac{vw}{c^{2}}\right)$,
o bien, con la misma aproximación, $w+v\left(I-\frac{I}{n^{2}}\right)$, lo cual 
concuerda con el resultado de Fizeau.} falla a favor de la ecuación \ref{eqnB} 
deducida de la teoría de la
relatividad, y además con gran exactitud. Según las últimas y excelentes mediciones de
Zeeman, la influencia de la velocidad de la corriente $v$ sobre la propagación de la luz
viene representada por la fórmula \ref{eqnB} con una exactitud superior al 1 por 100.

Hay que destacar, sin embargo, que H. A. Lorentz, mucho antes de establecerse la
teoría de la relatividad, dio ya una teoría de este fenómeno por vía puramente
electrodinámica y utilizando determinadas hipótesis sobre la estructura
electromagnética de la materia. Pero esta circunstancia no merma para nada el poder
probatorio del experimento, en tanto que \textit{experimentum crucis} a favor de la teoría de la
relatividad. Pues la Electrodinámica de Maxwell-Lorentz, sobre la cual descansaba la
teoría original, no está para nada en contradicción con la teoría de la relatividad. Esta
última ha emanado más bien de la Electrodinámica como resumen y generalización
asombrosamente sencillos de las hipótesis, antes mutuamente independientes, que servían 
de fundamento a la Electrodinámica.


\chapter{El Valor Heurístico de la Teoría de la Relatividad}

La cadena de ideas que hemos expuesto hasta aquí se puede resumir brevemente
como sigue. La experiencia ha llevado a la convicción de que, por un lado, el principio
de la relatividad (en sentido restringido) es válido, y por otro, que la velocidad de
propagación de la luz en el vacío es igual a una constante $c$. Uniendo estos dos
postulados resultá la ley de transformación para las coordenadas rectangulares $x$, $y$, $z$ y
el tiempo $t$ de los sucesos que componen los fenómenos naturales, obtenindose, no la
transformación de Galileo, sino (en discrepancia con la Mecánica clásica) la
transformación de Lorentz.

En este razonamiento desempeño un papel importante la ley de propagación de la
luz, cuya aceptación viene justificada por nuestro conocimiento actual. Ahora bien,
una vez en posesión de la transformación de Lorentz, podemos unir ésta con el
principio de relatividad y resumir la teoría en el enunciado siguiente:

Toda ley general de la naturaleza tiene que estar constituida de tal modo que se
transforme en otra ley de idéntica estructura al introducir, en lugar de las variables
espacio-temporales $x, y, z, t$ del sistema de coordenadas original $K$, nuevas variables
espacio-temporales $x', y', z', t'$ de otro sistema de coordenadas $K'$, donde la relación
matemática entre las cantidades con prima y sin prima viene dada por la
transformación de Lorentz. Formulado brevemente: \textit{«Las leyes generales de la naturaleza
son covariantes respecto a la transformación de Lorentz»}.

Esta es una condición matemática muy determinada que la teoría de la relatividad
prescribe a las leyes naturales, con lo cual se convierte en valioso auxiliar heurístico en
la búsqueda de leyes generales de la naturaleza. Si se encontrara una ley general de
la naturaleza que no cumpliera esa condición, quedara refutado por lo menos uno de
los dos supuestos fundamentales de la teoría. Veamos ahora lo que esta última ha
mostrado en cuanto a resultados generales.


\chapter{Resultados Generales de la Teoría}

De las consideraciones anteriores se echa de ver que la teoría de la relatividad 
(especial) ha nacido de la Electrodinámica y de la óptica. En estos campos no ha
modificado mucho los enunciados de la teoría, pero ha simplificado notablemente el
edificio teórico, es decir, la derivación de las leyes, y, lo que es incomparablemente
más importante, ha reducido mucho el número de hiptesis independientes sobre las
que descansa la teoría. A la teoría de Maxwell-Lorentz le ha conferido un grado tal de
evidencia, que aquélla se habra impuesto con carcter general entre los fsicos
aunque los experimentos hubiesen hablado menos convincentemente a su favor.

La Mecánica clásica precisaba de una modificación antes de poder armonizar con el
requisito de la teoría de la relatividad especial. Pero esta modificación afecta
únicamente, en esencia, a las leyes para movimientos rápidos en los que las
velocidades $v$ de la materia no sean demasiado pequeñas frente a la de la luz. Movimientos
tan rápidos sólo nos los muestra la experiencia en electrones e iones; en otros
movimientos las discrepancias respecto a las leyes de la Mecánica clásica son
demasiado pequeñas para ser detectables en la práctica. Del movimiento de los astros
no hablaremos hasta llegar a la teoría de la relatividad general. Según la teora de la
relatividad, la energía cinética de un punto material de masa $m$ no viene dado ya por la
conocida expresión

\[m\frac{v^{2}}{2}\]

\noindent sino por la expresión

\[\frac{mc^{2}}{\sqrt{I-\frac{v^{2}}{c^{2}}}}\]
 ~

Esta expresión se hace infinita cuando la velocidad $v$ se aproxima a la velocidad de la
luz $c$. Así pues, por grande que sea la energía invertida en la aceleración, la velocidad
tiene que permanecer siempre inferior a $c$. Si se desarrolla en serie la expresión de la
energía cinética, se obtiene:

\[mc^{2}+m\frac{v^{2}}{2}+\frac{3}{8}m\frac{v^{4}}{c^{2}}+\cdots\]
 ~

El tercer término es siempre pequeño frente al segundo (el único considerado en
la Mecánica clásica) cuando $v^{2}/c^{2}$ es pequeño comparado con $I$.

El primer término $mc^{2}$ no depende de la velocidad, por lo cual no entra
en consideración al tratar el problema de cómo la energía de un punto material
depende de la velocidad. Sobre su importancia teórica hablaremos más adelante. El
resultado más importante de índole general al que ha conducido la teoría de la
relatividad especial concierne al concepto de masa. La física prerrelativista conoce dos
principios de conservación de importancia fundamental, el de la conservación de la
energía y el de la conservación de la masa; estos dos principios fundamentales aparecen
completamente independientes uno de otro. La teoría de la relatividad los funde en
uno solo. A continuacin explicaremos brevemente cómo se llegá hasta ahí y cómo
hay que interpretar esta fusión.

El principio de relatividad exige que el postulado de conservación de la energía se
cumpla, no sólo respecto a \textit{un} sistema de coordenadas $K$, sino respecto a cualquier
sistema de coordenadas $K'$ que se encuentre con relación a $K$ en movimiento de
traslación uniforme (dicho brevemente, respecto a cualquier sistema de coordenadas
«de Galileo»). En contraposición a la Mecánica clásica, el paso entre dos de esos
sistemas viene regido por la transformación de Lorentz.

A partir de estas premisas, y en conjunción con las ecuaciones fundamentales de la
electrodinámica maxwelliana, se puede inferir rigurosamente, mediante
consideraciones relativamente sencillas, que: un cuerpo que se mueve con velocidad $v$
y que absorbe la energa $E_{0}$ en forma de radiación\footnote{$E_{0}$ es la energía
absorbida respecto a un sistema de coordenadas que se mueve con el cuerpo.} sin variar
por eso su velocidad, experimenta un aumento de energa en la cantidad:

\[\frac{E_{0}}{\sqrt{I-\frac{v^{2}}{c^{2}}}}\]
 ~

Teniendo en cuenta la expresión que dimos antes para la energía cinética, la energía del
cuerpo vendrá dada por:

\[\frac{\left(m+\frac{E_{0}}{c^{2}}\right)c^{2}}{\sqrt{I-\frac{v^{2}}{c^{2}}}}\]
 ~

\noindent el cuerpo tiene entonces la misma energía que otro de velocidad $v$ y masa

\[\left(m+\frac{E_{0}}{c^{2}}\right)\]
 ~

\noindent Cabe por tanto decir: si un cuerpo absorbe la energa $E_{0}$, su masa inercial crece en

\[\frac{E_{0}}{c^{2}}\]
 ~

\noindent la masa inercial de un cuerpo no es una constante, sino variable según la modificación
de su energa. La masa inercial de un sistema de cuerpos cabe contemplarla precisamente
como una medida de su energía. El postulado de la conservación de la masa de un sistema
coincide con el de la conservación de la energía y sólo es válido en la medida en que el
sistema no absorbe ni emite energía. Si escribimos la expresión de la energía en la
forma


\[\frac{mc^{2}+E_{0}}{\sqrt{I-\frac{v^{2}}{c^{2}}}}\]
 ~

\noindent se ve que el término $mc^{2}$, que ya nos llamó la atención con anterioridad,
no es otra cosa que la energía que poseía el cuerpo\footnote{Respecto a un sistema de 
coordenadas solidario con el cuerpo.} antes de absorber la energía $E_{0}$.

El cotejo directo de este postulado con la experiencia queda por ahora excluido,
porque las variaciones de energía $E_{0}$ que podemos comunicar a un sistema no son
suficientemente grandes para hacerse notar en forma de una alteración de la masa
inercial del sistema.

\[\frac{E_{0}}{c^{2}}\]
 ~

\noindent es demasiado pequeño en comparación con la masa $m$ que existía antes de la
variación de energía. A esta circunstancia se debe el que se pudiera establecer con
éxito un principio de conservación de la masa de validez independiente.

Una última observación de naturaleza teórica. El éxito de la interpretación de
Faraday-Maxwell de la acción electrodinámica a distancia a través de procesos
intermedios con velocidad de propagación finita hizo que entre los físicos arraigara la
convicción de que no existían acciones a distancia instantáneas e inmediatas del tipo de
la ley de gravitación de Newton. Según la teoría de la relatividad, en lugar de la acción
instantánea a distancia, o acción a distancia con velocidad de propagación infinita,
aparece siempre la acción a distancia con la velocidad de la luz, lo cual tiene que ver
con el papel teórico que desempeña la velocidad $c$ en esta teoría. En la segunda parte
se mostrará cómo se modifica este resultado en la teoría de la relatividad general.


\chapter{La Teoría de la Relatividad Especial y la Experiencia}

La pregunta de hasta qué punto se ve apoyada la teoría de la relatividad especial por
la experiencia no es fácil de responder, por un motivo que ya mencionamos al hablar
del experimento fundamental de Fizeau. La teoría de la relatividad especial cristalizá a
partir de la teoría de Maxwell-Lorentz de los fenómenos electromagnéticos, por lo cual
todos los hechos experimentales que apoyan esa teoría electromagntica apoyan
también la teoría de la relatividad. Mencionar aquí, por ser de especial importancia,
que la teoría de la relatividad permite derivar, de manera extremadamente simple y
en consonancia con la experiencia, aquellas influencias que experimenta la luz de las
estrellas fijas debido al movimiento relativo de la Tierra respecto a ellas. Se trata del
desplazamiento anual de la posición aparente de las estrellas fijas como
consecuencia del movimiento terrestre alrededor del Sol (aberración) y el influjo que
ejerce la componente radial de los movimientos relativos de las estrellas fijas respecto
a la Tierra sobre el color de la luz que llega hasta nosotros; este influjo se manifiesta
en un pequeño corrimiento de las rayas espectrales de la luz que nos llega desde una
estrella fija, respecto a la posicin espectral de las mismas rayas espectrales obtenidas
con una fuente luminosa terrestre (principio de Doppler). Los argumentos
experimentales a favor de la teoría de Maxwell-Lorentz, que al mismo tiempo son argumentos
a favor de la teoría de la relatividad, son demasiado copiosos como para
exponerlos aquí. De hecho, restringen hasta tal punto las posibilidades teóricas, que
ninguna otra teoría distinta de la de Maxwell-Lorentz se ha podido imponer frente a la
experiencia.

Sin embargo, hay dos clases de hechos experimentales constatados hasta ahora que la
teoría de Maxwell-Lorentz sólo puede acomodar a base de recurrir a una hipótesis
auxiliar que de suyo --es decir, sin utilizar la teoría de la relatividad-- parece extraña.

Es sabido que los rayos catódicos y los así llamados rayos (3 emitidos por sustancias
radiactivas constan de corpúsculos elctricos negativos (electrones)) de pequeñísima
inercia y gran velocidad. Investigando la desviación de estas radiaciones bajo la
influencia de campos elctricos y magnéticos se puede estudiar muy exactamente la ley
del movimiento de estos corpúsculos.

En el tratamiento teórico de estos electrones hay que luchar con la dificultad de que la
Electrodinámica por sí sola no es capaz de explicar su naturaleza. Pues dado que las
masas eléctricas de igual signo se repelen, las masas eléctricas negativas que
constituyen el electrón deberían separarse unas de otras bajo la influencia de su
interacción si no fuese por la acción de otras fuerzas cuya naturaleza nos resulta todavía
oscura\footnote{La teoría de la relatividad general propone la idea de que las masas
eléctricas de un electrón se mantienen unidas por fuerzas gravitacionales.}. 
Si suponemos ahora que las distancias relativas de las masas eléctricas que
constituyen el electrón permanecen constantes al moverse éste (unión rígida en el
sentido de la Mecánica clásica), llegamos a una ley del movimiento del electrón que no
concuerda con la experiencia. H. A. Lorentz, guiado por consideraciones puramente
formales, fue el primero en introducir la hipótesis de que el cuerpo del electrón
experimenta, en virtud del movimiento, una contracción proporcional a la expresión

\[\overline{\sqrt{I-\frac{v^{2}}{c^{2}}}}.\] 

Esta hipótesis, que electrodinámicamente no se justifica en modo alguno,
proporciona esa ley del movimiento que se ha visto confirmada con gran precisión
por la experiencia en los últimos años.

La teoría de la relatividad suministra la misma ley del movimiento sin necesidad de
sentar hipótesis especiales sobre la estructura y el comportamiento del electrón. Algo
análogo ocurría, como hemos visto en \S 13, con el experimento de Fizeau, cuyo
resultado lo explicaba la teoría de la relatividad sin tener que hacer hipótesis sobre la
naturaleza física del fluido.

La segunda clase de hechos que hemos señalado se refiere a la cuestión de si el
movimiento terrestre en el espacio se puede detectar o no en experimentos efectuados
en la Tierra. Ya indicamos en \S 5 que todos los intentos realizados en este sentido
dieron resultado negativo. Con anterioridad a la teoría relativista, la ciencia no podía
explicar fácilmente este resultado negativo, pues la situación era la siguiente. Los
viejos prejuicios sobre el espacio y el tiempo no permitían ninguna duda acerca de
que la transformación de Galileo era la que regía el paso de un cuerpo de referencia a
otro. Suponiendo entonces que las ecuaciones de Maxwell-Lorentz sean válidas para
un cuerpo de referencia $K$, resulta que no valen para otro cuerpo de referencia $K'$ que
se mueva uniformemente respecto a $K$ si se acepta que entre las coordenadas de $K$ y
$K'$ rigen las relaciones de la transformación de Galileo. Esto parece indicar que de
entre todos los sistemas de coordenadas de Galileo se destaca físicamente uno ($K$) que
posee un determinado estado de movimiento. Físicamente se interpretaba este
resultado diciendo que $K$ está en reposo respecto a un hipotético éter luminífero,
mientras que todos los sistemas de coordenadas $K'$ en movimiento respecto a $K$
estarían también en movimiento respecto al éter. A este movimiento de $K'$ respecto al
éter («viento del éter» en relación a $K'$) se le atribuían las complicadas leyes que
supuestamente valían respecto a $K'$. Para ser consecuentes, había que postular también
un viento del éter semejante con relación a la Tierra, y los físicos pusieron durante
mucho tiempo todo su empeño en probar su existencia.

Michelson halló con este propsito un camino que parecía infalible. Imaginemos dos
espejos montados sobre un cuerpo rígido, con las caras reflectantes mirándose de
frente. Si todo este sistema se halla en reposo respecto al éter luminífero, cualquier
rayo de luz necesita un tiempo muy determinado $T$ para ir de un espejo al otro y
volver. Por el contrario, el tiempo (calculado) para ese proceso es algo diferente ($T'$)
cuando el cuerpo, junto con los espejos, se mueve respecto al éter. ¡Es más!  Los
cálculos predicen que, para una determinada velocidad $v$ respecto al éter, ese tiempo
$T'$ es distinto cuando el cuerpo se mueve perpendicularmente al plano de los espejos
que cuando lo hace paralelamente. Aun siendo ínfima la diferencia calculada entre
estos dos intervalos temporales, Michelson y Morley realizaron un experimento de
interferencias en el que esa discrepancia tendra que haberse puesto claramente de
manifiesto. El resultado del experimento fue, no obstante, negativo, para gran desconcierto 
de los físicos. Lorentz y FitzGerarld sacaron a la teoría de este desconcierto,
suponiendo que el movimiento del cuerpo respecto al éter determinaba una
contracción de aquél en la dirección del movimiento y que dicha contracción
compensaba justamente esa diferencia de tiempos. La comparación con las consideraciones
de \S 12 demuestra que esta solución era también la correcta desde el punto de
vista de la teoría de la relatividad. Pero la interpretación de la situación según esta
última es incomparablemente más satisfactoria. De acuerdo con ella, no existe ningún
sistema de coordenadas privilegiado que dé pie a introducir la idea del éter, ni
tampoco ningún viento del éter ni experimento alguno que lo ponga de manifiesto. La
contracción de los cuerpos en movimiento se sigue aquí, sin hipótesis especiales, de
los dos principios básicos de la teoría; y lo decisivo para esta contracción no es el
movimiento en sí, al que no podemos atribuir ningn sentido, sino el movimiento
respecto al cuerpo de referencia elegido en cada caso. Así pues, el cuerpo que sostiene
los espejos en el experimento de Michelson y Morley no se acorta respecto a un
sistema de referencia solidario con la Tierra, pero sí respecto a un sistema que se halle
en reposo en relación al Sol.


\chapter{El Espacio Cuadridimensional de Minkowski}

El no matemático se siente sobrecogido por un escalofrío místico al oír la palabra
cuadridimensional, una sensación no disímil de la provocada por el fantasma de
una comedia. Y, sin embargo, no hay enunciado más banal que el que afirma que
nuestro mundo cotidiano es un continuo espacio-temporal cuadridimensional.

El \textit{espacio} es un continuo tridimensional. Quiere decir esto que es posible
describir la posición de un punto (en reposo) mediante tres números $x$, $y$, $z$ (coordenadas)
y que, dado cualquier punto, existen puntos arbitrariamente próximos cuya
posición se puede describir mediante valores coordenados (coordenadas)
$x_{1}$, $y_{1}$, $z_{1}$, que se aproximan arbitrariamente a las coordenadas 
$x$, $y$, $z$ del primero. Debido a esta última propiedad hablamos de un continuo; 
debido al carcter triple de las coordenadas, de tridimensional.

Análogamente ocurre con el universo del acontecer físico, con lo que Minkowski
llamara brevemente «mundo» o «universo», que es naturalmente cuadridimensional
en el sentido espacio-temporal. Pues ese universo se compone de sucesos individuales,
cada uno de los cuales puede describirse mediante cuatro números, a saber, tres
coordenadas espaciales $x$, $y$, $z$ y una coordenada temporal, el valor del tiempo $t$. El
universo es en este sentido también un continuo, pues para cada suceso existen
otros (reales o imaginables) arbitrariamente «próximos» cuyas coordenadas $x_{1}$, $y_{1}$, $z_{1}$, $t_{1}$
se diferencian arbitrariamente poco de las del suceso contemplado $x$, $y$, $z$, $t$. El que
no estemos acostumbrados a concebir el mundo en este sentido como un continuo
cuadridimensional se debe a que el tiempo desempeñá en la física prerrelativista un
papel distinto, más independiente, frente a las coordenadas espaciales, por lo cual nos
hemos habituado a tratar el tiempo como un continuo independiente. De hecho, en
la física clásica el tiempo es absoluto, es decir, independiente de la posición y del estado
de movimiento del sistema de referencia, lo cual queda patente en la última ecuación
de la transformación de Galileo ($t'=t$). La teoría de la relatividad sirve en bandeja
la visión cuadridimensional del «mundo», pues según esta teoría el tiempo es
despojado de su independencia, tal y como muestra la cuarta ecuación de la
transformacin de Lorentz:

\[t'=\frac{t-\frac{v}{c^{2}}x}{\sqrt{I-\frac{v^{2}}{c^{2}}}}\]
 ~

En efecto, según esta ecuación la diferencia temporal $\Delta t'$ de dos sucesos respecto a $K'$
no se anula en general, aunque la diferencia temporal $\Delta t'$ de aquellos respecto a $K$ sea
nula. Una distancia puramente espacial entre dos sucesos con relación a $K$ tiene como
consecuencia una distancia temporal de aquéllos con respecto a $K'$. La importancia del
descubrimiento de Minkowski para el desarrollo formal de la teoría de la relatividad no
reside tampoco aquí, sino en el reconocimiento de que el continuo cuadridimensional
de la teoría de la relatividad muestra en sus principales propiedades formales el
máximo parentesco con el continuo tridimensional del espacio geométrico euclídeo\footnote{
Cf. la exposición algo más detallada en el Apéndice.}. 
Sin embargo, para hacer resaltar del todo este parentesco es preciso sustituir las
coordenadas temporales usuales $t$ por la cantidad imaginaria $\sqrt{-I}ct$ 
proporcional a ellas. Las leyes de la naturaleza que satisfacen los requisitos de la
teoría de la relatividad (especial) toman entonces formas matemáticas en las que la
coordenada temporal desempeña exactamente el mismo papel que las tres
coordenadas espaciales. Estas cuatro coordenadas se corresponden exactamente, desde
el punto de vista formal, con las tres coordenadas espaciales de la geometría euclídea.
Incluso al no matemático le saltará a la vista que, gracias a este hallazgo puramente
formal, la teoría tuvo que ganar una dosis extraordinaria de claridad.

Tan someras indicaciones no dan al lector sino una noción muy vaga de las
importantes ideas de Minkowski, sin las cuales la teoría de la relatividad general,
desarrollada a continuación en sus líneas fundamentales, se habra quedado quizá en
pañales. Ahora bien, como para comprender las ideas fundamentales de la teoría de la
relatividad especial o general no es necesario entender con más exactitud esta materia,
sin duda de difcil acceso para el lector no ejercitado en la matemática, lo dejaremos
en este punto para volver sobre ello en las últimas consideraciones de este librito.


%PARTE II


\part{Sobre la Teoría de la Relatividad General}


\chapter{Principios de la Relatividad Especial y General}

La tesis fundamental alrededor de la cual giraban todas las consideraciones anteriores
era el principio de la relatividad \textit{especial}, es decir, el principio de la relatividad física de
todo movimiento \textit{uniforme}. Volvamos a analizar exactamente su contenido.

Que cualquier movimiento hay que entenderlo conceptualmente como un
movimiento meramente \textit{relativo} es algo que siempre fue evidente. Volviendo al ejemplo, 
tantas veces frecuentado ya, del terraplén y el vagón de ferrocarril, el hecho del
movimiento que aquí tiene lugar cabe expresarlo con igual razón en cualquiera de las
dos formas siguientes:
\begin{enumerate}
\item el vagón se mueve respecto al terraplén,
\item el terraplén se mueve respecto al vagón.
\end{enumerate}
En el caso 1) es el terraplén el que hace las veces de cuerpo de referencia; en el caso
2), el vagón. Cuando se trata simplemente de constatar o describir el movimiento es
teóricamente indiferente a qué cuerpo de referencia se refiera el movimiento. Lo cual
es, repetimos, evidente y no debemos confundirlo con la proposición, muchó o más
profunda, que hemos llamado principio de relatividad y en la que hemos basado
nuestras consideraciones.

El principio que nosotros hemos utilizado no se limita a sostener que para la
descripcin de cualquier suceso se puede elegir lo mismo el vagón que el terraplén
como cuerpo de referencia (por que también eso es evidente). Nuestro principio
afirma más bien que: si se formulan las leyes generales de la naturaleza, tal y como
resultan de la experiencia, sirvindose
\begin{enumerate}
\item del terraplén como cuerpo de referencia, 
\item del vagón como cuerpo de referencia,
\end{enumerate}
\noindent en ambos casos dichas leyes generales (p. ej., las leyes de la Mecánica o la ley de la
propagación de la luz en el vacío) tienen exactamente el mismo enunciado. Dicho de
otra manera: en la descripción \textit{física} de los procesos naturales no hay ningún cuerpo
de referencia $K$ o $K'$ que se distinga del otro. Este último enunciado no tiene que
cumplirse necesariamente a priori, como ocurre con el primero; no está contenido en
los conceptos de movimiento y cuerpo de referencia, ni puede deducirse de ellos,
sino que su verdad o falsedad depende sólo de la \textit{experiencia}.

Ahora bien, nosotros no hemos afirmado hasta ahora para nada la equivalencia de \textit{todos}
los cuerpos de referencia $K$ de cara a la formulación de las leyes naturales. El camino
que hemos seguido ha sido más bien el siguiente. Partimos inicialmente del supuesto
de que existe un cuerpo de referencia $K$ con un estado de movimieto respecto al
cual se cumple el principio fundamental de Galileo: un punto material abandonado a
su suerte y alejado lo suficiente de todos los demás se mueve uniformemente y en línea
recta. Referidas a $K$ (cuerpo de referencia de Galileo), las leyes de la naturaleza debían
ser lo más sencillas posible. Pero al margen de $K$, deberan ser privilegiados en este
sentido y exactamente equivalentes a $K$ de cara a la formulación de las leyes de la
naturaleza todos aquellos cuerpos de referencia $K'$ que ejecutan respecto a $K$ un
movimiento \textit{rectilneo, uniforme e irrotacional}: a todos estos cuerpos de referencia se los
considera cuerpos de referencia de Galileo. La validez del principio de la relatividad
solamente la supusimos para estos cuerpos de referencia, no para otros (animados de
otros movimientos). En este sentido hablamos del principio de la relatividad especial o
de la teoría de la relatividad especial.

En contraposición a lo anterior entenderemos por «principio de la relatividad
general» el siguiente enunciado: Todos los cuerpos de referencia $K$, $K'$, etc., sea cual
fuere su estado de movimiento, son equivalentes de cara a la descripción de la
naturaleza (formulación de las leyes naturales generales). Apresurámonos a señalar, sin
embargo, que esta formulación es preciso sustituirla por otra más abstracta, por razones
que saldrán a la luz más adelante.

Una vez que la introducción del principio de la relatividad especial ha salido airosa,
tiene que ser tentador, para cualquier espíritu que aspire a la generalización, el atreverse
a dar el paso que lleva al principio de la relatividad general. Pero basta una observación
muy simple, en apariencia perfectamente verosmil, para que el intento parezca en
principio condenado al fracaso. Imagínese el lector instalado en ese famoso vagón de
tren que viaja con velocidad uniforme. Mientras el vagón mantenga su marcha
uniforme, los ocupantes no notarán para nada el movimiento del tren; lo cual explica
asimismo que el ocupante pueda interpretar la situación en el sentido de que el vagón
está en reposo y que lo que se mueve es el terraplén, sin sentir por ello que violenta
su intuición. Y según el principio de la relatividad especial, esta interpretación está
perfectamente justificada desde el punto de vista físico.

Ahora bien, si el movimiento del vagón se hace no uniforme porque el tren frena
violentamente, pongamos por caso, el viajero experimentará un tirón igual de fuerte
hacia adelante. El movimiento acelerado del vagón se manifiesta en el comportamiento
mecánico de los cuerpos respecto a él; el comportamiento mecánico es distinto que en
el caso antes considerado, y por eso parece estar excluido que con relación al vagón en
movimiento no uniforme valgan las mismas leyes mecánicas que respecto al vagón en
reposo o en movimiento uniforme. En cualquier caso, está claro que en relación al
vagón que se mueve no uniformemente no vale el principio fundamental de Galileo.
De ahí que en un primer momento nos sintamos impelidos a atribuir, en contra del
principio de la relatividad general, una especie de realidad física absoluta al
movimiento no uniforme. En lo que sigue veremos, sin embargo, que esta inferencia
no es correcta.


\chapter{El Campo Gravitatorio}

A la pregunta de por qué cae al suelo una piedra levantada y soltada en el aire suele
contestarse «porque es atraída por la Tierra». La física moderna formula la respuesta
de un modo algo distinto, por la siguiente razón. A través de un estudio más detenido
de los fenómenos electromagnéticos se ha llegado a la conclusión de que no existe una
acción inmediata a distancia. Cuando un imán atrae un trozo de hierro, por ejemplo,
no puede uno contentarse con la explicación de que el imán actúa directamente sobre
el hierro a través del espacio intermedio vacío; lo que se hace es, según idea de
Faraday, imaginar que el imán crea siempre en el espacio circundante algo físicamente
real que se denomina campo magnético. Este campo magnético actúa a su vez sobre
el trozo de hierro, que tiende a moverse hacia el imán. No vamos a entrar aquí en la
justificación de este concepto interviniente que en sí es arbitrario. Señalemos tan sólo
que con su ayuda es posible explicar teóricamente de modo mucho más satisfactorio
los fenómenos electromagnéticos, y en especial la propagación de las ondas
electromagnéticas. De manera análoga se interpreta también la acción de la gravedad.

La influencia de la Tierra sobre la piedra se produce indirectamente. La Tierra
crea alrededor suyo un campo gravitatorio. Este campo actúa sobre la piedra y
ocasiona su movimiento de caída. La intensidad de la acción sobre un cuerpo decrece
al alejarse más y más de la Tierra, y decrece según una ley determinada. Lo cual, en
nuestra interpretación, quiere decir que: la ley que rige las propiedades espaciales del
campo gravitatorio tiene que ser una ley muy determinada para representar
correctamente la disminución de la acción gravitatoria con la distancia al cuerpo que
ejerce la acción. Se supone, por ejemplo, que el cuerpo (la Tierra, pongamos por caso)
genera directamente el campo en su vecindad inmediata; La intensidad y dirección del
campo a distancias más grandes vienen entonces determinadas por la ley que rige las
propiedades espaciales de los campos gravitatorios.

El campo gravitatorio, al contrario que el campo eléctrico y magnético, muestra una
propiedad sumamente peculiar que es de importancia fundamental para lo que sigue.
Los cuerpos que se mueven bajo la acción exclusiva del campo gravitatorio
experimentan una aceleración \textit{que no depende lo más mínimo ni del material ni del estado
físico del cuerpo}. Un trozo de plomo y un trozo de madera, por ejemplo, caen
exactamente igual en el campo gravitatorio (en ausencia de aire) cuando los
dejamos caer sin velocidad inicial o con velocidades iniciales iguales. Esta ley, que se
cumple con extremada exactitud, se puede formular también de otra manera sobre la
base de la siguiente consideración.

Según la ley del movimiento de Newton se cumple

\begin{center}
(Fuerza) = (masa inercial) $\times$ (aceleración), 
\par
\end{center}

\noindent donde la masa inercial es una constante caracterstica del cuerpo acelerado. Si la
fuerza aceleradora es la de la gravedad, tenemos, por otro lado, que

\[\mbox{(aceleración)}=\frac{\mbox{masa gravitacional}}{\mbox{masa inercial}}\times\mbox{intencidad del campo gravitacional}\]
 ~

Pues bien, si queremos que para un campo gravitatorio dado la aceleración sea
siempre la misma, independientemente de la naturaleza y del estado del cuerpo, tal y
como demuestra la experiencia, la relación entre la masa gravitatoria y la masa
inercial tiene que ser también igual para todos los cuerpos. Mediante adecuada
elección de las unidades puede hacerse que esta relación valga 1, siendo entonces
válido el teorema siguiente: la masa \textit{gravitatoria} y la masa \textit{inercial} de un cuerpo son
iguales.

La antigua mecánica \textit{registrá} este importante principio, pero no lo \textit{interpretá}. Una
interpretación satisfactoria no puede surgir sino reconociendo que la misma cualidad
del cuerpo se manifiesta como inercia o como gravedad, según las circunstancias.
En los párrafos siguientes veremos hasta qué punto es ese el caso y qué relación
guarda esta cuestión con el postulado de la relatividad general.


\chapter{La Igualdad Entre Masa Inercial y Masa Gravitatoria
Como Argumento a Favor del Postulado de la Relatividad General}

Imaginemos un trozo amplio de espacio vacío, tan alejado de estrellas y de grandes
masas que podamos decir con suficiente exactitud que nos encontramos ante el caso
previsto en la ley fundamental de Galileo. Para esta parte del universo es entonces
posible elegir un cuerpo de referencia de Galileo con respecto al cual los puntos en
reposo permanecen en reposo y los puntos en movimiento persisten constantemente en
un movimiento uniforme y rectilneo. Como cuerpo de referencia nos imaginamos un
espacioso cajón con la forma de una habitación; y suponemos que en su interior se
halla un observador pertrechado de aparatos. Para él no existe, como es natural,
ninguna gravedad. Tiene que sujetarse con cuerdas al piso, so pena de verse lanzado
hacia el techo al mínimo golpe contra el suelo.

Supongamos que en el centro del techo del cajón, por fuera, hay un gancho con
una cuerda, y que un ser --cuya naturaleza nos es indiferente-- empieza a tirar de ella
con fuerza constante. El cajón, junto con el observador, empezar a volar hacia arriba
con movimiento uniformemente acelerado. Su velocidad adquirirá con el tiempo cotas
fantásticas... siempre que juzguemos todo ello desde otro cuerpo de referencia del
cual no se tire con una cuerda.

Pero el hombre que está en el cajón ¿cómo juzga el proceso? El suelo del cajón le
transmite la aceleración Por presión contra los pies. Por consiguiente, tiene que
contrarrestar esta presión con ayuda de sus piernas si no quiere medir el suelo con su
cuerpo. Así pues, estar de pie en el cajón igual que lo está una persona en una
habitación de cualquier vivienda terrestre. Si suelta un cuerpo que antes sostenía en la
mano, la aceleración del cajón dejará de actuar sobre aquél, por lo cual se aproximará
al suelo en movimiento relativo acelerado. El observador se convencerá también de que
\textit{la aceleración del cuerpo respecto al suelo es siempre igual de grande, independientemente del
cuerpo con que realice el experimento.}

Apoyándose en sus conocimientos del campo gravitatorio, tal y como los hemos
comentado en el último epígrafe, el hombre llegará así a la conclusión de que se halla,
junto con el cajón, en el seno de un campo gravitatorio bastante constante. Por un
momento se sorprenderá, sin embargo, de que el cajón no caiga en este campo
gravitatorio, mas luego descubre el gancho en el centro del techo y la cuerda tensa
sujeta a él e infiere correctamente que el cajón cuelga en reposo en dicho campo.

¿Es lícito reírse del hombre y decir que su concepción es un error? Opino que, si
queremos ser consecuentes, no podemos hacerlo, debiendo admitir por el contrario
que su explicación no atenta ni contra la razón ni contra las leyes mecánicas
conocidas. Aun cuando el cajón se halle acelerado respecto al espacio de Galileo
considerado en primer lugar, cabe contemplarlo como inmóvil. Tenemos, pues, buenas
razones para extender el principio de relatividad a cuerpos de referencia que están
acelerados unos respecto a otros, habiendo ganado así un potente argumento a
favor de un postulado de relatividad generalizado.

Tómese buena nota de que la posibilidad de esta interpretación descansa en la
propiedad fundamental que posee el campo gravitatorio de comunicar a todos los
cuerpos la misma aceleración, o lo que viene a ser lo mismo, en el postulado de la
igualdad entre masa inercial y masa gravitatoria. Si no existiera esta ley de la naturaleza,
el hombre en el cajón acelerado no podra interpretar el comportamiento de los
cuerpos circundantes a base de suponer la existencia de un campo gravitatorio, y
ninguna experiencia le autorizaría a suponer que su cuerpo de referencia está «en
reposo».

Imaginemos ahora que el hombre del cajón ata una cuerda en la parte interior del
techo y fija un cuerpo en el extremo libre. El cuerpo hará que la cuerda cuelgue
«verticalmente» en estado tenso. Preguntémonos por la causa de la tensión. El hombre
en el cajón dirá: «Él cuerpo suspendido experimenta en el campo gravitatorio una fuerza
hacia abajo y se mantiene en equilibrio debido a la tensión de la cuerda; lo que
determina la magnitud de la tensión es la masa \textit{gravitatoria} del cuerpo suspendido».
Por otro lado, un observador que flote libremente en el espacio juzgará la situación así:
«La cuerda se ve obligada a participar del movimiento acelerado del cajón y lo
transmite al cuerpo sujeto a ella. La tensión de la cuerda es justamente suficiente para
producir la aceleración del cuerpo. Lo que determina la magnitud de la tensión en la
cuerda es la \textit{masa inercial} del cuerpo» En este ejemplo vemos que la extensión del
principio de relatividad pone de manifiesto la \textit{necesidad} del postulado de la igualdad
entre masa inercial y gravitatoria. Con lo cual hemos logrado una interpretación física
de este postulado.

El ejemplo del cajón acelerado demuestra que una teoría de la relatividad general ha
de proporcionar resultados importantes en punto a las leyes de la gravitación. Y en
efecto, el desarrollo consecuente de la idea de la relatividad general ha suministrado
las leyes que satisface el campo gravitatorio. Sin embargo, he de Prevenir desde este
mismo momento al lector de una confusión a que pueden inducir estas
consideraciones. Para el hombre del cajón existe un campo gravitatorio, pese a no
existir tal respecto al sistema de coordenadas inicialmente elegido. Dirase entonces que
la existencia de un campo gravitatorio es siempre meramente \textit{aparente}. Podra
pensarse que, independientemente del campo gravitatorio que exista, siempre cabra
elegir otro cuerpo de referencia de tal manera que respecto a él no existiese ninguno.
Pues bien, eso no es cierto para cualquier campo gravitatorio, sino sólo para aquellos
que poseen una estructura muy especial. Es imposible, por ejemplo, elegir un cuerpo
de referencia respecto al cual el campo gravitatorio de la Tierra desaparezca (en toda
su extensión).

Ahora nos damos cuenta de por qué el argumento esgrimido al final de \S 18 contra
el principio de la relatividad general no es concluyente. Sin duda es cierto que el
observador que se halla en el vagón siente un tirón hacia adelante como consecuencia
del frenazo, y es verdad que en eso nota la no uniformidad del movimiento. Pero nadie
le obliga a atribuir el tirón a una aceleracin «real» del vagón. Igual podra
interpretar el episodio así: «Mi cuerpo de referencia (el vagón) permanece constantemente 
en reposo. Sin embargo, (durante el tiempo de frenada) existe respecto
a él un campo gravitatorio temporalmente variable, dirigido hacia adelante. Bajo la
influencia de este último, el terraplén, junto con la Tierra, se mueve no uniformemente,
de suerte que su velocidad inicial, dirigida hacia atrás, disminuye cada vez más.
Este campo gravitatorio es también el que produce el tirón del observador»


\chapter{¿Hasta Qué Punto son Insatisfactorias las Bases de la Mecánica y de la
Teoría de la Relatividad Especial?}

Como ya hemos dicho en varias ocasiones, la Mecánica clásica parte del principio
siguiente: los puntos materiales suficientemente alejados de otros puntos materiales se
mueven uniformemente y en línea recta o persisten en estado de reposo. También
hemos subrayado repetidas veces que este principio fundamental sólo puede ser
válido para cuerpos de referencia $K$ que se encuentran en determinados estados de
movimiento y que se hallan en movimiento de traslación uniforme unos respecto a
otros. Con relación a otros cuerpos de referencia $K'$ no vale el principio. Tanto en la
Mecánica clásica como en la teoría de la relatividad especial se distingue, por tanto,
entre cuerpos de referencia $K$ respecto a los cuales son válidas las leyes de la naturaleza
y cuerpos de referencia $K'$ respecto a los cuales no lo son.

Ahora bien, ninguna persona que piense con un mínimo de lógica se dará por
satisfecha con este estado de cosas, y preguntará: ¿Cómo es posible que determinados
cuerpos de referencia (o bien sus estados de movimiento) sean privilegiados frente a
otros (o frente a sus estados de movimiento respectivos)? \textit{¿Cuál es la razón de ese
privilegio?} Para mostrar claramente lo que quiero decir con esta pregunta, me serviré de
una comparación.

Estoy ante un hornillo de gas. Sobre él se encuentran, una al lado de la otra, dos ollas
de cocina idénticas, hasta el punto de que podramos confundirlas. Ambas están
llenas de agua hasta la mitad. Advierto que de una de ellas sale ininterrumpidamente
vapor, mientras que de la otra no, lo cual me llamará la atención aunque jamás me haya
echado a la cara un hornillo de gas ni una olla de cocina. Si entonces percibo un algo que
brilla con luz azulada bajo la primera olla, pero no bajo la segunda, se desvanecerá mi
asombro aun en el caso de que jamás haya visto una llama de gas, pues ahora podré
decir que ese algo azulado es la causa, o al menos la \textit{posible} causa de la emanación de
vapor. Pero si no percibo bajo ninguna de las dos ollas ese algo azulado y veo que la
una no cesa de echar vapor mientras que en la otra no es así, entonces no saldré del
asombro y de la insatisfacción hasta que detecte alguna circunstancia a la que pueda
hacer responsable del dispar comportamiento de las dos ollas.

Anlogamente, busco en vano en la Mecánica clásica (o en la teoría de la relatividad
especial) un algo real al que poder atribuir el dispar comportamiento de los cuerpos
respecto a los sistemas $K$ y $K'$\footnote{La objeción adquiere especial contundencia
cuando el estado de movimiento del cuerpo de referencia es tal que para mantenerlo 
no requiere de ninguna influencia exterior, por ejemplo en el caso de que el cuerpo
de referencia rote uniformemente.}. Esta objeción la vio ya Newton, quien intentó en vano
neutralizarla. Pero fue E. Mach el que la detectó con mayor claridad, proponiendo como
solución colocar la Mecánica sobre fundamentos nuevos. La objeción solamente se
puede evitar en una física que se corresponda con el principio de la relatividad general,
porque las ecuaciones de una teoría semejante valen para cualquier cuerpo de referencia,
sea cual fuere su estado de movimiento.


\chapter{Algunas Conclusiones del Principio de la Relatividad General}

Las consideraciones hechas en \S 20 muestran que el principio de la relatividad
general nos permite deducir propiedades del campo gravitatorio por vía puramente
teórica. Supongamos, en efecto, que conocemos la evolución espacio-temporal de un
proceso natural cualquiera, tal y como ocurre en el terreno galileano respecto a un
cuerpo de referencia de Galileo $K$. En estas condiciones es posible averiguar mediante
operaciones puramente teóricas, es decir, por simples cálculos, cómo se comporta este
proceso natural conocido respecto a un cuerpo de referencia $K'$ que está acelerado con
relación a $K$. Y como respecto a este nuevo cuerpo de referencia $K'$ existe un campo
gravitatorio, el cálculo nos informa de cómo influye el campo gravitatorio en el
proceso estudiado.

Así descubrimos, por poner un caso, que un cuerpo que respecto a $K$ ejecuta un
movimiento uniforme y rectilneo (según el principio de Galileo), ejecuta respecto al
cuerpo de referencia acelerado $K'$ (cajón) un movimiento acelerado, de trayectoria
generalmente curvada. Esta aceleración, o esta curvatura, responde a la influencia que
sobre el cuerpo móvil ejerce el campo gravitatorio que existe respecto a $K'$. Que el
campo gravitatorio influye de este modo en el movimiento de los cuerpos es ya sabido,
de modo que la reflexión no aporta nada fundamentalmente nuevo.

Sí se obtiene, en cambio, un resultado nuevo y de importancia capital al hacer
consideraciones equivalentes para un rayo de luz. Respecto al cuerpo de referencia de
Galileo $K$, se propaga en línea recta con velocidad $c$. Respecto al cajón acelerado
(cuerpo de referencia $K'$), la trayectoria del mismo rayo de luz ya no es una recta, como
se deduce fácilmente. De aquí se infiere que \textit{los rayos de luz en el seno de campos
gravitatorios se propagan en general según líneas curvas}. Este resultado es de gran
importancia por dos conceptos.

En primer lugar, cabe contrastarlo con la realidad. Aun cuando una reflexión
detenida demuestra que la curvatura que predice la teoría de la relatividad general para
los rayos luminosos es ínfima en el caso de los campos gravitatorios que nos brinda
la experiencia, tiene que ascender a 1,7 segundos de arco para rayos de luz que pasan
por las inmediaciones del Sol. Este efecto debera traducirse en el hecho de que las
estrellas fijas situadas en las cercanías del Sol, y que son observables durante eclipses
solares totales, aparezcan alejadas de él en esa cantidad, comparado con la posición
que ocupan para nosotros en el cielo cuando el Sol se halla en otro lugar de la bóveda
celeste. La comprobación de la verdad o falsedad de este resultado es una tarea de la
máxima importancia, cuya solución es de esperar que nos la den muy pronto los
astrónomos\footnote{La existencia de la desviación de la luz exigida por la teoría fue comprobada fotográficamente durante el eclipse de Sol del 30 de mayo de 1919 por dos expediciones
organizadas por la Royal Society bajo la direccin de los astrónomos Eddington y Crommelin.}.

En segundo lugar, la consecuencia anterior demuestra que, según la teoría de la
relatividad general, la tantas veces mencionada ley de la constancia de la velocidad de
la luz en el vacío --que constituye uno de los dos supuestos básicos de la teoría de la
relatividad especial-- no puede aspirar a validez ilimitada, pues los rayos de luz
solamente pueden curvarse si la velocidad de propagación de ésta varía con la posición.
Cabra pensar que esta consecuencia da al traste con la teoría de la relatividad especial
y con toda la teoría de la relatividad en general. Pero en realidad no es así. Tan sólo
cabe inferir que la teoría de la relatividad especial no puede arrogarse validez en un
campo ilimitado; sus resultados sólo son válidos en la medida en que se pueda
prescindir de la influencia de los campos gravitatorios sobre los fenómenos (los
luminosos, por ejemplo).

Habida cuenta de que los detractores de la teoría de la relatividad han afirmado a
menudo que la relatividad general tira por la borda la teoría de la relatividad especial,
voy a aclarar el verdadero estado de cosas mediante una comparación. Antes de quedar
establecida la Electrodinámica, las leyes de la Electrostática pasaban por ser las leyes
de la Electricidad en general. Hoy sabemos que la Electrostática sólo puede explicar
correctamente los campos elctricos en el caso --que en rigor jamás se da-- de que las
masas eléctricas estén estrictamente en reposo unas respecto a otras y en relación al
sistema de coordenadas. ¿Quiere decir eso que las ecuaciones de campo
electrodinámicas de Maxwell hayan tirado por la borda a la Electrostática? ¡De ningún
modo! La Electrostática se contiene en la Electrodinámica como caso límite; las leyes
de esta última conducen directamente a las de aquélla en el supuesto de que los
campos sean temporalmente invariables. El sino más hermoso de una teoría física es
el de señalar el camino para establecer otra más amplia, en cuyo seno pervive como
caso límite.

En el ejemplo que acabamos de comentar, el de la propagación de la luz, hemos
visto que el principio de la relatividad general nos permite derivar por vía teórica la
influencia del campo gravitatorio sobre la evolución de fenómenos cuyas leyes son ya
conocidas para el caso de que no exista campo gravitatorio. Pero el problema más
atractivo de entre aquellos cuya clave proporciona la teoría de la relatividad general
tiene que ver con la determinación de las leyes que cumple el propio campo de
gravitación. La situación es aquí la siguiente.

Conocemos regiones espacio-temporales que, previa elección adecuada del cuerpo de
referencia, se comportan (aproximadamente) «al modo galileano», es decir, regiones en
las cuales no existen campos gravitatorios. Si referimos una región semejante a un
cuerpo de referencia de movimiento arbitrario $K'$, entonces existe respecto a $K'$ un
campo gravitatorio temporal y espacialmente variable\footnote{Esto se sigue por 
generalización del razonamiento expuesto en \S 20.}. La estructura de este campo
depende naturalmente de cómo elijamos el movimiento de $K'$. Según la teoría de la
relatividad general, la ley general del campo gravitatorio debe cumplirse para todos los
campos gravitatorios así obtenidos. Aun cuando de esta manera no se pueden
engendrar ni de lejos todos los campos gravitatorios, cabe la esperanza de poder
deducir de estos campos de clase especial la ley general de la gravitación. ¡Y esta
esperanza se ha visto bellísimamente cumplida! Pero desde que se vislumbró
claramente esta meta hasta que se llegó de verdad a ella hubo que superar una seria
dificultad que no debo ocultar al lector, por estar arraigada en la esencia misma
del asunto. La cuestión requiere profundizar nuevamente en los conceptos del
continuo espacio-temporal.


\chapter{El Comportamiento de Relojes y Reglas Sobre un Cuerpo de Referencia en Rotación}

Hasta ahora me he abstenido intencionadamente de hablar de la interpretación física
de localizaciones espaciales y temporales en el caso de la teoría de la relatividad
general. Con ello me he hecho culpable de un cierto desalio que, según sabemos
por la teoría de la relatividad especial, no es en modo alguno banal ni perdonable. Hora
es ya de llenar esta laguna; pero advierto de antemano que el asunto demanda no
poca paciencia y capacidad de abstracción por parte del lector.

Partimos una vez más de casos muy especiales y muy socorridos. Imaginemos una
región espacio-temporal en la que, respecto a un cuerpo de referencia $K$ que posea
un estado de movimiento convenientemente elegido, no exista ningún campo
gravitatorio; en relación a la región considerada, $K$ es entonces un cuerpo de
referencia de Galileo, siendo válidos respecto a él los resultados de la teoría de la
relatividad especial. Imaginemos la misma región, pero referida a un segundo cuerpo
de referencia $K'$ que rota uniformemente respecto a $K$. Para fijar las ideas,
supongamos que $K'$ es un disco circular que gira uniformemente alrededor de su
centro y en su mismo plano. Un observador sentado en posición excéntrica sobre el
disco circular $K'$ experimenta una fuerza que actúa en dirección radial hacia afuera y
que otro observador que se halle en reposo respecto al cuerpo de referencia original $K$
interpreta como acción inercial (fuerza centrífuga). Supongamos, sin embargo, que el
observador sentado en el disco considera éste como un cuerpo de referencia en
reposo, para lo cual está autorizado por el principio de relatividad. La fuerza que
actúa sobre él --y en general sobre los cuerpos que se hallan en reposo respecto al
disco-- la interpreta como la acción de un campo gravitatorio. La distribución espacial
de este campo no sera posible según la teoría newtoniana de la gravitación\footnote{El
campo se anula en el centro del disco y aumenta hacia fuera proporcionalmente a la 
distancia al punto medio.}. Pero como el observador cree en la teoría de la relatividad
general, no le preocupa este
detalle; espera, con razón, poder establecer una ley general de la gravitación que
explique correctamente no sólo el movimiento de los astros, sino también el campo de
fuerzas que él percibe.

Este observador, instalado en su disco circular, experimenta con relojes y reglas, con
la intención de obtener, a partir de lo observado, definiciones exactas para el
significado de los datos temporales y espaciales respecto al disco circular $K'$. ¿Qué
experiencias tendrá en ese intento?

Imaginemos que el observador coloca primero dos relojes de idéntica constitución,
uno en el punto medio del disco circular, el otro en la periferia del mismo, de manera
que ambos se hallan en reposo respecto al disco. En primer lugar nos preguntamos si
estos dos relojes marchan o no igual de rápido desde el punto de vista del cuerpo de
referencia de Galileo $K$, que no rota. Juzgado desde $K$, el reloj situado en el centro
no tiene ninguna velocidad, mientras que el de la periferia, debido a la rotación
respecto a $K$, está en movimiento. Según un resultado de \S 12, este segundo reloj
marchará constantemente más despacio --respecto a $K$-- que el reloj situado en el
centro del disco circular. Lo mismo debería evidentemente constatar el hombre del
disco, a quien vamos a imaginar sentado en el centro, junto al reloj que hay allí. Así
pues, en nuestro disco circular, y con más generalidad en cualquier campo
gravitatorio, los relojes marcharán más deprisa o más despacio según el lugar que
ocupe el reloj (en reposo). Por consiguiente, con ayuda de relojes colocados en reposo
respecto al cuerpo de referencia no es posible dar una definición razonable del tiempo.
Análoga dificultad se plantea al intentar aplicar aquí nuestra anterior definición de
simultaneidad, tema en el que no vamos a profundizar.

También la definición de las coordenadas espaciales plantea aquí problemas que en
principio son insuperables. Porque si el observador que se mueve junto con el disco
coloca su escala unidad (una regla pequeña, comparada con el radio del disco)
tangencialmente sobre la periferia de éste, su longitud, juzgada desde el sistema de
Galileo, será más corta que 1, pues según \S 12 los cuerpos en movimiento
experimentan un acortamiento en la dirección del movimiento. Si en cambio coloca la
regla en la dirección del radio del disco, no habrá acortamiento respecto a $K$. Por
consiguiente, si el observador mide primero el perímetro del disco, luego su
diámetro y divide estas dos medidas, obtendrá como cociente, no el conocido número
$\pi$= 3,14..., sino un número mayor\footnote{En todo este razonamiento hay que 
utilizar el sistema de Galileo $K$ (que no rota) como cuerpo de
coordenadas, porque la validez de los resultados de la teoría de la relatividad 
especial sólo cabe suponerla respecto a $K$ (en relación a $K'$ existe un campo 
gravitatorio).}, mientras que en un disco inmóvil respecto a $K$
debera resultar exactamente $\pi$ en esta operación, como es natural. Con ello queda ya
probado que los teoremas de la geometría euclídea no pueden cumplirse exactamente
sobre el disco rotatorio ni, en general, en un campo gravitacional, al menos si se
atribuye a la reglilla la longitud 1 en cualquier posición y orientación. También el
concepto de línea recta pierde con ello su significado. No estamos, pues, en
condiciones de definir exactamente las coordenadas $x$, $y$, $z$ respecto al disco,
utilizando el método empleado en la teoría de la relatividad especial. Y mientras las
coordenadas y los tiempos de los sucesos no estén definidos, tampoco tienen
significado exacto las leyes de la naturaleza en las que aparecen esas coordenadas.
Todas las consideraciones que hemos hecho anteriormente sobre la relatividad
general parecen quedar así en tela de juicio. En realidad hace falta dar un sutil rodeo
para aplicar exactamente el postulado de la relatividad general. Las siguientes
consideraciones prepararán al lector para este cometido.


\chapter{El Continuo Euclídeo y el No Euclídeo}

Delante de mí tengo la superficie de una mesa de mármol. Desde cualquier punto
de ella puedo llegar hasta cualquier otro a base de pasar un número (grande) de
veces hasta un punto «vecino», o dicho de otro modo, yendo de un punto a otro sin
dar «saltos». El lector (siempre que no sea demasiado exigente) percibirá sin duda con
suficiente precisión lo que se entiende aquí por «vecino» y «saltos». Esto lo expresamos
diciendo que la superficie es un continuo.

Imaginemos ahora que fabricamos un gran número de varillas cuyo tamaño sea
pequeo comparado con las medidas de la mesa, y todas ellas igual de largas. Por esto
último se entiende que se pueden enrasar los extremos de cada dos de ellas.
Colocamos ahora cuatro de estas varillas sobre la superficie de la mesa, de modo que
sus extremos formen un cuadrilátero cuyas diagonales sean iguales (cuadrado). Para
conseguir la igualdad de las diagonales nos servimos de una varilla de prueba. Pegados
a este cuadrado construimos otros iguales que tengan en común con él una varilla;
junto a estos últimos otros tantos, etc. Finalmente tenemos todo el tablero cubierto
de cuadrados, de tal manera que cada lado interior pertenece a dos cuadrados y
cada vértice interior, a cuatro.

El que se pueda llevar a cabo esta operación sin tropezar con grandísimas dificultades
es un verdadero milagro. Basta con pensar en lo siguiente. Cuando en un vértice
convergen tres cuadrados, están ya colocados dos lados del cuarto, lo cual determina
totalmente la colocación de los dos lados restantes de éste. Pero ahora ya no puedo
retocar el cuadriltero para igualar sus diagonales. Si lo son de por sí, será en virtud de
un favor especial de la mesa y de las varillas, ante el cual me tendré que mostrar
maravillado y agradecido. Y para que la construcción se logre, tenemos que asistir a
muchos milagros parecidos.

Si todo ha ido realmente sobre ruedas, entonces digo que los puntos del tablero
forman un continuo euclidiano respecto a la varilla utilizada como segmento. Si
destaco uno de los vértices de la malla en calidad de «punto de origen», cualquier otro
podré caracterizarlo, respecto al punto de origen, mediante dos números. Me basta
con especificar cuntas varillas hacia «la derecha» y cuntas luego hacia «arriba» tengo
que recorrer a partir del origen para llegar al vértice en cuestión. Estos dos números
son entonces las coordenadas cartesianas de ese vértice con respecto al sistema de
coordenadas determinado por las varillas colocadas.

La siguiente modificación del experimento mental demuestra que también hay
casos en los que fracasa esta tentativa. Supongamos que las varillas se dilatan con la
temperatura y que se calienta el tablero en el centro pero no en los bordes. Sigue
siendo posible enrasar dos de las varillas en cualquier lugar de la mesa, pero nuestra
construcción de cuadrados quedar ahora irremisiblemente desbaratada, porque las
varillas de la parte interior de la mesa se dilatan, mientras que las de la parte exterior,
no.

Respecto a nuestras varillas --definidas como segmentos unidad-- la mesa ya no es un
continuo euclidiano, y tampoco estamos ya en condiciones de definir directamente con
su ayuda unas coordenadas cartesianas, porque no podemos realizar la construcción
anterior. Sin embargo, como existen otros objetos sobre los cuales la temperatura de
la mesa no influye de la misma manera que sobre las varillas (o sobre los cuales no
influye ni siquiera), es posible, sin forzar las cosas, mantener aun así la idea de que la
mesa es un «continuo euclidiano», y es posible hacerlo de modo satisfactorio mediante
una constatación más sutil acerca de la medición o comparación de segmentos.

Ahora bien, si todas las varillas, de cualquier clase o material, mostraran \textit{idéntico}
comportamiento termosensible sobre la mesa irregularmente temperada, y si no
tuviéramos otro medio de percibir la acción de la temperatura que el comportamiento
geométrico de las varillas en experimentos análogos al antes descrito, entonces podra
ser conveniente adscribir a dos puntos de la mesa la distancia 1 cuando fuese posible
enrasar con ellos los extremos de una de nuestras varillas; porque ¿cómo definir si no
el segmento, sin caer en la más crasa de las arbitrariedades? En ese caso hay que abandonar,
sin embargo, el método de las coordenadas cartesianas y sustituirlo por otro
que no presuponga la validez de la geometra euclidiana\footnote{Nuestro problema se les 
plante a los matemáticos de la siguiente manera. Dada una superficie --por
ejemplo, la de un elipsoide-- en el espacio de medida tridimensional euclidiano, 
existe sobre ella una geometría bidimensional, exactamente igual que en el plano. Gauss
se plante el problema de tratar teóricamente esta geometra bidimensional sin utilizar el
hecho de que la superficie pertenece a un continuo euclidiano de tres dimensiones. Si 
imaginamos que \textit{en la superficie} (igual que antes sobre la mesa) realizamos
construcciones con varillas rígidas, las leyes que valen para ellas son distintas de 
las de la geometría euclidiana del plano. La superficie no es, respecto a las varillas,
un continuo euclidiano, ni tampoco se pueden definir coordenadas cartesianas \textit{en 
la superficie}. Gauss mostró los principios con arreglo a los cuales se pueden tratar
las condiciones geométricas en la superficie, señalando así el camino hacia el tratamiento
riemanniano de continuos no euclidianos multidimensionales. De ahí que los matemáticos tengan
resueltos desde hace mucho los problemas formales a que conduce el postulado de la relatividad
general.}. El lector advertirá que la situación aquí descrita se corresponde con aquella que
ha traído consigo el postulado de la relatividad general (\S 23).


\chapter{Coordenadas Gaussianas}

% Figura 4

%         __ u=1
%       _/
%  ____/    ___ u=2
%    / \___/    ___ u=3
% __/_  _/ \___/
%  /  \/__  _/  \_
%__/__/   \/__    \
% /   \__/    \_   v=3
%/   /  /\__    \
%   /  /    \   v=2
%           v=1

\begin{figure}[bthp]
\centering
\caption{}
\label{fig:4}
\unitlength 1mm
\begin{picture}(75.00,55.00)(0,12)
\qbezier(20.00,20.00)(35.00,50.00)(65.00,50.00) \qbezier(15.00,25.00)(25.00,55.00)(45.00,60.00)
\qbezier(30.00,20.00)(45.00,40.00)(70.00,40.00)
\qbezier(50.00,15.00)(40.00,30.00)(15.00,35.00) \qbezier(55.00,25.00)(45.00,40.00)(20.00,45.00)
\qbezier(65.00,30.00)(55.00,50.00)(25.00,55.00)
\put(34.00,16.00){\makebox(0,0)[cc]{P}}
\put(34.00,19.00){\vector(1,3){2}} \put(36.50,27.50){\circle*{1.50}}
\put(54.00,12.500){\makebox(0,0)[cc]{$v=1$}} \put(59.00,22.50){\makebox(0,0)[cc]{$v=2$}}
\put(69.00,27.50){\makebox(0,0)[cc]{$v=3$}} \put(52.00,61.00){\makebox(0,0)[cc]{$u=1$}}
\put(72.00,51.00){\makebox(0,0)[cc]{$u=2$}} \put(77.00,41.00){\makebox(0,0)[cc]{$u=3$}}
\end{picture} 
\end{figure}

Este tratamiento geométrico-analítico se puede conseguir, según Gauss, de la
siguiente manera. Imaginemos dibujadas sobre el tablero de la mesa un sistema de
curvas arbitrarias (vase Fig. \ref{fig:4}), que llamamos curvas $u$ y a cada una de las cuales
caracterizamos con un número. En la figura están dibujadas las curvas $u = 1$, $u = 2$ y $u= 3$.
Pero entre las curvas $u = 1$ y $u = 2$ hay que imaginarse dibujadas infinitas más,
correspondientes a todos los números reales que están comprendidos entre 1 y 2.
Tenemos entonces un sistema de curvas $u$ que recubren la mesa de manera
infinitamente densa. Ninguna curva $u$ corta a ninguna otra, sino que por cada punto
de la mesa pasa una curva y sólo una. A cada punto de la superficie de la mesa le
corresponde entonces un valor $u$ perfectamente determinado. Supongamos también
que sobre la superficie se ha dibujado un sistema de curvas $v$ que satisfacen las mismas
condiciones, que están caracterizadas de manera anloga por números y que pueden
tener tambin una forma arbitraria.

A cada punto de la mesa le corresponde así un valor $u$ y un valor $v$, y a estos dos
números los llamamos las coordenadas de la mesa (coordenadas gaussianas). El punto
$P$ de la figura, por ejemplo, tiene como coordenadas gaussianas $u = 3$; $v = 1$. A dos
puntos vecinos $P$ y $P'$ de la superficie les corresponden entonces las coordenadas

\begin{eqnarray*}
P: & u~~,~~v\\
P': & u+du,v+dv
\end{eqnarray*}

donde $du$ y $dv$ representan números muy pequeños. Sea $ds$ un número también muy
pequeño que representa la distancia entre $P$ y $P'$ medida con una reglilla. Según Gauss
se cumple entonces:

\[ds_{2}=g_{11}du^{2}+2g_{12}dudv=g_{22}dv^{2}\]

\noindent donde $g_{11}$, $g_{12}$, $g_{22}$ son cantidades que dependen de manera muy determinada de $u$ y de
$v$. Las cantidades $g_{11}$, $g_{12}$ y $g_{22}$ determinan el comportamiento de las varillas respecto a
las curvas $u$ y $v$, y por tanto también respecto a la superficie de la mesa. En el caso de
que los puntos de la superficie considerada constituyan respecto a las reglillas de
medida un continuo euclidiano --y sólo en ese caso-- será posible dibujar las curvas $u$
y $v$ y asignarles números de tal manera que se cumpla sencillamente

\[ds^{2}=du^{2}+dv^{2}\]

Las curvas $u$ y $v$ son entonces líneas rectas en el sentido de la geometría euclidiana, y
perpendiculares entre sí. y las coordenadas gaussianas serán sencillamente coordenadas
cartesianas. Como se ve, las coordenadas gaussianas no son más que una asignación de
dos números a cada punto de la superficie considerada, de tal manera que a puntos
espacialmente vecinos se les asigna valores numéricos que difieren muy poco entre
sí.

Estas consideraciones valen en primera instancia para un continuo de dos
dimensiones. Pero el método gaussiano se puede aplicar tambin a un continuo de tres,
cuatro o más. Con un continuo de cuatro dimensiones, por ejemplo, resulta la
siguiente representación. A cada punto del continuo se le asignan arbitrariamente
cuatro números $x_{1}$, $x_{2}$, $x_{3}$, $x_{4}$ que se denominan «coordenadas». Puntos vecinos se
corresponden con valores vecinos de las coordenadas. Si a dos puntos vecinos $P$ y $P'$ se
les asigna una distancia $ds$ físicamente bien definida, susceptible de ser determinada
mediante mediciones, entonces se cumple la fórmula:

\[ds^{2}=g_{11}dx_{1}^{2}+2g_{12}dx_{1}dx_{2}....g_{44}dx_{4}^{2}\]

\noindent donde las cantidades $g_{11}$, etc. tienen valores que varían con la posición en el continuo.
Solamente en el caso de que el continuo sea euclidiano será posible asignar las
coordenadas $x_{1}\ldots x_{4}$. a los puntos del continuo de tal manera que se cumpla
simplemente

\[ds^{2}=dx_{1}^{2}+dx_{2}^{2}+dx_{3}^{2}+dx_{4}^{2}\]

Las relaciones que se cumplen entonces en el continuo cuadridimensional son anlogas
a las que rigen en nuestras mediciones tridimensionales.

Señalemos que la representación gaussiana para $ds^{2}$ que acabamos de dar no siempre
es posible; sólo lo es cuando existan regiones suficientemente pequeñas del continuo
en cuestión que quepa considerar como continuos euclidianos. Lo cual se cumple
evidentemente en el caso de la mesa y de la temperatura localmente variable, por
ejemplo, porque en una porción pequeña de la mesa es prácticamente constante la
temperatura, y el comportamiento geomtrico de las varillas es \textit{casi} el que exigen las
reglas de la geometría euclidiana. Así pues, las discordancias en la construcción de
cuadrados del epígrafe anterior no se ponen claramente de manifiesto mientras la
operación no se extienda a una parte importante de la mesa.

En resumen, podemos decir: Gauss inventó un método para el tratamiento de
cualquier continuo en el que están definidas relaciones de medidas (distancia entre
puntos vecinos). A cada punto del continuo se le asignan tantos números (coordenadas
gaussianas) como dimensiones tenga el continuo. La asignación se realiza de tal modo
que se conserve la univocidad y de manera que a puntos vecinos les correspondan
números (coordenadas gaussianas) que difieran infinitamente poco entre sí. El sistema
de coordenadas gaussianas es una generalización lógica del sistema de coordenadas
cartesianas. También es aplicable a continuos no euclidianos, pero solamente cuando
pequeñas porciones del continuo considerado se comporten, respecto a la medida
definida (distancia), tanto más euclidianamente cuanto menor sea la parte del
continuo considerada.


\chapter{El Continuo Espacio-Temporal de la Teoría de la Relatividad Especial
Como Continuo Euclidiano}

Ahora estamos en condiciones de formular con algo más de precisión las ideas
de Minkowski que esbozamos vagamente en \S 17. Según la teoría de la relatividad
especial, en la descripción del continuo espacio temporal cuadridimensional gozan
de privilegio ciertos sistemas de coordenadas que hemos llamado «sistemas de
coordenadas de Galileo». Para ellos, las cuatro coordenadas $x$, $y$, $z$, $t$ que determinan
un suceso --o expresado de otro modo, un punto del continuo cuadridimensional--
vienen definidas físicamente de manera muy simple, como ya se explicó en la
primera parte de este librito. Para el paso de un sistema de Galileo a otro que se
mueva uniformemente respecto al primero son válidas las ecuaciones de la transformación
de Lorentz, que constituyen la base para derivar las consecuencias de la teoría de
la relatividad especial y que por su parte no son más que la expresión de la validez
universal de la ley de propagación de la luz para todos los sistemas de referencia de
Galileo.

Minkowski descubrió que las transformaciones de Lorentz satisfacen las sencillas
condiciones siguientes. Consideremos dos sucesos vecinos, cuya posición mutua en el
continuo cuadridimensional venga dada por las diferencias de coordenadas espaciales
$dx$, $dy$, $dz$ y la diferencia temporal $dt$ respecto a un cuerpo de referencia de Galileo $K$.
Respecto a un segundo sistema de Galileo, sean $dx'$, $dy'$, $dz'$, $dt'$ las correspondientes
diferencias para ambos sucesos. Entre ellas se cumple entonces siempre la condición\footnote{Cf.
Apéndice. Las relaciones (\ref{eqn:a16}) y (\ref{eqn:b1}) deducidas allí para las coordenadas valen también para
\textit{diferencias} de coordenadas, y por tanto para diferenciales de las mismas (diferencias
infinitamente pequeñas).}:

\[dx^{2}+dy^{2}+dz^{2}-c^{2}dt^{2}=dx'2+dy'2+dz'2-c^{2}dt'^{2}\]

Esta condición tiene como consecuencia la validez de la transformación de
Lorentz. Lo cual podemos expresarlo así: la cantidad

\[ds^{2}=dx^{2}+dy^{2}+dz^{2}-c^{2}dt^{2}\]

\noindent correspondiente a dos puntos vecinos del continuo espacio-temporal
cuadridimensional, tiene el mismo valor para todos los cuerpos de referencia
privilegiados (de Galileo). Si se sustituye $x$, $y$, $z$, $\sqrt{-I}\cdot ct$ , por
$x_{1}$, $x_{2}$, $x_{3}$, $x_{4}$, se obtiene el resultado de que

\[ds^{2}=dx_{1}^{2}+dx_{2}^{2}+dx_{3}^{2}+dx_{4}^{2}\]

\noindent es independiente de la elección del cuerpo de referencia. A la cantidad $ds$ la llamamos
«distancia» de los dos sucesos o puntos cuadridimensionales.

Así pues, si se elige la variable imaginaria $\sqrt{-I}\cdot ct$
en lugar de la $t$ real como variable temporal, cabe interpretar el continuo espacio-temporal
de la teoría de la relatividad especial como un continuo cuadridimensional
euclidiano, como se desprende de las consideraciones del último epígrafe.


\chapter{El Continuo Espacio-Temporal de la Teoría de la Relatividad no es un
Continuo Euclidiano}

En la primera parte de este opúsculo nos hemos podido servir de coordenadas
espacio-temporales que permitan una interpretación física directa y simple y que,
según \S 26, puedan interpretarse como coordenadas cartesianas cuadridimensionales.
Esto fue posible en virtud de la ley de la constancia de la velocidad de la luz, ley que,
sin embargo, según \S 21, la teoría de la relatividad general no puede mantener;
Llegamos, por el contrario, al resultado de que según aquélla la velocidad de la luz
depende siempre de las coordenadas cuando existe un campo gravitatorio. En \S 23
constatamos además, en un ejemplo especial, que la existencia de un campo
gravitatorio hace imposible esa definición de las coordenadas y del tiempo que nos
condujo a la meta en la teoría de la relatividad especial.

Teniendo en cuenta estos resultados de la reflexión, llegamos al convencimiento de
que, según el principio de la relatividad general, no cabe interpretar el continuo
espacio-temporal como un continuo euclidiano, sino que nos hallamos aquí ante el
caso que vimos para el continuo bidimensional de la mesa con temperatura localmente
variable. Así como era imposible construir allí un sistema de coordenadas cartesiano
con varillas iguales, ahora es también imposible construir, con ayuda de cuerpos
rígidos y relojes, un sistema (cuerpo de referencia) de manera que escalas y relojes que
sean fijos unos respecto a otros indiquen directamente la posición y el tiempo. Esta es
en esencia la dificultad con que tropezamos en \S 23.

Sin embargo, las consideraciones de \S 25 y \S 26 señalan el camino que hay que seguir
para superarla. Referimos de manera arbitraria el continuo espacio-temporal
cuadridimensional a coordenadas gaussianas. A cada punto del continuo (suceso) le
asignamos cuatro números $x_{1}$, $x_{2}$, $x_{3}$, $x_{4}$ (coordenadas) que no poseen ningún
significado físico inmediato, sino que sólo sirven para enumerar los puntos de una
manera determinada, aunque arbitraria. Esta correspondencia no tiene ni siquiera que
ser de tal carcter que obligue a interpretar $x_{1}$, $x_{2}$, $x_{3}$ como coordenadas «espaciales» y
$x_{4}$ como coordenada «temporal».

El lector quizá piense que semejante descripcin del mundo es absolutamente
insatisfactoria. ¿Qué significa asignar a un suceso unas determinadas coordenadas $x_{1}$, $x_{2}$,
$x_{3}$, $x_{4}$ que en sí no significan nada? Una reflexión más detenida demuestra, sin
embargo, que la preocupación es infundada. Contemplemos, por ejemplo, un punto
material de movimiento arbitrario. Si este punto tuviera sólo una existencia
momentánea, sin duración, entonces vendra descrito espacio-temporalmente a través
de un sistema de valores único $x_{1}$, $x_{2}$, $x_{3}$, $x_{4}$. Su existencia permanente viene, por
tanto, caracterizada por un número infinitamente grande de semejantes sistemas de
valores, en donde las coordenadas se encadenan ininterrumpidamente; al punto
material le corresponde, por consiguiente, una línea (unidimensional) en el continuo
cuadridimensional. Y a una multitud de puntos móviles les corresponden otras tantas
líneas en nuestro continuo. De todos los enunciados que atañen a estos puntos, los
únicos que pueden aspirar a realidad física son aquellos que versan sobre encuentros
de estos puntos. En el marco de nuestra representación matemática, un encuentro de
esta especie se traduce en el hecho de que las dos líneas que representan los correspondientes
movimientos de los puntos tienen en común un determinado sistema $x_{1}$, $x_{2}$, $x_{3}$, $x_{4}$
de valores de las coordenadas. Que semejantes encuentros son en realidad las
únicas constataciones reales de carcter espacio-temporal que encontramos en las
proposiciones físicas es algo que el lector admitirá sin duda tras pausada reflexión.

Cuando antes describamos el movimiento de un punto material respecto a un
cuerpo de referencia, no especificábamos otra cosa que los encuentros de este punto
con determinados puntos del cuerpo de referencia. Incluso las correspondientes
especificaciones temporales se reducen a constatar encuentros del cuerpo con relojes,
junto con la constatación del encuentro de las manillas del reloj con determinados
puntos de la esfera. Y lo mismo ocurre con las mediciones espaciales con ayuda de
escalas, como se verá a poco que se reflexione.

En general, se cumple lo siguiente: toda descripción física se reduce a una serie de
proposiciones, cada una de las cuales se refiere a la coincidencia espacio-temporal de
dos sucesos $A$ y $B$. Cada una de estas proposiciones se expresa en coordenadas
gaussianas mediante la coincidencia de las cuatro coordenadas $x_{1}$, $x_{2}$, $x_{3}$,
$x_{4}$. Por tanto, es cierto que la descripción del continuo espacio-temporal a través de coordenadas
gaussianas sustituye totalmente a la descripción con ayuda de un cuerpo de referencia,
sin adolecer de los defectos de este lítimo método, pues no está ligado al carácter
euclidiano del continuo a representar.


\chapter{Formulación Exacta del Principio de la Relatividad General}

Ahora estamos en condiciones de sustituir la formulación provisional del principio de
la relatividad general que dimos en \S 18 por otra que es exacta. La versión de entonces
--Todos los cuerpos de referencia $K$, $K'$, etc., son equivalentes para la descripción de
la naturaleza (formulación de las leyes generales de la naturaleza), sea cual fuere su
estado de movimiento-- es insostenible, porque en general no es posible utilizar
cuerpos de referencia rígidos en la descripción espacio-temporal en el sentido del
método seguido en la teoría de la relatividad especial. En lugar del cuerpo de referencia
tiene que aparecer el sistema de coordenadas gaussianas. La idea fundamental del
principio de la relatividad general responde al enunciado: \textit{«Todos los sistemas de
coordenadas gaussianas son esencialmente equivalentes para la formulación de las leyes
generales de la naturaleza».}

Este principio de la relatividad general cabe enunciarlo en otra forma que permite
reconocerlo aún más claramente como una extensión natural del principio de la
relatividad especial. Según la teoría de la relatividad especial, al sustituir las variables
espacio-temporales $x$, $y$, $z$, $t$ de un cuerpo de referencia $K$ (de Galileo) por las
variables espacio-temporales $x'$, $y'$, $z'$, $t'$ de un nuevo cuerpo de referencia $K'$ utilizando
la transformación de Lorentz, las ecuaciones que expresan las leyes generales de la
naturaleza se convierten en otras de la misma forma. Por el contrario, según la teoría de
la relatividad general, las ecuaciones tienen que transformarse en otras de la misma
forma al hacer \textit{cualesquiera sustituciones} de las variables gaussianas $x_{1}$, $x_{2}$, $x_{3}$, $x_{4}$; pues
toda sustitución (y no sólo la de la transformación de Lorentz) corresponde al paso de
un sistema de coordenadas gaussianas a otro.

Si no se quiere renunciar a la habitual representación tridimensional, podemos
caracterizar como sigue la evolución que vemos experimentar a la idea fundamental de
la teoría de la relatividad general: la teoría de la relatividad especial se refiere a
regiones de Galileo, es decir, aquellas en las que no existe ningún campo gravitatorio.
Como cuerpo de referencia actúa aquí un cuerpo de referencia de Galileo, es decir,
un cuerpo rígido cuyo estado de movimiento es tal que respecto a él es válido el
principio de Galileo del movimiento rectilneo y uniforme de puntos materiales
«aislados».

Ciertas consideraciones sugieren referir esas mismas regiones de Galileo a cuerpos
de referencia no galileanos también. Respecto a éstos existe entonces un campo
gravitatorio de tipo especial (\S 20 y \S 23).

Sin embargo, en los campos gravitatorios no existen cuerpos rígidos con propiedades
euclidianas; la ficción del cuerpo de referencia rígido fracasa, pues, en la teoría de la
relatividad general. Y los campos gravitatorios también influyen en la marcha de los
relojes, hasta el punto de que una definición física del tiempo con la ayuda directa de
relojes no posee ni mucho menos el grado de evidencia que tiene en la teoría de la
relatividad especial.

Por esa razón se utilizan cuerpos de referencia no rígidos que, vistos como un todo,
no sólo tienen un movimiento arbitrario, sino que durante su movimiento sufren
alteraciones arbitrarias en su forma. Para la definición del tiempo sirven relojes cuya
marcha obedezca a una ley arbitraria y todo lo irregular que se quiera; cada uno de estos
relojes hay que imaginrselo fijo en un punto del cuerpo de referencia no rígido, y
cumplen una sola condición: la de que los datos simultneamente perceptibles en relojes
espacialmente vecinos difieran infinitamente poco entre sí. Este cuerpo de referencia
no rígido, que no sin razón cabra llamarlo «molusco de referencia», equivale en
esencia a un sistema de coordenadas gaussianas, cuadridimensional y arbitrario. Lo que
le confiere al «molusco» un cierto atractivo frente al sistema de coordenadas gaussianas
es la conservación formal (en realidad injustificada) de la peculiar existencia de
las coordenadas espaciales frente a la coordenada temporal. Todo punto del molusco es
tratado como un punto espacial; todo punto material que esté en reposo respecto a él
será tratado como en reposo, a secas, mientras se utilice el molusco como cuerpo de
referencia. El principio de la relatividad general exige que todos estos moluscos se
puedan emplear, con igual derecho y éxito parejo, como cuerpos de referencia en la
formulación de las leyes generales de la naturaleza; estas leyes deben ser totalmente
independientes de la elección del molusco.

En la profunda restricción que se impone con ello a las leyes de la naturaleza reside la
sagacidad que le es inherente al principio de la relatividad general.


\chapter{La Solución del Problema de la Gravitación Sobre la Base
del Principio de la Relatividad General}

Si el lector ha seguido todos los razonamientos anteriores, no tendrá ya dificultad
ninguna para comprender los métodos que conducen a la solución del problema de la
gravitación.

Partimos de la contemplación de una región de Galileo, es decir, de una región en la
que no existe ningún campo gravitatorio respecto a un cuerpo de referencia de Galileo
$K$. El comportamiento de escalas y relojes respecto a $K$ es ya conocido por la teoría de
la relatividad especial, lo mismo que el comportamiento de puntos materiales
aislados; estos últimos se mueven en línea recta y uniformemente.

Referimos ahora esta región a un sistema de coordenadas gaussiano arbitrario, o
bien a un «molusco», como cuerpo de referencia $K'$. Respecto a $K'$ existe entonces un
campo gravitatorio $G$ (de clase especial). Por simple conversión se obtiene así el
comportamiento de reglas y relojes, así como de puntos materiales libremente
móviles, respecto a $K'$. Este comportamiento se interpreta como el comportamiento
de reglas, relojes y puntos materiales bajo la acción del campo gravitatorio $G$. Se
introduce entonces la hipótesis de que la acción del campo gravitatorio sobre escalas,
relojes y puntos materiales libremente móviles se produce según las mismas leyes aun
en el caso de que el campo gravitatorio reinante \textit{no} se pueda derivar del caso especial
galileano por mera transformación de coordenadas.

A continuación se investiga el comportamiento espacio-temporal del campo
gravitatorio $G$ derivado del caso especial galileano por simple transformación de
coordenadas y se formula este comportamiento mediante una ley que es válida
independientemente de $c$ no se elija el cuerpo de referencia (molusco) utilizado para
la descripción.

Esta ley no es todavía la ley \textit{general} del campo gravitatorio, porque el campo
gravitatorio $G$ estudiado es de una clase especial. Para hallar la ley general del
campo gravitatorio hace falta generalizar además la ley así obtenida; no obstante, cabe
encontrarla, sin ningún género de arbitrariedad, si se tienen en cuenta los siguientes
requisitos:
\begin{enumerate}
\item La generalización buscada debe satisfacer también el postulado de la relatividad general.
\item Si existe materia en la región considerada, entonces lo único que determina su acción generadora
de un campo es su masa inercial, es decir, según \S 15, su energía únicamente.
\item Campo gravitatorio y materia deben satisfacer juntos la ley de conservación de la energa (y del
impulso).
\end{enumerate}
El principio de la relatividad general nos permite por fin determinar la influencia del
campo gravitatorio sobre la evolución de todos aquellos procesos que en ausencia de
campo gravitatorio discurren según leyes conocidas, es decir, que están incluidos ya
en el marco de la teoría de la relatividad especial. Aquí se procede esencialmente por
el método que antes analizamos para reglas, relojes y puntos materiales libremente
móviles.

La teoría de la gravitación derivada así del postulado de la relatividad general no sólo
sobresale por su belleza, no sólo elimina el defecto indicado en \S 21 y del cual adolece
la Mecánica clásica, no sólo interpreta la ley emprica de la igualdad entre masa
inercial y masa gravitatoria, sino que ya ha explicado tambin dos resultados
experimentales de la astronomía, esencialmente muy distintos, frente a los cuales
fracasa la Mecánica clásica. El segundo de estos resultados, la curvatura de los rayos
luminosos en el campo gravitatorio del Sol, ya lo hemos mencionado; el primero tiene
que ver con la órbita del planeta Mercurio.

En efecto, si se particularizan las ecuaciones de la teoría de la relatividad general al
caso de que los campos gravitatorios sean débiles y de que todas las masas se muevan
respecto al sistema de coordenadas con velocidades pequeñas comparadas con la de la
luz, entonces se obtiene la teoría de Newton como primera aproximación; así pues,
esta teoría resulta aquí sin necesidad de sentar ninguna hipótesis especial, mientras que
Newton tuvo que introducir como hipótesis la fuerza de atracción inversamente
proporcional al cuadrado de la distancia entre los puntos materiales que interactúan. Si
se aumenta la exactitud del cálculo, aparecen desviaciones respecto a la teoría de
Newton, casi todas las cuales son, sin embargo, todavía demasiado pequeñas para ser
observables.

Una de estas desviaciones debemos examinarla aquí con especial detenimiento. Según
la teoría newtoniana, los planetas se mueven en torno al Sol según una elipse que
conservara eternamente su posición respecto a las estrellas fijas si se pudiera prescindir
de la influencia de los demás planetas sobre el planeta considerado, así como del
movimiento propio de las estrellas fijas. Fuera de estas dos influencias, la órbita del
planeta debera ser una elipse inmutable respecto a las estrellas fijas, siempre que la
teoría de Newton fuese exactamente correcta. En todos los planetas, menos en Mercurio,
el más próximo al Sol, se ha confirmado esta consecuencia --que se puede
comprobar con eminente precisión-- hasta el límite de exactitud que permiten los
métodos de observación actuales. Ahora bien, del planeta Mercurio sabemos desde
Leverrier que la elipse de su órbita respecto a las estrellas fijas, una vez corregida en
el sentido anterior, no es fija, sino que rota --aunque lentísimamente-- en el plano
orbital y en el sentido de su revolución. Para este movimiento de rotación de la
elipse orbital se obtuvo un valor de 43 segundos de arco por siglo, valor que es
seguro con una imprecisión de pocos segundos de arco. La explicación de este
fenómeno dentro de la Mecánica clásica sólo es posible mediante la utilización de
hipótesis poco verosímiles, inventadas exclusivamente con este propósito.

Según la teoría de la relatividad general resulta que toda elipse planetaria alrededor
del Sol debe necesariamente rotar en el sentido indicado anteriormente, que esta
rotación es en todos los planetas, menos en Mercurio, demasiado pequeña para poder
detectarla con la exactitud de observación hoy alcanzable, pero que en el caso de
Mercurio debe ascender a 43 segundos de arco por siglo, exactamente como se había
comprobado en las observaciones.

Al margen de esto, sólo se ha podido extraer de la teoría otra consecuencia
accesible a la contrastación experimental, y es un corrimiento, espectral de la luz que
nos envían las grandes estrellas respecto a la luz generada de manera equivalente (es
decir, por la misma clase de moléculas) en la Tierra. No me cabe ninguna duda de que
también esta consecuencia de la teoría hallará pronto confirmación.


%PARTE III

\part{Consideraciones Acerca del Universo como un Todo}


\chapter{Dificultades Cosmológicas de la Teoría Newtoniana}

Aparte del problema expuesto en \S 21, la Mecánica celeste clásica adolece de una
segunda dificultad teórica que, según mis conocimientos, fue examinada detenidamente
por primera vez por el astrónomo Seeliger. Si uno reflexiona sobre la pregunta
de cómo imaginar el mundo como un todo, la respuesta inmediata será seguramente la
siguiente. El universo es espacialmente (y temporalmente) infinito. Existen estrellas
por doquier, de manera que la densidad de materia será en puntos concretos muy
diversa, pero en todas partes la misma por término medio. Expresado de otro modo:
por mucho que se viaje por el universo, en todas partes se hallará un enjambre suelto
de estrellas fijas de aproximadamente la misma especie e igual densidad.

Esta concepción es irreconciliable con la teoría newtoniana. Esta última exige más
bien que el universo tenga una especie de centro en el cual la densidad de estrellas sea
máxima, y que la densidad de estrellas disminuya de allí hacia afuera, para dar paso, más
allá todavía, a un vacío infinito. El mundo estelar debería formar una isla finita en
medio del infinito ocano del espacio\footnote{\textit{Justificación}. Según la teoría 
newtoniana, en una masa $m$ van a morir una cierta cantidad de líneas de fuerza que 
provienen del infinito y cuyo número es proporcional a la masa $m$. Si la densidad 
de masa $\rho_{0}$ en el universo es por término medio constante, entonces una esfera 
de volumen $V$ encierra por término medio la masa $\rho_{0}V$. El número de líneas de
fuerza que entran a través de la superficie $F$ en el interior de la esfera es, por 
tanto, proporcional a $\rho_{0}V$. Por unidad de superficie de la esfera entra, pues, 
un número de líneas de fuerza que es proporcional a $\rho_{0}R$. La intensidad del campo
en la superficie tendera a infinito al crecer el radio de la esfera $R$, lo cual es imposible}.

Esta representación es de por sí poco satisfactoria. Pero lo es aún menos porque de
este modo se llega a la consecuencia de que la luz emitida por las estrellas, así como
algunas de las estrellas mismas del sistema estelar, emigran ininterrumpidamente hacia
el infinito, sin que jamás regresen ni vuelvan a entrar en interacción con otros objetos
de la naturaleza. El mundo de la materia, apelotonada en un espacio finito, ira
empobrecindose entonces paulatinamente.

Para eludir estas consecuencias Seeliger modificó la ley newtoniana en el sentido
de suponer que a distancias grandes la atracción de dos masas disminuye más deprisa
que la ley de $\frac{1}{r^{2}}$. Con ello se consigue que la densidad media de la materia sea constante en todas
partes hasta el infinito, sin que surjan campos gravitatorios infinitamente grandes, con lo
cual se deshace uno de la antipática idea de que el mundo material posee una especie
de punto medio. Sin embargo, el precio que se paga por liberarse de los problemas
teóricos descritos es una modificación y complicación de la ley de Newton que no se
justifican ni experimental ni teóricamente. Cabe imaginar un número arbitrario de leyes
que cumplan el mismo propósito, sin que se pueda dar ninguna razón para que una de
ellas prime sobre las demás; porque cualquiera de ellas está tan poco fundada en
principios teóricos más generales como la ley de Newton.


\chapter{La Posibilidad de un Universo Finito y Sin Embargo No Limitado}

Las especulaciones en torno a la estructura del universo se movieron también en otra
dirección muy distinta. En efecto, el desarrollo de la geometría no euclidiana hizo ver
que es posible dudar de la \textit{infinitud} de nuestro espacio sin entrar en colisión con las
leyes del pensamiento ni con la experiencia (Riemann, Helmholtz). Estas cuestiones
las han aclarado ya con todo detalle Helmholtz y Poincar, mientras que aquí yo no
puedo hacer más que tocarlas fugazmente.

Imaginemos en primer lugar un suceso bidimensional. Supongamos que unos seres
planos, provistos de herramientas planas --en particular pequeñas reglas planas y
rígidas-- se pueden mover libremente en un \textit{plano}. Fuera de él no existe nada para
ellos; el acontecer en su plano, que ellos observan en sí mismos y en sus objetos, es un
acontecer causalmente cerrado. En particular son realizables las construcciones de la
geometría euclidiana plana con varillas, por ejemplo la construccin reticular sobre la
mesa que contemplamos en \S 24. El mundo de estos seres es, en contraposición al
nuestro, espacialmente bidimensional, pero, al igual que el nuestro, de extensión
infinita. En él tienen cabida infinitos cuadrados iguales construidos con varillas, es
decir, su volumen (superficie) es infinito. Si estos seres dicen que su mundo es
plano, no dejará de tener sentido su afirmación, a saber, el sentido de que con sus
varillas se pueden realizar las construcciones de la geometría euclidiana del plano,
representando cada varilla siempre el mismo segmento, independientemente de su
posición.

Volvamos ahora a imaginarnos un suceso bidimensional, pero no en un plano, sino en
una superficie esférica. Los seres planos, junto con sus reglas de medida y demás
objetos, yacen exactamente en esta superficie y no pueden abandonarla; todo su mundo
perceptivo se extiende única y exclusivamente a la superficie esférica. Estos seres
¿podrán decir que la geometría de su mundo es una geometra euclidiana bidimensional
y considerar que sus varillas son una realización del segmento? No pueden, porque al
intentar materializar una recta obtendrán una curva, que nosotros, seres
«tridimensionales», llamamos círculo máximo, es decir, una línea cerrada de
determinada longitud finita que se puede medir con una varilla. Este mundo tiene asimismo
una superficie finita que se puede comparar con la de un cuadrado construido
con varillas. El gran encanto que depara el sumergirse en esta reflexión reside en
percatarse de lo siguiente: \textit{el mundo de estos seres es finito y sin embargo no tiene límites.}

Ahora bien, los seres esféricos no necesitan emprender un viaje por el mundo para
advertir que no habitan en un mundo euclideano, de lo cual pueden convencerse en
cualquier trozo no demasiado pequeño de la esfera. Basta con que, desde un punto,
tracen «segmentos rectos» (arcos de circunferencia, si lo juzgamos
tridimensionalmente) de igual longitud en todas direcciones. La unión de los extremos
libres de estos segmentos la llamarán «circunferencia». La razón entre el perímetro de
la circunferencia, medido con una varilla, y el dimetro medido con la misma varilla
es igual, según la geometría euclidiana del plano, a una constante $\pi$ que es
independiente del dimetro de la circunferencia. Sobre la superficie esférica, nuestros
seres hallaran para esta razón el valor

\[\pi\frac{\sin\frac{r}{R}}{\frac{r}{R}}\]

es decir, un valor que es menor que $\pi$, y tanto menor cuanto mayor sea el radio de la
circunferencia en comparación con el radio $R$ del «mundo esférico». A partir de esta
relación pueden determinar los seres esfricos el radio $R$ de su mundo, aunque sólo
tengan a su disposición una parte relativamente pequeña de la esfera para hacer sus
mediciones. Pero si esa parte es demasiado reducida, ya no podrán constatar que se
hallan sobre un mundo esférico y no sobre un plano euclidiano, porque un trozo
pequeño de una superficie esférica difiere poco de un trozo de plano de igual tamaño.

Así pues, si nuestros seres esféricos habitan en un planeta cuyo sistema solar ocupa
sólo una parte ínfima del universo esférico, no tendrán posibilidad de decidir si viven
en un mundo finito o infinito, porque el trozo de mundo que es accesible a su
experiencia es en ambos casos prácticamente plano o euclídeo. Esta reflexión
muestra directamente que para nuestros seres esfricos el perímetro de la
circunferencia crece al principio con el radio hasta alcanzar el perímetro del
universo, para luego, al seguir creciendo el radio, disminuir paulatinamente hasta
cero. La superficie del círculo crece continuamente, hasta hacerse finalmente igual a
la superficie total del mundo esférico entero.

Al lector quizá le extrañe que hayamos colocado a nuestros seres precisamente sobre
una esfera y no sobre otra superficie cerrada. Pero tiene su justificación, porque la
superficie esférica se caracteriza, frente a todas las demás superficies cerradas, por la
propiedad de que todos sus puntos son equivalentes. Es cierto que la relación entre el
perímetro $p$ de una circunferencia y su radio $r$ depende de $r$; pero, dado $r$, es igual para
todos los puntos del mundo esférico. El mundo esférico es una «superficie de
curvatura constante».

Este mundo esférico bidimensional tiene su homólogo en tres dimensiones, el espacio
esférico tridimensional, que fue descubierto por Riemann. Sus puntos son también
equivalentes. Posee un volumen finito, que viene determinado por su radio$R$
($2\pi^{2}R^{3}$). ¿Puede uno imaginarse un espacio esférico? Imaginarse un espacio no quiere
decir otra cosa que imaginarse un modelo de experiencias «espaciales», es decir, de experiencias
que se pueden tener con el movimiento de cuerpos «rígidos». En este
sentido sí que cabe imaginar un espacio esférico.

Desde un punto trazamos rectas (tensamos cuerdas) en todas direcciones y
marcamos en cada una el segmento $r$ con ayuda de la regla de medir. Todos los
extremos libres de estos segmentos yacen sobre una superficie esfrica. Su área ($A$)
podemos medirla con un cuadrado hecho con reglas. Si el mundo es euclidiano,
tendremos que $A=4\pi R^{2}$; si el mundo es esfrico, entonces $A$ será siempre menor que
$4\pi R^{2}$. $A$ aumenta con $r$ desde cero hasta un máximo que viene determinado por el
radio del universo, para luego disminuir otra vez hasta cero al seguir creciendo el
radio de la esfera $r$. Las rectas radiales que salen del punto origen se alejan al
principio cada vez más unas de otras, vuelven a acercarse luego y convergen otra vez
en el punto opuesto al origen; habrán recorrido entonces todo el espacio esférico. Es
fácil comprobar que el espacio esférico tridimensional es totalmente análogo al
bidimensional (superficie esférica). Es finito (es decir, de volumen finito) y no tiene
límites.

Señalemos que existe también una subespecie del espacio esférico: el «espacio
elíptico». Cabe concebirlo como un espacio esférico en el que los «puntos opuestos»
son idénticos (no distinguibles). Así pues, un mundo elíptico cabe contemplarlo, en
cierto modo, como un mundo esférico centralmente simtrico.

De lo dicho se desprende que es posible imaginar espacios cerrados que no tengan
límites. Entre ellos destaca por su simplicidad el espacio esfrico (o el elíptico), cuyos
puntos son todos equivalentes. Según todo lo anterior, se les plantea a los
astrónomos y a los físicos un problema altamente interesante, el de si el mundo en
que vivimos es infinito o, al estilo del mundo esférico, finito. Nuestra experiencia
no basta ni de lejos para contestar a esta pregunta. La teoría de la relatividad general
permite, sin embargo, responder con bastante seguridad y resolver de paso la
dificultad explicada en \S 30.


\chapter{La Estructura del Espacio Según la Teoría de la Relatividad General}

Según la teoría de la relatividad general, las propiedades geométricas del espacio no
son independientes, sino que vienen condicionadas por la materia. Por eso no es
posible inferir nada sobre la estructura geométrica del mundo a menos que la
reflexión se funde en el conocimiento del estado de la materia. Sabemos, por la
experiencia, que con una elección conveniente del sistema de coordenadas las
velocidades de las estrellas son pequeñas frente a la velocidad de propagación de la
luz. Así pues, si suponemos que la materia está en reposo, podremos conocer la
estructura del universo en una primera y tosqusima aproximación.

Por anteriores consideraciones sabemos ya que el comportamiento de reglas de medir
y relojes viene influido por los campos de gravitación, es decir, por la distribución de
la materia. De aquí se sigue ya que la validez exacta de la geometría euclidiana en
nuestro mundo es algo que no entra ni siquiera en consideración. Pero en sí es
concebible que nuestro mundo difiera poco de un mundo euclidiano, idea que viene
abonada por el hecho de que, según los cálculos, incluso masas de la magnitud de
nuestro Sol influyen mínimamente en la métrica del espacio circundante. Cabra
imaginar que nuestro mundo se comporta en el aspecto geométrico como una
superficie que está irregularmente curvada pero que en ningún punto se aparta
significativamente de un plano, lo mismo que ocurre, por ejemplo, con la superficie
de un lago rizado por débiles olas. A un mundo de esta especie podríamos llamarlo
con propiedad cuasi-euclidiano, y sera espacialmente infinito. Los cálculos indican, sin
embargo, que en un mundo cuasi-euclidiano la densidad media de materia tendra que
ser nula. Por consiguiente, un mundo semejante no podra estar poblado de materia
por doquier; ofrecera el cuadro insatisfactorio que dibujamos en \S 30.

Si la densidad media de materia en el mundo no es nula (aunque se acerque mucho
a cero), entonces el mundo no es cuasi-euclidiano. Los cálculos demuestran más
bien que, con una distribución uniforme de materia, debera ser necesariamente
esférico (o elíptico). Dado que la materia está distribuida de manera localmente no
uniforme, el mundo real diferirá localmente del comportamiento esférico, es
decir, será cuasi-esférico. Pero necesariamente tendrá que ser finito. La teoría
proporciona incluso una sencilla relación entre la extensión espacial del mundo y la
densidad media de materia en él.\footnote{Para el «radio» $R$ del mundo se obtiene la
ecuación

\[R^{2}=\frac{2}{\kappa \rho}\]

Utilizando el sistema cegesimal, tenemos que $2/\kappa=1,08\cdot10^{27}$;
$\rho$ es la densidad media de materia y $\kappa$ es una constante ligada con la constante
gravitacional de Newton.}


\part{Apéndice}

\appendix

%APPENDICE I
\chapter{Una Derivación Sencilla de la Transformación de Lorentz (Anexo a 11)}

Con la orientación relativa de los sistemas de coordenadas indicada en la Fig.\ref{fig:2}, los
ejes de abscisas de los dos sistemas coinciden constantemente. Aquí podemos desglosar
el problema y considerar primero únicamente sucesos que estén localizados en el eje
de las $X$. Un suceso semejante viene dado, respecto al sistema de coordenadas $K$,
por la abscisa $x$ y el tiempo $t$, y respecto a $K'$ por la abscisa $x'$ y el tiempo $t'$. Se trata
de hallar $x'$ y $t'$ cuando se conocen $x$ y $t$.

Una señal luminosa que avanza a lo largo del eje $X$ positivo se propaga según la
ecuación

\[x=ct\]
o bien,

\begin{equation}
x-ct=0\label{eqn:a1}
\end{equation}

Dado que la misma señal luminosa debe propagarse, también respecto a $K'$, con la
velocidad $c$, la propagación respecto a $K'$ vendrá descrita por la fórmula análoga

\begin{equation}
x'-ct'=0\label{eqn:a2}
\end{equation}

Aquellos puntos del espacio-tiempo (sucesos) que cumplen (\ref{eqn:a1}) tienen que cumplir
también (\ref{eqn:a2}), lo cual será el caso cuando se cumpla en general la relación

\begin{equation}
(x'-ct')=\lambda(x-ct)\label{eqn:a3}
\end{equation}

\noindent donde $\lambda$ es una constante; pues, según (\ref{eqn:a3}), la anulación de $(x-ct)$ 
conlleva la de $(x'-ct')$.

Un razonamiento totalmente análogo, aplicado a rayos de luz que se propaguen a
lo largo del eje $X$ negativo, proporciona la condición

\begin{equation}
(x'+ct')=\mu(x+ct)\label{eqn:a4}
\end{equation}

Si se suman y restan, respectivamente, las ecuaciones (\ref{eqn:a3}) y (\ref{eqn:a4}) introduciendo por
razones de comodidad las constantes 

\[a=\frac{\lambda+\mu}{2}\]

\noindent y

\[a=\frac{\lambda-\mu}{2}\]

\noindent en lugar de las constantes $\lambda$ y $\mu$, se obtiene

\begin{equation}
\left.\begin{array}{rcl}
x' & = & ax-bct\\
ct' & = & act-bx\end{array}\right\} \label{eqn:a5}
\end{equation}

Con ello quedara resuelto el problema, siempre que conozcamos las constantes $a$
y $b$; éstas resultan de las siguientes consideraciones.

Para el origen de $K'$ se cumple constantemente $x'=0$ de manera que, por la
primera de las ecuaciones (\ref{eqn:a5}):

\[x=\frac{bc}{a}t\]

Por tanto, si llamamos $v$ a la velocidad con que se mueve el origen de $K'$
respecto a $K$, tenemos que

\begin{equation}
v=\frac{bc}{a}\label{eqn:a6}
\end{equation}

El mismo valor de $v$ se obtiene a partir de (\ref{eqn:a5}), al calcular la velocidad de otro
punto de $K'$ respecto a $K$ o la velocidad (dirigida hacia el eje $X$ negativo) de un
punto $K$ respecto a $K'$. Por tanto, es posible decir en resumen que $v$ es la velocidad
relativa de ambos sistemas.

Además, por el principio de la relatividad, está claro que la longitud, juzgada
desde $K$, de una regla de medir unitaria que se halla en reposo respecto a $K'$ tiene
que ser exactamente la misma que la longitud, juzgada desde $K'$, de una regla unidad
que se halla en reposo respecto a $K$. Para ver qué aspecto tienen los puntos del eje
$X'$ vistos desde $K$ basta con tomar una «fotografía instantánea» de $K'$ desde $K$; lo
cual significa dar a $t$ (tiempo de $K$) un valor determinado, p. ej. $t=0$. De la primera
de las ecuaciones (\ref{eqn:a5}) se obtiene:

\[x'=ax\]

Así pues, dos puntos del eje $X'$ que medidos en $K'$ distan entre sí $\Delta x'=I$, tienen en
nuestra instantnea la separación:

\begin{equation}
\Delta x=\frac{I}{a}\label{eqn:a7}
\end{equation}

\noindent Pero si se toma la fotografa desde $K'(t'=0)$, se obtiene a partir de (\ref{eqn:a5}), por
eliminación de $t$ y teniendo en cuenta (\ref{eqn:a6}):

\[x'=a\left(I-\frac{v^{2}}{c^{2}}\right)x\]


\noindent   De aquí se deduce que dos puntos del eje $X$ que distan 1 (respecto a $K$) tienen en
nuestra instantnea la separación
\begin{equation}
\Delta x'=a\left(I-\frac{v^{2}}{c^{2}}\right)\label{eqn:a8}
\end{equation}

Teniendo en cuenta que, por lo que llevamos dicho, las dos fotografas deben ser
iguales, $\Delta x$ en (\ref{eqn:a7}) tiene que ser igual a $\Delta x'$ en (\ref{eqn:a8}), de modo que se obtiene:

\begin{equation}
a=\frac{I}{I-\frac{v^{2}}{c^{2}}}\label{eqn:a9}
\end{equation}

Las ecuaciones (\ref{eqn:a6}) y (\ref{eqn:a9}) determinan las constantes $a$ y $b$. Sustituyendo en (\ref{eqn:a5}) se
obtienen las ecuaciones cuarta y quinta de las que dimos en \S 11.

\begin{equation}
\left.\begin{array}{rcl}
x' & = & \frac{x-vt}{\sqrt{I-\frac{v^{2}}{c^{2}}}}\\
~\\
t' & = & \frac{t-\frac{v}{c^{2}}x}{\sqrt{I-\frac{v^{2}}{c^{2}}}}\end{array}\right\} \label{eqn:a10}
\end{equation}

Con ello hemos obtenido la transformación de Lorentz para sucesos localizados
en el eje $X$; dicha transformacin satisface la condición

\begin{equation}
x'^{2}-c^{2}t'^{2}=x^{2}-c^{2}t^{2}\label{eqn:a11}
\end{equation}

La extensión de este resultado a sucesos que ocurren fuera del eje $X$ se obtiene
reteniendo las ecuaciones (\ref{eqn:a10}) y añadiendo las relaciones

\begin{equation}
\left.\begin{array}{rcl}
y' & = & y\\
z' & = & z\end{array}\right\} \label{eqn:a12}
\end{equation}

Veamos ahora que con ello se satisface el postulado de la constancia de la velocidad
de la luz para rayos luminosos de dirección arbitraria, tanto para el sistema $K$ como
tambin para el $K'$.

Supongamos que en el instante $t=0$ se emite una señal luminosa desde el origen
de $K$. Su propagación obedece a la ecuación:

\[r=\sqrt{x^{2}+y^{2}+z^{2}}=ct\]

\noindent o bien, elevando al cuadrado

\begin{equation}
x^{2}+y^{2}+z^{2}=c^{2}t^{2}=0\label{eqn:a13}
\end{equation}

La ley de propagación de la luz, en conjunción con el postulado de la relatividad, exige
que la propagación de esa misma señal, pero juzgada desde $K'$, ocurra según la
fórmula correspondiente

\[r'=ct'\]

\noindent o bien,

\begin{equation}
x'^{2}+y'^{2}+z'^{2}-c^{2}t'^{2}=0\label{eqn:a14}
\end{equation}

Para que la ecuación (\ref{eqn:a14}) sea una consecuencia de (\ref{eqn:a13}), tiene que cumplirse que:

\begin{equation}
x'^{2}+y'^{2}+z'^{2}-c^{2}t'^{2}=\sigma(x^{2}+y^{2}+z^{2}-c^{2}t^{2})\label{eqn:a15}
\end{equation}

Puesto que la ecuacin (\ref{eqn:a10}) tiene que cumplirse para los puntos situados sobre el eje $X$,
ha de ser $\sigma=1$. Es fácil ver que la transformación de Lorentz cumple realmente la
ecuación (\ref{eqn:a15}) con $\sigma=1$, pues (\ref{eqn:a15}) es una consecuencia de (\ref{eqn:a10}) y (\ref{eqn:a12}), y por tanto tambin de (\ref{eqn:a10}) y (\ref{eqn:a12}). Con ello queda derivada la transformación de Lorentz.

Es preciso ahora generalizar esta transformación de Lorentz, representada por (\ref{eqn:a10}) y (\ref{eqn:a12}).
Evidentemente es inesencial que los ejes de $K'$ se elijan espacialmente paralelos a los de
$K$. Tampoco es esencial que la velocidad de traslación de $K'$ respecto a $K$ tenga la dirección
del eje $X$. La transformación de Lorentz, en este sentido general, cabe desglosarla
--como muestra un simple razonamiento-- en dos transformaciones; A saber:
transformaciones de Lorentz en sentido especial y transformaciones puramente
espaciales que equivalen a la sustitución del sistema de coordenadas rectangulares por
otro con ejes dirigidos en direcciones distintas.

Matemáticamente se puede caracterizar la transformación de Lorentz generalizada de la
siguiente manera:

Dicha transformación expresa $x'$, $y'$, $x'$, $t'$ mediante unas funciones homogneas y
lineales de $x$, $y$, $x$, $t$ que hacen que la relación

\begin{equation}
x'^{2}+y'^{2}+z'^{2}-c^{2}t'^{2}=x^{2}+y^{2}+z^{2}-c^{2}t^{2}\label{eqn:a16}
\end{equation}

\noindent se cumpla idénticamente. Lo cual quiere decir: si se sustituye a la izquierda $x'$, $y'$, $x'$, $t'$, etc.
por sus expresiones en $x$, $y$, $x$, $t$, entonces el miembro izquierdo de (\ref{eqn:a16}) es igual
al derecho.


% APPENDICE II
\chapter{El Mundo Cuadridimensional de Minkowski (Anexo a 17)}

La transformación de Lorentz generalizada puede caracterizarse de un modo aún más
sencillo si en lugar de $t$ se introduce como variable temporal la variable imaginaria
$\sqrt{-I}\cdot ct$ Si de acuerdo con esto ponemos

\begin{eqnarray*}
x_{1} & = & x\\
x_{2} & = & y\\
x_{3} & = & z\\
x_{4} & = & \sqrt{-I}\cdot ct
\end{eqnarray*}

y anlogamente para el sistema con primas $K'$, entonces la condicin que satisface
idnticamente la transformación será:

\begin{equation}
{x'_{1}}^{2}+{x'}_{2}^{2}+{x'}_{3}^{2}+{x'}_{4}^{2}=x_{1}^{2}+x_{2}^{2}+x_{3}^{2}+x_{4}^{2}\label{eqn:b1}
\end{equation}

\noindent Con la elección de «coordenadas» que acabamos de indicar, la ecuación (\ref{eqn:a16})
se convierte en la (\ref{eqn:b1}).

De (\ref{eqn:b1}) se desprende que la coordenada temporal imaginaria $x_{4}$ entra en la condición
de transformación en pie de igualdad con las coordenadas espaciales $x_{1}$, $x_{2}$, $x_{3}$. A eso
responde el que, según la teoría de la relatividad, el «tiempo» $x_{4}$ intervenga en las
leyes de la naturaleza en la misma forma que las coordenadas espaciales $x_{1},x_{2},x_{3}$

Minkowski llamó «universo» o «mundo» al continuo cuadridimensional descrito por
las «coordenadas» $x_{1}$, $x_{2}$, $x_{3}$, $x_{4}$, y «punto del universo» o «punto del mundo» al
suceso puntual. La física deja de ser un \textit{suceder} en el espacio tridimensional para
convertirse en cierto modo en un ser en el «mundo» cuadridimensional.

Este «mundo» cuadridimensional guarda un profundo parecido con el «espacio»
tridimensional de la geometría analtica (euclídea). Pues si en este último se introduce
un nuevo sistema de coordenadas cartesianas ($x'_{1}$, $x'_{2}$, $x'_{3}$) con el mismo origen,
entonces $x'_{1}$, $x'_{2}$, $x'_{3}$ son funciones homogéneas y lineales de $x_{1}$, $x_{2}$, $x_{3}$ que cumplen
idénticamente la ecuación

\[{x'}_{1}^{2}+{x'}_{2}^{2}+{x'}_{3}^{2}=x_{1}^{2}+x_{2}^{2}+x_{3}^{2}\]

La analogía con (12) es completa. El mundo de Minkowski cabe contemplarlo
formalmente como un espacio euclídeo cuadridimensional (con coordenada temporal
imaginaria); la transformación de Lorentz se corresponde con una rotación del
sistema de coordenadas en el «universo» cuadridimensional.


%APPENDICE III
\chapter{Sobre la Confirmacin de la Teoría de la Relatividad General por la
Experiencia}

Bajo una óptica epistemológica esquemática, el proceso de crecimiento de una ciencia
experimental aparece como un continuo proceso de inducción. Las teorías emergen
como resúmenes de una cantidad grande de experiencias individuales en leyes
empíricas, a partir de las cuales se determinan por comparación las leyes generales.
Desde este punto de vista, la evolución de la ciencia parece análoga a una obra de
catalogación o a un producto de mera empiria.

Esta concepción, sin embargo, no agota en modo alguno el verdadero proceso, pues
pasa por alto el importante papel que desempeñan la intuición y el pensamiento
deductivo en el desarrollo de la ciencia exacta. En efecto, tan pronto como una
ciencia sobrepasa el estadio más primitivo, los progresos teóricos no nacen ya de una
simple actividad ordenadora. El investigador, animado por los hechos experimentales,
construye más bien un sistema conceptual que se apoya lógicamente en un número por
lo general pequeño de supuestos básicos que se denominan axiomas. A un sistema
conceptual semejante lo llamamos teoría. La teoría obtiene la justificación de su
existencia por el hecho de conectar entre sí un número grande de experiencias
aisladas; en esto reside su verdad.

Frente a un mismo complejo de hechos de la experiencia puede haber diversas
teorías que difieran mucho entre sí. La coincidencia de las teorías en las consecuencias
accesibles a la experiencia puede ser tan profunda que resulte difcil encontrar otras,
también accesibles a la experiencia, respecto a las cuales difieran. Un caso semejante,
y de interés general, se da por ejemplo en el terreno de la biología, en la teoría
darwiniana de la evolución por selección en la lucha por la existencia y en aquella
otra teoría de la evolución que se funda en la hipótesis de la herencia de caracteres
adquiridos.

Otro caso semejante de profunda concordancia de las consecuencias es el de la
mecánica newtoniana, por un lado, y la teoría de la relatividad general, por otro. La
concordancia llega hasta tal punto que hasta ahora se han podido encontrar muy pocas
consecuencias de la teora de la relatividad general a las cuales no conduzca también la
física anterior, y eso a pesar de la radical diversidad de los supuestos básicos de una y
otra teoría. Vamos a contemplar aquí de nuevo estas importantes consecuencias y
comentar también brevemente las experiencias acumuladas hasta ahora al respecto.


\section{El Movimiento del Perihelio de Mercurio}

Según la mecánica newtoniana y la ley de gravitación de Newton, un único planeta
que girara en torno a un sol describira una elipse alrededor de él (o más exactamente,
alrededor del centro de gravedad común de ambos). El sol (o bien el centro de
gravedad común) yace en uno de los focos de la elipse orbital, de manera que la
distancia sol-planeta crece a lo largo de un año planetario hasta un máximo, para luego
volver a decrecer hasta el mnimo. Si en lugar de la ley de atracción newtoniana se
introduce en los cálculos otra distinta, entonces se comprueba que el movimiento
según esta nueva ley tendra que seguir siendo tal que la distancia sol-planeta
oscilase en un sentido y otro; pero el ángulo descrito por la línea sol-planeta
durante uno de esos períodos (de perihelio a perihelio) diferira de $360^{\circ}$. La curva de
la órbita no sera entonces cerrada, sino que llenara con el tiempo una porción anular
del plano orbital (entre el círculo de máxima y el de mínima distancia perihélica).

Según la teoría de la relatividad general, que difiere algo de la newtoniana, tiene
que haber también una pequeña desviación de esta especie respecto al movimiento
orbital previsto por Kepler-Newton, de manera que el ángulo descrito por el radio
sol-planeta entre un perihelio y el siguiente difiera de un ángulo completo de
rotacin (es decir, del ngulo $2\pi$, en la medida angular absoluta que es habitual en
física) en la cantidad

\[+\frac{24\pi^{3}a^{2}}{T^{2}e^{2}(I-e^{2})}\]

\noindent ($a$ es el semieje mayor de la elipse, $e$ su excentricidad, $c$ la velocidad de la luz, $T$ el
período de revolución). Expresado de otra manera: según la teora de la relatividad
general, el eje mayor de la elipse rota alrededor del Sol en el sentido del movimiento
orbital. Esta rotación es, de acuerdo con la teoría, de 43 segundos de arco cada 100
años en el caso del planeta Mercurio, mientras que en los demás planetas de nuestro
Sol sera tan pequeña que escapa a toda constatación.

Los astrónomos han comprobado efectivamente que la teoría de Newton no basta
para calcular el movimiento observado de Mercurio con la precisión que pueden
alcanzar hoy da las observaciones. Tras tener en cuenta todas las influencias
perturbadoras que ejercen los demás planetas sobre Mercurio, se comprobó
(Leverrier, 1859, y Newcomb, 1895) que en el movimiento del perihelio de la órbita
de Mercurio quedaba sin explicar una componente que no difiere perceptiblemente de
los +43 segundos por siglo que acabamos de mencionar. La imprecisin de este resultado
emprico, que concuerda con el resultado de la teoría general de la relatividad,
es de pocos segundos.


\section{La Desviación de la luz por el campo gravitacional}

En \S 22 explicamos que, según la teoría de la relatividad general, cualquier rayo de
luz tiene que experimentar en el seno de un campo gravitacional una curvatura que es
análoga a la que experimenta la trayectoria de un cuerpo al lanzarlo a través de ese
campo. De acuerdo con la teoría, un rayo de luz que pase al lado de un cuerpo celeste
sufrirá una desviacin hacia él; el ángulo de desviación $\alpha$, para un rayo luminoso
que pase a una distancia de $\Delta$ radios solares del Sol, debe ser de

\[\alpha=\frac{1.7\mbox{segundos de arco}}{\Delta}\]

Añadamos que, de acuerdo con la teoría, la mitad de esta desviación es producto del
campo de atracción (newtoniano) del Sol; la otra mitad, producto de la modificación
geométrica (curvatura) del espacio provocada por aquél.

Este resultado brinda la posibilidad de una comprobación experimental mediante
fotografas estelares tomadas durante un eclipse total de Sol. Es necesario esperar a este
fenómeno porque en cualquier otro momento la atmósfera, iluminada por la luz solar,
resplandece tanto que las estrellas próximas al Sol resultan invisibles. El fenmeno
esperado se deduce fácilmente de la Fig. \ref{fig:5}.

%figura 5

%              / D1
%             /
%        /   /
%       /   /
%      /   /
%      /D /
% S( )/--/
%     / /
% D1 / / D2
%    //
%    /
%  _/
% 
\begin{figure}[hbtp]
 \centering
\caption{}
\label{fig:5}
\begin{picture}(110,250)(0,30) \thicklines \put(38,138){\circle{15}}
\put(22,135){S}
\multiput(5,45)(15,15){2}{\line(1,1){10}} \multiput(30,70)(5,20){6}{\line(1,4){3}}
\multiput(30,70)(10,20){4}{\line(1,2){5}} \multiput(70,150)(5,20){6}{\line(1,4){3}}
\put(40,90){\vector(1,2){5}} \put(35,90){\vector(1,4){3}}
\put(90,230){\vector(1,4){3}}
\put(15,100){$R_{1}$} \put(50,90){$R_{2}$} \put(100,230){$R_{1}$}
\put(45,135){\line(3,-1){15}} \put(50,137){$\alpha$}
\end{picture} 
\end{figure}

Si no existiese el Sol Sí, cualquier estrella situada a distancia prácticamente infinita se
víera en la direccin $R_{1}$. Pero como consecuencia de la desviacin provocada por el Sol
se la ve en la dirección $R_{2}$, es decir, separada del centro del Sol un poco más de lo que
en realidad está.

La prueba se desarrolla en la práctica de la siguiente manera. Durante un eclipse
de Sol se fotografían las estrellas situadas en las inmediaciones de aquél. Se toma
además una segunda fotografía de las mismas estrellas cuando el Sol se halla en otro
lugar del cielo (es decir, algunos meses antes o después). Las imgenes estelares
fotografiadas durante el eclipse de Sol deben estar entonces desplazadas
radialmente hacia afuera (alejndose del centro del Sol) respecto a la fotografía de
referencia, correspondiendo el desplazamiento al ángulo $\alpha$.

Hemos de agradecer a la Astronomical Royal Society la contrastación de este
importante resultado. Sin dejarse turbar por la guerra ni por las consiguientes dificultades
de índole psicolgica, envió a varios de sus astrónomos más destacados
(Eddington, Crommelin, Davidson) y organizó dos expediciones con el fin de hacer
las fotografías pertinentes durante el eclipse de Sol del 29 de mayo de 1919 en
Sobral (Brasil) y en la isla Prncipe (África occidental). Las desviaciones relativas
que eran de esperar entre las fotografías del eclipse y las de referencia ascendían
tan sólo a unas pocas centsimas de milmetro. Así pues, las demandas que se
impuso a la precisión de las fotografías y a su medición no eran pequeñas.
El resultado de la medición confirmá la teoría de manera muy satisfactoria. Las
componentes transversales de las desviaciones estelares observadas y calculadas (en
segundos de arco) se contienen en la siguiente tabla:

%Tabla 01:
\[
\begin{array}{r|rr|rr}
\mbox{Numero de la Estrella} & \mbox{Primera} & \mbox{Coordenada} & \mbox{Segunda} & \mbox{Coordenada}\\
\hline  & \mbox{Observada} & \mbox{Calculada} & \mbox{Observada} & \mbox{Calculada}\\
11 & -0'19 & -0'22 & +0'16 & +0'02\\
5 & +0'29 & +0'31 & -0'46 & -0'43\\
4 & +0'11 & +0'10 & +0'83 & +0'73\\
3 & +0'22 & +0'12 & +1'00 & +0'87\\
6 & +0'10 & +0'04 & +0'57 & +0'40\\
10 & -0'08 & +0'09 & +0'35 & +0'32\\
2 & +'095 & +0'85 & -0'27 & -0'09
\end{array}
\]


\section{El Corrimiento al Rojo de las Rayas Espectrales}

En \S 23 se demuestra que en un sistema $K'$ que rota respecto a un sistema de
Galileo $K$, la velocidad de marcha de relojes en reposo y de idéntica constitución
depende de la posición. Vamos a examinar cuantitativamente esta dependencia. Un
reloj colocado a distancia $r$ del centro del disco tiene, respecto a $K$, la velocidad

\[V=\omega r\]

\noindent donde $\omega$ designa la velocidad de rotación del disco ($K'$) respecto a $K$. Si llamamos $v_{0}$ al
número de golpes del reloj por unidad de tiempo (velocidad de marcha) respecto a $K$
cuando el reloj está en reposo, entonces la velocidad de marcha $v$ del reloj cuando se
mueve con velocidad $v$ respecto a $K$ y está en reposo respecto al disco es, según \S 12,

\[v=v_{2}\sqrt{I-\frac{v^{2}}{c^{2}}}\]

\noindent que se puede escribir también, con suficiente precisión, así

\[v=v_{0}\left(I-\frac{1}{2}\frac{v^{2}}{c^{2}}\right)\]

\noindent o bien:

\[v=v_{0}\left(I-\frac{1}{c^{2}}\frac{\omega^{2}r^{2}}{2}\right)\]

Si llamamos $+\phi$ a la diferencia de potencial de la fuerza centrífuga entre el
lugar que ocupa el reloj y el punto medio del disco, es decir, al trabajo (con signo
negativo) que hay que aportar en contra de la fuerza centrífuga a la unidad de masa para
transportarla desde su posición en el disco móvil hasta el centro, entonces tenemos que

\[\phi=\frac{\omega^{2}r^{2}}{2}\]

\noindent Con lo cual resulta

\[v=v_{0}\left(I+\frac{\phi}{c^{2}}\right)\]

De aquí se desprende en primer lugar que dos relojes idénticos pero colocados a
diferente distancia del centro del disco marchan a distinta velocidad, resultado que
también es válido desde el punto de vista de un observador que gire con el disco.

Dado que --juzgado desde el disco-- existe un campo gravitacional cuyo
potencial es $\phi$, el resultado obtenido valdrá para campos gravitacionales en general. Y
como además un átomo que emite rayas espectrales es posible considerarlo como un
reloj, tenemos el siguiente teorema:

\textit{Un átomo absorbe o emite una frecuencia que depende del potencial del campo gravitatorio
en el que se encuentra.}

La frecuencia de un átomo que se halle en la superficie de un cuerpo celeste es algo
menor que la de un átomo del mismo elemento que se encuentre en el espacio libre (o
en la superficie de otro astro menor). 

Dado que $\phi=-K(M/r)$, donde $K$ es la constante de gravitacin newtoniana, $M$ la masa y $r$ 
el radio del cuerpo celeste, debera producirse un corrimiento hacia el rojo en las rayas espectrales
generadas en la superficie de las estrellas si se las compara con las generadas en la
superficie de la Tierra, concretamente en la cuantía

\[\frac{v_{0}-v}{v_{0}}=\frac{K}{c^{2}}\frac{M}{r}\]

En el Sol, el corrimiento al rojo que debera esperarse es de unas dos millonsimas de
longitud de onda. En el caso de las estrellas fijas no es posible hacer un cálculo fiable,
porque en general no se conoce ni la masa $M$ ni el radio $r$.

Que este efecto exista realmente o no es una cuestión abierta en cuya solución
trabajan actualmente con gran celo los astrónomos. En el caso del Sol es difcil juzgar
la existencia del efecto por ser muy pequeño. Mientras que Grebe y Bachem (Bonn)
--sobre la base de sus propias mediciones y de las de Evershed y Schwarzschild en la
así llamada banda cyan-- así como Perot (sobre la base de observaciones propias)
consideran probada la existencia del efecto, otros investigadores, especialmente W. H.
Julius y S. Sohn, son de la opinión contraria o no están convencidos de la fuerza
probatoria del anterior material empírico.

En las investigaciones estadísticas realizadas sobre las estrellas fijas no hay duda de que
existen por término medio corrimientos de las rayas espectrales hacia el extremo de las
ondas largas del espectro. Sin embargo, la elaboración que se ha hecho hasta ahora del
material no permite todava ninguna decisión acerca de si esos movimientos se deben
realmente al efecto de la gravitación. El lector podrá encontrar en el trabajo de E.
Freundlich Prfung der allgemeinen Relativittstheorie (\textit{Die Naturwissenschaften},
1919, H. 35, p. 520, Verlag Jul. Spinger, Berlín) una recopilación del material
emprico, junto a un análisis detenido desde el punto de vista de la cuestión que
aquí nos interesa.

En cualquier caso, los años venideros traerán la decisión definitiva. Si no
existiese ese corrimiento al rojo de las rayas espectrales debido al potencial
gravitatorio, la teoría de la relatividad general sería insostenible. Por otro lado, el
estudio del corrimiento de las rayas espectrales, caso de que se demuestre que su
origen está en el potencial gravitatorio, proporcionará conclusiones importantes sobre
la masa de los cuerpos celestes.


%APPENDICE IV
\chapter{La estructura del espacio en conexión con la teoría de la relatividad
general (Anexo a 32)}

Nuestro conocimiento sobre la estructura global del espacio (problema
cosmológico) ha experimentado, desde la aparición de la primera edición de este
librito, una evolución importante, que es preciso mencionar incluso en una exposición
de carcter divulgativo.

Mis iniciales consideraciones sobre este problema se basaban en dos hipótesis:
\begin{enumerate}
\item La densidad media de materia en todo el espacio
es distinta de 0 e igual en todas partes.
\item La magnitud (o el radio) del universo es independiente del tiempo.
\end{enumerate}

Estas dos hipótesis demostraron ser compatibles según la teoría de la relatividad
general, pero únicamente cuando se añadía a las ecuaciones de campo un término
hipotético que ni era exigido por la propia teoría ni tampoco parecía natural desde el
punto de vista teórico (término cosmológico de las ecuaciones de campo).

La hipótesis 2 me parecía a la sazón inevitable, pues por aquel entonces pensaba que,
de apartarse de ella, se caería en especulaciones sin límite.

Sin embargo, el matemático ruso Friedman descubrió, allí por los años veinte, que
desde el punto de vista puramente teórico era más natural otro supuesto diferente. En
efecto, Friedman se dio cuenta de que era posible mantener la hipótesis 1 sin
introducir en las ecuaciones de campo de la gravitación el poco natural término
cosmológico, siempre que uno se decidiese a prescindir de la hipótesis 2. Pues las
ecuaciones de campo originales admiten una solución en la que el radio del mundo
depende del tiempo (espacio en expansión). En este sentido cabe afirmar con Friedman
que la teoría exige una expansión del espacio.

Hubble demostró pocos años después, a través de sus investigaciones espectrales
en nebulosas extragalácticas, que las rayas espectrales emitidas por ellas muestran un
corrimiento al rojo que crece regularmente con la distancia de la nebulosa. Según los
conocimientos actuales, este corrimiento sólo cabe interpretarlo, en el sentido del
principio de Doppler, como un movimiento de expansión del sistema estelar entero,
tal y como, según el estudio de Friedman, exigen las ecuaciones de campo de la
gravitación. Así pues, en este sentido el descubrimiento de Hubble puede interpretarse
como una confirmación de la teoría.

Plantéase aquí, sin embargo, una curiosa dificultad. La interpretación (teóricamente
casi indudable) de los corrimientos de las rayas galácticas hallados por Hubble como
una expansión obliga a situar el origen de ésta hace tan sólo unos $10^{9}$ años,
mientras que la astronomía física tiene por probable que la evolución de las estrellas
y de los sistemas estelares necesitá tiempos mucho mayores. Hoy por hoy no está ni
mucho menos claro cómo superar estas incongruencias.

Señalemos también que la teoría del universo en expansión, junto con los datos
empricos de la astronomía, no permite ninguna decisión acerca de la finitud o
infinitud del espacio (tridimensional), mientras que la hipótesis estática original del
espacio había predicho un carcter cerrado (finitud) para el espacio.


\chapter{La Relatividad y el Problema del Espacio}

Es característico de la teoría de Newton el que tenga que atribuir al espacio y al
tiempo, y también a la materia, una existencia real independiente. Pues en la ley de
movimiento newtoniana aparece el concepto de aceleración, y la aceleración, en esta
teoría, sólo puede significar «aceleración respecto al espacio». El espacio
newtoniano hay que imaginárselo «en reposo», o al menos «no acelerado», para
que la aceleración que aparece en la ley del movimiento pueda contemplarse
como una magnitud con sentido. Y análogamente para el tiempo, que también
entra en el concepto de aceleración. El propio Newton, y aquellos de sus
coetáneos que gozaban de más sentido crítico, veían como algo perturbador el
hecho de tener que adscribir realidad física al espacio mismo y a su estado de
movimiento. Pero por aquel entonces no había otra salida si se quería atribuir a la
Mecánica un sentido claro.

El atribuir realidad física al espacio, y en especial al espacio vacío, es ya de por sí una
dura osadía. Los filósofos se han resistido una y otra vez, desde los tiempos más
antiguos, a cometerla. Descartes argumentaba más o menos así: el espacio es en esencia
igual a extensión. Pero la extensión va vinculada a los cuerpos; luego ningún espacio
sin cuerpos, es decir, no hay espacio vacío. El punto flaco de esta forma de inferencia
reside en primer lugar en lo siguiente: es cierto que el concepto de extensión debe su
origen a experiencias relativas a la posición (contacto) de cuerpos sólidos. Pero de
ahí no cabe inferir que el concepto de extensión no esté justificado en otros casos que
no hayan motivado la formación del concepto. Semejante ampliación de los conceptos
puede justificarse también indirectamente por su valor para la comprensión de
hallazgos empíricos. Por tanto, la afirmación de que la extensión va ligada a los cuerpos
es en sí infundada. Sin embargo, veremos más adelante que la teoría de la relatividad
general confirma la concepción de Descartes a través de un rodeo. Lo que llevó a
Descartes a una concepción tan curiosamente atrevida fue seguramente la sensación
de que a un objeto no «directamente experimentable»\footnote{Esta expresión hay que 
tomarla \textit{cum grano salis}.} como es el espacio no se le
podía atribuir ninguna realidad sin que hubiese una necesidad urgente de hacerlo.

El origen psicológico del concepto de espacio, o de su necesidad, no es ni mucho
menos tan evidente como pudiera parecerlo si nos dejásemos guiar por nuestros hábitos
de pensamiento. Los antiguos geómetras se ocuparon de objetos mentales (recta, punto,
superficie), pero no realmente del espacio en sí, como hizo más tarde la geometría
analítica. El concepto de espacio viene sin embargo sugerido por determinadas
experiencias primitivas. Imaginemos que fabricamos una caja. Dentro de ella se
pueden alojar objetos en determinada disposición, de manera que la caja se llene. La
posibilidad de semejantes disposiciones es una propiedad del objeto corpóreo caja,
algo que viene dado con la caja, el «espacio comprendido» en la caja. Es algo que
difiere según las cajas, algo que con toda naturalidad se lo imagina uno independiente
de si hay o no objetos en ellas. Cuando no hay objetos en la caja, su espacio aparece
«vacío».

Hasta aquí nuestro concepto de espacio va ligado a la caja. Sin embargo, se
comprueba que las posibilidades de alojamiento que constituyen el espacio de la caja
son independientes de qué grosor tengan las paredes. ¿No se puede hacer que el grosor
descienda a cero sin que al mismo tiempo se eche a perder el «espacio»? La naturalidad
de este proceso de paso al límite es evidente, subsistiendo ahora en nuestro
pensamiento el espacio sin caja, una cosa independiente que, sin embargo, parece tan
irreal cuando se olvida la procedencia del concepto. Se entiende que a Descartes le
repugnase contemplar el espacio como una cosa independiente de los objetos
corpóreos y que podía existir sin materia\footnote{El intento de Kant de sofocar el 
malestar negando la objetividad del espacio 
apenas puede tomarse en serio. Las posibilidades de alojamiento, encarnadas por el espacio
interior de la caja, son objetivas en el mismo sentido que lo son la propia caja y los objetos
que se pueden alojar en ella.}. (Lo cual no le impide, sin embargo, tratar el
espacio como concepto fundamental en su geometría analítica.) Una simple indicación
al vacío del termómetro de mercurio desarmó seguramente a los últimos cartesianos.
Pero no es de negar que incluso en este estadio primitivo hay algo de insatisfactorio en
el concepto de espacio, o en el espacio concebido como cosa real e independiente.

Las maneras en que se pueden alojar los cuerpos en el espacio (caja) constituyen el
objeto de la geometría euclidiana tridimensional, cuya estructura axiomática hace
fácilmente olvidar que se refiere a situaciones experimentables.

Una vez formado de la manera antes esbozada el concepto de espacio, en base a
experiencias sobre el «rellenado» de la caja, lo que tenemos es un espacio \textit{limitado}.
Pero \textit{esta limitación parece inesencial}, porque es evidente que siempre se puede introducir
una caja mayor que encierre a la menor. El espacio aparece así como algo que es
ilimitado.

No voy a hablar aquí de que las concepciones de la tridimensionalidad y la
«euclidicidad» del espacio proceden de experiencias (relativamente primitivas), sino
que consideraré primero el papel del concepto de espacio en la evolución del
pensamiento físico según otros puntos de vista.

Si una caja más pequeña $c$ se halla en reposo relativo en el interior del espacio hueco
de otra más grande $C$, entonces el espacio hueco o cavidad de $c$ es una parte de la
cavidad de $C$, y ambas cajas pertenecen al mismo «espacio» que las contiene. La
interpretación es, sin embargo, menos sencilla cuando $c$ se mueve respecto a $C$. Uno se
inclina entonces a pensar que $c$ encierra siempre el mismo espacio, pero ocupando una
porción variable del espacio $C$. Entonces es necesario atribuir a cada caja su espacio
particular (no concebido como limitado) y suponer que estos dos espacios se mueven
uno respecto al otro.

Antes de percatarnos de esta complicación, el espacio aparece como un medio
limitado (continente) en cuyo seno nadan los objetos corpóreos. Ahora, sin embargo,
hay que pensar que existen infinitos espacios que se hallan en mutuo movimiento. El
concepto de espacio como algo que existe objetivamente, con independencia de las
cosas, es propio ya del pensamiento precientífico, pero no así la idea de la existencia de
un número infinito de espacios en mutuo movimiento. Aunque esta idea es lógicamente
inevitable, no desempeñó durante mucho tiempo ningún papel destacado, ni siquiera
en el pensamiento científico.

¿Qué decir, sin embargo, del origen psicológico del concepto de tiempo? Este
concepto tiene indudablemente que ver con el hecho del «recordar», así como con la
distinción entre experiencias sensoriales y el recuerdo de las mismas. De suyo es
cuestionable que la distinción entre experiencia sensorial y recuerdo (o simple
imaginación) sea algo que nos venga dado de manera psicológicamente inmediata.
Cualquiera de nosotros conoce la duda entre si ha vivido algo con los sentidos o si sólo
lo ha soñado. Es probable que esta distinción no nazca sino como acto del
entendimiento ordenador.

Al «recuerdo» se le atribuye una vivencia que se reputa «anterior» a las «vivencias
presentes». Es éste un principio de ordenación conceptual para vivencias (imaginadas)
cuya viabilidad da pie al concepto de tiempo subjetivo, es decir, ese concepto de
tiempo que remite a la ordenación de las vivencias del individuo.

\section{Objetivación del Concepto de Tiempo}

Ejemplo. La persona A («yo») tiene la vivencia «cae
un rayo». La persona A vivencia al mismo tiempo un comportamiento de la persona B
que establece una conexión entre este comportamiento y la propia vivencia de «cae un
rayo». Es así como A atribuye a B la vivencia «cae un rayo». En la persona A nace la
idea de que en ese «cae un rayo» participan también otras personas. El «cae un rayo»
no se concibe ya como una vivencia exclusivamente personal, sino como vivencia (o
finalmente sólo como «vivencia potencial») de otras personas. De este modo nace la
idea de que «cae un rayo», que en origen apareció en la conciencia como «vivencia»,
puede interpretarse ahora también como un «suceso» (objetivo). Pero la esencia de
todos los sucesos es aquello a lo que nos referimos cuando hablamos del «mundo
real de afuera».

Hemos visto que tendemos a atribuir a las vivencias una ordenación temporal del tipo:
Si $\beta$ es posterior a $\alpha$ y $\gamma$ posterior a $\beta$, entonces y también
es posterior a $\alpha$ (seriación de las «vivencias»). ¿Qué ocurre en este aspecto con los
sucesos que hemos asignado a las vivencias? Lo inmediato es suponer que existe una 
ordenación temporal de los sucesos y que esa ordenación coincide con la de las vivencias.
Eso es lo que se supuso con carácter general —e inconscientemente— hasta que se hicieron valer
ciertas dudas escépticas\footnote{La ordenación temporal de vivencias adquirida por
vía acústica puede, por ejemplo, diferir de la
ordenación temporal adquirida visualmente, con lo cual no cabe identificar sin más la 
ordenación temporal de los sucesos con la ordenación temporal de las vivencias.}.

Para acceder a una objetivación del mundo hace falta otra idea constructiva: el suceso
(\textit{event}) está localizado también en el espacio, no sólo en el tiempo.

En lo que antecede hemos intentado relatar cómo se puede establecer una relación
psicológica entre los conceptos de espacio, tiempo y suceso, por una parte, y las
vivencias, por otra. Contemplados lógicamente, son creaciones libres de la
inteligencia humana, herramientas del pensamiento que deben servir para relacionar
vivencias y comprenderlas así mejor. El intento de tomar conciencia de las fuentes
empíricas de estos conceptos básicos muestra hasta qué punto estamos realmente
ligados a estos conceptos. De este modo nos hacemos conscientes de nuestra libertad,
cuyo uso razonable en caso de necesidad es siempre un asunto duro.

A este esquema relativo al origen psicológico de los conceptos de espacio-tiempo-\textit{event}
(los llamaremos brevemente «tipo espacio», en contraposición a los conceptos de la
esfera psicológica) tenemos que añadir algo esencial. Hemos conectado el concepto de
espacio con vivencias con cajas y con el alojamiento de objetos corpóreos dentro de
ellas. Esta formación conceptual presupone ya, por tanto, el concepto de objeto
corpóreo (p. ej., «caja»). Y en este contexto también desempeñan el papel de objetos
corpóreos las personas que hubo que introducir para la formación de un concepto
objetivo de tiempo. Se me antoja, por tanto, que la formación del concepto de objeto
corpóreo debe preceder a nuestros conceptos de tiempo y espacio.

Todos estos conceptos «tipo espacio» pertenecen ya al pensamiento precientífico,
junto a conceptos de la esfera psicológica, como dolor, meta, propósito, etc. El
pensamiento físico, y el de las ciencias naturales en general, se caracteriza por
pretender arreglárselas en principio con conceptos «tipo espacio» \textit{únicamente} y aspirar
a expresar con ellos todas las relaciones regulares. El físico intenta reducir colores y
tonos a vibraciones; el fisiólogo, pensamiento y dolor a procesos nerviosos, de tal
modo que lo psíquico como tal queda eliminado del nexo causal del ser, es decir, no
aparece por ningún lado como eslabón independiente en las relaciones causales. Esta
actitud, que considera teóricamente posible la comprensión de todas las relaciones
mediante el empleo exclusivo de conceptos «tipo espacio», es seguramente lo que se
entiende actualmente por «materialismo» (después de que la «materia» haya perdido su
papel como concepto fundamental).

¿Por qué es necesario bajar los conceptos fundamentales del pensamiento científico
de sus campos olímpicos platónicos e intentar desvelar su origen terrestre? Respuesta:
para liberarlos del tabú que llevan colgado y conseguir así mayor libertad en la
formación de conceptos. El haber introducido esta reflexión crítica es mérito
imperecedero de D. Hume y E. Mach en primera línea.

La ciencia ha tomado los conceptos de espacio, tiempo y objeto corpóreo (con el
importante caso especial «cuerpo sólido») del pensamiento precientífico, los ha
precisado y los ha modificado. Su primer logro importante fue la creación de la
geometría euclidiana, cuya formulación axiomática no debe hacernos olvidar su origen
empírico (posibilidades de alojamiento de cuerpos sólidos). De origen empírico es
también, en particular, la tridimensionalidad del espacio, así como su carácter
euclidiano (es posible llenarlo con «cubos» idénticos sin dejar resquicio).

La sutileza del concepto de espacio se vio acrecentada por el descubrimiento de que
no existen cuerpos totalmente rígidos. Todos los cuerpos se deforman elásticamente y
cambian de volumen al variar la temperatura. Por eso, los objetos cuyas posibles
colocaciones pretende describir la geometría euclídea no se pueden especificar al
margen del contenido de la física. Mas, dado que la física tiene que hacer uso de la
geometría desde el momento en que establece sus conceptos, el contenido empírico de
la geometría no puede ser especificado y contrastado sino en el marco de la física
como un todo.

En este contexto hay que mencionar también el atomismo y su concepción de la
divisibilidad finita, pues los espacios de extensión subatómica no se pueden medir.
El atomismo obliga también a abandonar teóricamente la idea de superficies limítrofes
neta y estáticamente definidas en cuerpos sólidos. En rigor no existen entonces leyes
independientes para las posibilidades de alojamiento de cuerpos sólidos, ni siquiera en
el terreno macroscópico.

A pesar de todo, nadie pensó en abandonar el concepto de espacio, porque parecía
imprescindible en ese sistema global de la ciencia natural tan magníficamente
acreditado. Mach fue el único que en el siglo XIX pensó seriamente en eliminar el
concepto de espacio, intentando sustituirlo por el concepto del conjunto de las
distancias actuales de todos los puntos materiales. (E hizo este intento con el fin de
llegar a una concepción satisfactoria de la inercia.)

\section{El Campo}

El espacio y el tiempo desempeñan en la mecánica newtoniana un papel
doble. En primer lugar, como soporte o marco para el acontecer físico, respecto al
cual los sucesos vienen descritos por las coordenadas espaciales y el tiempo. La materia
es vista en esencia como compuesta de «puntos materiales» cuyos movimientos
constituyen el acontecer físico. Cuando se la concibe como continua es en cierto
modo con carácter provisional y en aquellos casos en los que no se quiere o no se
puede descubrir la estructura discreta. Entonces se dispensa el tratamiento de puntos
materiales a pequeñas partes (elementos de volumen) de la materia, al menos en la
medida en que se trate simplemente de movimientos y no de procesos cuya reducción
a movimientos no fuese posible o conveniente (p. ej., variaciones de temperatura,
procesos químicos). El segundo papel del espacio y del tiempo era el de «sistema
inercial». De entre todos los sistemas de referencia imaginables, los inerciales se
distinguían por el hecho de que respecto a ellos era válido el principio de inercia.

Lo esencial en esto es que lo «físicamente real», imaginado como independiente de
los sujetos que lo vivencian, se interpretaba—al menos en teoría— como compuesto de
espacio y tiempo, por un lado, y de puntos materiales permanentemente existentes y
en movimiento respecto a aquéllos, por otro. La idea de la existencia independiente del
espacio y del tiempo cabe expresarla drásticamente así: Si desapareciera la materia,
quedarían únicamente el espacio y el tiempo (como una especie de escenario para el
acontecer físico).

La superación de este punto de vista resultó de una evolución que al principio no
parecía guardar ninguna relación con el problema del espacio-tiempo: la aparición del
\textit{concepto de campo} y su aspiración final de sustituir el concepto de partícula (punto
material). En el marco de la física clásica, el concepto de campo se instaló como
concepto auxiliar en aquellos casos en que se trataba la materia como un continuo.
En el estudio de la conducción del calor en un sólido, por poner un caso, el estado se
describe especificando la temperatura en cada punto del cuerpo y en cada instante de
tiempo. Matemáticamente quiere decir: la temperatura $T$ es representada como
expresión matemática (función) de la coordinación espacial con el tiempo $t$ (campo de
temperaturas). La ley de la conducción del calor se representa como una relación local
(ecuación diferencial) que comprende todos los casos especiales de aquélla. La
temperatura es aquí un sencillo ejemplo del concepto de campo: una magnitud (o un
complejo de magnitudes) que es función de las coordenadas y del tiempo. Otro
ejemplo es la descripción del movimiento de un fluido. En cada punto y en cada
instante existe una velocidad que viene descrita cuantitativamente por sus tres
«componentes» respecto a los ejes de un sistema de coordenadas (vector). Las
componentes de la velocidad en un punto (componentes del campo) son también
aquí funciones de las coordenadas ($x$ y $z$) y del tiempo ($t$).

Los campos mencionados se caracterizan por aparecer únicamente en el interior de
una masa ponderable; lo único que pretenden es describir un estado de esa materia.
Allí donde no había materia no podía existir tampoco --de acuerdo con la génesis del
concepto-- ningún campo. En el primer cuarto del siglo XIX se comprobó, sin
embargo, que los fenómenos de interferencia y movimiento de la luz admitían una
explicación asombrosamente nítida si se interpretaba la luz como un campo de
ondas, completamente análogo al campo de oscilaciones mecánicas en un sólido
elástico. Fue entonces necesario introducir un campo que pudiese existir incluso en
ausencia de materia ponderable, en el vacío.

Este estado de cosas creó una situación paradójica, porque el concepto de campo,
de acuerdo con su origen, parecía limitarse a describir estados en el interior de un
cuerpo ponderable. Lo cual parecía tanto más seguro cuanto que existía la convicción
de que todo campo había que concebirlo como un estado mecánicamente
interpretable, presuponiendo eso la presencia de materia. Se vio así la necesidad de
suponer por doquier, incluso en ese espacio que hasta entonces se reputaba vacío, la
existencia de una materia que se denominó «éter».

La forma en que el concepto de campo se sacudió el yugo impuesto por un sustrato
material pertenece a los procesos psicológicamente más interesantes en la evolución
del pensamiento físico. En la segunda mitad del siglo XIX, y a raíz de las investigaciones
de Faraday y Maxwell, se vio cada vez más claro que la descripción de los procesos
electromagnéticos con ayuda de la idea del campo era muy superior a un tratamiento
basado en conceptos de puntos mecánicos. Maxwell, gracias a la introducción del
concepto de campo en la Electrodinámica, consiguió predecir la existencia de las ondas
electromagnéticas, cuya fundamental identificación con las ondas luminosas era
indudable, aunque sólo fuese por la igualdad de sus velocidades de propagación.

Como consecuencia de ello, la Óptica quedó absorbida en principio por la
Electrodinámica. Uno de los efectos psicológicos de este imponente éxito fue que el
concepto de campo adquirió paulatinamente mayor autonomía frente al marco
mecanicista de la física clásica.

Pese a todo, se dio en un principio por supuesto que los campos electromagnéticos
había que interpretarlos como estados del éter, y se intentó con gran celo explicar estos
estados como mecánicos. Tuvieron que fracasar estas tentativas una y otra vez para que
se empezara a renunciar poco a poco a la interpretación mecánica, persistiendo sin
embargo el convencimiento de que los campos electromagnéticos eran estados del éter.
Así estaban las cosas hacia la vuelta del siglo.

La teoría del éter trajo consigo la pregunta de cómo se comporta mecánicamente el
éter frente a los cuerpos ponderables. ¿Participa de los movimientos de los cuerpos o
están sus partes en reposo mutuo? Muchos fueron los experimentos ingeniosos que se
realizaron para dirimir esta cuestión. Como hechos que eran importantes en este
contexto entraban también en consideración la aberración de las estrellas fijas como
consecuencia del movimiento anual de la Tierra, así como el «efecto Doppler»
(influencia del movimiento relativo de las estrellas fijas sobre la frecuencia de la luz
que llega hasta nosotros y que posee una frecuencia de emisión conocida). Los
resultados de estos hechos y experimentos (salvo uno, el experimento de Michelson-
Morley) los explicó H. A. Lorentz con la hipótesis de que el éter no participa de los
movimientos de los cuerpos ponderables y de que las partes del éter no tienen
absolutamente ningún movimiento relativo mutuo. El éter aparecía así en cierto modo
como la encarnación de un espacio absolutamente en reposo. Pero la investigación de
Lorentz dio además otros frutos. Explicó los procesos electromagnéticos y ópticos
entonces conocidos en el interior de los cuerpos ponderables, suponiendo para ello
que el influjo de la materia ponderable sobre el campo eléctrico (y a la inversa) se
debe exclusivamente a que las partículas de la materia portan cargas eléctricas que
participan del movimiento de las partículas. En relación con el experimento de
Michelson-Morley demostró H. A. Lorentz que su resultado no estaba al menos en
contradicción con la teoría del éter en reposo.

Pese a todos estos éxitos tan hermosos, el estado de la teoría no era del todo
satisfactorio, por la siguiente razón. La Mecánica clásica, de la cual no "cabía dudar
que era válida con gran aproximación, postula la equivalencia de todos los sistemas
inerciales (o espacios inerciales) para la formulación de las leyes de la naturaleza
(invariancia de las leyes de la naturaleza respecto al paso de un sistema inercial a otro).
Los \textit{experimentos} electromagnéticos y ópticos demostraron lo mismo con gran
exactitud, mientras que el fundamento de la \textit{teoría} electromagnética postulaba el
privilegio de un sistema inercial especial, a saber, el del éter luminífero en reposo.
Esta concepción del fundamento teórico era demasiado insatisfactoria. ¿No cabía
alguna modificación de éste que respetara --como la Mecánica clásica-- la
equivalencia de los sistemas inerciales (principio de la relatividad especial)?

La respuesta a esta pregunta es la teoría de la relatividad especial, que toma de la de
Maxwell-Lorentz la hipótesis de la constancia de la velocidad de la luz en el vacío. Para
hacer que esta hipótesis sea compatible con la equivalencia de los sistemas inerciales
(principio de la relatividad especial) hay que abandonar el carácter absoluto de la
simultaneidad; aparte de eso, se siguen de ahí las transformaciones de Lorentz para el
tiempo y para las coordenadas espaciales, que permiten pasar de un sistema inercial a
otro. El contenido entero de la teoría de la relatividad especial se contiene en el
postulado siguiente: las leyes de la naturaleza son invariantes respecto a las
transformaciones de Lorentz. La importancia de este requisito reside en que restringe
de manera muy determinada las posibles leyes de la naturaleza.

¿Cuál es la postura de la teoría de la relatividad especial frente al problema del
espacio? Ante todo hay que guardarse de la opinión de que fue esta teoría la que
introdujo el carácter cuadridimensional de la realidad. También en la Mecánica clásica
vienen localizados los sucesos (\textit{events}) mediante cuatro números, tres coordenadas
espaciales y otra temporal; la totalidad de los «sucesos» físicos se concibe, pues, como
inmersa en una variedad continua cuadridimensional. Pero, según la Mecánica clásica,
este continuo cuadridimensional se descompone objetivamente en un tiempo uni-
dimensional y en secciones espaciales tridimensionales que sólo contienen sucesos
tridimensionales. Esta descomposición es la misma para todos los sistemas inerciales.
La simultaneidad de dos sucesos determinados respecto a un sistema inercia! implica la
simultaneidad de estos sucesos respecto a todos los sistemas inercia-les. Esto es lo que
debe entenderse cuando se dice que el tiempo de la Mecánica clásica es absoluto. En
la teoría de la relatividad especial ya no es así. La idea del conjunto de sucesos que son
simultáneos a otro determinado existe en relación a un determinado sistema inercial,
pero ya no con independencia de la elección del sistema inercial. El continuo
cuadridimensional no se descompone ya objetivamente en secciones que contienen
todos los sucesos simultáneos; el «ahora» pierde para el mundo, espacialmente extenso,
su significado objetivo. De ahí que haya que concebir espacio y tiempo,
objetivamente indisolubles, como un continuo cuadridimensional si se quiere expresar
el contenido de las relaciones objetivas sin arbitrariedades convencionales y
prescindibles.

La teoría de la relatividad especial, al demostrar la equivalencia física de todos los
sistemas inerciales, puso de manifiesto el carácter insostenible de la hipótesis del éter
en reposo. Hubo que renunciar por eso a la idea de interpretar el campo
electromagnético como estado de un sustrato material. El campo se convierte así en
un elemento irreducible de la descripción física, e irreducible en el mismo sentido que
el concepto de materia en la teoría newtoniana.

Hasta aquí hemos centrado la atención en el tema de hasta qué punto la teoría de la
relatividad especial modificó los conceptos de espacio y tiempo. Vamos a fijarnos
ahora en aquellos elementos que la teoría tomó de la mecánica clásica. Al igual que en
ésta, en la relatividad especial las leyes de la naturaleza sólo aspiran a validez cuando la
descripción espacio-temporal se basa en un \textit{sistema inercial}. El principio de inercia y el
de la constancia de la velocidad de la luz solamente son válidos respecto a un sistema
inercial. También las leyes del campo aspiran a tener sentido y validez respecto a sis-
temas inerciales únicamente. Por consiguiente, al igual que en la Mecánica clásica, el
espacio es, también aquí, una componente independiente de la representación de lo
físicamente real. El espacio (inercial) —o con más exactitud, este espacio, junto con el
correspondiente tiempo— es lo que queda al suprimir mentalmente la materia y el
campo. Esta estructura cuadridimensional (espacio de Minkowski) se concibe como
soporte de la materia y del campo. Los espacios inerciales, con sus correspondientes
tiempos, son sólo sistemas de coordenadas cuadridimensionales privilegiados que se
relacionan entre sí a través de transformaciones de Lorentz lineales. Dado que en esta
estructura cuadridimensional ya no hay secciones que representen objetivamente el
«ahora», el concepto de «ocurrir» y «devenir» no es que quede eliminado
completamente, pero sí se complica. Parece, por tanto, más natural imaginar lo física-
mente real como un ser cuadridimensional en lugar de contemplarlo, como hasta
entonces, como el \textit{devenir} de un ser tridimensional.

Este espacio cuadridimensional rígido de la teoría de la relatividad especial es en
cierto modo el homólogo cuadridimensional del éter tridimensional rígido de H. A.
Lorentz. Para esta teoría vale también el enunciado: la descripción de los estados físicos
presupone el espacio como algo que viene dado de antemano y que lleva una existencia
independiente. Quiere decirse que esta teoría tampoco elimina el recelo de Descartes
en punto a la existencia autónoma, incluso \textit{a priori}, del «espacio vacío». El mostrar
hasta qué punto la teoría de la relatividad general supera estas reservas es la verdadera
meta de estas reflexiones elementales.

\section{El Concepto de Espacio en la Teoría de la Relatividad General}

Esta teoría nació en principio del intento de comprender la igualdad entre masa inercial 
y masa gravitatoria. Se parte de
un sistema inercial $S_{1}$ cuyo espacio está físicamente vacío. Quiere decir esto que en la
porción de espacio considerada no existe ni materia (en el sentido usual) ni un campo en
el sentido de la teoría de la relatividad especial. Sea $S_{2}$ un segundo sistema de referencia
uniformemente acelerado respecto a $S_{1}$. $S_{2}$ no es, pues, un sistema inercial. 

Respecto a $S_{2}$, cualquier masa de prueba se movería aceleradamente, y además
independientemente de su constitución física y química. Respecto a $S_{2}$ existe por tanto
un estado que --al menos en primera aproximación-- no cabe distinguir de un campo
gravitacional. El estado de cosas que se percibe es por tanto compatible con la
siguiente concepción: también $S_{2}$ es equivalente a un «sistema inercial», pero respecto a
$S_{2}$ existe un campo gravitacional (homogéneo) (cuyo origen no nos preocupa en este
contexto). Así pues, si se incluye el campo gravitacional en el marco de las
consideraciones, entonces el sistema inercial pierde su significado objetivo, siempre y
cuando este «principio de equivalencia» se pueda extender a cualquier movimiento
relativo de los sistemas de referencia. Si es posible fundamentar en estas ideas básicas
una teoría consistente, entonces satisfará de por sí el hecho, empíricamente muy bien
fundado, de la igualdad entre masa inercial y gravitatoria.

Cuadridimensionalmente, el paso de $S_{1}$ a $S_{2}$ se corresponde con una transformación
no lineal de las cuatro coordenadas. Se plantea entonces la pregunta: ¿qué
transformaciones no lineales deben permitirse?, o bien ¿cómo debe generalizarse la
transformación de Lorentz? Para responder a esta pregunta es decisiva la siguiente
reflexión.

Al sistema inercial de las teorías anteriores se le atribuye la propiedad de que las
diferencias de coordenadas se miden por medio de reglas «rígidas» (en reposo) y las
diferencias temporales mediante relojes (en reposo). El primer supuesto se
complementa con la hipótesis de que para las posibilidades de colocación relativa de
las reglas en reposo valen los teoremas sobre «segmentos» de la geometría euclidiana.
De los resultados de la teoría de la relatividad especial se infiere entonces, mediante
consideraciones elementales, que esta interpretación física directa de las coordenadas
se echa a perder para sistemas de referencia ($S_{2}$) acelerados respecto a sistemas
inerciales ($S_{1}$). Mas en ese caso las coordenadas sólo expresan ya el orden de lo «yuxtapuesto» 
(y con ello el grado de dimensiones del espacio), pero no las propiedades
métricas del espacio. De esta manera se llega a extender las transformaciones a
cualesquiera transformaciones continuas\footnote{Sirva aquí esta manera de expresarnos,
aunque no sea exacta.}. Esto es lo que implica la teoría de la
relatividad general. Las leyes de la naturaleza tienen que ser covariantes respecto a
cualesquiera transformaciones continuas de las coordenadas. Este requisito (en
conjunción con el de la máxima simplicidad lógica de las leyes) restringe las
posibles leyes naturales de un modo incomparablemente más fuerte que el principio
de la relatividad especial.

  El razonamiento se basa esencialmente en el campo como concepto independiente.
Pues las condiciones que prevalecen respecto a $S_{2}$ se interpretan como campo
gravitacional, sin que se plantee la cuestión de la existencia de masas que engendren el
campo. Y este razonamiento permite también comprender por qué las leyes del campo
gravitacional puro están conectadas más directamente con la idea de la relatividad

general que las leyes para campos de clase general (cuando existe un campo
electromagnético, por ejemplo). Pues tenemos buenas razones para suponer que el
espacio de Minkowski «libre de campo» representa un caso especial permitido por las
leyes de la naturaleza, y en concreto el caso especial más sencillo que cabe imaginar.
Un espacio semejante se caracteriza, en relación a su propiedad métrica, por el hecho
de que $dx_{1}^{2} + dx_{2}^{2} + dx_{3}^{2}$ es el cuadrado de la distancia espacial, medida con una regla
unidad, entre dos puntos infinitesimalmente próximos de una sección espacial
tridimensional (teorema de Pitágoras), mientras que $dx_{4}$ es la distancia temporal
--medida con una unidad de tiempo conveniente-- entre dos sucesos con ($x_{1}$, $x_{2}$, $x_{3}$)
comunes. De aquí se deduce --como es fácil mostrar con ayuda de las
transformaciones de Lorentz-- que la cantidad

\begin{equation}
ds^{2}=dx_{1}^{2}+dx_{2}^{2}+dx_{3}^{2}-dx_{4}^{2} \label{eqn:e1}
\end{equation}

\noindent posee un significado métrico objetivo. Matemáticamente se corresponde este hecho con la circunstancia de que $ds^{2}$ es invariante respecto a transformaciones de Lorentz.

Si, en el sentido del principio de la relatividad general, se somete ahora este
espacio a una transformación de coordenadas arbitraria pero continua, esa cantidad
objetivamente significativa se expresa en el nuevo sistema de coordenadas por la
relación

\begin{equation}
ds^{2}=g_{ik}dx_{i}dx_{k},\label{eqn:e2}
\end{equation}

\noindent donde hay que sumar en los subíndices $i$ y $k$ en todas sus combinaciones 11, 12, ...
hasta 44. Ahora bien, las $g_{ik}$ ya no son constantes, sino funciones de las coordenadas, y
vienen determinadas por la transformación arbitrariamente elegida. A pesar de ello, las
$g_{ik}$ no son funciones arbitrarias de las nuevas coordenadas, sino precisamente funciones
tales que la forma (\ref{eqn:e2}) pueda transformarse de nuevo en la forma (\ref{eqn:e1}) mediante una
transformación continua de las cuatro coordenadas. Para que esto sea posible, las
funciones $g_{ik}$ tienen que cumplir ciertas ecuaciones generalmente covariantes que B.
Riemann derivó más de medio siglo antes del establecimiento de la teoría de la
relatividad general («condición de Riemann»). Según el principio de equivalencia, (\ref{eqn:e2})
describe en forma generalmente covariante un campo gravitacional de tipo especial,
siempre que las $g_{ik}$ cumplan la condición de Riemann.

Así pues, la ley para el campo gravitacional puro de tipo general debe cumplir las
siguientes condiciones. Debe satisfacerse cuando se satisface la condición de
Riemann; pero debe ser más débil, es decir, menos restrictiva que la condición de
Riemann. Con ello queda prácticamente determinada por completo la ley de campo
de la gravitación pura, cosa que no vamos a fundamentar aquí con más detalle.

Ahora ya estamos preparados para ver hasta qué punto el paso a la teoría de la
relatividad general modifica el concepto de espacio. Según la Mecánica clásica y según
la teoría de la relatividad especial, el espacio (espacio-tiempo) tiene una existencia
independiente de la materia o del campo. Para poder describir aquello que llena el
espacio, aquello que depende de las coordenadas, hay que imaginar que el espacio-tiempo,
o el sistema inercial con sus propiedades métricas, viene dado desde un
principio, porque si no carecería de sentido la descripción de «aquello que llena el
espacio» \footnote{Si se suprime mentalmente aquello que llena el espacio (p. ej., el campo),
queda todavía el espacio métrico según (\ref{eqn:e1}), que también sería determinante 
para el comportamiento inercial de un cuerpo de prueba introducido en él.}. Por el contrario,
según la teoría de la relatividad general, el espacio no tiene
existencia peculiar al margen de «aquello que llena el espacio», de aquello que
depende de las coordenadas. Sea, por ejemplo, un campo gravitacional puro descrito
por las $g_{ik}$ (como funciones de las coordenadas) mediante resolución de las ecuaciones
gravitacionales. Si suprimimos mentalmente el campo gravitatorio, es decir, las
funciones $g_{ik}$, lo que queda no es algo así como un espacio del tipo (\ref{eqn:e1}), sino que no
queda absolutamente \textit{nada,} ni siquiera un «espacio topológico». Pues las funciones $g_{ik}$
describen no sólo el campo, sino al mismo tiempo también la estructura y propiedades
topológicas y métricas de la variedad. Un espacio del tipo (\ref{eqn:e1}) es, en el sentido de la
teoría de la relatividad general, no un espacio sin campo, sino un caso especial del
campo $g_{ik}$ \textit{para}, el cual las $g_{ik}$ (para el sistema de coordenadas empleado, que en sí no
tiene ningún significado objetivo) poseen valores que no dependen de las coordenadas;
el espacio vacío, es decir, un espacio sin campo, no existe.

Así pues, Descartes no estaba tan confundido al creerse obligado a excluir la
existencia de un espacio vacío. Semejante opinión parece ciertamente absurda
mientras uno sólo vea lo físicamente real en los cuerpos ponderables. Es la idea del
campo como representante de lo real, en combinación con el principio de la relatividad
general, la que muestra el verdadero meollo de la idea cartesiana: no existe espacio
«libre de campo».


\section{Teoría de la Gravitación Generalizada}

La teoría del campo gravitacional puro, asentada
sobre el firme de la teoría de la relatividad general, es fácilmente accesible porque
podemos confiar en que el espacio de Minkowski «libre de campo» con la métrica de
(\ref{eqn:e1}) tiene que corresponderse con las leyes generales del campo. A partir de este caso
especial se sigue la ley de gravitación mediante una generalización prácticamente exenta
de toda arbitrariedad. La ulterior evolución de la teoría no está tan unívocamente
determinada por el principio de la relatividad general; en los últimos decenios ha
habido intentos en distintas direcciones. Todos ellos tienen en común la interpretación
de lo físicamente real como campo, siendo éste una generalización del campo
gravitacional y la ley del campo una generalización de la ley para el campo
gravitacional puro. Creo que ahora, tras largos tanteos, he hallado la forma más natural
para esta generalización\footnote{La generalización cabe caracterizarla del siguiente modo. El campo 
gravitacional puro de los gik posee, de acuerdo con su derivación a partir del
«espacio de Minkowski» vacío, la propiedad de simetría $g_{ik} = _{gki}(g_{12} = g_{21}, etc.$).
El campo generalizado es de la misma clase, pero sin esa propiedad de simetría. 
La derivación de la ley del campo es completamente análoga a la del caso especial de
la gravitación pura.}; pero hasta la fecha no he logrado averiguar si esta ley
generalizada resiste o no la confrontación con los hechos experimentales.

Para las consideraciones generales que anteceden es secundario conocer la ley del
campo concreta. La cuestión principal es actualmente la de si una teoría de campo
como la que aquí nos interesa puede siquiera llevarnos al objetivo. Nos referimos a
una teoría que describa exhaustivamente lo físicamente real (con inclusión del espacio
cuadridimensional) mediante un campo. La presente generación de físicos se inclina
por contestar negativamente a esta pregunta; opinan, en concordancia con la forma
actual de la teoría cuántica, que el estado de un sistema no se puede caracterizar
directa sino sólo indirectamente, mediante especificación de la estadística de las
medidas realizadas en el sistema; prevalece la convicción de que la naturaleza dual
(corpuscular y ondulatoria), confirmada experimentalmente, sólo puede alcanzarse
mediante un debilitamiento semejante del concepto de realidad. Mi opinión es que
nuestros conocimientos reales no justifican una renuncia teórica de tan largo alcance, y
que no se debería dejar de estudiar hasta el final el camino de la teoría de campos
relativista.


\newpage{}

~\\
 $K$ = sistema de coordenadas \\
 $x,y$ = coordenadas de dos dimenciones \\
 $x,y,z$ = coordenadas de tres dimenciones \\
 $x,y,z,t$ = coordenadas de cuatro dimenciones \\


~\\
 $t$ = tiempo \\
 $I$ = distancia \\
 $v$ = velocidad \\


~\\
 $F$ = fuerza \\
 $G$ = campo gravitacional 

\end{document}
