\documentclass[
	%% tamaño papel
	letterpaper,
	%% tamaño fuentes 
	12pt,
	%% incia capitulos en paginas pares ó impares
	oneside]
	% estilo del documento
	{article}
%% manejo del idioma español
\usepackage[spanish]{babel}
%% Fontspec provides an interface to system fonts and Opentype features
\usepackage{fontspec}

%% Load default features. An asterisk means they are supposed to be
%% enabled by default, but engines may behave differently.
\defaultfontfeatures{%
 RawFeature={%
   ,+cv06  % narrow guillemets
   }%
}

%% Finally set the main font
\setmainfont{EB Garamond}

%% Load Microtype with default settings. This will use the
%% EB-Garamond protrusion definitions if present.
%% [kerning=false, tracking=false, protrusion=true, expansion=false]
\usepackage{microtype}

\usepackage{lettrine} 

% permite utilizar colors predefinidos
\usepackage[
	usenames,
	dvipsnames]{color}
% define parametros para hyperlinks en el text
\usepackage[
	% añade colores a links
	colorlinks=true,
	% define color a  ulizar en links internos
	linkcolor=BrickRed,
	% define color a  ulizar en links externos (urls)
	urlcolor=Bittersweet,
	% los numeros en el toc son links
	linktocpage]{hyperref}
	% metadatos del archivo pdf
	\hypersetup{
		% titulo
		pdftitle={¿Cuánta tierra necesita un hombre?},
		% autor
		pdfauthor={Lev Nikoláievich Tolstói},
		% palabras cable
		pdfkeywords={Filosófico, Infantil, Relato},
		% creador del documento
		pdfcreator={Juan Pablo Neira Durán},
		% pdfproducer={Salvatore Mazzarino}, % producer of the document
		%pdfstartview={FitH},    % fits the width of the page to the window
		}
%% ~~~~~~~~~~~~~~~~~~~~~~~~~~~~~~~~~~~~~~~~~~~~~~~~~~~~~
%%             \\\ COMIENZA EL DOCUMENTO ///
%% ~~~~~~~~~~~~~~~~~~~~~~~~~~~~~~~~~~~~~~~~~~~~~~~~~~~~~
\begin{document}
\title{\Huge ¿Cuánta tierra necesita un hombre?}
\author{\emph{Lev Nikoláievich Tolstói}~\footnote{Traducción de Víctor Gallego}}
\date{\addfontfeature{Color=980000}
	  \Huge  ~ }
\maketitle
\section*{\centering \centering I}

\lettrine[lines=2, lhang=0.33, loversize=0.25]{U}{na}
hermana mayor fue al campo a visitar a su hermana menor. La mayor
vivía en la ciudad y estaba casada con un comerciante; la menor, mujer
de un campesino, residía en la aldea. Las hermanas bebieron té y
charlaron. La mayor empezó a alabar las ventajas de vivir en la ciudad,
comentando qué espaciosa y limpia era su casa, qué bien vestidos iban,
qué elegantes prendas lucían sus hijos, cuántas cosas buenas comían y
bebían, cómo iba en carroza, acudía al teatro e iba de paseo.

La menor, sintiéndose ofendida, empezó a menospreciar la vida de los
comerciantes y a ponderar la de los campesinos.

—No cambiaría mi vida por la tuya —dijo—. Será todo lo gris que
quieras, pero no sabemos lo que es el miedo. Es verdad que vuestro
estilo de vida es más refinado, pero no es menos cierto que, aunque
algunas veces obtenéis grandes ganancias, al día siguiente podéis
perderlo todo. Recuerda lo que dice el proverbio: «La ganancia es
hermana de la pérdida». A menudo sucede que hoy eres rico y mañana estás
mendigando un pedazo de pan. En cambio, la vida del campesino es más
segura: modesta, pero larga; nunca seremos ricos, pero siempre tendremos
qué comer.

Entonces la mayor dijo:

—¡Ya! ¡En compañía de cerdos y terneros! ¡Sin ninguna elegancia ni
modales! Por mucho que se afane tu marido, viviréis entre estiércol y
entre estiércol moriréis; y la misma suerte conocerán vuestros hijos.

—¡Qué se le va a hacer! —replicó la menor—. Nuestras labores lo
exigen. Pero en cambio nuestra posición es más firme; no tenemos que
inclinarnos ante nadie y a nadie tememos. Vosotros, en la ciudad, vivís
rodeados de toda clase de tentaciones; hoy todo va bien, pero mañana el
demonio puede tentar a tu marido con las cartas, el vino o una hermosa
mujer. Y todo se convertirá en polvo. ¿Acaso no sucede así a menudo?

Pajom, el dueño de la casa, estaba tumbado en lo alto de la estufa y
escuchaba lo que decían las mujeres.

—Es la pura verdad —exclamó—. Ocupados desde pequeños en cultivar
a nuestra madre tierra, no tenemos tiempo de pensar siquiera en
tonterías. ¡La única pena es que disponemos de poca tierra! ¡Si tuviera
toda la que quisiera, no tendría miedo de nadie, ni siquiera del diablo!

Las mujeres acabaron de beber el té, charlaron un rato de vestidos,
recogieron la vajilla y se fueron a la cama.

El diablo se había sentado detrás de la estufa y lo había escuchado
todo. Se había alegrado mucho de que la mujer del campesino hubiera
inducido a su marido a alabarse: se había jactado de que, si tuviese
mucha tierra, no temería ni siquiera al diablo.

«De acuerdo —pensó el diablo—. Haremos una apuesta tú y yo: te daré
mucha tierra y gracias a ella te tendré en mi poder».

\section*{\centering II}

Cerca de la aldea vivía una pequeña propietaria, dueña de una hacienda
de ciento veinte \emph{desiatinas}. Antes siempre había vivido en paz
con los mujiks, sin perjudicarlos en modo alguno. Pero un día contrató
como administrador a un soldado retirado, que empezó a abrumarlos con
multas. Por muy atento que estuviera Pajom, tan pronto un caballo se
metía en un campo de avena como una vaca se colaba en el huerto o las
terneras entraban en los prados; y cada vez recibía una multa.

Pajom pagaba y luego, en casa, insultaba y pegaba a los suyos. Aquel
verano tuvo tantos quebraderos de cabeza por culpa de ese administrador
que se alegró cuando llegó el momento de encerrar el ganado en los
establos; aunque le molestaba tener que procurarse forraje, al menos
estaría libre de temores.

Durante el invierno corrió la voz de que la señora quería vender la
tierra y ya estaba en tratos con el posadero del camino real. Los
campesinos recibieron la noticia con no poca inquietud. «Si el posadero
se queda con la tierra —pensaban— nos acribillará a multas;
estaremos aún peor que con la señora. No podemos vivir sin esa tierra;
la compraremos entre todos».

Así pues, una asamblea de campesinos fue a ver a la señora para rogarle
que no vendiera la tierra al posadero y le ofrecieron pagar un precio
más alto. La señora aceptó. Los campesinos trataron de concertarse para
comprar toda la tierra; se reunieron una vez y después otra, pero no se
pusieron de acuerdo. El diablo sembró la discordia entre ellos y no
fueron capaces de alcanzar un compromiso. Entonces los campesinos
decidieron comprar parcelas individuales, cada cual según sus medios. La
señora aceptó también esa solución. Pajom se enteró de que su vecino
había comprado veinte \emph{desiatinas} a la señora, que había aceptado
aplazar la mitad del pago hasta el año siguiente. Lleno de envidia,
pensó: «Comprarán toda la tierra y yo me quedaré sin nada». Entonces
decidió hablar con su mujer.

—Todos compran —dijo—. También nosotros deberíamos comprar unas
diez \emph{desiatinas}. Así no podemos seguir: ese administrador va a
acabar con nosotros a fuerza de multas.

Se pusieron a pensar en lo que podrían hacer para comprar esa tierra.
Habían ahorrado cien rublos; vendieron el potro y la mitad de las
colmenas, mandaron al hijo a trabajar y Pajom pidió un préstamo a su
cuñado; de ese modo lograron reunir la mitad del dinero.

Una vez amasada esa suma, Pajom eligió una parcela de quince
\emph{desiatinas} con un bosquecillo y fue a tratar con la señora.
Llegaron a un acuerdo, se estrecharon la mano y Pajom entregó una señal.
Luego fueron a la ciudad para firmar el acta de compraventa; él entregó
la mitad del dinero y se comprometió a pagar el resto en dos años.

Así fue como Pajom adquirió esa tierra. Compró semillas a préstamo y
sembró. La cosecha fue tan buena que al cabo de un año consiguió saldar
las deudas con la señora y con su cuñado. Y Pajom se convirtió en
propietario: araba, sembraba y segaba heno en su propia tierra; talaba
sus propios árboles y sacaba a pastar al ganado a sus propios prados.
Cuando iba a arar sus campos o se quedaba mirando los sembrados y las
praderas, su corazón se exultaba de alegría. Hasta tenía la impresión de
que las hierbas y las flores eran diferentes ahora. Antes, cuando pasaba
por aquellas tierras, le parecían como las demás; ahora se le antojaban
completamente distintas.

\section*{\centering III}

Pajom estaba muy satisfecho con su vida. Todo podría haber ido bien,
pero los campesinos vecinos empezaron a hollar sus sembrados y sus
prados. Les pidió por favor que no lo hicieran, pero no hubo manera: tan
pronto los pastores dejaban pasar las vacas a los prados como los
caballos que pastaban de noche entraban en sus sembrados. Al principio
Pajom los echaba y perdonaba a los propietarios, pero luego perdió la
paciencia y fue a quejarse al tribunal del distrito. Sabía que el
comportamiento de los campesinos obedecía a su pobreza, que no lo hacían
con mala intención, pero pensó: «No puedo dejar así las cosas; si no,
acabarán con todo. Hay que darles una lección».

Así pues, con la ayuda del tribunal, les dio una lección y luego otra;
uno o dos campesinos fueron condenados a pagar una multa. Sus vecinos
empezaron a cogerle ojeriza; volvieron a causar estragos en sus campos,
esta vez a propósito. Una vez uno de ellos entró en su bosquecillo y
taló diez jóvenes tilos para aprovechar la corteza. Al pasar un día por
el bosque, Pajom creyó ver algo blanco. Se acercó y vio los troncos por
el suelo, al lado de los tocones. Si al menos hubiera cortado los de los
bordes y hubiese dejado uno aquí y allá, pero el muy canalla había
cortado uno detrás de otro. Pajom se enfureció. «Ah, si pudiera saber
quién ha sido —pensó— se lo haría pagar». Tras darle muchas vueltas,
llegó a la conclusión de que solo podía haber sido Siomka. Fue al patio
de Siomka a echar un vistazo, pero no descubrió nada y acabó discutiendo
con él. No obstante, plenamente convencido de su culpabilidad, puso una
denuncia. Juzgaron a Siomka, pero el tribunal lo absolvió por falta de
pruebas. Pajom se ofendió aún más y riñó con los jueces y con el jefe de
la aldea.

—Estáis confabulados con los ladrones —dijo—. Si respetarais la
justicia, no soltaríais a esos maleantes.

Pajom discutió con los jueces y con los vecinos, que le amenazaron con
prender fuego a su casa. En definitiva, aunque Pajom tenía muchas más
tierras, su posición era peor que antes.

Por esa época corrió el rumor de que la gente emigraba a lugares nuevos.
«No tengo ninguna razón para marcharme de mis tierras —pensó Pajom—,
pero si algunos de nuestros vecinos se fueran, viviríamos con más
holgura. Me quedaría con sus tierras y ampliaría mis propiedades.
Entonces viviríamos mejor. Ahora padecemos demasiadas estrecheces».

Un día en que se hallaba en casa llamó a la puerta un mujik que pasaba
por la aldea. Pajom le ofreció un lecho donde dormir, le dio de comer y
charló con él. Entre otras cosas Pajom le preguntó de dónde venía. El
mujik le dijo que venía de más allá del Volga, donde había estado
trabajando. Poco a poco el mujik le contó que mucha gente se estaba
estableciendo en aquellos lugares.

—Han venido campesinos de fuera, se han inscrito en el Registro y han
recibido diez \emph{desiatinas} por cabeza —dijo—. Es una tierra tan
buena que si siembras centeno crece paja, hasta alcanzar la altura de un
caballo, y tan grueso que cinco puñados forman un haz. Un mujik pobre de
solemnidad —añadió—, que llegó sin un céntimo en el bolsillo, ahora
tiene seis caballos y dos vacas.

Muy excitado, Pajom, pensó: «¿Por qué pasar apuros y estrecheces aquí
cuando se puede vivir mejor en otro lugar? Venderé mis tierras y mi casa
y con ese dinero me estableceré y llevaré mi propia hacienda. Aquí, con
tantas apreturas, no hay quien viva. Pero antes es preciso que vaya a
enterarme de todo en persona».

Ese mismo verano preparó lo necesario y partió. Descendió por el Volga
en un vapor hasta Samara y a partir de allí cubrió a pie unas
cuatrocientas verstas. Llegó al lugar y comprobó que todo lo que había
oído era cierto. Los campesinos vivían con holgura; cada hombre recibía
diez \emph{desiatinas} y en el Registro inscribían de buena gana a los
recién llegados. Si alguien llegaba con dinero, además de la parcela que
se le asignaba, podía comprar, con derecho a perpetuidad, toda la tierra
que quisiera. La tierra de mejor calidad se vendía a un precio de tres
rublos la \emph{desiatina}. ¡Podía uno comprar cuanto se le antojara!

Una vez enterado de todo, Pajom regresó a su casa en otoño y empezó a
vender cuanto tenía. Vendió la tierra con beneficio, vendió la casa,
vendió todo el ganado, se dio de baja en el Registro y, cuando llegó la
primavera, partió con su familia a esos nuevos lugares.

\section*{\centering IV}

Una vez allí, Pajom se inscribió en el Registro de una gran aldea.
Ofreció de beber a los ancianos y puso en orden todos los papeles. Como
su familia se componía de cinco personas, le entregaron cincuenta
\emph{desiatinas} de tierra en campos diferentes, con los pastos aparte.
Pajom se estableció y compró ganado. Ahora tenía tres veces más tierra
que antes, contando solo la que le habían asignado. Y era una tierra
estupenda para el cultivo del cereal. Su situación era diez veces mejor.
Había gran abundancia de pastos y de tierras de labor y podía tener todo
el ganado que quisiese.

Al principio, mientras se ocupaba de la construcción de la casa y de
todos los preparativos, estaba muy contento; pero una vez que se
acostumbró, también esa tierra le pareció poca. El primer año Pajom
sembró trigo en la tierra asignada y obtuvo una buena cosecha. Le
hubiera gustado sembrar más, pero había poca para distribuir y la que
tenía ya no le servía, pues en esas regiones el trigo se siembra en
tierras incultas o cubiertas de hierba; siembran un año o dos y luego
dejan la tierra en barbecho hasta que vuelve a cubrirse de hierba. Eran
muchos los que querían esa tierra y no había suficiente para todos. Así
pues, surgían disputas. Los más ricos querían cultivarlas y los más
pobres se las arrendaban a los comerciantes a cambio del pago de la
contribución. Pajom quería sembrar más tierra. Al segundo año fue a ver
a un mercader y arrendó tierras por un año. En suma, pudo sembrar más y
obtuvo una buena cosecha, pero aquellas tierras estaban lejos de la
aldea: había que cubrir quince verstas con los carros. Al cabo de algún
tiempo Pajom advirtió que algunos campesinos-comerciantes de los
alrededores vivían en granjas y se enriquecían. «No estaría mal si yo
también pudiera comprar tierras a perpetuidad y construirme una granja
—se dijo—. Así lo tendría todo a la puerta de casa». A partir de ese
momento Pajom no pensó en otra cosa.

Vivió de ese modo por espacio de tres años. Arrendaba tierras y sembraba
trigo. Esos años las cosechas fueron buenas y Pajom empezó a ganar
dinero. Vivía bien, pero le molestaba pagar cada año el arriendo de la
tierra y tener que luchar por ella; porque, allí donde había una buena
parcela, acudían enseguida otros campesinos y la acaparaban toda; así
que, si no llegaba a tiempo se quedaba sin tierra para sembrar. El
tercer año arrendó a medias con un mercader un prado de unos campesinos;
habían empezado a ararlo cuando los campesinos les pusieron un pleito y
el trabajo se perdió. «Si hubiera tenido mi propia tierra —pensaba—,
no habría tenido que rendir cuentas a nadie y me habría ahorrado todos
estos disgustos».

Y empezó a informarse de dónde podía comprar tierra a perpetuidad. Al
poco tiempo conoció a un mujik que había comprado quinientas
\emph{desiatinas}, pero se había arruinado y las vendía a un buen
precio. Pajom entabló negociaciones con él. Tras mucho regatear, se
pusieron de acuerdo en una suma de mil quinientos rublos, mitad al
contado y mitad a plazos. Habían cerrado ya el acuerdo, cuando un día un
comerciante de paso se detuvo en casa de Pajom para dar de comer a los
caballos. Bebieron un poco de té y charlaron. El comerciante le contó
que venía de la lejana región de los bashkirios, donde había comprado
cinco mil \emph{desiatinas} de tierra por mil rublos. Pajom le hizo
algunas preguntas y el comerciante dijo lo siguiente:

—Solo hay que ganarse la voluntad de los ancianos. Les he regalado
batas y alfombras por valor de cien rublos, además de una caja de té; y
he dado vino a los que les gusta la bebida. De ese modo he comprado la
tierra a veinte kopeks la \emph{desiatina.} —Le enseñó el acta de
compraventa y añadió—: La tierra está a la orilla de un río y toda la
estepa está cubierta de hierba.

Pajom le hizo más preguntas y el comerciante dijo:

—Hay tanta tierra que no podrías recorrerla en un año. Y toda
pertenece a los bashkirios, que son tan inocentes como corderos. Se
puede conseguir la tierra casi de balde.

«¿Por qué voy a pagar mil rublos por quinientas \emph{desiatinas}
—pensó Pajom— y a contraer una deuda, cuando con esa misma cantidad
puedo conseguir allí toda la tierra que se me antoje?».

\section*{\centering V}

Pajom preguntó al comerciante cómo podía llegar hasta allí y, en cuanto
lo acompañó a la puerta, empezó a hacer los preparativos para el viaje.
Confió la casa a su mujer y partió acompañado de un trabajador. Al pasar
por la ciudad, compraron una caja de té, regalos y vino, como el
comerciante le había aconsejado. Recorrieron unas quinientas verstas y
al séptimo día llegaron a un campamento bashkirio. Todo era como el
mercader le había dicho. Los bashkirios vivían en la estepa, a la orilla
del río, en \emph{kibitkas} de fieltro. No cultivaban la tierra ni
comían pan. Su ganado y sus caballos vagaban en rebaños por la estepa.
Tenían los potros atados a las \emph{kibitkas} y dos veces al día
llevaban allí las yeguas, cuya leche utilizaban para elaborar
\emph{kumis}. Las mujeres batían el \emph{kumis} y preparaban queso; los
hombres no hacían nada: bebían \emph{kumis} y té, comían carne de
cordero y tocaban el pífano. De aspecto saludable y ánimo alegre,
pasaban el verano de fiesta. Eran ignorantes y no hablaban ruso, pero se
mostraban acogedores con los forasteros.

En cuanto vieron a Pajom, salieron de sus \emph{kibitkas} y le rodearon.
Encontraron un intérprete. Pajom les dijo que había venido para comprar
tierra. Los bashkirios se alegraron mucho, llevaron a Pajom a una de las
mejores \emph{kibitkas}, le hicieron sentarse sobre alfombras, le
pusieron debajo cojines de plumas, se acomodaron a su alrededor y
empezaron a agasajarlo con té y \emph{kumis}. Mataron un cordero y le
dieron de comer. Pajom cogió los regalos que llevaba en el carro y los
distribuyó entre los bashkirios; a continuación dividió el té entre
todos. Los bashkirios se alegraron mucho, charlaron entre ellos y luego
pidieron al intérprete que tradujera sus palabras.

—Me ordenan que te diga —dijo el interprete— que les has caído
bien y que tenemos por costumbre agasajar a nuestros huéspedes de todas
las maneras posibles e intercambiar regalos con ellos. Tú nos has hecho
varios obsequios; ahora debes decirnos qué es lo que más te gusta de lo
que tenemos para que podamos ofrecértelo.

—Lo que más me gusta es vuestra tierra —dijo Pajom—. La nuestra es
escasa y está agotada; entre vosotros, en cambio, la tierra es buena y
abundante. Nunca la había visto igual.

El intérprete tradujo. Los bashkirios estuvieron deliberando un buen
rato. Pajom no comprendía lo que decían, pero veía que estaban alegres,
porque gritaban y reían. Luego guardaron silencio y se quedaron mirando
a Pajom, mientras el intérprete decía:

—Me piden que te comunique que, a cambio de tus regalos, te entregarán
toda la tierra que desees. No tienes más que indicarnos cuál quieres y
será tuya.

Los bashkirios se pusieron a hablar de nuevo, discutiendo entre ellos
alguna cuestión. Pajom preguntó qué estaban diciendo y el intérprete le
contestó:

—Unos aseguran que primero hay que consultar con el jefe y que no se
puede hacer nada en su ausencia, mientras otros opinan que no es
necesario su consentimiento.

\section*{\centering VI}

Mientras los bashkirios discutían, llegó un hombre con un gorro de piel
de zorro.

Todos guardaron silencio y se pusieron en pie. El intérprete dijo:

—Es el jefe.

Sin perder tiempo, Pajom sacó la mejor bata que llevaba y se la ofreció,
así como cinco libras de té. El jefe aceptó los regalos y se sentó en el
puesto de honor. A continuación los bashkirios empezaron a decirle algo.
El jefe los escuchó, hizo una señal con la cabeza para que se callasen y
se puso a hablar con Pajom en ruso.

—Pues claro —dijo—. Elige la que más te guste. Hay tierra de
sobra.

«Pero ¿cómo hago para coger toda la que quiera? —pensó Pajom—. Hay
que ponerlo por escrito de algún modo. De otro modo, pueden decirme que
es mía y luego quitármela».

—Os agradezco vuestras amables palabras —dijo—. Tenéis mucha
tierra y yo solo necesito una poca. Pero me gustaría saber cuál es mía.
Quisiera medirla de algún modo y poner por escrito que me pertenece.
Porque la vida y la muerte están en manos de Dios. Vosotros sois buenos
y me la dais; pero tal vez vuestros hijos me la quiten.

—Tienes razón —dijo el jefe—. Se puede poner por escrito.

—He oído que hace poco vino a veros un mercader —continuó Pajom—,
al que también ofrecisteis un poco de tierra y con el que firmasteis un
acta de compraventa. Me gustaría hacer lo mismo.

El jefe comprendió lo que quería.

—Se puede hacer así —dijo—. Tenemos un escribiente. Iremos a la
ciudad y pondremos todos los sellos necesarios.

—¿Y cuál será el precio? —preguntó Pajom.

—Tenemos un solo precio: mil rublos por jornada.

Pajom no comprendió.

—¿Qué clase de medida es una jornada? ¿Cuántas \emph{desiatinas}
tiene?

—Nosotros no sabemos contar de ese modo —dijo el jefe—. Vendemos
por jornadas. Toda la tierra que consigas recorrer en una jornada será
tuya, al precio de mil rublos.

Pajom se sorprendió.

—En un día entero se puede recorrer mucha tierra —dijo.

El jefe se echó a reír.

—¡Toda será tuya! —dijo el jefe—. Pero con una condición: si antes
del anochecer no has vuelto al punto de partida, perderás el dinero.

—¿Y cómo vamos a marcar los lugares por los que pase? —preguntó
Pajom.

—Nos colocaremos en el lugar de partida y nos quedaremos allí,
mientras tú vas y vuelves. Llevarás un azadón para hacer señales donde
sea necesario; harás un agujero en cada extremo y dejarás al lado un
montón de hierba; más tarde nosotros pasaremos con el arado de un
agujero a otro. Puedes hacer el recorrido que quieras, pero debes
regresar al punto de partida antes de que se ponga el sol. Todo el
terreno que logres abarcar será tuyo.

Pajom se puso muy contento. Decidieron empezar por la mañana temprano.
Estuvieron hablando un rato, tomaron más \emph{kumis}, comieron un poco
de cordero y volvieron a beber té. Cuando se hizo de noche, los
bashkirios ofrecieron a Pajom un lecho de plumas y se separaron.
Prometieron reunirse al amanecer, para llegar al lugar señalado antes de
la salida del sol.

\section*{\centering VII}

Pajom se tendió en el lecho de plumas, pero no pudo conciliar el sueño.
Seguía pensando en la tierra. «Marcaré una parcela muy grande. En una
jornada puedo recorrer unas cincuenta verstas. En esta época un día dura
tanto que parece un año. Y en cincuenta verstas hay un montón de tierra.
La peor la venderé o se la dejaré a los mujiks y yo me quedaré con la
mejor y la cultivaré con mis propias manos. Compraré dos bueyes para el
arado y contrataré al menos dos trabajadores; sembraré medio centenar de
verstas y dejaré el resto para que paste el ganado», pensaba.

Pajom no pegó ojo en toda la noche, pero justo antes del amanecer se
quedó adormilado y tuvo un sueño. Estaba tumbado en esa misma
\emph{kibitka} y oía que alguien se estaba riendo fuera. Quiso saber de
quién se trataba y se levantó. Cuando salió de la \emph{kibitka} vio al
jefe de los bashkirios; estaba sentado y, sujetándose la panza con las
dos manos, se balanceaba y se reía a carcajadas. Pajom se acercó y le
preguntó:

—¿De qué te ríes?

Entonces se dio cuenta de que no era el jefe de los bashkirios, sino el
mercader que había pasado recientemente por su casa y le había hablado
de esas tierras. Pero en cuanto le preguntó si llevaba mucho tiempo
allí, advirtió que ya no era el mercader, sino aquel mujik que se había
presentado en su casa mucho tiempo antes, procedente del Volga. Por
último vio que tampoco era el mujik, sino el diablo en persona, con
cuernos y pezuñas; estaba allí sentado, riéndose a carcajadas, delante
de un hombre descalzo, vestido solo con camisa y pantalón. Pajom miró
atentamente para ver quién era ese hombre y se dio cuenta de que estaba
muerto y de que era él. Se despertó horrorizado. «¡Hay que ver qué cosas
sueña uno!», pensó. Miró a su alrededor y a través de la puerta abierta
vio que empezaba a clarear. «Hay que despertar a la gente —se dijo—.
Es hora de partir». Se levantó, llamó a su trabajador, que dormía en el
carro, le ordenó que enganchara y se fue a despertar a los bashkirios.

—Ya es hora de que vayamos a la estepa a medir la tierra —dijo.

Los bashkirios se levantaron y se reunieron; al poco rato llegó también
el jefe. Entonces se pusieron a beber \emph{kumis} y ofrecieron té a
Pajom, pero este no quería perder más tiempo.

—Si hay que ir, vamos —dijo—. Ya es hora.

\section*{\centering VIII}

Los bashkirios se reunieron y partieron, unos montados a caballo y otros
en carros.

Pajom cogió un azadón y se instaló en su propio carro, acompañado de su
trabajador. Cuando llegaron a la estepa, empezaba a amanecer. Subieron a
una colina, que en bashkirio se llama \emph{shijan}. Se apearon de los
carros, descabalgaron y se reunieron. El jefe se acercó a Pajom y,
señalando la estepa con la mano, dijo:

—Toda la tierra que abarcas con la vista es nuestra. Elige la que
quieras.

Los ojos de Pajom resplandecieron: toda la tierra estaba cubierta de
hierba, era lisa como la palma de la mano y negra como la semilla de la
amapola; en las hondonadas se veían hierbas de distintas clases, que
llegaban hasta el pecho.

El jefe se quitó el gorro de piel de zorro y lo dejó en el suelo.

—Esta será la marca —dijo—. Partirás de aquí y aquí volverás. Y
toda la tierra que recorras será tuya.

Pajom sacó el dinero, lo puso sobre el gorro, se quitó el caftán y se
quedó solo con la chaqueta sin mangas; luego se ciñó bien el cinturón
bajo la panza, se estiró, se metió en el seno una bolsa de pan, ató al
cinto una garrafa de agua, se ajustó las botas, cogió el azadón de manos
de su trabajador y se dispuso a partir. Estuvo un momento pensando por
dónde empezar, pues toda la tierra le parecía buena. «Da lo mismo
—decidió—: iré hacia levante». Se colocó de cara al sol y,
desperezándose, esperó a que despuntase en el horizonte. «No debo perder
ni un segundo se dijo. Con la fresca se camina mejor». En cuanto surgió
el sol, Pajom se echó el azadón al hombro y se internó en la estepa.

Caminaba con paso intermedio, ni deprisa ni despacio. Después de
recorrer una versta, se detuvo, cavó un agujero, puso un montón de
hierba sobre otro para que se viese bien, y siguió adelante. Había
entrado en calor y se movía con mayor ligereza. Al cabo de un rato, cavó
otro agujero.

Pajom miró a su alrededor. A la luz del sol se veía bien la colina y la
gente que estaba allí, así como el destello de las ruedas de los carros.

Pajom intuyó que había recorrido ya unas cinco verstas. Sintió calor, se
quitó la chaqueta, se la echó al hombro y siguió adelante. Recorrió
otras cinco verstas. El calor apretaba. Echó un vistazo al sol: era hora
de desayunar.

«Ha transcurrido ya el primer cuarto de la jornada —se dijo Pajom—.
Aún es pronto para dar la vuelta. Voy a descalzarme». Se sentó, se quitó
las botas, se las ató al cinto y reemprendió la marcha. Ahora iba más
ligero. «Recorreré otras cinco verstas y luego giraré a la izquierda
—pensó—. Este lugar es muy bueno y da pena dejarlo. Cuanto más
avanzas, mejor es». Y siguió en línea recta. Cuando se volvió, apenas
pudo divisar la colina; los hombres parecían hormigas y se distinguía un
leve resplandor.

«Bueno —pensó Pajom—, por esta parte he cogido bastante; hay que
torcer. Además, estoy empapado en sudor y tengo sed». Se detuvo, cavó un
agujero un poco más grande, puso unos trozos de hierba, desató la
garrafa, bebió y giró a la izquierda. Después de mucho caminar, llegó a
un lugar cubierto de hierba más alta; el calor se volvió sofocante.

Empezaba a sentirse cansado; miró el sol y vio que era la hora de comer.
«Tengo que descansar un rato», pensó. Pajom se detuvo y se sentó. Comió
un poco de pan y bebió agua pero no se tumbó. «Si me tumbo, me quedaré
dormido», se dijo. Estuvo sentado un rato y luego reanudó la marcha. Al
principio caminaba a buen paso. La comida le había dado fuerzas. Pero
hacía muchísimo calor y tenía sueño. Sin embargo, siguió caminando,
mientras pensaba: «Aguanta unas horas y vivirás como un rey el resto de
tu vida».

Caminó también mucho en esa dirección y estaba ya a punto de girar a la
izquierda cuando vio que un poco más lejos había una hondonada húmeda;
le dio pena dejarla. «Ahí se dará bien el lino», se dijo. Y siguió en
línea recta. Atravesó la hondonada, cavó un agujero y torció, creando de
ese modo una segunda esquina. Se volvió a mirar la colina: el calor lo
había vuelto todo borroso; algo parecía estremecerse en el aire y a
través de la neblina, apenas se vislumbraba a los hombres: debían de
estar a quince verstas. «He cogido dos partes muy largas —pensó
Pajom—. Esta tiene que ser más corta». Caminó un poco en esa
dirección, apretando el paso. Echó un vistazo al sol: estaba empezando a
declinar, y de la tercera parte solo había recorrido dos verstas. Hasta
el lugar de partida quedaban unas quince. «No —pensó—, aunque quede
una parcela irregular, debo seguir en línea recta, sin coger demasiado.
De todas formas, tengo tierra de sobra». Cavó a toda prisa un agujero y
se dirigió en línea recta hacia la colina.

\section*{\centering IX}

Empezaba a sentirse cansado. Estaba empapado en sudor y tenía los pies
descalzos, llenos de heridas y magulladuras; las piernas apenas le
sostenían. Le hubiera gustado descansar, pero no podía, pues no llegaría
a tiempo antes del ocaso. El sol no esperaba; no hacía más que bajar y
bajar. «Ah —pensó—, ¿no me habré equivocado y habré abarcado
demasiado? ¿Y si no llego a tiempo?». Contempló la colina y echó un
vistazo al sol: quedaba mucho para llegar al punto de partida y el sol
estaba ya cerca del horizonte.

Siguió caminando, a pesar del cansancio, apretando cada vez más el paso.
Pero por más que andaba, seguía estando lejos. Finalmente echó a correr.
Arrojó la chaqueta, las botas, la garrafa y el gorro, quedándose solo
con el azadón, en el que se apoyaba. «Ah —pensó— he sido demasiado
codicioso y lo he echado todo a perder; no lograré llegar antes de la
puesta del sol». Y ese miedo hacía que respirara aún peor. Pajom corría,
con la camisa y los pantalones pegados al cuerpo por el sudor; tenía la
boca completamente seca. El pecho se le dilataba como el fuelle de una
fragua, el corazón le latía como un martillo y no sentía ni sus propias
piernas. Aterrorizado, Pajom pensó: «Mientras no muera de agotamiento».

Tenía miedo de morir, pero no podía detenerse. «He corrido tanto —se
dijo— que, si me detengo ahora, dirán que soy tonto». Siguió
corriendo; cuando llegó más cerca oyó que los bashkirios chillaban y
gritaban. Al oírlos, el corazón le latió aún más deprisa. Pajom hizo
acopio de sus últimas fuerzas y siguió corriendo, mientras el sol se
acercaba al horizonte, cubierto de niebla, grande, rojo, ensangrentado.
Estaba a punto de desaparecer, pero ya no le quedaba mucho para llegar
al punto de partida. Podía ver a los hombres en la colina, que agitaban
los brazos y le animaban. Distinguía el gorro de piel de zorro en el
suelo, con el dinero encima; el jefe estaba sentado en el suelo y se
sujetaba la panza con las manos. Pajom se acordó de su sueño: «Tengo
mucha tierra, pero quién sabe si Dios me dejará vivir en ella
—pensó—. Ah, estoy perdido. No llegaré a tiempo».

Echó un vistazo: el sol había alcanzado la tierra; una de sus partes
había desaparecido ya y la otra se recortaba como un arco contra el
horizonte. Con las últimas fuerzas que le quedaban, Pajom aceleró el
paso, inclinando tanto el cuerpo hacia delante que las piernas apenas
conseguían seguirlo y a cada paso estaba a punto de caer. Justo cuando
llegaba a la colina, se hizo de noche. Miró a su alrededor y vio que el
sol ya se había puesto. Pajom gimió. «Todos mis esfuerzos han sido en
vano». Estuvo a punto de detenerse, pero oyó que los bashkirios
continuaban chillando; entonces se dio cuenta de que, aunque allí abajo
reinaba la oscuridad, desde lo alto de la colina aún podía verse el sol.
Pajom tomó aliento y subió corriendo por la ladera. En lo alto aún había
luz. Lo primero que vio fue el gorro. Delante de él estaba sentado el
jefe, riéndose a carcajadas y sujetándose la panza con las manos. Pajom
se acordó de su sueño y gimió; las piernas le fallaron, cayó de bruces y
alcanzo el gorro con las manos.

—¡Bravo! —gritó el jefe—. ¡Has ganado mucha tierra!

El trabajador de Pajom se acercó corriendo y quiso levantarlo, pero un
reguero de sangre le corría por la boca: estaba muerto.

Los bashkirios chasquearon la lengua para expresar su tristeza.

El trabajador cogió el azadón, cavó una tumba lo suficientemente grande
para alojar a su amo y lo enterró. Tres \emph{arshines} de la cabeza a
los pies le bastaron.
\end{document}
\hypersetup{
    pdftoolbar=true,        % show Acrobat’s toolbar?
    pdfmenubar=true,        % show Acrobat’s menu?
    pdffitwindow=false,     % window fit to page when opened
    pdftitle={title},    
    pdfsubject={Subject},   % subject of the document
    pdfcreator={Salvatore Mazzarino},   % creator of the document

    pdfnewwindow=true,      % links in new window
    colorlinks=false,       % false: boxed links; true: colored links
    linkcolor=red,          % color of internal links (change box color with linkbordercolor)
    citecolor=green,        % color of links to bibliography
    filecolor=magenta,      % color of file links
    urlcolor=cyan           % color of external links
}